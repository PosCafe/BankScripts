\documentclass[12pt,a4paper,tocpage=plain,floatnumber=continuous,chapter=TITLE,appendix=nobox,font=plain, header=ruled,hyperindex=false]{abnt}
\usepackage[utf8x]{inputenc}
\usepackage[portuges, brazil]{babel}
\usepackage{tikz}
\usetikzlibrary{positioning}
\usetikzlibrary{shadows}
\usepgflibrary{arrows}
\usepackage{bm}
\usepackage{amsmath}
\usepackage{amsfonts}
\usepackage{amssymb}
\usepackage{lmodern}
\usepackage{courier}
\usepackage{indentfirst}
\usepackage{latexsym,amstext,amsxtra,amsopn}
\usepackage{hyperref}
\usepackage[alf,abnt-full-initials=no,abnt-etal-list=0,abnt-emphasize=bf,abnt-thesis-year=final,abnt-year-extra-label=yes,abnt-etal-cite=2]{abntcite}
\usepackage{graphicx}
\usepackage[font=small,labelfont=small,textfont=small]{caption}
\usepackage[singlelinecheck=true,margin=10pt,font=small]{subfig}
\usepackage{rotating}
\usepackage{marvosym}
\usepackage{remreset}
\usepackage{setspace}
\usepackage{listings}
\usepackage{adjustbox}
\usepackage{multirow}
\usepackage{pdfpages}
%\usepackage[hang,center]{subfigure}
\usepackage{makeidx}
\usepackage{tabularx,array,colortbl}
\usepackage[section]{placeins} %too many floats!!
%\usepackage[table]{xcolor}
%\usepackage[style=list,number=none,border=none,header=none,cols=2]{glossary}
\usepackage[style=long,number=none,border=none,header=none, cols=2 ]{glossary}
%\usepackage{glossaries}
%\usepackage[subentrycounter,seeautonumberlist]{glossaries}
%\usepackage{lineno}
%Numeração das equações
\usepackage{remreset}
\makeatletter\@removefromreset{equation}{chapter}\makeatother

%defini glossarios
\newglossarytype[agl]{sigla}{sig}{sgl}
\newglossarytype[sgl]{simbolo}{sim}{sbl}
%\newglossary[agl]{sigla}{sig}{sgl}{Siglas}
%\newglossary[sgl]{simbolo}{sim}{sbl}{Simbolos}

%\usepackage{geometry}
%\geometry{tmargin= 3cm,bmargin= 2cm,lmargin= 3cm,rmargin=2cm}
\makeglossary
%\makeglossaries
\makeindex
\makeatletter
\@removefromreset{footnote}{chapter}
\makeatother
\author{Evandro Moimaz Anselmo}
\renewcommand{\ABNTchapterfont}{\bfseries}
\renewcommand{\ABNTchaptersize}{\normalsize}
\renewcommand{\ABNTanapsize}{\normalsize}
\renewcommand{\ABNTsectionfontsize}{\normalsize} 
\renewcommand{\ABNTsectionfont}{\rm} 
\renewcommand{\ABNTsubsectionfontsize}{\normalsize}
\renewcommand{\ABNTsubsectionfont}{\rm\bfseries}
\renewcommand{\captionfont}{\small}
%%códigos fontes
%\renewcommand{\lstlistingname}{Código fonte}
%\renewcommand{\lstlistlistingname}{Lista de Códigos Fonte}
%\renewcommand*\thelstlisting{\arabic{lstlisting}}
\renewcommand{\theequation}{\arabic{equation}}

%\renewcommand{\descriptionwidth}{0.85\textwidth}
%\renewcommand{\glossaryalignment}{\textwidth}
\hypersetup{pdfborder=000,colorlinks=false,pdftitle={Morfologia das tempestades elétricas na América do Sul}, pdfcreator={Evandro Moimaz Anselmo}, pdfproducer={Evandro Moimaz Anselmo}, pdfkeywords={Raios, tempestades elétricas, precipitação tridimensional}}
%\hypersetup{colorlinks, citecolor=black,filecolor=black,linkcolor=black,urlcolor=black}
%\hypersetup{colorlinks=false}

%\hypersetup{backref=true,pdfpagemode=UseOutlines,colorlinks=true,a5paper,breaklinks=true,hyperindex,linkcolor=blue,
%anchorcolor=black,citecolor=green,filecolor=magenta,menucolor=red,pagecolor=red,urlcolor=cyan,bookmarks=true,
%bookmarksopen=true,pdfpagelayout=SinglePage,pdfpagetransition=Dissolve}
\hypersetup{bookmarksopen=true,backref=true,pdfpagemode=true,hyperindex,bookmarks=true}

\lstset{ 
language=SQL,                % choose the language of the code
extendedchars=true,
basicstyle=\scriptsize,       % the size of the fonts that are used for the code
numbers=left,                   % where to put the line-numbers'z
numberstyle=\footnotesize,      % the size of the fonts that are used for the line-numbers
stepnumber=2,                   % the step between two line-numbers. If it's 1 each line will be numbered
numbersep=8pt,                  % how far the line-numbers are from the code
numberstyle=\tiny,
%backgroundcolor=\color{white},  % choose the background color. You must add \usepackage{color}
showspaces=false,               % show spaces adding particular underscores
showstringspaces=false,         % underline spaces within strings
showtabs=false,                 % show tabs within strings adding particular underscores
frame=trBL,			% adds a frame around the code
frameround=fttt,
tabsize=2,			% sets default tabsize to 2 spaces
captionpos=b,			% sets the caption-position to bottom
breaklines=true,		% sets automatic line breaking
breakatwhitespace=false,	% sets if automatic breaks should only happen at whitespace
escapeinside={\%*}{*)},        % if you want to add a comment within your code
basicstyle=\fontfamily{pcr}\fontseries{m}\selectfont\footnotesize,
commentstyle=\ttfamily\color{gray}
%columns=fullflexible
}
%%%
\setcounter{table}{0}
\setcounter{figure}{0}
\begin{document}
%\linenumbers
\autor{Evandro Moimaz Anselmo}
\titulo{Morfologia das tempestades elétricas na América do Sul}
\orientador[Orientador:\\]{Prof. Dr. Carlos Augusto Morales Rodriguez }

\comentario{Tese ao departamento de Ciências Atmosféricas, realizada como pré-requisito para obtenção do título de Doutor em Ciências.}
\instituicao{Instituto de Astronomia, Geofísica e Ciências Atmosféricas da Universidade de São Paulo}
\local{São Paulo - SP}
\data{2014}
\capa
\folhaderosto
%\begin{folhadeaprovacao}
%Dissertação titulada como \textit{Estudo da razão entre o número de relâmpagos intranuvens e nuvem-solo para sistemas convectivos na cidade de Campo Grande - MS}, defendida por Evandro Moimaz Anselmo e aprovada em 29 de Maio de 2009 na cidade de Campo Grande - MS, Universidade Federal de Mato Grosso do Sul, Departamento de Física, pela seguinte banca examinadora: 
%\setlength{\ABNTsignthickness}{0.4pt}
%\setlength{\ABNTsignskip}{2cm}
%\assinatura{Prof. Dr. Widinei Alver Fernandes\\ Universidade Federal de Mato Grosso do Sul \\ Orientador}
%\assinatura{Prof. Ph. D. Moacir Lacerda\\ Universidade Federal de Mato Grosso do Sul \\ Co-orientador}
%\assinatura{Prof. Ph. D. Carlos Augusto Morales \\ Universidade de São Paulo}
%\assinatura{Prof. Dr. Roberto Ferreira dos Santos\\ Universidade Federal de Mato Grosso do Sul}
%\end{folhadeaprovacao}
\pretextualchapter{}
\vfill
\begin{flushright}
\textit{\large Tese}
\end{flushright}

\pretextualchapter{AGRADECIMENTOS}


\begin{resumo}
Resumo

\bigskip
\bigskip
Palavras-chave: relâmpagos, tempestades, monitoramento.
\end{resumo}

\begin{abstract}
\paragraph*{}
Abstract

\bigskip
\bigskip
Key-words: lightning, storms, tracking.
\end{abstract}

\listoffigures
\listoftables
%\lstlistoflistings
%\sumario
\cleardoublepage
\phantomsection
%\ABNTaddcontentsline{lot}{chapter}{SUMÁRIO}
\tableofcontents


\makesigla
\renewcommand{\glossaryname}{LISTA DE SIGLAS E ABREVIATURAS}
\cleardoublepage
\phantomsection
\printsigla
\makesimbolo
\renewcommand{\glossaryname}{LISTA DE S\'{I}MBOLOS}
\cleardoublepage
\phantomsection
\printsimbolo

%\printglossaries
\chapter{INTRODUÇÃO}

A região tropical das Américas, África e Continente Marítimo são consideradas como as chaminés globais de descargas elétricas atmosféricas -- raios. Em \citeonline{whipple1929} já se observava que no horário de maior atividade de tempestades elétricas sobre a América do Sul (AS), é quando há a maior intensidade do campo elétrico de bom tempo. Portanto a América do Sul é a chaminé dominante no processo de manutenção do circuito elétrico atmosférico global \cite{williams2004}.


%As tempestades elétricas são observadas em regiões aonde ocorre levantamento de ar por convecção, pela topografia elevada, propagação de frentes quando combinados com mecanismos de suporte de umidade como aumento da temperatura superficial de oceanos, brisa marítima, brisa de rio e evapotranspiração de florestas.
%Observando a climatologia da ocorrência de descargas elétricas atmosféricas sobre a Terra pelos diversos sistemas de detecção de descargas em operação, STARNET, WWLN, WSI, LIS-TRMM, … é evidente que as tempestades elétricas concentram-se sobre os continentes, indicando que a convecção e a topografia são fatores dominantes que contornam a problemática da microfísica da eletrificação atmosférica.
%A região tropical da América do Sul, Africa e Continente Marítimo são também conhecidas como chaminés globais de descargas elétricas atmosféricas. Whipple, F.J.W., (1929) já observava que a América do Sul é a chaminé dominante para a manutenção do Circuito Elétrico Atmosférico Global. 


Em trabalhos que mostram a densidade global de raios para a região tropical conforme \citeonline{albrecht2009tropical,cecil2014gridded}, observa-se que as maiores densidades de raios do globo ocorrem sobre a América do Sul e África.

Estando os países da América do Sul situados em uma das regiões de maior atividade elétrica atmosférica do globo, saber quando e aonde as tempestades elétricas ocorrem, bem como, quais os locais em que os sistemas são mais eficientes na produção de raios, torna-se fundamental para o planejamento da infraestrutura dos países Sul-americanos, no sentido de garantir segurança no transporte aéreo, fluvial e terrestre, nas linhas de transmissão de dados e de energia elétrica, etc, setores estratégicos que quando paralisados devidos aos danos causados pela queda de raios refletem em prejuízos em cascata em todos os setores econômicos. 

Por exemplo, uma falha no sistema de distribuição de energia elétrica pode cessar a energia elétrica de um bairro, cidade, etc. Pode causar queima de equipamentos eletroeletrônicos devido sobre tensão elétrica, causar quedas na rede de internet, o que pode paralisar  setores como: educação, pesquisa, comércio e industrias. Também gera grande número de ações judiciais indenizatórias contra as operadoras do sistema de energia, sobrecarregando o sistema judiciário. No Brasil, as empresas prestadoras de serviços de  fornecimento de energia elétrica ao consumidor lideram as reclamações nos PROCONs ao lado de empresas de telecomunicações, evidenciando a falta de infraestrutura destes setores. Em \citeonline{pinto2005arte,noticiainpe2007}, estima-se prejuízos na ordem de 1 bilhão de dólares anuais em função da densidade de raios observada apenas sobre o Brasil. 
  
%e melhor lidar com problemas como enchentes rápidas, chuvas 
%de granizo, tornados e visar uma melhor forma de gerir recursos naturais.
%Williams E. R. e Sátori G., (2004) buscaram entender a maior resposta da Curva de Carnegie associada a atividade de tempestades na América do Sul fazendo um estudo comparativo entre as regiões da bacia Amazônica e bacia do Congo. Sobre a maior bacia hidrográfica do Continente Africano, as taxa de raios por km$^2$ por ano são maiores enquanto que os sistemas precipitantes sobre a bacia Amazônica, observa-se menor densidade de raios porém maior volume de chuva, indicando que as formações estratiformes no continente sul-americano também funcionam como baterias do Circuito Elétrico Global, em que carga negativa é transferida para a Terra por meio das gotas de chuva carregadas (SOULA et al., 2003).


No entanto, é possível presenciarmos situações de chuva torrencial porém sem que haja raios, pois a capacidade de gerar raios em uma tempestade elétrica não depende apenas dos processos de nucleação e colisão coalescência, depende de como ocorrerá o crescimento de cristais de gelo na nuvem, exigindo condições atmosféricas em que a água possa coexistir na fase líquida, gasosa e sólida (região mista), que são possíveis em regiões atmosféricas com temperaturas entre 0 $^{\circ}$C e -40 $^{\circ}$ \cite{Takahashi1978,williams1991mixed,korolev2007}.

Desta forma, o processo de eletrificação dos hidrometeoros até a formação de um raio, depende da capacidade do ar quente e úmido da superfície romper a estabilidade atmosférica e atingir altitude entre 4--15 km, regiões acima da isoterma de 0 $^{\circ}$C. Portanto, as tempestades elétricas podem indicar condições dinâmicas e termodinâmicas associadas a convecção profunda na atmosfera \cite{doswell2001,zipser2006}.

Além dos raios que atingem o solo -- raios nuvem-solo -- causando danos a sociedade, trabalhos como em \citeonline{macgorman1989,carey1998,williams1999} mostram que condições de tempo severo como: frentes de rajadas com velocidade superior a 92.6 km h$^{-1}$, queda de granizo com diâmetro maior do que 1.9 cm ou tornados, são geralmente precedidas de um salto na taxa de raios das tempestades elétricas governado por raios que não atingem o solo -- raios intra-nuvens -- indicando intenso crescimento de hidrometeoros na região mista.      

Neste sentido, técnicas de seleção de sistemas meteorológicos a partir de dados de sensoriamento remoto, combinadas com o monitoramento da taxa de raios totais\footnote{O termo raios totais faz referencia a todos os tipos de raios, tanto os raios intranuvens quanto os nuvem-solo} dos sistemas são de grande importância na identificação e previsão de curto prazo de tempo severo. 

Diversos estudos definiram sistemas meteorológicos fazendo o agrupamento de regiões na superfície a partir de limiares de temperatura de brilho observadas em satélite. \citeonline{mapes1993}, utilizaram esta metodologia e também fazem uma síntese de trabalhos que buscaram selecionar clusters de nuvens a partir de limiares de temperatura de brilho em infravermelho, como por exemplo \citeonline{Maddox1980} que observou a ocorrência de duas regiões: uma com temperatura de brilho $\leqslant$ -32°C (241K) e área $\geqslant$ 100,000 km$^2$;  outra região menor, no interior da região maior, com temperatura de brilho $\leqslant$ -52°C (221K) e área $\geqslant$ 50,000 km$^2$, associadas com Sistemas Convectivos de Meso-escala (SCM) nos Estados Unidos.

\citeonline{morales2003} desenvolveram um algoritmo hidro-estimador estudando regiões de temperatura de brilho em infravermelho e a precipitação observada pelo radar a bordo do satélite \textit{Tropical Rainfall Measuring Mission} -- TRMM. Dados da \textit{Sferics Timing and Ranging Network} (STARNET) foram utilizados e clusters com raios e sem raios foram identificados. Foi observado que as descargas localizadas pela STARNET, em 90\% dos casos, estiveram associados a regiões com temperatura de brilho menores do que 258 K.


\sigla{name={TRMM},description={\textit{Tropical Rainfall Measuring Mission}}}
\sigla{name={STARNET},description={\textit{Sferics Timing and Ranging Network}}}

%Porém a radiação infravermelha observada por satélites, corresponde apenas a irradiação do topo das nuvens. Nuvens finas, com formação acima da isoterma de 0°C, como por exemplo as nuvem cirrus, podem cobrir grandes extensões e não estar associadas a precipitação nem descargas elétricas.


Em \citeonline{houze1993} Linhas de Instabilidades (LI) foram definidas observando extensões com chuva contínua observada por radar. Em \citeonline{MohrZipser1996} Sistemas Convectivos de Meso-escala (SCM) sobre os trópicos foram observados a partir do espalhamento radiativo em micro-ondas (85-GHz PCT), em que regiões contínuas $\geqslant$ 2000 km$^2$ com PCT $\leqslant$ 250 K foram os principais critérios para a identificação dos sistemas.

\sigla{name={SCM},description={\textit{Sistema Convectivo de Meso-escala}}}
\sigla{name={LI},description={\textit{Linha de Instabilidade}}}

Combinando dados do PR e TMI abordo do TRMM, \citeonline{Nesbitt2000} desenvolveu uma metodologia para selecionar sistemas precipitantes denominados como \textit{Precipitation Features} (PF), em que o principal critério de seleção foi identificar área contínua de chuva na superfície, seja estimada pelas observações de radar ou micro-ondas quando os sistemas estiveram fora da varredura do PR.

\sigla{name={PF},description={\textit{Precipitation Features}}}


\citeonline{cecil2005} mostram que apenas 2.4\% das PFs observadas pelo TRMM em todo globo entre 12/1997--10/2000 possuíram atividade elétrica. Na AS as PFs classificadas com as maiores taxas de raios, concentraram-se na região da Bacia do Prata associadas a Sistemas Convectivos de Meso-escala \cite{Velasco1987,Durkee2009}.   


Com base também nos dados das PFs,  \citeonline{zipser2006} identificou os locais das tempestades mais severas em um estudo global. Apenas 0.1\% da amostragem das PFs em um período de 7 anos foram observadas taxa de raios acima de 32.9 por minuto.  Entre as regiões do globo com as maiores concentrações de PFs que indicaram valores extremos (0.001\%), seja de taxa de raios, seja de mínimas temperaturas de brilho (85 e 37 GHz) ou de máxima altitude com 40 dBZ de fator de refletividade do radar, encontram-se na região Sul da AS que engloba a Bacia do Prata e o extremo Norte da Cordilheira dos Andes que abrange a Colômbia e região do Lago Maracaibo na Venezuela. 
%Conforme aumenta a intensidade das tempestades elétricas, menor será sua probabilidade de ocorrência. Apenas 0.1\% das PFs observou-se taxa de raios acima de 32.9 por minuto. 

Nesta tese, faz-se a identificação de sistemas que possuem atividade elétrica, ou seja, tempestades elétricas, apenas sobre a América do Sul  a partir das observações orbitais do satélite TRMM, mais especificamente do sensor de raios (LIS), radiômetro no infravermelho (VIRS) e o radar de precipitação (PR) a bordo do da satélite entre os anos de 1998 e 2011. Desta forma cria-se um banco de dados de tempestades elétricas do TRMM.

Com este subconjunto de dados do TRMM, é estudada a sazonalidade e o ciclo diurno das tempestades elétricas, bem como a densidade geográfica de raios e de tempestades elétricas, buscando evidenciar regiões ou estações do ano em que as tempestades elétricas possuem processo de eletrificação mais eficientes.

A intensidade das tempestades elétricas é estudada com base na taxa de raios e aspectos morfológicos como: dimensões relacionadas a sua extensão e a estrutura tridimensional da precipitação observada pelo PR, buscando identificar qual é a taxa de raios que corresponde potencialmente a condições de convecção profunda e consequentemente de tempo severo, conforme cada localidade da extensa região da AS.








%------------------------------------------------------\\
%\textit{Acho que essa parte de baixo é Metodologia... ou discussão dos resultados}

%A grande extensão territorial do Brasil na America do Sul  
%m base nos experimentos do LBA, investigar a morfologia dos sistemas na pré monsão, na região amazônica. 

%As descargas na região amazônica parecem estar mais associadas com os processos de inibição do que precipitação... Em uma região tropical dominada por processos quentes, as descargas podem indicar inibição de colisão coalescência, desenvolvimento de fase fria e menos precipitação.

%mas no período úmido de leste raios relacionam-se mais com a precipitação.

%...A Rachel já fez uma boa discussão sobre a microfísica dos sistemas da amazônia, períodos seco úmido e de transição. As tempestades foram organizadas em clusters, estudou-se o ciclo de vida, a ocorrência de raios em áreas desmatadas e com floresta/outras, e foi explicado a microfísica dos sistemas basicamente com: taxa de raios positivos e negativos, eco tops, VIL, CAPE, CINE. Falta explotar a os CFADS para essa região. Como varia a probabilidade de ocorrência por altitude dos perfis de refletividade nos períodos seco de transição e úmido.

%Em desenvolvimento ...
%As tempestades mais eficientes estão mostrando tamanhos diversos. Tanto cluster grandes quanto pequenos podem ser eficientes. Não é uma relação que depende apenas da área, ou da fração convectiva ou estratiforme. Os raios relacionam-se com o ciclo de vida, que no caso é aleatório. Depende do ciclo de vida

%Relação exponencial entre max fl no pixel versus fl-rate/km2. Maior a concentração de raios em um único pixel, menos eficiente e mais chuva. Talvez uma reintensificação do sistema maduro. Mas existem duas categorias de sistemas :
%1 – os mais eficiente com área ~10^3 e chuva ~10^5
%2 – as com maior fl/rate no pixel, menos eficientes e com chuva ~10^9



%Com o experimento de campo LBA (Large-Scale Biosphere-Atmosphere Experiment in Amazonia) realizado na região de Rondônia entre janeiro e fevereiro de 1999, foi possível identificar alguns fatores importantes que regulam a precipitação na região Amazônica. Além disso o LBA foi importante para validação de dados do satélite TRMM que são amplamente utilizados nesta pesquisa \cite{silva2002lba,williams2002,albrecht2011}.

%Silva Dias M. A. F. et al, (2002), fazem uma síntese dos principais resultados e objetivos do LBA, entre estes destaco os estudos de Anagnostou e Morales, (2002), \citeonline{Carvalho2002} que mostram dois regimes de vento em 700 mb, de Leste e de Oeste, em que observou-se maior precipitação convectiva e atividade elétrica durante o regime de ventos de Leste. Petersen W. A. et al, (2002), investigaram como que esses dois regimes de vento (Leste-Oeste) observados durante o LBA em Rondônia, influenciam no número de descargas elétricas observadas pelo LIS (Ligthning Image Sensor), não apenas para região Amazônica mas para toda a América do Sul durante 4 verões entre 1997 e 2000.

%A variação intra-sazonal da atividade elétrica durante o período chuvoso mostrou-se evidente. \citeonline{petersen2002trmm},  identificaram regiões de extremos opostos de atividade elétrica que devem estar associados ao mecanismos de manutenção da monção na América do Sul, principalmente com a dinâmica que envolve Zona de Convergência do Atlântico Sul (ZCAS) \cite{CarvalhoJones2002,Carvalho2002}.   

\section{FUNDAMENTOS DA OBSERVAÇÃO DA PRECIPITAÇÃO POR RADAR}
\label{chuvaEtemperatura}

A medida da precipitação por radar, consiste na emissão de um feixe eletromagnético ($P_t$) e na análise da potência do eco ($P_r$) gerado pelo sinal emitido.

O sistema eletrônico de um radar envia um feixe eletromagnético periódico com frequência e a largura ($H$) de pulso definidos de modo a avaliar o eco de obstáculos em distâncias distintas. Assim pode-se distinguir o eco do sinal emitido para diferentes elementos de volume ($\Delta V$) do feixe do radar.

Um elemento de volume ($\Delta V$) de um feixe de radar está associado com a resolução radial da medida, que depende de H, pois a largura do pulso (H) é proporcional a distância mínima ($\frac{H}{2}$) para que dois alvos sejam distinguíveis.
 
Considerando um feixe eletromagnético com azimute de $\phi$ na vertical e $\varphi$ na horizontal, pode-se considerar que $\Delta V$ a uma distância $r$ do radar terá o volume na forma do cilindro elíptico

\begin{equation}
\Delta V = \pi \dfrac{r\phi}{2} \dfrac{r\varphi}{2} \dfrac{H}{2} .
\end{equation}
 
A potência recebida $P_r$ do sinal espalhado referente a ao volume iluminado $\Delta V$, pode ser expressa como
\begin{equation}
P_r = \dfrac{P_t G^2 \lambda^2 \phi \varphi H}{ 512 (2\ln2)\pi^2 r^2} \sum_{i=1, 2, 3 ... }^{\Delta V} \sigma_i ,
\label{radar1}
\end{equation}
em que, $G$ é o ganho da antena, $\lambda$ o comprimento de onda, $r$ a distância do alvo, $\sigma$ a seção transversal de retro-espalhamento das partículas de nuvem e o fator $2\ln2$ corresponde ao efeito de lóbulo no sinal eletromagnético captado \cite{battan1973}.

A Refletividade do Radar 
\begin{equation}
\eta = \sum_{i=1, 2, 3 ... }^{V_{m}} \sigma_i,
\label{refletividade}
\end{equation}
é a somatória da seção de retro-espalhamento dos hidrometeoros a cada elemento de volume iluminado $\Delta V$ (a cada $gate$) do radar.

Considerando que o parâmetro de tamanho ($\alpha$), que é a relação entre a área da seção transversal do hidrometeoro ($2\pi R_{h}$) precipitável na atmosfera e o comprimento de onda da radiação emitida pelo radar
\begin{equation}
\alpha = \dfrac{2\pi R_{h} }{\lambda},
\label{parametroTam} 
\end{equation}
corresponda a um valor de $\alpha$ $<<$ 1, então o espalhamento do feixe do radar pode ser considerando como Rayleigh. Neste caso, a seção transversal de retro-espalhamento pode ser escrita como 
\begin{equation}
\sigma = \dfrac{\lambda^2 \alpha^6}{\pi} K^2,
\label{sigma}
\end{equation}
em que $K^2$ corresponde ao índice de refração dos hidrometeoros. Como a equação \ref{parametroTam} depende do raio $R_h$, podemos reescrever a equação \ref{sigma} considerando o diâmetro 
\begin{equation}
D_h = 2R_h.
\label{diametro}
\end{equation}

Então, substituindo as equações \ref{parametroTam} e \ref{diametro} em \ref{sigma}, obtemos que
\begin{equation}
\sigma = \dfrac{\pi^5 K^2  }{ \lambda^4 } D_h^6.
\label{sigma2}
\end{equation}

Considerando o hidrometeoro como esférico, podemos relacionar o $D_h$ com a sua quantidade de massa, sendo  
\begin{equation}
D_h = \left( \dfrac{6 M_h}{\pi \rho} \right)^{\frac{1}{3}},
\label{dh}
\end{equation}
em que, $\rho$ é a densidade e $M_h$ a massa do hidrometeoro.

Então substituindo a equação \ref{dh} em \ref{sigma2}, obtém-se
\begin{equation}
\sigma = \dfrac{36 \pi^3 K^2  }{ \lambda^4 \rho^2 M_h^2} .
\label{sigma3}
\end{equation}

Portanto, observe que a Refletividade do Radar (equação  \ref{refletividade}), depende  de uma relação entre quantidade de massa $M_h$, densidade $\rho$ e também do índice de refração $K^2$ dos hidrometeoros, conforme mostra a equação \ref{sigma3}.

Porém o radar mede apenas a potência do sinal retro-espalhado ($P_r$). Pode-se ter uma ideia da concentração de obstáculos espalhadores, porém não podemos ter certeza a respeito da massa, índice de refração e densidade dos obstáculos. 

O que se faz para estimar a chuva é considerar que todos os obstáculos associados ao espalhamento do feixe do radar são gotas de água esféricas. Neste caso, podemos combinar as equações \ref{radar1} e \ref{sigma3}, obtendo
%\begin{equation}
%P_r = V_{m} \dfrac{P_t G^2 \lambda^2}{2\ln2(4\pi)^3 r^4}   \dfrac{\pi^5 K^2  }{ \lambda^4 } \sum_{i=1, 2, 3 ... }^{V_{m}}  D_{h_i}^6.      
%\end{equation} 
%\begin{equation}
%P_r = \dfrac{P_t G^2 \lambda^2 \phi \varphi H}{ 512 (2\ln2)\pi^2 r^2} \dfrac{\pi^5 K^2  }{ \lambda^4 }  \sum_{i=1, 2, 3 ... }^{\Delta V}  D_{h_i}^6.   
%\end{equation} 
\begin{equation}
P_r = \dfrac{\pi^3 P_t G^2  \phi \varphi H }{ 512 (2\ln2) \lambda^2 } \dfrac{ K^2  }{ r^2 }  \sum_{i=1, 2, 3 ... }^{\Delta V}  D_{h_i}^6, 
\end{equation} 
sendo o Fator de Refletividade do Radar
\begin{equation}
Z =  \sum_{i=1, 2, 3 ... }^{\Delta V}  D_{h_i}^6.
\label{fz}
\end{equation}
Observe que a cada medida de $P_r$, algumas variáveis serão sempre constantes 
\begin{equation}
C = \dfrac{\pi^3 P_t G^2  \phi \varphi H }{ 512 (2\ln2) \lambda^2 } .
\end{equation}
Então,
\begin{equation}
P_r = C \dfrac{K^2}{r^2}  Z.
\end{equation}

Desta maneira, a partir da $P_r$ pelo radar, sabendo a distância $r$ do alvo e considerando que a chuva é composta de gotas esféricas de água, então podemos saber $Z$ 
\begin{equation}
Z = \dfrac{P_r r^2}{C K^2},
\label{zsimples}
\end{equation}
pois, $K^2=0.931$ para água líquida.


Conforme mostra a equação \ref{fz}, $Z$ depende de $D_h$, portanto conforme a equação \ref{dh}, podemos relacionar o diâmetro $D_h$ dos hidrometeoros com a massa ou o volume de chuva.  

É conveniente converter o Fator de Refletividade do Radar ($Z$), para uma escala em decibel, portanto os valores de eco de radar geralmente são expressos em unidade de decibel do Fator de Refletividade (dBZ). 

Aplicando $10\log_{10}$ na equação \ref{zsimples}, temos que

\begin{align}
10\log_{10}(Z)  &= 10\log_{10} \left( \dfrac{P_r r^2}{C K^2} \right)\\
dBZ &= 10\left[ \log_{10}(P_r r^2) - \log_{10}(C K^2)     \right]\\
      &=  10 \lbrace \log_{10}(P_r)+ 2\log_{10}(r) - [ \log_{10}(C) + \log_{10}(K^2)]  \rbrace,
\end{align}
portanto,
\begin{equation}
dBZ =  10\log_{10}(P_r) + 20\log_{10}(r) - 10\log_{10}(C) - 10\log_{10}(K^2)
\end{equation}

\section{FUSÃO DO GELO DE NUVEM E O CRESCIMENTO DOS HIDROMETEOROS}

Ao observar a precipitação por radar no perfil atmosférico, a varredura do feixe irá atingir regiões de altitude com temperaturas abaixo de 0 $^{\circ}$C. Nestes casos, a potência $P_r$ estará associada ao espalhamento em gelo de nuvem. Para quantificar o gelo precipitando, podemos considerar que 
\begin{equation}
dBZ_{\mathrm{gelo}} =  10\log_{10}(P_r) + 20\log_{10}(r) - 10\log_{10}(C) - 10\log_{10}(K_{\mathrm{gelo}}^2).
\label{zg}
\end{equation}

Portanto quando o feixe espalha em altitudes abaixo da isoterma de 0 $^{\circ}$C, pressupõe-se que $P_r$ estará associada a quantidade de água líquida. Então, para quantificar a água precipitando, podemos considerar que
\begin{equation}
dBZ_{\mathrm{agua}} =  10\log_{10}(P_r) + 20\log_{10}(r) - 10\log_{10}(C) - 10\log_{10}(K_{\mathrm{agua}}^2).
\label{za}
\end{equation}

Subtraindo as equações \ref{zg} e \ref{za}, temos
\begin{equation}
dBZ_{\mathrm{gelo}} - dBZ_{\mathrm{agua}} = - 10\log_{10}(K_{\mathrm{gelo}}^2 )+ 10\log_{10}(K_{\mathrm{agua}}^2).
\end{equation}

Sabendo que $K_{\mathrm{agua}}^2 = 0.931$ e $K_{\mathrm{gelo}}^2= 0.197$, então 
\begin{equation}
dBZ_{\mathrm{gelo}} - dBZ_{\mathrm{agua}} =   6.7dBZ,
\end{equation}
mostrando que devido ao índice de refração do gelo ser menor do que o índice de refração da água ($K_{\mathrm{agua}}^2 > K_{\mathrm{gelo}}^2$), ao considerar $K_{\mathrm{gelo}}^2$, conforme mostra a equação \ref{zg}, haverá um acréscimo de potência de  6.7 dBZ em relação a considerar $K_{\mathrm{agua}}^2$, como mostra a equação \ref{za}.

Porém, na observação tridimensional da precipitação, podemos conhecer as distâncias $r$ dos alvos espalhadores, mas não podemos afirmar sobre a temperatura da atmosfera para cada distância $r$ do radar, bem como se haverá água super-resfriada acima de 0 $^{\circ}$C ou gelo sólido caindo na superfície. Então os dados brutos das observações de radar, consideram $K^2$ como constante, geralmente $K_{\mathrm{agua}}^2 = 0.931$  para toda a estrutura tridimensional da precipitação observada.

Portanto em observações de radar no perfil de altitude, a região ou camada de derretimento (fusão do gelo) é bastante marcada, pois, identifica-se um aumento de $\simeq$7 dBZ no Fator de Refletividade do Radar devido a mudança do índice de refração.

Em \citeonline{Fabry1995}, é mostrado que processos como a agregação, acreção e colisão coalescência, podem ser estudados em função da espessura da camada de derretimento e flutuações nos valores do Fator de Refletividade $Z$ no perfil atmosférico. 

A espessura da camada de derretimento está relacionada com o lapse-rate da atmosfera \cite[p.~462]{mason1971_2ed}. Em uma atmosfera instável, com e convecção profunda e precipitação convectiva, a camada de a transição de fase de gelo para a água liquida é perturbada por correntes ascendentes. A mudança do índice de refração da água não ocorre apenas em torno de 0 $^{\circ}$C, pois no ambiente convectivo teremos água super-resfriada em temperaturas de -15 $^{\circ}$C, o que intensifica o processo de acreção podendo gerar gelo sólido que cai derretendo até a superfície. Nestes casos espera-se uma camada de derretimento mais espessa.

Considerando um regime de precipitação estratiforme, que é governado por processos de agregação, será observado um aumento acentuado no fator de refletividade do radar em logo abaixo da isoterma de 0 $^{\circ}$C associado ao derretimento de flocos de neve, que denomina-se banda brilhante. Neste caso espera-se uma camada de derretimento menos espessa, pois os flocos de neve possuem velocidade terminal e densidade inferior as partículas de gelo compacto (ganizo ou saraiva/$graupel$) portanto derretem mais rapidamente, ou seja, percorrem um caminho menor durante o derretimento. 

\simbolo{name={$^{\circ}$C},description={Grau Celcius}} 


%\begin{xalignat}{3}
%\mathbf{n} \cdot \mathbf{E} = 0 && &e  && \mathbf{n} \cdot \mathbf{B} = 0.
%\end{xalignat}


%\begin{equation}
%K^2 = \left( \dfrac{m^2-1}{m^2+2}\right)^2
%\end{equation}

Também, sabendo que $Z$ é proporcional ao diâmetro dos hidrometeoros $D_h$ elevado a 6 potência, os processos de crescimento de flocos de neves, granizo e gotas, são marcados por aumentos exponenciais no Fator de Refletividade do Radar $Z$ no perfil de altitude. 

%Considerando um regime de precipitação estratiforme, que é governado por processos de agregação, será observado um aumento acentuado no fator de refletividade do radar em logo abaixo da isoterma de 0 $^{\circ}$C associado ao derretimento de flocos de neve. \simbolo{name={$^{\circ}$C},description={Grau Celcius}} 

Acima da região de derretimento, um aumento abruto nos valores de $Z$ podem indicar processo de crescimento de cristais de gelo e de gelo compacto. Enquanto que abaixo da região de derretimento, os acréscimos nos valores de $Z$ podem indicar processos de colisão coalescência ou, decréscimos devido a evaporação. 







% durante o caminho que a precipitação percorre até a superfície ou temperaturas acima de 0°C.
%Na figura \ref{fabry}, \citeonline{Fabry1995}
%e o trabalho de 
%\begin{figure}[hbp]
%  \centering{
%  \subfloat[\cite{Fabry1995}]{{\includegraphics[scale=0.25]{img/ilustracoes/fabry}} \label{fabry}}
%  \subfloat[\cite{Takahashi2002}]{{\includegraphics[scale=0.35]{img/ilustracoes/takahashi}} \label{taka}}
%  }
%\caption{Fabry Taka}
%\label{fabyTaka} 
%\end{figure} 

%Consequentemente, a taxa de raios associa-se com a intensidade convectiva devido a acreção\footnote{A acreção é o processo de \textit{rimming} descrito no trabalho de \citeonline{Takahashi1978}.} ser o processo mais eficiente de eletrificação de nuvens, principalmente quando há presença de flocos de neve embebidos na região de fase mista \cite{Takahashi1978,Takahashi2002}. 


\section{OBJETIVOS... PROPOSTA...}

\begin{itemize}

\item Criar um banco de dados de tempestades elétricas do TRMM. 

\item Criar mapas que identifique a densidade de tempestades elétricas e de raios sobre a América do Sul.

\item Descrever o ciclo diurno e o ciclo anual das tempestades elétricas do TRMM.

\item Classificar a intensidade das tempestades elétricas com base na taxa de raios e no estudo da frequência de ocorrência do Fator de Refletividade do radar por temperatura e por altura. 

\end{itemize}

\chapter{METODOLOGIA}
\label{metodologia}

A Metodologia consiste fundamentalmente na construção de um subconjunto de dados das observações dos sensores VIRS, LIS e PR abordo do satélite TRMM, durante o período entre 1998 e 2011. Além dos dados satelitais, as reanálises 2 do \textit{National Centers for Environmental Prediction -- Department of Energy
} (NCEP--DOE) em níveis de pressão foram utilizadas para calcular os valores de temperatura nos níveis correspondentes a altitude do PR.
\sigla{name={NCEP--DOE},description={\textit{National Centers for Environmental Prediction -- Department of Energy}}} 



As informações dos diferentes sensores foram combinadas de maneira a identificar sistemas denominados como tempestades elétricas, definidas como nuvens as quais possuíram pelo menos um raio -- \textit{flash} -- detectado pelo LIS. 

%A seguir são apresentadas as principais características do TRMM 
%Para melhor entender as implicações que envolvem a construção de uma base de dados de sistemas individualmente a partir das observações do TRMM, inicialmente descreve-se algumas das principais características operacionais do satélite TRMM.

\section{O SATÉLITE TRMM}
\label{metodologiaTRMM}

O satélite \textit{Tropical Rainfall Measuring Mission} -- TRMM  faz parte de uma missão conjunta entre a \textit{National Aeronautics and Space Administration} (NASA) e a \textit{Japan Aerospace Exploration Agency} (JAXA),  com o objetivo de estimar a distribuição espaço-temporal da chuva e do fluxo de calor latente para a região tropical e subtropical terrestre. Estas informações são fundamentais para avaliar modelos atmosféricos globais e climáticos principalmente quando se trata de previsão do tempo e clima nos trópicos \cite{kummerok1998,simpson1988}.

\sigla{name={NASA},description={\textit{National Aeronautics and Space Administration}}}   
\sigla{name={JAXA},description={\textit{Japan Aerospace Exploration Agency}}}
  

O satélite TRMM foi lançado em 28 de novembro de 1997 entrando em órbita circular de 350 km de altitude com inclinação de 35$^{\circ}$ e período de 90 minutos. Originalmente a missão teria 3 anos, porém, em agosto de 2011 sua órbita foi elevada até 402,5 km de altitude e o tempo  de vida da missão foi prolongado. Devido as suas características orbitais, o TRMM sobrevoa 2 vezes ao dia em uma região de 10$^{\circ}$ $\times$ 10$^{\circ}$ de latitude e longitude \cite{simpson1988}. Em 2014, verificou-se que o combustível do satélite TRMM está no fim e a manobra de reentrada na atmosfera e aterrissagem deverá ocorrer por volta de abril de 2015, conforme notícia divulgada \cite{TRMMgoodbye}.
  
Os instrumentos a bordo do TRMM são: radar de precipitação (\textit{Precipitation Radar} -- PR), radiômetro de microondas (\textit{TRMM Microwave Imager} -- TMI), radiômetro no visível e no infravermelho (\textit{Visible and Infrared Scanner} -- VIRS), radiômetro para medir a energia radiante da terra e das nuvens (\textit{Clouds and the Earth's Radiant Energy System} -- CERES) e sensor para imageamento de relâmpagos (\textit{Lightning Imaging Sensor} -- LIS). A figura \ref{figtrmm}, ilustra algumas das principais características de varreduras \cite{kummerok1998}.

\begin{figure}[!ht]
  \centering{
  {{\includegraphics[height=14.5cm]{img/TRMM/sensorPackageTraduzido}}}
  }
\caption{Ilustração do satélite TRMM e as  principais características de varredura dos sensores (adaptada de \citeonline{kummerok1998}).}
\label{figtrmm} 
\end{figure} 


\subsection{Radar de Precipitação}

O radar de precipitação (PR) é o principal instrumento do satélite TRMM. Se trata do primeiro radar meteorológico lançado no espaço sendo a maior inovação apresentada pela missão TRMM. Os objetivos do PR são prover a estrutura tridimensional da precipitação e quantificar as taxas de precipitação sobre os continentes e oceanos.  \cite{kummerok1998}. 


\begin{table}[!ht]
\caption{Principais parâmetros do sinal eletromagnético transmitido e recebido pelo PR (adaptada de\citeonline{kummerok1998,trmmhandbook}).}
\label{PRparametros}
\centering
\small
\newcommand{\grayline}{\rowcolor[gray]{.88}}
\renewcommand {\tabularxcolumn }[1]{ >{\arraybackslash }m{#1}}
\newcolumntype{W}{>{\centering\arraybackslash}X}
\begin{tabularx}{\textwidth}{l W } %{|p{10cm}|X|X|X|X|X|X|X|X| }
\hline\hline 
  Item & Especificações \\[1.5pt]
\hline
\grayline Frequência & 13,796, 13,802 GHz\\[1.5pt]
Sensibilidade & $\leq$0,7 mm h$^{-1}$ (Sinal/Ruído por pulso $\simeq$ 0 dB)\\[1.5pt]
\grayline  Transmissor/Receptor: & \\[1.5pt]
\grayline {~~~~~~~~~} Tipo & SSPA and LNA (128 channels)\\[1.5pt]
\grayline {~~~~~~~~~} Potência máxima & $\geq$500 W \\[1.5pt]
\grayline {~~~~~~~~~} Largura do pulso & 1,6 $\mu$s $\times$ 2 canais \\[1.5pt]
\grayline {~~~~~~~~~} Frequência de repetição do pulso (PRF) & 2776 Hz \\[1.5pt]
\hline 
\end{tabularx}
\end{table}


Os principais parâmetros relacionados ao feixe eletromagnético do PR são listados na tabela \ref{PRparametros}. Em relação a antena do PR, possui uma largura de feixe de 0,71$^{\circ}$ $\times$ 0,71$^{\circ}$ disposta em um painel com abertura de 2,0 m $\times$ 2,0 m. Sua varredura transversal (\textit{cross-track}) de $\pm$17$^{\circ}$ é composta por 49 feixes. Após/antes a elevação do satélite TRMM, o PR observa uma faixa na superfície de 247/215 km e resolução horizontal no nadir de 5,0/4,3 km. Verticalmente o PR registra 80 medidas ao longo de uma faixa de 20 km a partir da superfície, com resolução de 250 m. O PR realiza $\simeq$9150 varreduras por órbita, o que corresponde a uma matriz tridimensional de 49 $\times$ 80 $\times$ 9150   \textit{gates}: 49 feixes na varredura horizontal, com 80 níveis verticais e $\simeq$9150 varreduras horizontais.

As medidas de potência recebida ($P_r$), associadas com a secção transversal de retro-espalhamento sem correção de atenuação e  \textit{ground clutter}, são armazenadas no produto 1B21. A equação do radar conforme as especificações do PR é aplicada nos dados de $P_r$ e são convertidos para o fator de refletividade ($Z_m$) em dBZ, definindo então o produto 1C21.

\simbolo{name={$Z_m$},description={Fator de refletividade sem correção de atenuação e \textit{ground clutter}}}

No produto 2A25 calcula-se a taxa de precipitação (R). É quando faz-se necessário corrigir o efeito de atenuação e \textit{ground clutter} de $Z_m$ obtendo o fator refletividade corrigido ($Z_c$) \cite{iguchi1994atenua,meneghini2000,iguchi2000rain}. No cálculo de R são utilizadas as classificações dos perfis verticais como convectivo, estratiforme e outros do produto 2A23 \cite{awaka1997}. 
\simbolo{name={$Z_c$},description={Fator de refletividade corrigido}}
\simbolo{name={R},description={Taxa de precipitação}}

%Os dados do PR desta pesquisa fazem parte do produto 2A25, que depende do 1C21, 2A21 e 2A23. No produto 1C21, informações a respeito da  degradação do feixe com a distância e do tipo de superfície (oceano, costa ou continente) são adicionadas. A partir do produto 1C21, no produto 2A25, a correção por atenuação do feixe é aplicada nos valores de $Z_m$, quando se obtém os valores de refletividade corrigida por atenuação $Z_c$ bem como a taxa de precipitação em cada instante do campo de visão do PR \cite{iguchi1994atenua,meneghini2000,iguchi2000rain}.   
%O produto 2A25 possui também as informações sobe o tipo de chuva, as quais são utilizadas nesta pesquisa. Esta classificação é gerada no produto 2A23 e incorporadas no 2A25, em que basicamente identifica-se a presença ou não de sinais de banda brilhante no perfil e busca classificar cada perfil vertical de $Z_m$ como convectivo estratiforme ou outros \cite{awaka1997}. 
% \cite{meneghini2000,iguchi2000rain,iguchi2009,PRv7}.
%Considerando as características do TRMM, no produto 
%Para esta pesquisa serão utilizados os dados do produto 2A25, o qual corrige a atenuação da refletividade do radar medida ($Z_m$) \simbolo{name={$Z_m$},description={Refletividade do radar medida}} e a partir do fator de refletividade corrigido por atenuação $Z_c$, estima a estrutura tridimensional da precipitação no instante da observação, bem como a taxa de precipitação em cada célula da resolução (46 $\times$ 80) do PR \cite{PRv7}. 
%Além da estrutura tridimensional da precipitação do produto 2A25, nesta tese, utiliza-se das classificações do tipo de chuva do produto 2A23: convectivo, estratiforme, outros, etc \cite{2A25,PRv7}.   
\sigla{name={PR},description={Radar de precipitação, do inglês \textit{Precipitation Radar}}}
\sigla{name={PRF},description={Frequência de repetição do pulso, do inglês \textit{Pulse Repetition Frequency}}}
\simbolo{name={$f$},description={Distância focal}}

\subsection{Sensor imageador de raios}

O imageador de raios (LIS) é um sensor óptico capaz de detectar e localizar raios individualmente, a partir da emissão óptica resultante da dissociação, excitação e recombinação dos constituintes atmosféricos durante uma descarga atmosférica. 

O sistema de imageamento do LIS é constituído por um telescópio com razão focal de $f/1,6$ expandindo o feixe luminoso observado, que passa por um filtro de interferência no comprimento de onda de 777,4 nm e com largura de banda de 1 nm e atinge uma matriz de 128 $\times$ 128 CCDs\footnote{O CCD (\textit{charge-coupled device}) é um dispositivo eletrônico que mede corrente elétrica gerada por efeito fotoelétrico amplamente utilizado para obter imagens digitalmente.}. Acoplada a matriz de CCDs, uma lente angular proporciona um campo de visão panorâmico de 80$^{\circ}$ $\times$ 80$^{\circ}$, que corresponde a uma área de 600 km $\times$ 600 km na superfície terrestre. Um pixel do campo de visão do LIS possui resolução entre 5 km no nadir e até 10 km nas regiões mais externas da sua varredura. Para identificar os raios o LIS utiliza um sistema de amostragem que captura 500 imagens por segundo \cite{christian2000LISalgorithm,boccippio1996science,trmmhandbook}. 

%\begin{figure}[!hb]
%  \centering
%  {{\includegraphics[height=6.cm]{img/TRMM/lis_pic.gif}}}
%\caption{LIS}
%\label{Lispic}
%\end{figure} 

Conforme descrito em \citeonline{christian2000LISalgorithm}, a identificação dos raios depende do brilho difuso e transiente observado no topo das tempestades elétricas. Dependendo das posições das CCDs que são sensibilizadas e do intervalo de tempo entre os brilhos subsequentes, o algoritmo de processamento de imagens do LIS identifica  eventos, grupos e os raios. Os eventos são as posições das CCDs que ``brilham", os grupos são os agrupamentos de eventos que podem ser associados com as descargas de retorno -- \textit{strokes} -- dos raios. O raio -- \textit{flash} -- é o agrupamento espaço-temporal de grupos de eventos. Os raios do LIS correspondem aos raios totais observados na atmosfera, intranuvens e nuvem-solo.

A figura \ref{LisImagemProcessa}, ilustra como é feita a identificação dos  eventos, grupos até a caracterização de um raio. Observe, na figura \ref{evgrfla} que em $t=0$ ms, as CCDs, 1, 2 e 3  foram sensibilizadas e o algoritmo definiu o grupo de evento $a$ candidato a ser um raio $A$.  Quando $t=100$ ms, as CCDs, 4, 5 e 6, são sensibilizadas e temos 2 grupos de eventos, $a$ e $b$, espacialmente e temporalmente ($<$330 ms) próximos, portanto, os grupos $a$ e $b$ integram o mesmo raio $A$, como ilustra a figura \ref{evgrflb}. Quando $t = 350$ ms, as CCDs 9 e 10 são sensibilizadas como mostra a figura \ref{evgrflc}. As CCDs 9 e 10 não estão próximas dos grupos $a$ e $b$ que compõe o raio $A$, portanto, o algoritmo define o grupo $d$ e um novo candidato a raio $B$. Na figura \ref{evgrfld} o raio $B$ possui mais dois grupos, $e$ e $f$. Na figura \ref{evgrfle}, note que a CCD 13 coincide com a posição 2, inicialmente em \ref{evgrfla}, mas como o intervalo de tempo entre as figuras \ref{evgrfla} e \ref{evgrfle} é superior a 330 ms, o grupo $g$ também definiu o novo raio $C$ \cite{christian2000LISalgorithm}.  

     
O LIS possui a capacidade de identificar descargas nuvem-solo e intranuvens, tanto no período diurno quanto noturno. Conforme \citeonline{boccippio1996science} a eficiência de detecção de raios do LIS é maior no período noturno, com 93$\pm$4\%, enquanto que no período diurno é de 73$\pm$11\%.  Com a velocidade orbital de 11 km s$^{-1}$, o sensor LIS possui um campo de visão que permite a observação de um ponto na Terra por até $\simeq$100 segundos no nadir, tempo suficiente para a estimativa da taxa de raios de uma tempestade \cite{christianTM,trmmhandbook}.

\sigla{name={LIS},description={Sensor imageador de raios, do inglês \textit{Lightning Imaging Sensor}.}}

\begin{figure}[!ht]
  \centering{
  \subfloat[]{{\includegraphics[height=3.3cm]{img/TRMM/time0_ptbr}} \label{evgrfla}}
  \subfloat[]{{\includegraphics[height=3.3cm]{img/TRMM/time100_ptbr}}\label{evgrflb}}
  
  \subfloat[]{{\includegraphics[height=3.3cm]{img/TRMM/time350_ptbr}}\label{evgrflc}}
  \subfloat[]{{\includegraphics[height=3.3cm]{img/TRMM/time400_ptbr}}\label{evgrfld}}

  \subfloat[]{{\includegraphics[height=3.3cm]{img/TRMM/time700_ptbr}} \label{evgrfle}}  
  }
\caption{Ilustração do algoritmo de identificação de eventos, grupos e os raios do LIS \cite{christian2000LISalgorithm}.}
\label{LisImagemProcessa} 
\end{figure} 


\subsection{Radiômetro no visível e infravermelho}

O Radiômetro no visível e infravermelho (VIRS) é um radiômetro de varredura transversal de $\pm$45$^{\circ}$, fazendo que seja observada uma faixa de 720 km  na superfície terrestre com uma resolução de 2,11 km no nadir. Após a elevação do satélite, o VIRS passou a observar uma faixa de 833 km com 2,4 km de resolução no nadir. O VIRS mede a radiância em 5 bandas espectrais entre 0,63--12,03 $\mu$m, conforme mostra a tabela \ref{canaisVirs} \cite{kummerok1998}.
%: com comprimentos de onda de 0.63 $\mu$m e  1.61 $\mu$m, faixa do visível; 3.75 $\mu$m, infravermelho próximo; 10.8 $\mu$m e 12 $\mu$m, infravermelho. 

\begin{table}[!ht]
\caption{Canais do VIRS e objetivos das medidas de radiância espectral conforme cada comprimento de onda ($\lambda$) (adaptada de\citeonline{kummerok1998,trmmhandbook}).}
\label{canaisVirs}
\centering
\small
\newcommand{\grayline}{\rowcolor[gray]{.88}}
\renewcommand {\tabularxcolumn }[1]{ >{\arraybackslash }m{#1}}
\newcolumntype{W}{>{\centering\arraybackslash}X}
\begin{tabularx}{\textwidth}{W W W W W W} %{|p{10cm}|X|X|X|X|X|X|X|X| }
\hline\hline 
  & Canal 1 & Canal 2 & Canal 3 & Canal 4 & Canal 5\\[1.5pt]
\hline
\grayline $\lambda$ ($\mu$m) & 0,623$\pm$0,088 & 1,610$\pm$0,055 & 3,784$\pm$0,340 & 10,826$\pm$1,045 & 12,028$\pm$1,055 \\[1.5pt]
%Objetivos & identificar nuvens no período diurno & Identificar diferenças entre água e gelo & Vapor & Temperatura de topo de nuvem& Vapor \\[1.5pt]
\hline 
\end{tabularx}
\end{table}

O conjunto de dados das radiâncias espectrais calibradas, a partir das observações periódicas de referências ópticas como a Lua, o Sol e uma cavidade de corpo negro abordo do satélite, representam o produto 1B01 \cite{kummerok1998}.

Nesta pesquisa, utilizamos apenas a radiância do canal 4 que corresponde ao comprimento de onda ($\lambda$) de 10,8 $\mu$m, pois considerando a Lei de Planck, os dados de radiância de $\lambda$=10,8 $\mu$m podem ser convertidos em temperatura de brilho ($T_b$) e a emissão do topo das nuvens associados a emissão de um corpo negro. 

\sigla{name={VIRS},description={Radiômetro no visível e infravermelho, do inglês \textit{Visible and InfraRed Scanner}}}
\simbolo{name={$\lambda$},description={Comprimento de onda}}
\simbolo{name={$T_b$},description={temperatura de brilho}}
%\subsection{Radiômetro de microondas}

%O TMI (\textit{TRMM Microwave Imager}) é um radiômetro passivo multicanal, 10,65 GHz, 19,35 GHz, 21,3 GHz, 37 GHz, e 85,5 GHz, com dupla polarização. Possui uma varredura cônica combinada com movimento de rotação de sua antena, a qual observa regiões elipsoidais quando projetadas na superfície \cite{kummerok1998}. Sua resolução horizontal varia entre 6-50 km, dependendo do ângulo entre o feixe e o nadir, e varredura de ~760 km \cite{trmmhandbook}. 
%\sigla{name={TMI},description={\textit{TRMM Microwave Imager}}}



\section{REANÁLISES (R2) DO NCEP-DOE}

Os dados de altura geopotencial e temperatura são utilizados para converter a altura do feixe do PR em um eixo de temperatura. Para tanto, utiliza-se dos dados das reanálises 2 (R2) do NCEP-DOE.  \sigla{name={R2},description={Reanálises 2 do NCEP-DOE}} 

As reanálises são um conjunto de campos meteorológicos  consistidos dinamicamente e termodinamicamente em um modelo de circulação atmosférica global a partir de dados de radio-sonda, aviões e satélites \cite{kalnay1996ncep}. Os campos disponíveis são: magnitude e direção de ventos, temperatura, umidade relativa, altura geopotencial entre outros.

O projeto R2  -- \textit{NCEP-DOE Atmospheric Model Intercomparison Project (AMIP-II) reanalysis} -- representa correções aplicadas no projeto das reanálises R1, que busca corrigir erros humanos e erros de versões anteriores de modelos atmosféricos utilizados no processo de integração e assimilação que envolve a construção das reanálises \cite{kanamitsu}.


\sigla{name={R1},description={Reanálises do NCEP-NCAR}}
\sigla{name={NCEP--NCAR},description={\textit{National Centers for Environmental Prediction -- National Center for Atmospheric Research }}}
\sigla{name={NCEP--DOE},description={\textit{National Centers for Environmental Prediction -- Department of Energy}}}
\sigla{name={NCEP},description={\textit{National Centers for Environmental Prediction}}}

\section{DADOS}

Os dados referentes as observações do TRMM foram obtidos a partir do servidor de FTP da NASA (ftp://disc2.nascom.nasa.gov) e do NCEP (ftp://ftp.cdc.noaa.gov).

Foram utilizadas os dados de temperatura e altura geopotencial em 17 níveis de pressão das reanálises 2 do NCEP-DOE e os arquivos orbitais do TRMM, produto 1B01 e 2A25 ambos na versão 7, para o período entre 1998 e 2011. 

%Nesta etapa um conjunto de \textit{scripts} foi desenvolvido para download e verificação de integridade dos dados baixados. No total o volume de dados atingiu 28 TB.  %PR 16,5TB / VIRS 10TB / TMI 1,4TB /   
%órbitas
%primeira (1998) = 00539
%ultima (2011) = 80471
%total = 79932 (observados nos diretórios = 79924) 
%sobre a AS, subset = 63613(VIRS) 61229(PR)           


Os dados do LIS refente ao tempo de visada (\textit{view time}), eventos, grupos e raios foram concedidos pela pesquisadora \citeonline{rachel}, que processou estes dados na NASA anteriormente a esta pesquisa. 

No total, os dados brutos desta pesquisa, representaram um volume de  aproximadamente 30 terabytes. 

Para este trabalho de pesquisa os dados do TRMM e da R2 foram amostrados sobre a região limitada entre 10N-40S e 91W-30W, que abrange toda a extensão da América do Sul. 
%Portanto foi feito um recorte nos dados orbitais apenas para esta região que cobre toda a América do Sul, o que reduziu bastante o volume de dados a serem utilizados e tornou o processamento possível perante a infraestrutura computacional do IAG-USP.

Na tabela \ref{varsTRMM} são apresentadas as medidas extraídas de cada sensor do TRMM e respectiva fonte.


\begin{table}[!h]
\centering
\small
\caption{Variáveis dos produtos do TRMM que foram utilizadas na identificação e descrição das tempestades elétricas.}
\label{varsTRMM}
\renewcommand {\tabularxcolumn }[1]{ >{\arraybackslash }m{#1}}
\newcolumntype{W}{>{\centering\arraybackslash }X}
\begin{tabularx}{\textwidth}{ p{7cm} W W }
\hline
\hline
\textbf{Variável} & \textbf{Sensor TRMM} & \textbf{Produto} \\[1.5pt]
\hline
 Latitude & VIRS & 1B01 \\[1.5pt]
Longitude & VIRS & 1B01 \\[1.5pt]
 Radiância (10,8 $\mu$m) & VIRS & 1B01 \\[1.5pt]
Latitude & PR & 2A25 \\[1.5pt]
 Longitude & PR & 2A25 \\[1.5pt]
Fator de refletividade $Z_c$ & PR & 2A25 \\[1.5pt]
 Tipo de chuva  &  PR  & 2A23 \\[1.5pt]
Latitude eventos/grupos/raios & LIS &  \cite{rachel} \\[1.5pt]
 Longitude eventos/grupos/raios&LIS& \cite{rachel} \\[1.5pt]
Tempo de visada 0,25$^{\circ}$ $\times$ 0,25$^{\circ}$ & LIS &  \cite{rachel} \\

\hline
\end{tabularx} 
\end{table} 
 

\section{TEMPESTADES ELÉTRICAS}
\label{identificaTempestades}


%Após uma análise ponto a ponto, buscando associar cada raio com um perfil de refletividade do PR, partimos para uma análise de grupo, buscando identificar quais as tempestades elétricas que representam maior intensidade convectiva.

As tempestades elétricas, como já definido anteriormente, são nuvens que durante o seu ciclo de vida apresentaram pelo menos um raio.

Dessa maneira, para criar o banco de dados de nuvens de tempestades elétricas deste trabalho de pesquisa, a equação de Planck foi aplicada nos dados de radiância espectral do produto 1B01, canal 4 do VIRS (10,8 $\mu$m) e as regiões com temperatura de brilho ($T_b$) inferiores à 258 K e com pelo menos um raio do LIS observado, definiram as tempestades elétricas \cite{morales2003}.


% Após, o algoritmo verifica se houve raios detectados pelo LIS na mesma área da nuvem. Havendo pelo menos um raio, o sistema era classificado como uma tempestade elétrica. 

A partir do agrupamento dos sistemas, \textit{clusters} com $T_b \leq 258$ K, o algorítimo extrai as variáveis listadas na tabela \ref{varsTRMM} refentes as observações do PR e LIS para a mesma região em que o \textit{cluster} de tempestade elétrica foi observado.

Como os sensores do TRMM possuem diferentes resoluções espaciais, técnicas numéricas de mudança de base foram utilizadas para projetar as observações orbitais do VIRS, PR e LIS em uma grade regular com 0,05$^{\circ}$ $\times$ 0,05$^{\circ}$ de resolução, de maneira à verificar as medidas do PR, LIS e VIRS para uma mesma tempestade elétrica. 

Cada tempestade elétrica identificada foi armazenada na forma de um arquivo HDF contendo medidas coincidentes do VIRS, LIS e PR. 

Inicialmente foram identificadas {154 189} tempestades elétricas. Entretanto, 331 tempestades elétricas não corresponderam a um único sistema convectivo ou multicelular, pois esses núcleo convectivos com raios estavam embebidos em grandes sistemas como Frentes e a ZCAS. 

Portanto, foi feita uma redefinição nos 331 sistemas enormes considerando a temperatura de brilho limiar para definição dos \textit{clusters} de nuvens de 221 K. Regiões com temperatura de brilho em infravermelho inferiores a 221 K são consideradas como a parte mais ativa dos sistemas convectivos de meso-escala identificados em \citeonline{Maddox1980}. 


Na figura \ref{nuvem221}, temos a representação de um dos sistemas considerados como enorme. Note que, na parte superior e inferior da figura \ref{nuvem221}, há informações referentes a data e hora em que o sistema foi observado, número de raios/eventos (FL/EV), fração do sistema observado pelo PR, área do sistema (A), semi-eixo maior (a), menor (b), distância focal (2c) e excentricidade (e) de uma elipse ajustada as dimensões do sistema, a qual está plotada sobre a região geográfica. A barra de cores corresponde as temperaturas de brilho do topo da tempestade elétrica. Os valores de FTA e FT na parte superior da figura \ref{nuvem221} serão definidos em \ref{metodoFtaFt}, próxima seção.
% Este sistema foi dividido em 12 sistemas menores, com limiar de temperatura de brilho de 221 K e ocorrência de raio. 

\begin{figure}[hb]
\centering
\includegraphics[height=1.0cm]{img/grids/nucleosRaios/colorbar_virs}\\
\includegraphics[height=11.cm,trim=0 47cm 0 0,clip]{img/topSevero/EnormesInvalidas/018_Enormes_40791_0001}
\caption{Nuvem de tempestade elétrica considerada enorme.}  
\label{nuvem221}
\end{figure}

Com a recategorização destes sistemas enormes, o número total de tempestades elétricas que passaram a integrar esta pesquisa é de {157 592}.

\simbolo{name={A},description={Nas figuras que ilustram as tempestades elétricas, refere-se a área total ($A_t$) da tempestade elétrica.}}
\simbolo{name={a},description={Semi-eixo maior}}
\simbolo{name={b},description={Semi-eixo menor}}
\simbolo{name={2c},description={Distância focal}}
\simbolo{name={e},description={Excentricidade}}



\section{SEVERIDADE: TAXA DE RAIOS}
\label{metodoFtaFt}

Condições de tempo severo, como frentes de rajadas, queda de granizo e tornados estão associados com um aumento abrupto na taxa de raios total das tempestades elétricas, principalmente governado por raios intra-nuvens \cite{macgorman1989,carey1998,williams1999}.   

Portanto, a taxa de raios ([min$^{-1}$]) do LIS, observada sobre a área que define uma tempestade elétrica pode indicar condições de tempo severo. Os sistemas precipitantes, (\textit{precipitation features} -- PFs) com as maiores 
taxas de raios por minuto em \citeonline{cecil2005,zipser2006}, corresponderam aos sistemas com os maiores volumes de chuvas, mínimas temperaturas de brilho em mico-ondas, máximos valores de refletividade do PR, e de acordo com estes autores pode ser indicativo da presença de fortes correntes ascendentes. 

Neste trabalho de pesquisa a taxa de raios das tempestades elétricas será avaliada a partir de dois índices:

\begin{itemize}
\item FT -- A taxa de raios no tempo, sendo a razão entre o número de raios ($N_{fl}$) e o tempo médio ($VT_m$) de observação do LIS sobre a tempestade elétrica, conforme descreve a equação \ref{eqFT}.
\end{itemize}

\begin{equation}
FT = \frac{N_{fl} }{VT_m} 60 ~[\mathrm{minuto^{-1}}]  
\label{eqFT}  
\end{equation}
%31557600 ano

\begin{itemize}
\item FTA -- A taxa de raios por tempo normalizada pela área da tempestade elétrica, sendo a razão entre o número de raios ($N_{fl}$), o tempo médio ($VT_m$) observação do LIS e a extensão em área ($A_t$) da tempestade elétrica observada, conforme descreve a equação \ref{eqFTA}.
\end{itemize}

\begin{equation}
FTA = \frac{N_{fl} }{VT_m A_t } 60 ~[\mathrm{minuto^{-1}~km^{-2}}]
\label{eqFTA}
\end{equation}

\simbolo{name={$N_{fl}$},description={Número de flashes }}
\simbolo{name={$VT_m$},description={Tempo médio de visada do LIS}}
\simbolo{name={$A_t$},description={Área da tempestade elétrica}}

Note que o fator 60 que multiplica tanto a equação \ref{eqFT} quanto a \ref{eqFTA} é aplicado para converter o tempo de visada do LIS de segundos ($VT_m$) para minutos de observações, pois, em geral a severidade é quantificada em raios por minuto.

%Para cada tempestade elétrica foram calculados os dois índices que podem estar associados com a severidade, o FT e FTA, conforme as equações \ref{eqFT} e \ref{eqFTA}. 

\simbolo{name={$FT$},description={Taxa de raios por tempo $[raios~minuto^{-1}]$}} \simbolo{name={$FTA$},description={Taxa de raios por tempo por área $[raios~dia^{-1}~km^{-2}]$}}.


\section{DENSIDADE GEOGRÁFICA DE RAIOS E TEMPESTADES ELÉTRICAS}
\label{metodoPass}


A densidade de tempestades elétricas e também de raios projetadas sobre a AS, busca proporcionar ao leitor não apenas uma visão geográfica dos locais em que se tem maior ocorrência de raios, mas também identificar os locais e períodos do ano em que as tempestades elétricas apresentam processos de eletrificação mais eficazes.  

O que se torna fundamental na construção destes mapas é considerar quantas vezes, ou qual o tempo em que o satélite ficou observando cada parte da região de estudo, pois uma determinada região pode ter muito mais amostragens do que outras, por causa das características orbitais do TRMM. Portanto, qualquer análise de densidade geográfica com dados do TRMM que não considere o número de passagens ou tempo em que o sensor observou a região projetada na superfície será tendenciosa. 

Mesmo que o satélite TRMM visite o mesmo lugar do globo duas vezes por dia em função de sua órbita inclinada 35$^{\circ}$ e velocidade de 7,3 km s$^{-1}$, entre o período de 1998--2011, o satélite passou {10 000} vezes mais sobre a região extra-topical do que na região tropical, como mostra a figura \ref{VirsVT} que apresenta o número de orbitas sobrevoadas pelo VIRS em cada ponto da grade regular de 0,25$^{\circ}$  $\times$ 0,25$^{\circ}$ na América do Sul. Este efeito decorre da região aonde o satélite atinge a latitude máxima.

\begin{figure}[!hb]
  \centering
  {{\includegraphics[height=13.5cm]{img/grids/passagens_virs_1998-2011}}}
\caption{Número de observações do VIRS em cada região de 0,25$^{\circ}$  $\times$ 0,25$^{\circ}$.}
\label{VirsVT}
\end{figure} 

%\label{gridAmostragem} 

Agora levando em consideração o tempo de amostragem do LIS (\textit{view time}), figura \ref{lisVT}, o número de dias de amostragem em cada ponto da grade de 0,25$^{\circ}$  $\times$ 0,25$^{\circ}$ projetada sobre a região de estudo, revela que durante os 14 anos o LIS observou 10 dias a mais na latitude 34$^{\circ}$ Sul do que em 0$^{\circ}$. Logo, se estas regiões com maior tempo de amostragem forem eletricamente ativas, é de se esperar um alto número raios observados.

\begin{figure}[!ht]
  \centering
  {{\includegraphics[height=13.5cm]{img/grids/vt_trmm}} }
  \caption{Tempo de amostragem (\textit{View time}) do LIS ente 1998-2011 (0,25$^{\circ}$  $\times$ 0,25$^{\circ}$).}
\label{lisVT}
\end{figure} 

As figuras  \ref{lisVT} e \ref{VirsVT} representam duas matrizes que correspondem aos pontos de uma grade igualmente espaçada (grade regular), com 0.25$^{\circ}$ de resolução, projetada sobre a América do Sul. A matriz $\mathbf{VT}_{lis}$, figura \ref{lisVT}, do tempo de amostragem do sensor LIS sobre a superfície e a matriz $\mathbf{VT}_{virs}$, figura \ref{VirsVT}, do número de vezes que o VIRS sobrevoou cada ponto de grade na superfície, são utilizadas para normalizar as medidas de raios observados pelo LIS e tempestades elétricas definidas por meio do  canal 4 do VIRS.  

\simbolo{name={$\mathbf{VT}_{lis}$},description={Matriz do tempo total da visada do sensor LIS sobre a superfície}}

Portanto a partir destas grandezas podemos calcular a densidade de raios ($\mathbf{DE}_{fl}$) e a densidade de tempestades elétricas ($\mathbf{DE}_{te}$) em função do tempo de amostragem.

Para tanto, é necessário calcular o número total de raios observados $\mathbf{FL}_{lis}$ e o número total de tempestades elétricas $\mathbf{P}_{te}$ em cada ponto da grade de 0,25$^{\circ}$ $\times$ 0,25$^{\circ}$. A matriz $\mathbf{FL}_{lis}$ é ilustrada na figura \ref{taxatotalraios}, e a matriz $\mathbf{P}_{te}$ na figura \ref{taxaTotalTe}.

%Com as mesmas dimensões e resolução de grade que o tempo de observação e o número de passagens do satélite foram acumulados em duas matrizes, os raios foram acumulados na matriz ($\mathbf{FL}_{lis}$) e todos os píxeis do VIRS com radiância espectral associada com temperaturas de brilho inferiores a 258 K e que definiram as áreas das tempestades elétricas, foram acumulados na matriz ($\mathbf{P}_{te}$) que representa os locais com maior cobertura de nuvens de tempestades elétricas.
%A matriz $\mathbf{FL}_{lis}$ projeta sobre a América do Sul está representada na figura \ref{gridFL} e a matriz $\mathbf{P}_{te}$, na figura \ref{gridTe}.

Na figura \ref{taxaTotalTe} é notável o alto número de sistemas na região Sul da AS, mas este máximo não indica maior ocorrência de tempestades elétricas e sim maior frequência de passagem do satélite TRMM. 

%Portanto, faz-se necessário normalizar as medidas do TRMM pelo tempo de amostragem ou número de passagens.

\begin{figure}[!ht]
  \centering
  \subfloat[]{{\includegraphics[height=13.5cm]{img/grids/densEspacial_19982011acumuladoFlashPolyfill}} } 
\caption{Número total de raios (\textit{flashes}) observados pelo LIS em cada região de 0,25$^{\circ}$  $\times$ 0,25$^{\circ}$,  entre 1998--2011.}
\label{taxatotalraios}
\end{figure}   
  
\begin{figure}[!ht]
  \centering 
  {{\includegraphics[height=13.5cm]{img/grids/densEspacial19982011acumuladoTempestadesPolyfill}}}
\caption{Número  total de tempestade elétrica observadas em cada região de 0,25$^{\circ}$  $\times$ 0,25$^{\circ}$,  entre 1998--2011.}
\label{taxaTotalTe}
\end{figure} 

Mesmo que as matrizes representem pontos em uma grade com espaçamento angular regular, as áreas de cada ponto de grade não são iguais, pois a  o comprimento de arco de 0,25$^{\circ}$ na direção zonal depende da latitude da região. Assim a matriz que corresponde a área da grade regular ($\mathbf{A}_g$) foi calculada e considera nos cálculos de densidades.


A densidade de raios ($\mathbf{DE}_{fl}$) é calculada conforme a equação \ref{defl}, que apresenta a razão entre $\mathbf{FL}_{lis}$, $\mathbf{VT}_{lis}$ e $\mathbf{A}_g$ multiplicada por 24 horas $\times$ 60 minutos $\times$ 60 segundos $\times$ 365,25 dias, o que converte o tempo de observação do LIS de segundos para anos. A densidade de raios, portanto, é uma grandeza que representa o número de raios por ano por quilômetro quadrado ([ano$^{-1}$] [km$^{-2}$]).

\begin{equation}
\mathbf{DE}_{fl} = \frac{\mathbf{FL}_{lis}}{\mathbf{VT}_{lis} \mathbf{A}_g} 31557600     
\label{defl}
\end{equation}

No mesmo caminho a densidade de tempestades elétricas ($\mathbf{DE}_{te}$) é obtida. Porém, conforme descrito em \ref{metodologiaTRMM}, o tempo de amostragem do LIS e do VIRS são distintos. Enquanto o LIS é um sistema de imageamento, o VIRS é um radiômetro que realiza varreduras durante a trajetória do satélite. Portanto, na obtenção da densidade de tempestades elétricas é considerado o número de vezes que o VIRS sobrevoou cada ponto da grade de 0,25$^{\circ}$  $\times$ 0,25$^{\circ}$. Desta forma, $\mathbf{DE}_{te}$ é obtida conforme a equação \ref{dete}, que define uma grandeza que representa o número de tempestades elétricas por órbita por quilômetro quadrado ([km$^{-2}$]).

%Na figura \ref{VirsVT} temos apenas o acumulado de passagens do VIRS, porém, é conveniente converter o número de vezes que o VIRS observou cada ponto da grade de 0.25$^{\circ}$  $\times$ 0.25$^{\circ}$ em unidade de tempo, pois desta forma teremos as mesmas dimensões físicas tanto para a $\mathbf{DE}_{te}$ quanto para a $\mathbf{DE}_{fl}$.
%Conforme descrito em \ref{metodologiaTRMM}, em função da velocidade e órbita do satélite TRMM, é possível sobrevoar uma mesma região tropical duas vezes por dia. Então podemos considerar que cada ponto da grande de 0.25$^{\circ}$  $\times$ 0.25$^{\circ}$ da figura \ref{VirsVT} é observado pelo VIRS 2 vezes por dia, portanto se multiplicarmos 2 passagens por 365.25 dias podemos concluir o satélite observa cada ponto da grade aproximadamente 730.5 vezes por ano.
%Porém a constante de conversão de tempo na equação \ref{dete} é diferente da equação \ref{defl}, pois o tempo que o VIRS observou a AS, foi estimado a partir do número de vezes que o satélite passou sobre a AS e considerando que cada ponto de grade na orbita foi observado por 90 segundos. 



% a densidade de tempestade elétrica $\mathbf{DE}_{te}$, como descreve a equação \ref{dete} é normalizada pelo número de vezes que o VIRS sobrevoou cada ponto da grade de 0.25$^{\circ}$  $\times$ 0.25$^{\circ}$, o que corresponde a dimensão física do número de [sistemas] por  [observação] por [quilômetro quadrado] (sistemas observaçoes$^{-1}$ km$^{-2}$). Essa grandeza revela, por exemplo, que a cada 10,000 observações do VIRS, temos entre 1-4 tempestades elétricas observadas na América do Sul.

\begin{equation}
\mathbf{DE}_{te} = \frac{\mathbf{P}_{te}}{\mathbf{VT}_{virs} \mathbf{A}_g}    
\label{dete}
\end{equation}

\section{MORFOLOGIA DA ESTRUTURA TRIDIMENSIONAL DA PRECIPITAÇÃO}
\label{metodologiaPrec3D}

O estudo para descrever a morfologia da precipitação foi realizado com base nas observações do PR, buscando avaliar como a precipitação está distribuída nos níveis de altitude e como os perfis de $Z_c$ estão associados com os processos de crescimento de hidrometeoros e consequentemente de eletrificação.  

\subsection{Distribuições de probabilidades com a altitude}

A partir dos perfis de $Z_c$ selecionados pelo algoritmo de identificação de tempestades elétricas, foi estudada a probabilidade de ocorrência de $Z_c$ por altitude, através dos Diagramas de Contorno de Frequência por Altitude (CFAD) \cite{yuter1995}.
\sigla{name={CFAD},description={Diagrama de Contorno de Frequência por Altitude}}

Conforme descrevem \citeonline{yuter1995}, primeiramente obteve-se uma função de densidade de probabilidade com duas variáveis ($f_{pdf}(x,y)$), cuja a dimensão $x$ correspondeu à valores de $Z_{c}$ e $y$ os nível de altitude do PR. Neste estudo, a função $f_{pdf}(x,y)$ foi representada numericamente por uma matriz bidimensional com a granularidade de 1 dBZ para cada 250 m de altitude.

\simbolo{name={$f_{pdf}(x,y)$},description={Função densidade de probabilidade com duas variáveis}}


Para a obtenção dos diagramas de probabilidade normalizados por nível de altitude, cada nível $y$ da função $f_{pdf}(x,y)$ foi normalizado pelo número total de ocorrências de valores de $Z_c$ distribuídos em $x$. Os níveis $y$ de altitude com número total de ocorrência de $Z_c$ em $x$, menor do que 10\% do nível de máxima ocorrência, foram desconsiderados dos contornos de probabilidade em todos os CFADs.

Com base na $f_{pdf}(x,y)$ que definiu cada CFAD, foi calculada a função densidade de probabilidade cumulativa $f_{cdf}(x,y)$ de $Z_c$ por altitude, que originaram os Diagramas de Contorno de Frequência Cumulativa por Altitude (CCFAD).  \sigla{name={CCFAD},description={Diagramas de Contorno de Frequência Cumulativa por Altitude}}   

\simbolo{name={$f_{cdf}(x,y)$},description={Função densidade de probabilidade cumulativa com duas variáveis}}

Os CCFADs auxiliam a investigar quais as diferenças entre os perfis de $Z_c$ associados à diferentes quantis da amostra de probabilidade, elucidando ainda mais as informações contidas nos CFADs.

%Então, objetivando uma análise dos processos de crescimento de hidrometeoros no perfil atmosférico e na mesma óptica de trabalhos como \citeonline{Takahashi1978,Saunders1999,Takahashi2002,avila2009}, ou seja, em função de diferentes condições de temperatura, nesta pesquisa construímos o diagrama denominado como Diagrama de Contorno de Frequência por Temperatura (CFTD), \sigla{name={CFTD},description={Diagrama de Contorno de Frequência por Temperatura}}.

\subsection{Distribuições de probabilidades em função da  temperatura}

A distribuição vertical da precipitação está associada com o desenvolvimento vertical, porém o tipo de hidrometeoro de uma  determinada altitude é função da temperatura, probabilidade de colisão e da razão de saturação naquela altura \cite[p.~263]{mason1971_2ed}. Os processos de eletrificação por outro lado dependem do conteúdo de água líquida, temperatura, velocidade terminal, velocidade vertical, tamanho e tipo das partículas \cite{Takahashi1978,Saunders1999,Takahashi2002,avila2009}.

Portanto saber como o perfil da precipitação varia com a temperatura é preponderante para identificar qual mecanismo de crescimento dos hidrometeoros está dominando conforme o desenvolvimento e a severidade das tempestades elétricas. Neste sentido foi elaborado o Diagrama de Contorno de Frequência por Temperatura (CFTD). \sigla{name={CFTD},description={Diagrama de Contorno de Frequência por Temperatura}} 

%Nos CFTDs, os níveis de temperatura não correspondem as condições controladas em laboratório, e sim às variações de temperatura do perfil atmosférico. 

Para converter os níveis de altitude do PR em níveis de temperatura, foram utilizados os dados de reanálises (R2) do NCEP--DOE entre 1998--2011, em 17 níveis de pressão, da altura geopotencial e temperatura, correspondentes aos horários mais próximos e anteriores ao horário de cada tempestades elétrica observada pelo TRMM.

Os perfis de altura geopotencial e temperatura mais próximos ou coincidentes com cada região de tempestade elétrica observada pelo VIRS, foram extraídos. A partir dos 17 níveis verticais das R2, os 80 níveis de temperaturas associados aos 80 níveis de altitude das observações do PR foram calculados através de um método de mínimos quadrados.

Desta maneira, obteve-se a função $f_{pdf}(x,y)$, cuja a dimensão $x$ correspondeu à valores de $Z_{c}$ e $y$ os nível de temperaturas estimados a partir das R2 do NCEP--DOE. A função $f_{pdf}(x,y)$ de $Z_c$ por temperatura, foi representada por uma matriz bidimensional com a granularidade de 1 dBZ para cada 2 $^{\circ}$C. Nos CFTDs, os níveis superiores e inferiores foram definidos para temperaturas entre 20° C e -50° C.

Também foi calculada a função $f_{cdf}(x,y)$ de $Z_c$ por temperatura, que originaram os Diagramas de Contorno de Frequência Cumulativa por Temperatura (CCFTD).  \sigla{name={CCFTD},description={Diagramas de Contorno de Frequência Cumulativa por Temperatura}}   

\chapter{MARCO DAS TEMPESTADES ELÉTRICAS NA AMÉRICA DO SUL}


O Marco das tempestades elétricas descreve os locais e quando estes sistemas ocorrem na América do Sul. Determina-se a sazonalidade, o ciclo diurno, a distribuição espacial de raios e das tempestades elétricas. 

\section{CICLO DIURNO E CICLO ANUAL}

Utilizando a base de dados de tempestades elétricas construída nesta pesquisa, foi estudada a frequência de ocorrências dos sistemas no decorrer das horas do dia, figura \ref{ciclodiurnototal}, e meses do ano, figura \ref{cicloanualtotal}. Deste modo, obtivemos na figura \ref{diurnoanual}, o ciclo diurno e anual das tempestades elétricas por meio da distribuição de probabilidade de ocorrências.


\begin{figure}[!hb]
  \centering{
  \subfloat[Ciclo diruno.]{{\includegraphics[scale=0.95]{img/ciclos/ciclodiurno19982011total}} \label{ciclodiurnototal}}
  \subfloat[Ciclo anual.]{{\includegraphics[scale=0.95]{img/ciclos/cicloanual19982011total}} \label{cicloanualtotal}}
  }
\caption{Ciclo diurno e anual das tempestades elétricas observadas em hora local. Os valores de probabilidade foram normalizados pelo total dos 154,189 sistemas identificados.}
\label{diurnoanual} 
\end{figure} 

A figura \ref{ciclodiurnototal},  mostra que, entre 14h e 15h as tempestades elétricas são mais prováveis, indicando que o aquecimento da superfície do continente e o aumento da camada limite planetária no decorrer do dia são ingredientes que podem aumentar a probabilidade de ocorrência em até 4,6 vezes em relação aos horários de menor fluxo de calor sensível para a atmosfera. Enquanto o TRMM observou 2312 tempestades elétricas às 9h (hora local), às 14h foram observadas 13,877.

No ciclo anual, conforme mostra a figura \ref{cicloanualtotal}, observa-se que a estação de tempestades elétricas na América do Sul possui dois picos, um em outubro e outro em março, porém contempla os meses de outubro, novembro, dezembro, janeiro, fevereiro e março. A maior probabilidade de ocorrência esteve associada ao mês de outubro, que concentrou 16,961 tempestades elétricas observadas em 14 anos. % Entre janeiro e março as tempestades elétricas tiveram probabilidade de 0.3\% menor do que no início da estação, em outubro.   


O ciclo diurno também foi estudado para cada região de 10 por 10 graus, como mostra a figura \ref{diurno}. 



\begin{figure}[!hb]
\centering{\includegraphics[scale=1.5]{img/ciclos/ciclodiurno10x1019982011localtime}}  
\caption{Ciclo diurno em hora local para as tempestades elétricas observadas em cada região de 10 por 10 graus. Os valores de probabilidade são mostrados em porcentagem e foram normalizados pelo total de 154,189 sistemas observados.}
\label{diurno}
\end{figure}


Mesmo que em uma análise geral mostre a importância do aquecimento superficial do continente para a ocorrência de tempestades elétricas, sistemas noturnos sobre a Colômbia e Venezuela são bastante frequentes. Na figura \ref{diurno}, entre 0$^{\circ}$--10$^{\circ}$ Norte e 80$^{\circ}$--70$^{\circ}$ Oeste, às 0h em hora local, temos o maior valor de probabilidade (0.4\%) de tempestades elétricas noturnas da América do Sul, o que representou um número de 617 sistemas observados em 14 anos, apenas entre 0h e 00:59h.


A circulação de vale e montanha associada com a topografia elevada na Colômbia, principalmente a região do Parque Nacional Natural Paramillo, e o Lago Maracaibo na Venezuela, combinados com a atuação da Zona de Convergência Intertropical (ZCIT), promovem condições para o desenvolvimento de tempestades elétricas noturnas de maneira mais eficiente do que as demais regiões. \sigla{name={ZCIT},description={Zona de Convergência Intertropical}}

No Oceano Pacífico, entre 0$^{\circ}$--10$^{\circ}$ Norte e 90$^{\circ}$--80$^{\circ}$ Sul abrangendo o Parque Nacional da Ilha do Coco na Costa Rica e parte das ilhas Galápagos no Equador, foi a região oceânica com a maior probabilidade de ocorrência de tempestades elétricas. Esta possui um ciclo diurno duplo de tempestades elétricas. Elas ocorrem às 4h, em hora local, e as 14h. A maior probabilidade de ocorrência (0.15\%) foi observada às 4h, que correspondeu à 231 sistemas.

No Pacífico Sul, as tempestades elétricas são mais raras do que as demais regiões devido a atuação permanente da subsidência da Célula de Hadley, que modula a Alta Subtropical do Pacífico Sul, responsável também por regiões como o Deserto do Atacama e parte do semi-árido Argentino.

Na região do Atlântico Subtropical, a probabilidade de tempestades elétricas é maior do que no Atlântico Norte. A passagem de sistemas transientes entre 40$^{\circ}$--30$^{\circ}$ Sul e 50$^{\circ}$--30$^{\circ}$ Oeste e 30$^{\circ}$--20$^{\circ}$ Sul e 40$^{\circ}$--30$^{\circ}$ Oeste, gera maior número de tempestades elétricas oceânicas do que com a atuação da ITCZ no Atlântico Tropical. Observa-se também que nas regiões oceânicas o ciclo diurno das tempestades elétricas indica maior atividade noturna e não às 14-15h igual no continente.


O pico de atividade de tempestades elétricas durante o ciclo diurno, figura \ref{diurno}, ocorreu entre 10$^{\circ}$--0$^{\circ}$ Sul e 70$^{\circ}$--50$^{\circ}$ Oeste e 20$^{\circ}$--10$^{\circ}$ Sul e 60$^{\circ}$--50$^{\circ}$ Oeste. Em cada uma destas três caixas, observou-se a probabilidade de aproximadamente 0.8\% entre as 14h e 15h, mostrando que em toda esta região, o TRMM observou uma média de 3 tempestades elétricas a cada 2 dias, apenas nessas duas horas.


Entre 30$^{\circ}$--20$^{\circ}$ Sul e 60$^{\circ}$--50$^{\circ}$ Oeste, na figura \ref{diurno}, região de grande atividade de Sistemas Convectivos de Meso-escala (MCS) conforme descrevem \citeonline{Durkee2009}, encontra-se um máximo durante a tarde e os sistemas noturnos tiveram probabilidade de ocorrência 2.7 vezes menor do que os valores encontrados sobre os vales na Colômbia e Venezuela, mostrando que a ocorrência dos MCS ao Sul da América do Sul com ciclo de vida maior do que 9h ou com formação noturna, não possuem probabilidade de ocorrência que destaca-se em relação as demais regiões continentais, mesmo neste banco de dados composto apenas por tempestades elétricas. Na figura \ref{anual}, há um máximo de atividade em outubro que antecede a estação de tempestades elétricas entre dezembro e março. O máximo é observado em janeiro com 1234 sistemas identificados na região.\sigla{name={MCS},description={Sistemas Convectivos de Meso-escala}}


A tabela \ref{caracEstacao} mostra os meses de duração das estações de tempestades elétricas com base no estudo mostrado na figura \ref{anual}. Os períodos em que a probabilidade de ocorrência de sistemas foram superiores à 0.7 do máximo observado na região, foram considerados como os períodos das estações de tempestades elétricas. %Regiões como as linhas 19 e 20 da tabela \ref{caracEstacao} mostram que houveram apenas 32 tempestades elétricas em 14 anos, portanto 


\begin{figure}[!ht]
\centering{\includegraphics[scale=1.5]{img/ciclos/cicloanual10x1019982011localtime}}  
\caption{Ciclo anual em hora local para as tempestades elétricas observadas em cada região de 10 por 10 graus. Os valores de probabilidade são mostrados em porcentagem e foram normalizados pelo total de 154,189 sistemas observados. As linhas horizontais cortam o valor de 0.7 do máximo de probabilidade, utilizado como limiar para definir o início e fim das estações de tempestades elétricas.}
\label{anual}
\end{figure}



\begin{table}[!ht]
\caption{Principais características do ciclo anual de probabilidade de ocorrência de tempestades elétricas observadas entre 1998-2011, em cada região de 10 por 10 graus.}
\label{caracEstacao}
\centering
\small
\newcommand{\grayline}{\rowcolor[gray]{.88}}
\renewcommand {\tabularxcolumn }[1]{ >{\arraybackslash }m{#1}}
\newcolumntype{W}{>{\centering\arraybackslash}X}
\begin{tabularx}{\textwidth}{p{0.6cm} p{3.5cm} W W W W} %{|p{10cm}|X|X|X|X|X|X|X|X| }
\hline\hline 
\grayline  & Localização & Número de sistemas & Estação (meses) & Duração (meses) & Máximo\\[1.5pt]
\hline
1&0$^{\circ}$--10$^{\circ}$N, 90$^{\circ}$--80$^{\circ}$O& 4159  & Abr--Set &6& Abr\\[1.5pt]\grayline
2&0$^{\circ}$--10$^{\circ}$N, 80$^{\circ}$--70$^{\circ}$O& 14,047 & Mar--Nov &9& Out\\[1.5pt]
3&0$^{\circ}$--10$^{\circ}$N, 70$^{\circ}$--60$^{\circ}$O& 11,787 & Jul--Out &4& Set\\[1.5pt]\grayline
4&0$^{\circ}$--10$^{\circ}$N, 60$^{\circ}$--50$^{\circ}$O&  4868 & Jul--Set &3& Ago\\[1.5pt]
5&0$^{\circ}$--10$^{\circ}$N, 50$^{\circ}$--40$^{\circ}$O& 645 & Mai--Jun, Set--Nov &5& Set\\[1.5pt] \grayline
6&0$^{\circ}$--10$^{\circ}$N, 40$^{\circ}$--30$^{\circ}$O& 821 & Out--Dez &3& Dez\\[1.5pt]

7&10$^{\circ}$--0$^{\circ}$S, 90$^{\circ}$--80$^{\circ}$O& 217 & Mar &1& Mar\\[1.5pt]\grayline
8&10$^{\circ}$--0$^{\circ}$S, 80$^{\circ}$--70$^{\circ}$O& 9721 & Set--Dez &4& Out\\[1.5pt]
9&10$^{\circ}$--0$^{\circ}$S, 70$^{\circ}$--60$^{\circ}$O& 12,168 & Ago--Nov &4& Out\\[1.5pt]\grayline
10&10$^{\circ}$--0$^{\circ}$S, 60$^{\circ}$--50$^{\circ}$O& 12,231 & Set--Dez &4& Out\\[1.5pt]
11&10$^{\circ}$--0$^{\circ}$S, 50$^{\circ}$--40$^{\circ}$O& 7731 & Jan--Abr, Nov--Dez &6&  Jan\\[1.5pt]\grayline
12&10$^{\circ}$--0$^{\circ}$S, 40$^{\circ}$--30$^{\circ}$O& 1349 & Jan--Abr &4&  Mar\\[1.5pt]

13&20$^{\circ}$--10$^{\circ}$S, 90$^{\circ}$--80$^{\circ}$O& 1  &   --0--  & 1 & Mai \\[1.5pt]\grayline
14&20$^{\circ}$--10$^{\circ}$S, 80$^{\circ}$--70$^{\circ}$O& 4254 & Jan--Fev,  Set--Dez  &6& Out\\[1.5pt]
15&20$^{\circ}$--10$^{\circ}$S, 70$^{\circ}$--60$^{\circ}$O& 8585 & Jan--Mar, Set--Dez &7& Out\\[1.5pt]\grayline
16&20$^{\circ}$--10$^{\circ}$S, 60$^{\circ}$--50$^{\circ}$O& 10,414 & Jan--Mar,  Out--Dez &6& Out\\[1.5pt]
17&20$^{\circ}$--10$^{\circ}$S, 50$^{\circ}$--40$^{\circ}$O& 8201 & Jan--Mar, Out--Dez &6&  Jan\\[1.5pt]\grayline
18&20$^{\circ}$--10$^{\circ}$S, 40$^{\circ}$--30$^{\circ}$O& 611 & Jan--Mar &3&  Mar\\[1.5pt]

19&30$^{\circ}$--20$^{\circ}$S, 90$^{\circ}$--80$^{\circ}$O& 32 & Mai--Jun &2&  Mai\\[1.5pt]\grayline
20&30$^{\circ}$--20$^{\circ}$S, 80$^{\circ}$--70$^{\circ}$O& 32 & Fev, Mai, Jul--Ago &4&  Fev, Mai,  Jul\\[1.5pt]
21&30$^{\circ}$--20$^{\circ}$S, 70$^{\circ}$--60$^{\circ}$O& 5558 & Dez--Mar &4& Jan\\[1.5pt]\grayline
22&30$^{\circ}$--20$^{\circ}$S, 60$^{\circ}$--50$^{\circ}$O& 8676 & Dez--Mar &4& Jan\\[1.5pt]
23&30$^{\circ}$--20$^{\circ}$S, 50$^{\circ}$--40$^{\circ}$O& 5996 & Dez--Mar &4& Jan\\[1.5pt]\grayline
24&30$^{\circ}$--20$^{\circ}$S, 40$^{\circ}$--30$^{\circ}$O& 1849 & Fev--Mar, Mai, Dez &4&  Mar\\[1.5pt]

25&40$^{\circ}$--30$^{\circ}$S, 90$^{\circ}$--80$^{\circ}$O& 258 & Jun &1&  Jun \\[1.5pt]\grayline
26&40$^{\circ}$--30$^{\circ}$S, 80$^{\circ}$--70$^{\circ}$O& 370 & Jan--Mar, Mai--Jun &5&  Jan\\[1.5pt]
27&40$^{\circ}$--30$^{\circ}$S, 70$^{\circ}$--60$^{\circ}$O& 7638 & Dez--Jan &2&  Jan\\[1.5pt]\grayline
28&40$^{\circ}$--30$^{\circ}$S, 60$^{\circ}$--50$^{\circ}$O& 5403 & Dez--Mar &4&  Jan\\[1.5pt]
29&40$^{\circ}$--30$^{\circ}$S, 50$^{\circ}$--40$^{\circ}$O& 2966 & Jan--Set &9&  Abr\\[1.5pt]\grayline
30&40$^{\circ}$--30$^{\circ}$S, 40$^{\circ}$--30$^{\circ}$O& 2288 & Abr--Jun  &3& Mai\\[1.5pt]


\hline 
\end{tabularx}
\end{table}

 
%No ciclo anual mostrado na figura \ref{anual}, observa-se uma clara diferença sazonal na ocorrência de tempestades elétricas entre os dois Hemisférios. Sobre o Hemisfério Norte as tempestades ocorrem principalmente entre os meses de abril e agosto, enquanto no Hemisfério Sul entre setembro e março, apesar das características individuais de cada região como por exemplo dois ou três picos de atividade durante a estação.
%O deslocamento da ZCIT durante o verão setentrional, inverte a estação de tempestades elétricas entre 10$^{\circ}$ Sul e 10$^{\circ}$ Norte, diminuindo o número de sistemas no inverno austral.

Na região entre o clima semi-árido na Argentina e parte da Bacia do Prata, entre 40$^{\circ}$--20$^{\circ}$ Sul e 70$^{\circ}$--60$^{\circ}$ Norte, figura \ref{anual}, local das tempestades mais severas e convecção mais profunda da América do Sul como apontam \citeonline{cecil2005, Romatschke2010}, foi encontrada uma estação de tempestades elétricas bastante definida entre dezembro e janeiro, sendo que entre maio e agosto, a probabilidade de ocorrência de sistemas é praticamente 0\%.%, pois é no verão que o jato de baixos níveis, que traz umidade da Amazônia, se intensifica e dispara os processos de eletrificação nesta região.

As estações de tempestades elétricas se configuram conforme o Sistema de Monção da América do Sul (SAMS)   \sigla{name={SAMS},description={Sistema de Monção da América do Sul}}. Na região central da América do Sul, observa-se que com o aumento da temperatura da superfície entre julho e setembro, o máximo de precipitação começa a se deslocar do Hemisfério Norte para o Hemisfério Sul e desta forma iniciando a estação chuvosa meridional pela região Oeste da Bacia Amazônica \cite{Zhou1998,grimm2003nino,reboita2010regimes,Marengo2012}.

Entre 10$^{\circ}$--0$^{\circ}$ Sul e 80$^{\circ}$--60$^{\circ}$ Norte,  na figura \ref{anual}, observa-se que o pico da estação de tempestades elétricas ocorreu em outubro, nos primeiros passos da estação chuvosa da América do Sul. Porém o máximo de precipitação nesta região ocorre depois de 4 ou 5 meses. 

Em \citeonline{Petersen2001}, o estudo realizado referente a estrutura tridimensional da precipitação observada pelo TRMM sobre a região Central da Amazônia, mostrou que a convecção mais profunda ocorre também na transição do período seco para o chuvoso, exatamente quando começa a reversão sazonal do vento em baixos níveis associado ao SAMS conforme apontam \citeonline{Zhou1998}.

Com o início do verão austral, o máximo de precipitação caminha até a região Centro Oeste e Sudeste do Brasil. Em janeiro, o SAMS se configura mais ativamente com a atuação da Zona de Convergência do Atlântico Sul (SACZ) \sigla{name={SACZ},description={Zona de Convergência do Atlântico Sul}} e intensificação do Jato de Baixos Níveis (JBN). \sigla{name={JBN},description={Jato de Baixos Níveis}} A atuação do JBN, principalmente nas regiões abaixo de 20$^{\circ}$ Sul, ativa a estação chuvosa e de tempestades elétricas em sincronismo. 

Durante abril e maio, o SAMS vai se desconfigurando e o máximo de chuva começa a retornar para o Hemisfério Norte caminhando de Sudeste para o Nordeste do Brasil e subindo pelo lado Leste da Bacia Amazônica. Neste retorno é que ocorrem os máximos de precipitação em toda a região da Bacia Amazônica, porém o máximo de ocorrência de tempestades elétrica ocorreu na vinda da estação chuvosa para o Hemisfério Sul.

Na região Nordeste do Brasil, entre 10$^{\circ}$--0$^{\circ}$ Sul e 40$^{\circ}$--30$^{\circ}$ Norte, o máximo de chuva ocorre juntamente com o máximo de ocorrência de tempestades elétricas, depois da atuação da SACZ no continente.

%As probabilidades de ocorrência mostradas nas figuras \ref{anual} e \ref{diurno} descrevem bem comportamento dos ciclos diurno e anual, porém os valores de probabilidades são tendenciosos nas regiões em que o satélite passou mais tempo observando, conforme é descrito em \ref{metodoPass}. Pois as densidades de probabilidade em cada região de 10 por 10 graus foram obtidas apenas considerando as latitudes e longitudes médias dos sistemas e a hora, minuto, segundo, dia, mês e ano da observação.

\section{DENSIDADES ESPACIAIS}

Considerando o método descrito em \ref{metodoPass}, referente ao cálculo da densidade espacial de tempestades elétricas, equação \ref{dete}, e densidade espacial de raios, equação \ref{defl}, nesta seção será possível avaliar se as regiões aonde ocorrem o maior número de sistemas, correspondem as regiões com maior número de raios.

Na figura \ref{tempestadestotal}, observa-se que as regiões de máxima ocorrência de tempestades elétricas estão situadas sobre a Colômbia e região central da Bacia Amazônica, abrangendo a parte brasileira, colombiana, venezuelana e peruana.



\begin{figure}[!ht]
 \centering{
  \subfloat[Densidade espacial total de tempestades elétricas. Os valores correspondem ao número de sistemas por ano por quilômetro quadrado em cada pronto da grade de 0.25 graus.]{\includegraphics[scale=0.88]{img/DensidadeTempestades/densEspacial19982011TotalTempestadesPolyfill} \label{tempestadestotal}}
  \subfloat[Densidade espacial total de raios. Os valores correspondem ao número de raios por ano por quilômetro quadrado em cada pronto da grade de 0.25 graus.]{\includegraphics[scale=0.88]{img/TaxaFlash/densEspacial_19982011totalTaxaFlashPolyfill}\label{raiosTotal}}
  }
\caption{Densidade espacial de tempestades elétricas e raios observados entre 1998 e 2011.}
\label{tempesRaios}
\end{figure}

Mesmo que os sistemas com as maiores taxas de raios no tempo observados pelo TRMM, estejam mais concentrados no Sul da América do Sul conforme mostram \citeonline{cecil2005,zipser2006}, as tempestades elétricas são bem mais frequentes à Noroeste da AS.

A atuação da ITCZ combinada com a convergência de umidade e liberação de calor latente e sensível na Floresta Amazônica, além que regular o SAMS, são os principais propulsores de tempestades elétricas da América do Sul. %que associa-se intimamente com o SAMS,

No entanto, os mecanismos de eletrificação são bem mais eficientes nas tempestades elétricas no Sul da AS, sobe a Bacia do Prata. Na figura \ref{tempesRaios}, observa-se que na Amazônia, as regiões com taxa de raios superiores 30 (ano$^{-1}$ km$^{-2}$), possuíram taxa de sistemas acima de 120 (ano$^{-1}$ km$^{-2}$), enquanto na região da Argentina e Paraguai, as mesmas taxas de raios são atingidas com uma taxa de sistemas em torno de 40 (ano$^{-1}$ km$^{-2}$).

Nas figuras \ref{TaxaFlash} e \ref{DensidadeTempestadesSazonal}, a densidade espacial de raios e de tempestades elétricas, foi calculada para os períodos associados a cada estação do ano: dezembro, janeiro e fevereiro (DJF), março, abril e maio (MAM), junho, julho e agosto (JJA) e setembro, outubro e novembro (SON). A tabela \ref{EstacaoQtd} mostra o acumulado de sistemas observados em cada estação do ano.
\sigla{name={DJF},description={Dezembro, janeiro e fevereiro}} 
\sigla{name={MAM},description={Março, abril e maio}}
\sigla{name={JJA},description={Junho, julho e agosto}}
\sigla{name={SON},description={Setembro, outubro e novembro}}


\begin{figure}[!ht]
  \centering{
  \subfloat[DJF]{{\includegraphics[scale=0.88, trim=0 7 0 0, clip]{img/TaxaFlash/densEspacial_19982011djfTaxaFlashPolyfill}} \label{txDJF}}
  \subfloat[MAM]{{\includegraphics[scale=0.88, trim=0 7 0 0, clip]{img/TaxaFlash/densEspacial_19982011mamTaxaFlashPolyfill}} \label{txMAM}}

  \subfloat[JJA]{{\includegraphics[scale=0.88, trim=0 7 0 0, clip]{img/TaxaFlash/densEspacial_19982011jjaTaxaFlashPolyfill}} \label{txJJA}}
  \subfloat[SON]{{\includegraphics[scale=0.88, trim=0 7 0 0, clip]{img/TaxaFlash/densEspacial_19982011sonTaxaFlashPolyfill}} \label{txSON}}
  }    
  \caption{Densidade espacial sazonal de raios.}
\label{TaxaFlash}
\end{figure} 


Na primavera austral (SON), início do SAMS, a intensificação dos alísios vindos do Atlântico Norte, e o aumento gradativo da evapotranspiração na Floresta Amazônica vão intensificando o transporte de umidade da bacia do Amazônias para a bacia do Prata \cite{marengo2004}.  Esse processo de início da configuração do SAMS provoca a estação com a maior taxa de raios do continente Sul Americano, e esta, ocorre em regiões no centro no continente principalmente a Leste da Cordilheira dos Andes: na Amazônia Central, Argentina, Paraguai e Sul do Brasil.


Neste período destaca-se a taxa de raios sobre o Lago Maracaibo durante SON, na Venezuela, que no acumulado dos 14 anos atingiu o valor de 30 raios por mês de observação por quilômetro quadrado em cada ponto da gade de 0.25 graus. Em \citeonline{albrecht2009tropical}, a região do Lago Maracaibo foi apontada como o máximo global das observações do TRMM. 


\begin{table}[!h]
\caption{Total de tempestades elétricas observadas entre 1998-2011, para cada período de três meses associados as estações do ano.}
\label{EstacaoQtd}
\centering
\small
\newcommand{\grayline}{\rowcolor[gray]{.88}}
\renewcommand {\tabularxcolumn }[1]{ >{\arraybackslash }m{#1}}
\newcolumntype{W}{>{\centering\arraybackslash}X}
\begin{tabularx}{\textwidth}{W W} %{|p{10cm}|X|X|X|X|X|X|X|X| }
\hline \hline 
Estação & Número de sistemas \\[1.5pt]
 \hline
\grayline DJF & 44,534\\[1.5pt]
MAM & 36,096\\[1.5pt]
\grayline JJA & 16,786\\[1.5pt] 
SON & 56,773\\[1.5pt]
\hline 
\end{tabularx}
\end{table}

%Durante DJF a circulação do JBN trazendo umidade da Amazônia é predominante. É quando é ativado os processos de eletrificações de nuvens entre 40$^{\circ}$--20$^{\circ}$ Sul e 70$^{\circ}$--60$^{\circ}$ Norte.

Durante DJF, os máximos de raios são observados em Mato Grosso do Sul; Sul de Mato Grosso; Sudeste Brasileiro, entre costa de Santa Catarina e o Vale do Ribeira em São Paulo, região de fronteira entre São Paulo, Minas Gerais e Rio de Janeiro, aonde localiza-se o Parque Nacional Itatiaia e o Pico das Agulhas Negras; interior de São Paulo; Goias; e na Bacia do Rio Tocantis. Apesar de observarmos o maior número de raios durante a estação de transição entre seca e chuvosa, essas regiões Centrais e Sudeste da AS possuem os processos de eletrificação regulados durante a estação chuvosa. 

Em \citeonline{petersen2002trmm}, é mostrado que mesmo que se tenha observado diminuição na taxa de raios e redução da intensidade convectiva durante o regime de vento de Oeste no experimento TRMM \textit{Large-scale Biosphere Atmosphere Experiment in Amazonia} (LBA), em outras regiões da AS durante o período chuvoso, há um aumento da taxa de raios.

Considerando que o regime de ventos de Leste e Oeste identificado no LBA está associado com as fases ativas e inativas do SAMS conforme descrevem \citeonline{carvalho2002intraseasonal}, pode-se considerar que as máximas taxas de raios apresentadas na figura \ref{txDJF} são moduladas pelas variações na circulação sinóptica associadas com o processos de formação e dissipação da SACZ \cite{petersen2002trmm,albrecht2011,silva2002lba}.

Durante MAM, quando o máximo de chuvas começa a retornar para o Hemisfério Norte, observamos as tempestades elétricas bastante concentradas na região Norte e Nordeste da AS, como mostra a figura \ref{tempestadesMAM}. Neste período, principalmente nas regiões das cidade de Belém, estado do Maranhão, Piauí, Rio Grande do Norte e Paraíba,  ocorrem: os máximos de chuva, os máximos de densidade de raios e os máximos de densidade de tempestades elétricas. Esse sincronismo não é comum.

Ao comparar as figuras \ref{TaxaFlash} e \ref{DensidadeTempestadesSazonal} observa-se que as regiões de máxima densidade espacial de raios não são as regiões de máxima densidade de tempestades elétricas. Os máximos de raios ficam situados em regiões de transição, deslocados dos máximos de sistemas, reforçando a hipótese de \citeonline{williams2002}, em que se espera maior atividade elétrica de nuvem em um ambiente de transição entre seco e úmido.

Por exemplo, a maior área continua da América do Sul com taxas anuais de raios superiores a 20 raios por ano por quilômetro quadrado, como mostra a figura \ref{raiosTotal}, ocorre na região Sul da AS. Tanto na figura \ref{tempestadestotal} quanto na figura \ref{DensidadeTempestadesSazonal}, podemos observar um forte gradiente de sistemas nesta região, que marca a transição entre o clima Desértico no Deserto do Atacama e Semi-árido na Argentina para o clima Subtropical úmido, promovendo um ambiente de transição seco/úmido permanente para os sistemas que iniciam-se principalmente na região da Serra de Córdoba na Argentina e se propagam para Noroeste.

\begin{figure}[!ht]
  \centering{
  \subfloat[DJF]{{\includegraphics[scale=0.88, trim=0 7 0 0, clip]{img/DensidadeTempestades/densEspacial19982011djfTempestadesPolyfill}} \label{tempestadesDJF}}
  \subfloat[MAM]{{\includegraphics[scale=0.88, trim=0 7 0 0, clip]{img/DensidadeTempestades/densEspacial19982011mamTempestadesPolyfill}} \label{tempestadesMAM}}

  \subfloat[JJA]{{\includegraphics[scale=0.88, trim=0 7 0 0, clip]{img/DensidadeTempestades/densEspacial19982011jjaTempestadesPolyfill}} \label{tempestadesJJA}}
  \subfloat[SON]{{\includegraphics[scale=0.88, trim=0 7 0 0, clip]{img/DensidadeTempestades/densEspacial19982011sonTempestadesPolyfill}} \label{tempestadesSON}}
  
  }    
  \caption{Densidade espacial sazonal das tempestades elétricas.}
\label{DensidadeTempestadesSazonal}
\end{figure} 

A partir do estudo das densidades de tempestades elétricas e raios, a figura \ref{eficiencia},  representa as regiões em que as tempestades elétricas são mais eficientes na produção de raios. Foi calculada a taxa de raios por tempestade elétrica por ano por quilômetro quadrado. Os maiores valores desta dimensão que associa-se com eficiência espacial que cada região de 0.25 graus tem em produzir raios, representam os locais em que se tem menor número de sistemas em relação ao número de raios durante os 14 anos de dados.

A região da bacia do Prata é a maior extensão contínua com os maiores valores de eficiência espacial de produção de raios. Porém destacam-se regiões menores como no Vale do Ribeira em São Paulo, Pico das Agulhas Negras em Minas Gerais, região serrana do Rio de Janeiro, parte Sul do Tocantis, parte Leste e Norte do Pará e Leste do estado do Amazonas. Estas regiões podem estar associadas com regiões de tempo severo. Locais em que a topografia ou a circulação local intensifica os sistemas.

Na região do  Parque Nacional Natural Paramillo na Colômbia e no Lago Maracaibo na Venezuela, a taxa de raios por em cada área de tempestade de 0.25 graus mostra valores com a mesma magnitude de regiões na Bacia do Prata, mesmo que o número de raios e de sistemas produzidos ao Norte sejam maiores.

Algumas regiões no pico da Cordilheira dos Andes são bastante eficientes, principalmente na região da cidade de Cochabamba na Bolívia.

\begin{figure}[!h]
\centering{\includegraphics[scale=0.88]{img/TaxaFlashTempestade/densEspacial19982011totalEficienciaPolyfill}}  
\caption{Eficiencia de tempestade}
\label{eficiencia}
\end{figure}
%\chapter{Morfologia da estrutura 3D da precipitação}

\begin{figure}
  \centering{
  \subfloat[CFAD]{{\includegraphics[scale=1.3]{img/precipitacao3d/cfad10_chuvatotal_total_nuvem}} \label{cfadtotal}} \\
  \subfloat[CCFAD]{{\includegraphics[scale=1.3]{img/precipitacao3d/ccfad_10deg_chuvatotal_total_nuvem}} \label{ccfadtotal}}
  }    
  \caption{Diagramas de Contorno de Frequência por Altitude (CFADs) para tempestades elétricas ordenadas pelo 90° percentil dos índices FTA e FT, para a América do Sul em cada 10$^{\circ}$  $\times$ 10$^{\circ}$. Em cada caixa pode-se verificar a porcentagem (\%) de perfis convectivos, estratiformes e outros, respectivamente, (P) o numero de perfis computados, (L) o número de ocorrência de refletividade no nível de máxima ocorrência e (H) o nível de máxima ocorrência.}
\label{cfadstotal}
\end{figure} 


\chapter{TEMPESTADES ELÉTRICAS SEVERAS}

Conforme descrito em \ref{metodoFtaFt}, as taxas de raios das tempestades elétricas neste trabalho de pesquisa estão associadas aos índices FTA e FT. Nesta seção identifica-se qual desses índices podem melhor associar-se com a intensidade convectiva das tempestades elétricas.
%, ou seja, os sistemas com as maiores taxas de raios por minuto (FT) ou os sistemas com as maiores taxas de raios por minuto por quilômetro quadrado (FTA) da sua extensão. 

Para o estudo de intensidade dos sistemas, foram selecionados apenas as tempestades elétricas as quais possuíram $VT_m$ maior ou igual a 1 minuto e com pelo menos um pixel do campo de visão do PR contido na área do sistema, totalizando 94,733 tempestades elétricas do TRMM. %Como a intensidade convectiva dos sistemas é avaliada com base, principalmente, na morfologia da estrutura tridimensional da precipitação e na taxa de raios, 

As equações \ref{eqFT} e \ref{eqFTA} foram aplicadas nas 94,733 tempestades elétricas selecionadas, e então estudada as distribuições de probabilidades dos índices FTA e FT (figuras \ref{pdfFTAFT} e \ref{cdfFTAFT}). Conforme mostra a figura \ref{pdfFTAFT}, trata-se de distribuições exponenciais de probabilidade. Os valores de FTA e FT para cada quantil da amostragem de tempestades elétricas, tanto para as mais frequentes quanto as mais raras, podem ser verificados por meio da distribuição cumulativa de probabilidade de FTA e FT mostradas na figura \ref{cdfFTAFT}. 

\begin{figure}[!ht]
  \centering
  \includegraphics[height=9cm]{img/FtaFt/pdf_FTA_FT}      
  \caption{Densidade de probabilidade de FTA e FT.} 
   \label{pdfFTAFT} 
\end{figure}

\begin{figure}[!hb]
  \centering 
  \includegraphics[height=9cm]{img/FtaFt/cdf_FTA_FT} 
  \caption{Densidade de probabilidade cumulativa de FTA e FT.}
  \label{cdfFTAFT}
\end{figure}
%\label{seriesFtaFt}

Os sistemas potencialmente severos são selecionados pelo 90\textsuperscript{\underline{o}} percentil das amostras de probabilidades de FTA e FT, que estão associado aos maiores e mais raros valores ocorridos. Portanto, pressupõe-se que as tempestades elétricas as quais provavelmente causaram chuva de granizo, rajadas de ventos com queda de árvores e construções ou tornados estão associadas aos sistemas com valores extremos de FTA ou FT.

O grupo das tempestades elétricas com FTA extremo, possuíram valores entre {29.3--1258.7 $\times$ 10$^{-4}$} raios por minuto por quilômetro quadrado (minuto$^{-1}$ km$^{-2}$), enquanto as tempestades elétricas com extremos de FT possuíram valores entre {47.2--1283.6} raios por minuto (minuto$^{-1}$). 



Observe que o valor mínimo de FT foi de 0.6 raios por minuto. Em \citeonline{cecil2005}, considera-se que, a mínima taxa de raios no tempo para as PFs é de 0.7 raios por minutos. Porém a resolução espacial da projeção do tempo de visada do LIS utilizada nesta tese possui resolução de 0.25$^{\circ}$ $\times$ 0.25$^{\circ}$  \cite{albrecht2009tropical,albrecht2011b}. Ao considerar a velocidade e altura da órbita do satélite, o tempo de observação do LIS em um ponto de 0.25$^{\circ}$ $\times$ 0.25$^{\circ}$ na superfície terrestre pode atingir até 102 segundos na região zenital. Então, as tempestades elétricas que possuíram apenas 1 raio e $VT_m$ de $\simeq$100 segundos, tiveram o mínimo valor de FT de 0.6 raios min$^{-1}$, sendo esta, a mínima taxa de raios no tempo detectável em uma tempestades elétricas do TRMM desta pesquisa.

% Na figura \ref{percetilFtaFt}, temos a série de FTA e FT ordenada, e a linha tracejada vertical corta o 90\% percentil dos índices. 

%\begin{figure}[!ht]
%  \centering
%  \includegraphics[height=6cm]{img/FtaFt/90thFtaFt}	 
%  \caption{90\textsuperscript{\underline{o}} percentil de FTA e FT.}
%  \label{percetilFtaFt}
%\end{figure}


\section{ÁREA E TEMPERATURA DO TOPO DA NUVEM}

Observa-se que os extremos de FTA e FT correspondem a sistemas com tamanhos bem distintos. Conforme é mostrado na figura \ref{size}, verifica-se que as máximas probabilidades de ocorrência de tempestades elétricas associadas com os extremos de FTA, ocorrem em sistemas com área 3 ordens de grandeza menor do que nos extremos de FT.

\begin{figure}[!ht]
  \centering
  \includegraphics[height=9cm,trim=0 0 215 0,clip]{img/tb/TbAreas}   
  \caption{Densidade de probabilidade de extensão em área das tempestades elétricas com extremos de FTA e FT.}
  \label{size}  
\end{figure}


%\begin{figure}[!hb]
%  \centering{  
%  \subfloat[Densidade de probabilidade de extensão em área.] { \includegraphics[height=7.5cm,trim=0 0 215 0,clip]{img/tb/TbAreas} \label{size}} \\
%  \subfloat[Densidade de probabilidade de temperatura de brilho em infravermelho.]{ \includegraphics[height=7.5cm,trim=220 0 0 0,clip]{img/tb/TbAreas} \label{tb}} 
%  }
%  \label{t_tb}
%  \caption{Estudo das frequências de ocorrências de tempestades elétricas selecionas pelo 90\textsuperscript{\underline{o}} percentil dos índices de FT e FTA, por extensão em área e por temperatura de brilho de topo das nuvens.}
%\end{figure}

%As tempestades elétricas com valores extremos de FT são maiores em extensão.
Na figura \ref{areaFTAFTA}, pode-se observar que os sistemas com tamanho entre 10$^2$--10$^3$ km$^2$, não ultrapassam 20 raios min$^{-1}$ de FT. As tempestades elétricas com FT superior a 100 raios por minuto, possuíram tamanho entre 10$^{4}$--10$^{6}$ km$^2$. Note que há tendencia de aumento exponencial de FT conforme aumenta a extensão das tempestades elétricas, no  entanto, FTA tende a diminui exponencialmente com o aumento da área das tempestades elétricas.

%Uma tempestade elétrica com 10$^5$ km$^2$, terá maior número de descargas observadas durante o tempo de visada do LIS do que uma com 10$^2$ km$^2$.

\begin{figure}[!ht]
  \centering
  \includegraphics[height=9cm]{img/FtaFt/area_FTA_FT}   
  \caption{Dispersão entre as áreas das tempestades elétricas e os valores de FTA e FT. As linhas horizontais marcam os valores de FTA (cor preta) e FT (cor azul) referente ao 90\textsuperscript{\underline{o}} percentil das respectivas amostragem.}
  \label{areaFTAFTA}  
\end{figure}

\begin{figure}[!hb]
  \centering 
  \includegraphics[height=9cm]{img/FtaFt/volChuva_FTA_FT}
  \caption{Dispersão entre o volume de chuva das tempestades elétricas e os valores de FTA e FT.  As linhas horizontais marcam os valores de FTA (cor preta) e FT (cor azul) referente ao 90\textsuperscript{\underline{o}} percentil das respectivas amostragem.}
  \label{volchuvaFTAFT}
\end{figure}

%\begin{figure}[!hb]
%  \centering 
%  \subfloat[areas versus taxa de raios]{ \includegraphics[height=7.5cm]{img/FtaFt/area_FTA_FT}\label{areaFTAFTA}} \\
%  \subfloat[volume de chuva]{\includegraphics[height=7.5cm]{img/FtaFt/volChuva_FTA_FT}\label{pdfFt}} 
%  \caption{Dispersão referente aos índices FTA e FT.}
%  \label{areaFtaFt}
%\end{figure} 
%Quando a densidade espacial de descargas aumenta muito em uma região com centenas de quilômetros quadrados, em torno de 100 descargas intranuvens para uma nuven-solo, como por exemplo os maiores valores de $Z=IC/CG$ mostrados por \cite{evandro2009} na região de Campo Grande - MS no Brasil, a capacidade do LIS de identificar brilhos transientes provavelmente fica comprometida devido a resolução horizontal da CCD.

Ao normalizar a taxa de raios no tempo por $A_t$, o número de raios fica diluído na extensão do sistema, evidenciando que os maiores valores de FTA correspondem aos sistemas com as maiores densidades espaciais de raios, cuja a área e o número de raios são menores do que nos sistemas com extremos de FT.

Na figura \ref{volchuvaFTAFT}, observa-se que conforme aumenta FT o volume de chuva das tempestades elétricas também aumenta exponencialmente, de maneira semelhante ao aumento de FT com a área (figura \ref{areaFTAFTA}). Para FTA, há um comportamento inverso. Conforme aumenta FTA, o volume de chuva dos sistemas diminui.  

A frequência de ocorrência dos pixeis de temperaturas de brilho, associados a radiância espectral em infravermelho observada pelo VIRS, os quais definiram as áreas das tempestades elétricas, é mostrado na figura \ref{tb}. Observa-se que o maior valor de probabilidade para a curva das tempestades elétricas com índice extremo de FTA, possui temperatura de topo de nuvens aproximadamente 10 K mais frias do que nas tempestades elétricas com extremos de FT, indicando que a convecção nos sistemas extremos de FTA é mais profunda na atmosfera na maioria das observações.

\begin{figure}[!ht]
  \centering 
  \includegraphics[height=9cm,trim=220 0 0 0,clip]{img/tb/TbAreas}
  \caption{Densidade de probabilidade de temperatura de brilho em infravermelho (VIRS 10.8$\mu$m) do topo das nuvens das tempestades elétricas com extremos de FTA e FT.}
  \label{tb}
\end{figure}

\citeonline{morales2003} ao desenvolver a \textit{Sferics Infrared Rainfall Technique} (SIRT), mostram que as regiões com temperatura de brilho inferior a 215 K e com ocorrência de \textit{sferics} foram as regiões categorizadas como de maior precipitação associada. Nesta tese, ao selecionar as tempestades elétricas com índice extremo de FTA, os maiores valores de probabilidade de ocorrência, conforme é mostrado na figura \ref{tb}, concentram-se em temperaturas de brilho abaixo de 215 K.

\sigla{name={SIRT},description={\textit{Sferics Infrared Rainfall Technique} }}

Nas figuras \ref{pdffracaoFTA}, \ref{cdffracaoFTA}, \ref{pdffracaoFT} e  \ref{cdffracaoFT}, apresenta-se o estudo das probabilidades das frações de chuva para as tempestades elétricas com valores extremos de FTA e FT. As curvas denominadas como convectivo, estratiforme e outros, correspondem a fração de área de chuva associada aos perfis do PR classificados como convectivo, estratiforme e outros em relação a área total de chuva do sistema. A curva denominada na legenda como chuva total corresponde a fração da área de chuva em relação a área total ($A_t$) da tempestade elétrica. A curva denominada como varredura do PR, mostra a fração da $A_t$ que esteve dentro do alcance da varredura do PR.

\begin{figure}[!ht]
  \centering
  \includegraphics[height=9cm]{img/FtaFt/fracaoChuva_pdf_topFTA}   
  \caption{Densidade de probabilidade das frações de áreas de chuva das tempestades elétricas com extremos de FTA, que foram classificadas (2A23) como convectiva (vermelha), estratiforme (bege) e outros (verde), em relação a toda a área de chuva observada pelo PR; das frações de chuva total (preta), que corresponde as frações das áreas de chuva observadas pelo PR em relação a $A_t$ das tempestades elétricas e também das frações das áreas das tempestades elétricas contidas na varredura do PR (azul).}
  \label{pdffracaoFTA}  
\end{figure}

\begin{figure}[!ht]
  \centering 
  \includegraphics[height=9cm]{img/FtaFt/fracaoChuva_cdf_topFTA}
  \caption{Densidade de probabilidade cumulativa das frações de áreas de chuva das tempestades elétricas com valores extremos de FTA.}
  \label{cdffracaoFTA}
\end{figure}


\begin{figure}[!ht]
  \centering
  \includegraphics[height=9cm]{img/FtaFt/fracaoChuva_pdf_topFT}   
  \caption{Densidade de probabilidade das frações de áreas de chuva das tempestades elétricas com extremos de FT, que foram classificadas (2A23) como convectiva (vermelha), estratiforme (bege) e outros (verde), em relação a toda a área de chuva observada pelo PR; das frações de chuva total (preta), que corresponde as frações das áreas de chuva observadas pelo PR em relação a $A_t$ das tempestades elétricas e também das frações das áreas das tempestades elétricas contidas na varredura do PR (azul).}
  \label{pdffracaoFT}  
\end{figure}

\begin{figure}[!ht]
  \centering 
  \includegraphics[height=9cm]{img/FtaFt/fracaoChuva_cdf_topFT}
  \caption{Densidade de probabilidade cumulativa das frações de áreas de chuva das tempestades elétricas com valores extremos de FT.}
  \label{cdffracaoFT}
\end{figure}

Avaliando a densidade de probabilidade da fração convectiva e fração estratiforme das tempestades elétricas, figuras \ref{pdffracaoFTA} e \ref{pdffracaoFT}, verifica-se que para os extremos de FTA as tempestades elétricas são mais frequentemente observadas com 70\% de área convectiva e 30\% de área estratiforme, enquanto que para os extremos de FT, 20\% de fração convectiva e 75\% de fração estratiforme.
%juntamente com a as respectivas distribuições de probabilidade acumulativa,

Verifica-se nas figuras \ref{pdffracaoFT} e \ref{cdffracaoFT} que para a maioria dos sistemas com FT extremo, o PR conseguiu observar 25\% da $A_t$ das tempestades elétricas. Nas figuras \ref{pdffracaoFTA} e \ref{cdffracaoFTA} das frações de chuva dos sistemas com FTA extremo, o PR observou com maior frequência 100\% da área das tempestades elétricas. Portanto a fração de chuva total dos extremos de FT (figura \ref{pdffracaoFT}) possui um valor de apenas $\simeq$10\% devido a varredura do PR ser menor do que a do VIRS e as tempestades elétricas com os maiores valores de FT abrangerem uma extensão que ultrapassa o alcance do PR. 


Os sistemas selecionados pelo 90\textsuperscript{\underline{o}} percentil do índice FT possuem maior extensão em área e maior volume de chuva. São sistemas com vasta extensão estratiforme conforme descrevem \citeonline{Rasmussen2011}. As regiões das tempestades elétricas com precipitação convectiva, as quais tem potencial de gerar chuva de granizo, frentes de rajada e tornados, ocupam área bem menor do que as áreas com  precipitação estratiforme \cite{Jr2007}.

A maior fração convectiva e menor tamanho das tempestades elétricas com FTA extremos sugerem sistemas em estágio de maturação. Conforme as tempestades elétricas vão entrando em estágio maduro e dissipativo, vão ganhando área de chuva estratiforme e podem começar a se enquadrar no grupo dos extremos de FT. 

%.....
%Para avaliar qual dos índices representaram a maior severidade de tempo, a morfologia da estrutura 3D da precipitação foi estudada por meio dos diagramas CFAD, CCFAD, CFTD e CCFTD. %para os 10\% das amostras de FT e FTA com os maiores valores.
%......

\section{SEVERIDADE COM BASE NA ESTRUTURA 3D DA PRECIPITAÇÃO}

Nesta etapa iremos avaliar a intensidade convectiva com base nos perfis de $Z_c$ do PR, contidos nos sistemas com índices extremos de FTA e FT. 

Nas figuras \ref{ftacfadwithout}, \ref{ftcfadwithout},  \ref{ftacfadwith} e \ref{ftcfadwith} foram calculados os CFADs das tempestades elétricas com índices FTA e FT extremos, para cada região de 10 por 10 graus na superfície sobre a AS. Para localizar a caixa de 10 por 10 graus em que cada sistema esteve contido, foi considerado a latitude e longitude do centro geométrico da área definida por cada sistema. Desta forma, foram obtidas as amostragens de probabilidade de FTA e FT, da mesma maneira que mostrado na figura \ref{pdfFTAFT} ou \ref{cdfFTAFT}, porém para cada região de 10$^{\circ}$ $\times$ 10$^{\circ}$. Assim, o estudo da precipitação tridimensional é feito apenas para as tempestades elétricas com valores de FTA ou FT extremos referentes a cada região, ou seja, referente ao 90\textsuperscript{\underline{o}} percentil de cada  amostragem de FTA e FT regionalizada a cada 10$^{\circ}$ $\times$ 10$^{\circ}$.


\begin{figure}[!ht]
  \centering{  
  \subfloat[]{\includegraphics[height=1.0cm]{img/grids/nucleosRaios/colorbar_virs}\label{barravirs}}\\
  \subfloat[]{\includegraphics[height=6.cm,trim=0 47cm 0 0,clip]{img/grids/nucleosRaios/001_topFTA_25471_0003} \label{nr1}} 
  \subfloat[]{\includegraphics[height=6.cm,trim=0 47cm 0 0,clip]{img/grids/nucleosRaios/002_topFTA_36502_0001} \label{nr2}} \\
  \subfloat[]{\includegraphics[height=6.cm,trim=0 47cm 0 0,clip]{img/grids/nucleosRaios/003_topFTA_03444_0001} \label{nr3}} 
  \subfloat[]{\includegraphics[height=6.cm,trim=0 47cm 0 0,clip]{img/grids/nucleosRaios/004_topFTA_05694_0005} \label{nr4}} 
  }
  \caption{Núcleos de raios das tempestades elétricas. Os pontos na cor  verde são os eventos e os símbolos de positivo na cor preta são os raios. Os pixeis em vermelho são as regiões dos núcleos de raios, definidas a partir da projeção dos eventos em uma grade regular de 0.05$^{\circ}$.} %(0.05$^{\circ}$ $\times$ 0.05$^{\circ}$)
\label{nucleosRaios}
\end{figure}



%----------------------------------------
\begin{sidewaysfigure}%[!H]
\centering
\includegraphics[width=19.5cm]{img/precipitacao3d/severo/percentil/90th/cfad10_semraio_topFTA_percentil}
\caption{CFADs para os extremos de FTA. Porção da precipitação sem raios.}
\label{ftacfadwithout}
\end{sidewaysfigure} 


\begin{sidewaysfigure}%[!H]
\centering
\includegraphics[width=19.5cm]{img/precipitacao3d/severo/percentil/90th/cfad10_semraio_topFT_percentil}
\caption{CFADs para os extremos de FT	. Porção da precipitação sem raios.}
\label{ftcfadwithout}
\end{sidewaysfigure} 
%----------------------------------------

%\begin{figure}[!ht]
%  \centering
%  \includegraphics[height=13.5cm]{img/precipitacao3d/severo/percentil/90th/cfad10_semraio_topFTA_percentil}
% \caption{CFADs para os extremos de FTA. Porção da precipitação sem raios.}
% \label{ftacfadwithout}
%\end{figure} 

\begin{sidewaysfigure}%[!H]
  \centering
  \includegraphics[width=19.5cm]{img/precipitacao3d/severo/percentil/90th/cfad10_comraio_topFTA_percentil}
  \caption{CFADs para os extremos de FTA. Porção da precipitação com raios.}
  \label{ftacfadwith}   
\end{sidewaysfigure} 


\begin{sidewaysfigure}%[!H]
  \centering
  \includegraphics[width=19.5cm]{img/precipitacao3d/severo/percentil/90th/cfad10_comraio_topFT_percentil}
  \caption{CFADs para os extremos de FT. Porção da precipitação com raios.}
  \label{ftcfadwith}   
\end{sidewaysfigure} 

%\begin{figure}[!ht]
%  \centering
%  \includegraphics[height=13.5cm]{img/precipitacao3d/severo/percentil/90th/cfad10_comraio_topFTA_percentil}
%  \caption{CFADs para os extremos de FTA. Porção da precipitação com raios.}
%  \label{ftacfadwith}   
%\end{figure} 

	

As posições geográficas dos eventos do LIS e dos perfis de $Z_c$ válidos do PR, foram projetadas em uma grade regular de 0.05 graus. Os perfis de $Z_c$ projetados em pontos de grade em que tiveram eventos do LIS, definiram as regiões aqui denominadas como precipitação dos núcleos de raios. A figura \ref{nucleosRaios}, mostra as regiões dos núcleos de raios das tempestades elétricas. Observe que na parte superior e inferior das figuras \ref{nr1}, \ref{nr2}, \ref{nr3} e \ref{nr4}, há informações referentes a data e hora em que o sistema foi observado, número de raios/eventos (FL/EV), fração do sistema observado pelo PR, área do sistema (A), semi-eixo maior (a), menor (b), distância focal (2c) e excentricidade (e) de uma elipse ajustada as dimensões do sistema. A barra de cores na figura \ref{barravirs}, corresponde as temperaturas de brilho ($<$258K) associada a radiância espectral em infravermelho observada pelo VIRS.


Os CFADs foram calculados para a precipitação dos núcleos de raios das tempestades elétricas, figuras \ref{ftacfadwith} e \ref{ftcfadwith}, como também para a precipitação fora dos núcleos de raios, figuras \ref{ftacfadwithout} e \ref{ftcfadwithout}.

Note que no canto superior direito de cada CFAD temos alguns valores estatísticos que representam: (\%)  a porcentagem de perfis convectivos, estratiformes e outros, respectivamente; (P) o número de perfis do PR computados; (L) o número de ocorrência de $Z_c$ no nível de altitude de máxima ocorrência; (H) o nível de altitude, em quilômetros, aonde ocorreu o máximo de ocorrências de $Z_c$; (N) o número de tempestades elétricas computadas.

\simbolo{name={\%},description={Nos diagramas, CFAD, CCFAD, CFTD e CCFTD, representam: a porcentagem de perfis convectivos, estratiformes e outros, respectivamente}}
\simbolo{name={P},description={Nos diagramas, CFAD, CCFAD, CFTD e CCFTD, representam: número de perfis do PR computados}}
\simbolo{name={L},description={Nos diagramas, CFAD, CCFAD, CFTD e CCFTD, representam: o número de ocorrência de $Z_c$ no nível de altitude de máxima ocorrência}}
\simbolo{name={H},description={Nos diagramas, CFAD, CCFAD, CFTD e CCFTD, representam: o nível de altitude, em quilômetros, aonde ocorreu o máximo de ocorrências de $Z_c$}}
\simbolo{name={N},description={Nos diagramas, CFAD, CCFAD, CFTD e CCFTD, representam: o número de tempestades elétricas computadas}}

Comparando os CFADs da chuva com e sem raios, representados para os extremos de FTA nas figuras \ref{ftacfadwithout} e \ref{ftacfadwith} e para os extremos de FT, nas figuras \ref{ftacfadwithout} e \ref{ftcfadwithout}, é evidente que a fração das tempestades elétricas sem raios é a porção de menor intensidade convectiva e a fração eletricamente ativa é a região de maior velocidade vertical. Os níveis de contorno de probabilidades dos CFADs da precipitação sem raios possuem suas máximas altitudes aproximadamente 3 quilômetros abaixo das máximas altitudes atingidas pelos contornos dos CFADs da precipitação com raios. A porção sem raios dos sistemas possuíram maior percentual de perfis estratiformes e menores valores de $Z_c$ com os contornos de probabilidades entre 1-10\%, em todos os níveis de altitude.

%\begin{figure}[!ht]
%  \centering
%  \includegraphics[height=13.5cm]{img/precipitacao3d/severo/percentil/90th/cfad10_semraio_topFT_percentil}%
% \caption{CFADs para os extremos de FT. Porção da precipitação sem raios.}
% \label{ftcfadwithout}
%\end{figure} 

%\begin{figure}[!ht]
%  \centering
%   \adjustbox{trim={0\width} {0.435\height} {0\width} {0\height} , clip}%
%   {\includegraphics[width=\textwidth]{img/precipitacao3d/severo/percentil/90th/cCumFad_10deg_semraio_topFTpercentil}}
% \caption{CCFDs para os extremos de FT entre 20S-10N e 90W-30W. Porção da precipitação sem raios.}
% \label{ftccfadwithout}
%\end{figure} 


Se avaliarmos apenas os níveis de contorno com probabilidade entre 2-3.7\% (cor verde), observa-se que os máximos de $Z_c$ associados à chuva na superfície da porção sem raios, figuras \ref{ftacfadwithout} e \ref{ftcfadwithout}, não ultrapassaram 40 dBZ em nenhuma região da AS, enquanto que para a porção de chuvas com raios, figuras \ref{ftacfadwith} e \ref{ftcfadwith}, os valores de $Z_c$ entre 0-2 km de altitude registram valores entre 45-50 dBZ. A fração da precipitação eletricamente ativa possui maior percentual de perfis convectivos e com maiores valores de $Z_c$ associado aos contornos de probabilidade, confirmando a correlação positiva entre a taxa de raios e o volume de chuva \cite{Petersen1998}.

%A convecção é mais ativa nas regiões dos núcleos de raios, aonde a precipitação está associadas com frentes de rajadas, chuvas de granizo e enchentes rápidas. Fora dos núcleos de raios temos a parte da precipitação mais estratiforme, composta por hidrometeoros que não possuem velocidade terminal suficiente para precipitar nos núcleos de raios, e caem mais afastados da região eletricamente ativa.     % Dependendo principalmente das condições de calor umidade e cisalhamento vertical do vento as células 


% \caption{Diagramas de Contorno de Frequência por Altitude (CFADs). Em cada CFAD pode-se verificar: a porcentagem (\%) de perfis convectivos, estratiformes e outros, respectivamente; (P) o numero de perfis do PR computados, (L) o número de ocorrência de refletividade no nível de máxima ocorrência e (H) o nível de máxima ocorrência.}

A figura \ref{ftcfadwithout} mostra que a precipitação sem raios dos extremos de FT na região tropical, entre 20S-10N e 90W-30W, possui banda brilhante marcada entre 4-5 km de altitude, principalmente nos perfis com probabilidade de ocorrência entre 2-5.3\%, nas cores de contorno em verde e amarelo. 

Podemos observar a banda brilhante entre 20S-10N e 90W-30W  dos sistemas com  extremo de FT na porção sem raios de maneira mais elucidativa por meio dos CCFADs da figura \ref{ftccfadwithout}, os quais evidenciam que entre o 12\textsuperscript{\underline{o}} e o 95\textsuperscript{\underline{o}} percentil da amostragem de probabilidade de $Z_c$ por altitude, há uma queda no valor de $Z_c$ logo abaixo de 5 quilômetros de altitude em cada região de 10 por 10 graus. 

No entanto, para a região entre 20S-10N e 90W-30W, ao avaliar os CFADs da figura \ref{ftacfadwithout} ou CCFADs da figura\ref{ftaccfdsubtrop}, que representam a porção sem raios da precipitação tridimensional dos sistemas com índice extremo de FTA, não se observa banda brilhante marcada nos contornos de probabilidade de $Z_c$ por altitude. Há um aumento contínuo de $Z_c$, conforme os níveis de altitude diminuem, sem a diminuição abrupta de $Z_c$ logo abaixo de 5 quilômetros, mostrando que os processos de acreção e colisão coalescência são ativos, indicando maior velocidade vertical no ambiente dos sistemas extremos de FTA do que no ambiente dos sistemas extremos de FT. 

%observa-se que, entre 20S-10N e 90W-30W,  
%a banda brilhante é evidente apenas nas regiões costeiras e oceânicas, nas caixas entre 0-10N  %e 90-80W, entre 10-0S e 40-30W e entre 20-10S e 70-60W.

As chuvas na superfície associadas com a precipitação sem raios das tempestades elétricas entre 20S-10N e 90W-30W, referente aos extremos de FT têm maiores valores de probabilidades com valores de $Z_c$ mais moderados do que quando compara-se com os extremos de FTA, os quais possuem perfis de $Z_c$ com maior aleatoriedade, mas podem atingir valores em dBZ superiores. Observe as diferenças entre as figuras \ref{ftaccfdsubtrop} e figura \ref{ftccfadwithout}. Note como entre 20S-10N e 90W-30W os contornos de probabilidade, principalmente entre 0.3-3.7\% representados pelas cores em azul e verde, são mais alargados na chuva sem raios dos extremos de FTA, do que na chuva sem raios dos extremos de FT. Na chuva sem raios dos sistemas com extremos de FT, os contornos da figura \ref{ftcfadwithout} são mais estreitos, indicando menor aleatoriedade nos valores de $Z_c$ observados.

\begin{sidewaysfigure}%[!H]
  \centering
  \includegraphics[width=19.5cm]{img/precipitacao3d/severo/percentil/90th/cCumFad_10deg_semraio_topFTApercentil}
  \caption{CCFDs para os extremos de FTA. Porção da precipitação sem raios.}
  \label{ftaccfdsubtrop}   
\end{sidewaysfigure} 

\begin{sidewaysfigure}%[!H]
  \centering
  \includegraphics[width=19.5cm]{img/precipitacao3d/severo/percentil/90th/cCumFad_10deg_semraio_topFTpercentil}
  \caption{CCFDs para os extremos de FT. Porção da precipitação sem raios.}
  \label{ftccfadwithout}   
\end{sidewaysfigure} 

%\begin{figure}[!ht]
%  \centering  
%  \adjustbox{trim={.0\width} {.04\height} {0\width} {.565\height},clip}%
%  \centering  
%  \adjustbox{trim={.349\width} {.045\height} {.322\width} {.565\height},clip}%  
%  {\includegraphics[width=27cm] {img/precipitacao3d/severo/percentil/90th/cCumFad_10deg_semraio_topFTApercentil}}
% \caption{CCFDs para os extremos de FTA entre 40-20S e 70-50W. Porção da precipitação sem raios.}
% \label{ftaccfdsubtrop}
%\end{figure} 



Na região entre 40-20S e 70-50W que engloba a Bacia do Prata, a banda brilhante foi menos evidente nos contornos de probabilidade associados a estrutura tridimensional da precipitação fora dos núcleos de raios, tanto para os extremos de FT, figura \ref{ftccfadwithout}, quanto para os extremos de FTA, figura \ref{ftacfadwithout}. 

Entre 40-20S e 70-50W, a porção sem raios da chuva dos extremos de FTA mostra que entre 0-5 km de altitude, a probabilidade de valores inferiores de $Z_c$ em relação as porções sem raios dos extremos de FT é maior. Observe como a mediana das amostras de probabilidades, marcada pela linha de contorno na cor preta no 50\textsuperscript{\underline{o}} percentil do CCFAD em cada caixa de 10 por 10 graus nas figuras \ref{ftccfdsubtrop} e \ref{ftaccfdsubtrop}, indica maior taxa de precipitação entre 0-5 km na porção sem raios das tempestades elétricas com índice FT extremo, figura \ref{ftccfdsubtrop}, mesmo que a estatística na parte superior direita de cada CCFAD indique maior percentual de perfis convectivos para a porção sem raios dos extremos de FTA, figura \ref{ftaccfdsubtrop}.


%, evidenciando que a chuva sem raios dos sistemas com as maiores taxas de raios no tempo é mais severa nesta região.

%\begin{figure}[!ht]
%  \centering  
%  \adjustbox{trim={.349\width} {.045\height} {.%322\width} {.565\height},clip}%
%  {\includegraphics[width=27cm] {img/precipitacao3d/severo/percentil/90th/cCumFad_10deg_semraio_topFTpercentil}}
% \caption{CCFDs para os extremos de FT entre 40-20S e 70-50W. Porção da precipitação sem raios.}
% \label{ftccfdsubtrop}
%\end{figure} 


Porém, a precipitação contida fora dos núcleos de raios dos sistemas extremos selecionados pelo índice FTA, situados entre 40-20S e 70-50W,  e que é explicitada por meio dos CFADs da figura \ref{ftacfadwithout}, revela que a probabilidade entre 0.001-2\%, representados pelas cores de contorno em preto e azul, atingem valores superiores de $Z_c$ do que quando compara-se com os sistemas extremos de FT, na figura \ref{ftcfadwithout}, também entre 40-20S e 70-50W.

Apesar da mediana das probabilidades dos CFADs mostrarem que a precipitação entre 0-5 km de altitude foi mais intensa para os sistemas extremos de FT e localizados entre 40-20S e 70-50W, ao avaliar os contornos de probabilidade cumulativa dos CCFADs na figura \ref{ftaccfdsubtrop}, referente ao estudo da estrutura tridimensional da precipitação fora dos núcleos de raios dos extremos de FTA, observa-se que, acima do 80\textsuperscript{\underline{o}} percentil os valores de $Z_c$ foram superiores em relação aos sistemas com índice extremo de FT, na figura \ref{ftccfdsubtrop}.

Os CFADs referentes as tempestades elétricas selecionadas por FTA possuem contornos de probabilidade em níveis de altitude mais elevados do que os CFADs dos sistemas selecionados por FT, tanto para a porção com raios quanto para a porção sem raios da precipitação dos sistemas.

%A diferença mais notável pode ser observada entre a figura \ref{ftacfadwithout} e \ref{ftcfadwithout} para 0S-10S e 50W-60W, que abrange principalmente o estado do Pará, e parte do Amazônas, Tocantis e Mato Grosso. O CFAD em \ref{ftacfadwithout} define valores de probabilidade em altitude 2 km mais elevada do que em \ref{ftcfadwithout}.


%Nas regiões entre 10N-0S e 70W-80W e entre 20S-40S e 50W-60W, em que \cite{cecil2005} apontam como região das tempestades mais severas na América do Sul, os CFADs em \ref{ftacfadwith} e \ref{ftacfadwithout} possuem contornos de probabilidade aproximadamente 1 km mais elevado do que em \ref{ftcfadwith} e \ref{ftcfadwithout}.

Como o último nível de altitude dos CFADs deste trabalho é limitado por altitudes com até 10\% de L, a maior definição de probabilidades de ocorrência de $Z_c$ em altitude para as tempestades selecionadas pelo índice FTA, indica que a convecção é mais intensa nos extremos de FTA do que nos extremos de FT. Principalmente quando observamos a morfologia da estrutura tridimensional da precipitação dos núcleos de raios, para os extremos do FTA e FT, expressa nos CFADs das figuras \ref{ftacfadwith} e \ref{ftcfadwith}, aonde os perfis de precipitação são classificados majoritariamente como convectivos.

A precipitação é bem mais frequente próxima da superfície, entre 0-3 km de altitude. Acima da região de mistura, a precipitação é mais rara de ocorrer. Em \cite{liu2008}, é mostrado que a densidade espacial de sistemas com no mínimo 20 dBZ em 2 km de altitude é globalmente maior do que os sistemas que atingem 20 dBZ em níveis superiores de altitude.


%A região de 10 por 10 graus, a qual o valor de H marcado no topo direito de cada CFAD, é menor para a precipitação dos núcleos de raios dos sistemas com extremo de FTA, figura \ref{ftacfadwith}, do que para  a precipitação dos núcleos de raios dos sistemas com extremos de FT, figura \ref{ftcfadwith}, e mesmo assim, o CFAD dos extremos de FTA, figura \ref{ftacfadwith}, possuiu maior altitude nos níveis de contorno de probabilidade de $Z_c$, o índice FTA mostra que a chuva  
%esteve associado com maior severidade de tempo do que FT.
%Pois, mesmo que a refletividade mais ocorrente esteja abaixo da região de mistura, a precipitação também é frequente conforme o aumento da altitude, mostrando que nestas regiões, os sistemas com índice FTA extremo têm maior número de ocorrência de chuva 
%mais chuvas na superfície e também maior precipitação acima de 10 km de altitude.
%com bastante representatividade estatística.
%maior quantidade de hidrometeoros na região de mistura e

Por exemplo na região do Panamá, Colômbia e Equador, entre 10N-0S e 70W-80W, o CFAD da figura \ref{ftacfadwith} possui contornos de probabilidade até 16 km de altitude. Na figura \ref{ftcfadwith}, os níveis de contorno param em 15 km.

A precipitação tridimensional observada nos núcleos de raios, explicitada nos contorno dos CFADs a cada 10 graus na figura \ref{ftacfadwith}, referente ao índice FTA, mostram valores de refletividade entre 1-3 dBZ maiores do que na figura \ref{ftcfadwith}, referente ao índice FT, principalmente quando observa-se os contornos de probabilidade de $Z_c$ acima de 5 km de altitude. Para a precipitação entre 1-2 km de altitude os valores são mais semelhantes entre as tempestades elétricas selecionadas por FTA e FT. 

%Porém, nos sistemas extremos de FTA, figura \ref{ftacfadwith}, há um estreitamento da região de contorno com os maiores valores de probabilidade associada a chuva na superfície, entre 3-5\%. Entre 20S-40S e 40-70W, o estreitamente é maior do que as demais regiões mostrando que as chuvas possuem maior probabilidade de estarem associadas com valores de 45 dBZ em \ref{ftacfadwith}.      

As mais baixas probabilidades de $Z_c$ observadas nos CFADs das figuras \ref{ftacfadwith} e \ref{ftcfadwith}, estão associadas com a estrutura tridimensional da precipitação mais severa. Observe os contornos de probabilidade entre 0.001-0.5\%. Estes níveis de contorno revelam os valores de $Z_c$ da precipitação mais rara entre os sistemas com índice extremo de FTA e FT, os quais provavelmente estiveram associados com enchentes rápidas, alta taxa de raios, chuva de granizo, fortes rajadas de vento e até mesmo ocorrência de tornados em algumas regiões. 
% nas figuras \ref{ftacfadwith} e \ref{ftcfadwith}

Os valores maiores valores de $Z_c$ foram registrados na figura \ref{ftacfadwith} entre 20S-40S e 40W-70W, sobre a Bacia do Rio da Prata, que abrange o Sul do Brasil, Paraguai, Uruguai e Argentina. A dinâmica de formação de Sistemas Convectivos de Meso-escala, como é discutido em \cite{Velasco1987} e \cite{Durkee2009}, somados com efeitos de topografia, como por exemplo na região da Serra de Córdoba na Argentina, a qual \cite{Rasmussen2011} mostram grande ocorrência de convecção profunda, promoveram sistemas em que a estrutura tridimensional da precipitação dos núcleos de raios atingiram valores de $Z_c$ superiores a 45 dBZ entre 10-15 km de altitude e chuvas na superfície com $Z_c$ acima de 55 dBZ, como mostram os contornos de probabilidade entre 0.001-0.5\%.

%Os processos de eletrificação dos extremos de FTA pode estar associada com a presença de granizo, cristais de gelo e gotas água super-resfriada, enquanto que para os extremos de FT, 




\subsection{A precipitação dos sistemas severos e o perfil atmosférico de temperatura.}

Os diagramas CCFTD e CFTD, descritos em \ref{chuvaEtemperatura}, são expostos nas figuras \ref{ccftd_fta_com}, \ref{ccftd_ft_com}, \ref{cftd_fta_com} e \ref{cftd_ft_com}, associados as tempestades elétricas com índice extremo de FTA e FT, apenas em suas porções com raios.

A partir dos CCFTDs das figuras \ref{ccftd_fta_com} e \ref{ccftd_ft_com}, iremos avaliar a intensidade convectiva dos sistemas com índice extremo de FTA e FT em determinadas regiões, com base na velocidade de crescimento ou decrescimento dos valores de $Z_{c}$ associados os contornos de probabilidade do 30\textsuperscript{\underline{o}}, 50\textsuperscript{\underline{o}}, 70\textsuperscript{\underline{o}} e 95\textsuperscript{\underline{o}} percentil das amostras de probabilidades expressas nos CFTDs das figuras \ref{cftd_fta_com} e \ref{cftd_ft_com}.	

\begin{sidewaysfigure}%[!H]
\centering
\includegraphics[width=19.5cm]{img/precipitacao3d/severo/percentil/90th/cftd_10deg_comraio_topFTApercentil}
\caption{CFTDs para os extremos de FTA. Porção da precipitação com raios.}
\label{cftd_fta_com}
\end{sidewaysfigure} 

\begin{sidewaysfigure}%[!H]
\centering
\includegraphics[width=19.5cm]{img/precipitacao3d/severo/percentil/90th/ccftd_10deg_comraio_topFTApercentil}
\caption{CCFTDs para os extremos de FTA. Porção da precipitação com raios.}
\label{cftd_fta_com}
\end{sidewaysfigure} 

\begin{sidewaysfigure}%[!H]
\centering
\includegraphics[width=19.5cm]{img/precipitacao3d/severo/percentil/90th/cftd_10deg_comraio_topFTpercentil}
\caption{CFTDs para os extremos de FT. Porção da precipitação com raios.}
\label{cftd_ft_com}
\end{sidewaysfigure} 

\begin{sidewaysfigure}%[!H]
\centering
\includegraphics[width=19.5cm]{img/precipitacao3d/severo/percentil/90th/ccftd_10deg_comraio_topFTpercentil}
\caption{CCFTDs para os extremos de FT. Porção da precipitação com raios.}
\label{ccftd_ft_com}
\end{sidewaysfigure} 

Então, para a região central da Bacia do Rio Amazonas, entre 10-0S e 70-60W, extraí-se as linhas de contorno do CCFTD referentes as probabilidades acumulativas de 30\%, 50\%, 70\% e 95\%. Desta forma obtemos quatro funções $f(x)=y$, \simbolo{name={$f(x)=y$},description={Função de uma variável}} em que $y$ corresponde aos valores de $Z_c$ e $x$ o perfil atmosférico de temperatura. Fazendo a derivada $\dfrac{dy}{dx}$ pode-se avaliar taxa de variação de $Z_c$ por temperatura (dBZ/\textsuperscript{o}C), para diferentes regimes de chuva, das mais frequentes até as mais raras, como mostra a figura \ref{deriv_amazonas}.

\begin{figure}[!ht]
  \centering
  \includegraphics[height=9cm]{img/precipitacao3d/deriv_ccftd/deriv_contornos_cdf_2_1}
  \caption{Taxa de variação de $Z_c$ no perfil de temperatura atmosférico para a região central da Bacia do Rio Amazonas, entre 10-0S e 70-60W.}
  \label{deriv_amazonas}  
\end{figure} 

% nos diferentes quartis do CCFTD dos extremos de FTA,  figura \ref{ccftd_fta_com},  e dos extremos de FT, figura \ref{ccftd_ft_com}. 

Na figura \ref{deriv_amazonas}, observa-se que a taxa de aumento de $Z_c$ em torno de -40 \textsuperscript{o}C e -15 \textsuperscript{o}C, é maior para os extremos de FTA, porém em torno de 0 \textsuperscript{o}C, a taxa de aumento de $Z_c$ é maior para os extremos de FT, mostrando que os hidrometeoros dos sistemas extremos de FTA, crescem em regiões mais frias do que nos extremos de FT.

% agregação e acreção é maior para os sistemas extremos de FTA e na região de derretimento, em torno de 0 \textsuperscript{o}C, o efeito da banda brilhante é mais acentuado para os extremos de FT.  

O aumento do fator de refletividade em torno de 0 \textsuperscript{o}C está associado a mudança do índice de refração da água devido a sua fusão. Já o aumento do fator de refletividade em torno de -40 \textsuperscript{o}C e -15 \textsuperscript{o}C representam o crescimento de hidrometeoros por agregação e acreção \cite{Fabry1995,Takahashi1978}.

Note na figura \ref{deriv_amazonas}, como a precipitação do 95\textsuperscript{\underline{o}} percentil de probabilidade de ocorrência tanto para FTA quanto para FT, é o regime de precipitação mais severa. Pois, há o crescimento de $Z_c$ em torno de -10 \textsuperscript{o}C e -15 \textsuperscript{o}C e não há banda brilhante, indicando precipitação a partir de granizo\footnote{Em \cite{Fabry1995}, este tipo de perfil é discutido como chuva a partir de gelo compacto.}. No 30\textsuperscript{\underline{o}}, 50\textsuperscript{\underline{o}} e 70\textsuperscript{\underline{o}} percentil dos extremos de FT, o efeito da banda brilhante associada ao derretimento é mais evidente do que para os extremos de FTA. 

Quando comparamos a região central da Bacia do Rio Amazonas, com a região central da Bacia do Rio da Prata, entre 30-20S e 60-50W, a microfísica de eletrificação se mostra diferente em cada local. Observa-se que no 50\textsuperscript{\underline{o}} percentil, a taxa de crescimento de $Z_c$ entre -40 \textsuperscript{o}C e -20 \textsuperscript{o}C é maior para a região da Bacia do Prata, figura \ref{deriv_prata}, do que para a região da Bacia Amazônica, figura \ref{deriv_amazonas}, tanto para os sistemas extremos de FTA quando para os sistemas extremos de FT, indicando maior crescimento de flocos de neve na precipitação severa sobre a Bacia do Prata. 


\begin{figure}[!ht]
  \centering
  \includegraphics[height=9cm]{img/precipitacao3d/deriv_ccftd/deriv_contornos_cdf_3_3}
  \caption{Taxa de variação de $Z_c$ no perfil de temperatura atmosférico para a região central da Bacia do Rio da Prata, entre 30-20S e 60-50W.}
  \label{deriv_prata}  
\end{figure} 

Apesar do 95\textsuperscript{\underline{o}} percentil mostrar maiores taxas de dBZ/\textsuperscript{o}C, em -15 \textsuperscript{o}C tanto para FTA quanto FT sobre a Bacia do Rio Amazonas, na figura \ref{deriv_amazonas}, do que sobre a Bacia do Rio da Prata, na figura \ref{deriv_prata}, os contorno de probabilidade acumulativa de 95\% nos CCFTD das figuras \ref{ccftd_fta_com} e \ref{ccftd_ft_com}, em -15 \textsuperscript{o}C, mostram valores de $Z_c$ de aproximadamente 3 dBZ superiores na região da Bacia Platina. Mesmo que o 95\textsuperscript{\underline{o}} percentil mostre maior crescimento de hidrometeoros na região mista sobre a Bacia Amazônica, a precipitação do 95\textsuperscript{\underline{o}} percentil na Bacia do Prata foi mais severa, pois possui maiores valores de $Z_c$.

O aumento abrupto de $Z_c$ associado a fusão da água, entre 30-20S e 60-50W, figura \ref{deriv_prata}, principalmente do 50\textsuperscript{\underline{o}} e 70\textsuperscript{\underline{o}} percentil, ocorrem em -4 \textsuperscript{o}C, enquanto que, entre 10-0S e 70-60W, figura \ref{deriv_amazonas}, o aumento de $Z_c$ ocorre mais próximo de 0 \textsuperscript{o}C, o que indica maior presença de água super-resfriada associada ao processo de derretimento da precipitação entre 30-20S e 60-50W, região da Bacia do Rio da Prata.  

Na região da Bacia do Prata, representada na figura \ref{deriv_prata}, o efeito de banda brilhante também é mais pronunciado para a precipitação com raios dos extremos de FT, o que mostra que as regiões eletricamente ativas da precipitação dos sistemas com índice extremo de FTA é menos estratificada do que nos extremos de FT, em ambas as Bacias Hidrológicas: da Prata e do Amazonas.

\newpage
\section{SEVERIDADE REGIONALIZADA}

Aqui, o estudo da densidade de probabilidade de FTA e FT, conforme mostrado na figura \ref{pdfFTAFT}, foi feito para os sistemas ocorridos em cada região de 2.5 por 2.5 graus de latitude e longitude entre 40N-10S e 90-30W. Verifica-se a distribuição geográfica, dos valores de FTA e FT mais frequentes e mais raros conforme cada localidade.

Buscando identificar quais dos índices, FTA ou FT foi mais sensível para indicar a intensidade convectiva das tempestades elétricas, torna-se interessante verificar quais são as regiões aonde sistemas com os menores valores de FTA e FT são mais frequentes.

Nas figuras \ref{extremosInfFTA} e \ref{extremosInfFT}, temos os valores de FTA e FT  para o 5\textsuperscript{\underline{o}} e 10\textsuperscript{\underline{o}} percentil, das distribuições de probabilidades regionalizadas a cada 2.5 por 2.5 graus.

\begin{figure}
  \subfloat[\textsuperscript{\underline{o}} percentil de FTA]{{\includegraphics[height=6.5cm, trim=0 0 0 0, clip]{img/DistEspacialPercentis/FTA/distEspacialValor005thFta}} \label{5oFta}}\\
  \subfloat[10\textsuperscript{\underline{o}} percentil de FTA]{{\includegraphics[height=6.5cm, trim=0 0 0 0, clip]{img/DistEspacialPercentis/FTA/distEspacialValor010thFta}} \label{10oFta}} 
  \caption{Distribuição espacial dos valores do 5\textsuperscript{\underline{o}} e 10\textsuperscript{\underline{o}} percentil da amostra de probabilidade do índice FTA a cada região de 2.5 por 2.5 graus de latitude e longitude.}
\label{extremosInfFTA}
\end{figure} 

A linha de contorno na cor preta em cada mapa apresentado nesta seção, corresponde ao valor do percentil determinado para a análise regional, porém, é referente a amostragem total exposta na figura \ref{cdfFTAFT}.

No ambiente oceânico e costeiro, as tempestades elétricas mais frequentes devem possuir menores índices de severidade do que no continente, pois na costa e oceano observa-se as maiores probabilidades de ocorrência de chuva quente \cite{Liu2009}. 
%o aquecimento da superfície durante o ciclo diurno é menor e o processo de colisão coalescência é dominante em relação aos processos que envolvem a formação de gelo de nuvem 

Nas figuras \ref{5oFta} e \ref{10oFta}, os contornos com valores de 0.05 $\times$ 10$^{-4}$ e 0.12 $\times$ 10$^{-4}$ raios minutos$^{-1}$
km$^{-2}$ respectivamente, demarcam claramente a divisão entre a convecção oceânica e a continental. O Oceano Verde, conceito associado a convecção durante o regime de ventos de Oeste na estação chuvosa Amazônica, discutido por \citeonline{silva2002lba,williams2002}, é bastante evidente. A região central da Bacia Amazônica possui os  valores de FTA na mesma ordem de magnitude e no mesmo percentil das densidades de probabilidades de FTA regionalizadas das tempestades elétricas oceânicas e costeiras.

Os valores de FT associados ao 5\textsuperscript{\underline{o}} e 10\textsuperscript{\underline{o}} percentil, mostrados nas figuras \ref{5oFt} e \ref{10oFt}, revelam os menores valores de FT no centro do continente, principalmente nas regiões continentais fora da área de atuação da ZCIT e de sistemas transientes subtropicais.

\begin{figure}[!ht]
  \centering{
  \subfloat[5\textsuperscript{\underline{o}} percentil de FT]{{\includegraphics[height=6.5cm, trim=0 0 0 0, clip]{img/DistEspacialPercentis/FT/distEspacialValor005thFt}} \label{5oFt}}\\
  \subfloat[10\textsuperscript{\underline{o}} percentil de FT]{{\includegraphics[height=6.5cm, trim=0 0 0 0, clip]{img/DistEspacialPercentis/FT/distEspacialValor010thFt}} \label{10oFt}}
  }    
  \caption{Distribuição espacial dos valores do 5\textsuperscript{\underline{o}} e 10\textsuperscript{\underline{o}} percentil da amostra de probabilidade do índice FT a cada região de 2.5 por 2.5 graus de latitude e longitude.}
\label{extremosInfFT}
\end{figure} 

Para avaliar a distribuição geográfica dos extremos superiores dos índices FTA e FT, verificou-se os valores do 95\textsuperscript{\underline{o}} e 99\textsuperscript{\underline{o}} percentil das amostragens, os quais são expostos nos mapas das figuras \ref{extremosSupFTA} e \ref{extremosSupFT}.

%: regionalizadas, representados nas cores das figuras \ref{extremosSupFTA} e \ref{extremosSupFT}; e totais (ver figura \ref{seriesFtaFt}), representados pela linha de contorno das figuras \ref{extremosSupFTA} e \ref{extremosSupFT}.

Observa-se na figura \ref{95oFta} que os sistemas com índice FTA  superior à 52.76 $\times$ 10$^{-4}$ raios minutos$^{-1}$
km$^{-2}$, são considerados de severidade extrema, pois correspondem a valores superiores ao valor do 95\textsuperscript{\underline{o}} percentil da amostragem total de FTA da figura \ref{pdfFta}. Porém, em regiões no interior do continente, os valores de FTA do 95\textsuperscript{\underline{o}} percentil das amostragens regionalizadas, atingiram  111.97 $\times$ 10$^{-4}$ raios minutos$^{-1}$ km$^{-2}$.
%, mostrando que nestas regiões, valores de 52.76 $\times$ 10$^{-4}$ raios minutos$^{-1}$ km$^{-2}$ são mais frequentes do que nas regiões em que passa a linha de contorno de 52.76 $\times$ 10$^{-4}$ raios minutos$^{-1}$ km$^{-2}$.
%raios por minutos por quilômetros quadrado

\begin{figure}[!ht]
\centering
{\includegraphics[height=13.5cm, trim=0 0 0 0, clip]{img/DistEspacialPercentis/FTA/distEspacialValor095thFta}} 
\caption{Distribuição espacial dos valores do 95\textsuperscript{\underline{o}} percentil da amostra de probabilidade do índice FTA a cada região de 2.5 por 2.5 graus de latitude e longitude.}
\label{95oFta}
\end{figure} 
  
\begin{figure}[!ht]
\centering  
{\includegraphics[height=13.5cm, trim=0 0 0 0, clip]{img/DistEspacialPercentis/FTA/distEspacialValor099thFta}}
\caption{Distribuição espacial dos valores do  99\textsuperscript{\underline{o}} percentil da amostra de probabilidade do índice FTA a cada região de 2.5 por 2.5 graus de latitude e longitude.}
\label{99oFta}
\end{figure} 


Os valores do 99\textsuperscript{\underline{o}} percentil na figura \ref{99oFta}, mostram que no Leste do estado do Amazonas, no Acre e Tocantis e Sudeste do Peru e Norte da Bolívia, regiões estas que compõem o Oceano Verde, a severidade extrema de FTA possui valores entre 148.93-230.00  $\times$ 10$^{-4}$ raios minutos$^{-1}$ km$^{-2}$,  valores que correspondem aos mais extremos do continente Sul-americano. 

Mesmo que a Floresta Amazônica seja um Oceano Verde para atmosfera durante as fases ativas do SAMS, durante o regime de ventos de Leste na estação chuvosa que associa-se as fases inativas da SAMS e durante a estação de transição seca-úmida (SON), ``o Mar Verde"  ~fica revolto. Mesmo que a Floresta Amazônica dialogue com a precipitação como um oceano, este oceano possui temperatura superficial média na classe das maiores temperaturas superficiais continentais globais e está cercado por um vasto continente. Portanto, tem a capacidades de gerar tempestades elétricas extremamente severas, mostrando que a interação entre a Floresta Amazônica e a atmosfera é bastante diversificada.

As regiões dos maiores valores do 95\textsuperscript{\underline{o}} e 99\textsuperscript{\underline{o}} percentil do índice FTA, os quais são  expostos nas figuras \ref{95oFta} e \ref{99oFta}, são principalmente: a Bacia do Rio da Prata, a região Leste Amazônia e as regiões  do planalto Brasileiro, que se estendem por quase todo o país.

Observa-se que os sistemas mais severos da América do Sul ocorrem associados ao relevo nas regiões entre o Pantanal Mato-grossense e o Planalto Central Brasileiro, entre as Bacias dos Rios: Xingu, Araguaia e Tocantis e também o Planalto Central Brasileiro, entre a Bacia do Rio Paraná e o Planalto Meridional Brasileiro, aonde está localizado os planaltos e chapadas da Bacia do Paraná. Nestas regiões os sistemas severos possuem índice FTA superiores à 80 $\times$ 10$^{-4}$ raios minutos$^{-1}$ km$^{-2}$, como mostram as cores das figuras \ref{95oFta} e \ref{99oFta}. 

Note que para saber aproximadamente o número de raios produzidos pelos sistemas extremos de FTA temos que multiplicar o índice FTA pela área do sistema. Por exemplo, a equação \ref{FTAkm2}, descreve que nas regiões em que os sistemas extremos possuem 100 $\times$ 10$^{-4}$ raios minutos$^{-1}$ km$^{-2}$, um sistemas severo com área de 10$^3$ km$^2$ então possui 10 raios observados pelo LIS em 1 minuto.  

\begin{equation}
100 \times 10^{-4} \left[ \frac{\mathrm{raios}}{\mathrm{minutos}~\mathrm{km}^2} \right]  10^3 [ \mathrm{km}^2 ] = 10 \left[ \frac{\mathrm{raios}}{\mathrm{minutos}}\right]  
\label{FTAkm2}
\end{equation}

Os mapas das figuras \ref{95oFt} e \ref{99oFt}, mostam que nas Bacias: do Rio da Prata principalmente, do Rio Araguaia, Rio Xingu e Rio Tocantis, são locais em que os sistemas possuem os maiores índices de FT tanto no 95\textsuperscript{\underline{o}} quanto no 99\textsuperscript{\underline{o}} percentil.

\begin{figure}[!ht]
\centering
{\includegraphics[height=13.5cm, trim=0 0 0 0, clip]{img/DistEspacialPercentis/FT/distEspacialValor095thFt}} 
\caption{Distribuição espacial dos valores do 95\textsuperscript{\underline{o}} percentil da amostra de probabilidade do índice FT a cada região de 2.5 por 2.5 graus de latitude e longitude.}
\label{95oFt}
\end{figure} 
  
\begin{figure}[!ht]
\centering  
{\includegraphics[height=13.5cm, trim=0 0 0 0, clip]{img/DistEspacialPercentis/FT/distEspacialValor099thFt}}
\caption{Distribuição espacial dos valores do  99\textsuperscript{\underline{o}} percentil da amostra de probabilidade do índice FT a cada região de 2.5 por 2.5 graus de latitude e longitude.}
\label{99oFt}
\end{figure}

Os maiores valores do 95\textsuperscript{\underline{o}} e 99\textsuperscript{\underline{o}} percentil do índice FT, figuras \ref{95oFt} e \ref{99oFt},  ficam situados na região Sul da América do Sul, compatível com a região em que \citeonline{cecil2005}, apontam como o local das tempestades categoria 5, ou seja, das mais severas do globo.
\chapter{CONCLUSÃO}

Cria-se uma metodologia para caracterizar as tempestades elétricas observadas na AS a partir das medidas integradas dos sensores LIS, PR e VIRS a bordo do TRMM, durante o período de janeiro de 1998 a dezembro de 2011. As tempestades elétricas foram definidas por pixeis contíguos com $T_b$ $\leq$ 258 K do canal 4 do VIRS com pelo menos um raio do LIS. Para tempestades elétricas consideras enormes, foi utilizado o limiar de $T_b$ $\leq$ 221 K, pois as mesmas faziam parte de sistemas frontais ou ZCAS. A partir deste procedimento foi criado um banco de dados de 157~592 tempestades elétricas do TRMM que são compostos basicamente de: distribuição de $T_b$ e respetivas latitude e longitudes do VIRS, perfis verticais de $Z_c$, classificação convectiva, estratiforme e outros, taxa de precipitação na superfície e respectivas latitudes e longitudes, localização (latitude e longitude) dos eventos, grupos e raios,  e tempo de visada com resolução de 0,25$^{\circ}$ $\times$ 0,25$^{\circ}$.

Com base neste subconjunto de dados de tempestades elétricas do TRMM, foi avaliado o Marco das tempestades elétricas na AS, a partir do estudo do ciclo diurno, ciclo anual, distribuição geográfica de densidades de raios e tempestades elétricas.


Para estudar a severidade das tempestades elétricas foram utilizados dois índices de taxas de raios, uma vez que na literatura os extremos são avaliados em termos da taxa de raios no tempo, mas não avaliam a eficiência da produção de raios. Portanto foi utilizado a taxa de raios no tempo -- FT -- e a taxa de raios no tempo normalizado pela área do sistema -- FTA. O FT é comumente utilizado  para prognosticar a ocorrência de granizo e tornados \cite{williams1999,goodman1988,schultz2011,gatlin2010}. Já o FTA foi pensado com o intuito de analisar a eficiência da tempestade em produzir raios. 

Os processos microfísicos que levam as tempestades elétricas a terem mais ou menos raios, foram investigados utilizando os CFADs \cite{yuter1995}. Entretanto, a América do Sul cobre uma vasta extensão territorial que vai do equador até os sub-trópicos e a altura das isotermas podem variar significativamente e como os processos de eletrificação de nuvens  dependem essencialmente da temperatura \cite{Takahashi1978}, faz-se necessário converter a base de altura dos CFADs para temperaturas. Logo criou-se o diagrama CFTD que proporciona uma compreensão a respeito das mudanças de $Z_c$ em função da temperatura do perfil atmosférico, o que auxiliou na identificação de água super-resfriada, cristais de gelo, agregados, \textit{graupel} e granizo, que são responsáveis pela transferência de cargas dentro das nuvens \cite{Takahashi1978,saunders2008}.


\section{MARCO DAS TEMPESTADES ELÉTRICAS}

No o ciclo diurno das tempestades elétricas observadas sobre a AS, observa-se que 40\% dos sistemas ocorreram entre 13h e 17h (HL), mostrando que o aquecimento diurno e o aumento da camada limite planetária no decorrer do dia, aumenta a probabilidade de ocorrência em 5,4 vezes (às 14h HL) em relação ao horário de menor ocorrência, às 9h HL.

No continente observa-se um predomínio da atividade elétrica entre às 13h e 17h, enquanto que no oceano existe uma variação dependendo da proximidade com o continente, mas em geral, sobre o oceano, observa-se dois máximos: um por volta das 20h e outro entre 4--5h. Quando  próximo do continente o pico das 20h se desloca para às 15h.  

Apesar das tempestades elétricas sobre a AS possuírem um ciclo diurno bem definido, entre 0$^{\circ}$--10$^{\circ}$ Norte e 80$^{\circ}$--70$^{\circ}$ Oeste, observou-se a maior probabilidade ($\simeq$0,4\%) de tempestades elétricas noturnas da América do Sul, entre 0h e 00:59h. A circulação de vale e montanha associada com a topografia elevada na Colômbia, principalmente a região do Parque Nacional Natural Paramillo, e o Lago Maracaibo na Venezuela, e a atuação da Zona de Convergência Intertropical (ZCIT), promovem condições para o desenvolvimento de tempestades elétricas noturnas de maneira mais eficiente do que as demais regiões da AS \cite{burgesser2012}.

Sobre o oceano costeiro, temos que entre 0$^{\circ}$--10$^{\circ}$ Norte e 90$^{\circ}$--80$^{\circ}$ Oeste (Panamá e Sul da Costa Rica,  Oceano Pacífico que engloba o Parque Nacional da Ilha do Coco na Costa Rica, incluindo ilhas Galápagos no Equador) foi observado a maior probabilidade de ocorrência de tempestades elétricas da AS, com pico de ocorrência às 4h e outro às 14h. Possivelmente o aquecimento superficial durante o dia e as trocas de energia na forma de calor entre o oceano e a atmosfera explicam esta distribuição bimodal.

A maior atividade horária de tempestades elétricas ( 0,8\%), ocorreu entre 10$^{\circ}$--0$^{\circ}$ Sul e 70$^{\circ}$--50$^{\circ}$ Oeste e 20$^{\circ}$--10$^{\circ}$ Sul e 60$^{\circ}$--50$^{\circ}$ Oeste, durante às 14h e 16h.

Apesar dos SCMs apresentarem um ciclo diurno com atividade noturna \cite{Velasco1987, Durkee2009, machado1998}, entre 30$^{\circ}$--20$^{\circ}$ Sul e 60$^{\circ}$--50$^{\circ}$ Oeste, as tempestades elétricas apresentaram um máximo por volta das 15h.


%\section{CICLO ANUAL}

No ciclo anual, a estação de tempestades elétricas na América do Sul se configura entre outubro e março e possui dois picos: janeiro, durante o verão austral; e outubro, período de transição entre a estação seca e chuvosa, quando se observa a maior atividade de tempestades elétricas. 

Distribuições bimodais de atividade elétrica, março e outubro, estão restritas a parte Norte da AS entre 0-10N e 80-70 Oeste e entre a Amazônia, definidas pelas regiões 10S-0 e 80-50 Oeste. 

A parte Nordeste da AS, entre 0-10 Norte e 70-50 Oeste, é marcada por um máximo em agosto, verão do hemisfério Norte. No sul da AS,  entre 40$^{\circ}$--20$^{\circ}$ Sul e 70$^{\circ}$--60$^{\circ}$ Oeste, região da Argentina e Bacia do Prata, foi encontrada a estação de tempestades elétricas mais curta (2 meses), e uma estação quase sem raios durante o inverno austral.

A região com a estação de maior duração de tempestades elétricas,  9 meses (março a novembro), foi sobre a Colômbia e parte Oeste da Venezuela que abrange até o lago Maracaibo.  


%\section{DISTRIBUIÇÃO GEOGRÁFICA}

Referente as distribuição de densidades geográficas, as maiores densidades de tempestades elétricas situam-se na parte Norte e Noroeste da AS, ou seja, na Colômbia associado ao extremo Norte da Cordilheira dos Andes e ao Norte/Noroeste da Floresta Amazônica respectivamente, com valores entre 3,5-4,7 $\times$ 10$^{-4}$ km$^{-2}$. Além disso observa-se uma vasta região com densidades superiores a  2,5 $\times$ 10$^{-4}$ km$^{-2}$, que conta com regiões de topografia elevada como à Noroeste do Lago Titicaca no Peru e algumas regiões do Planalto Brasileiro como sobre a Serra Catarinense e o Parque Nacional das Emas ao Sudoeste de Goias, além de grande parte da região Amazônica, Colômbia, Venezuela e Panamá.  

Em termos de sazonabilidade temos que a primavera apresenta a maior atividade de tempestades elétricas (57~861) seguidas pelo verão (46~077), outono (36~804) e inverno (16~850). 

Durante o verão a máxima atividade é encontrada ao Sul da Amazônia se estendendo pela parte central e Sudeste do Brasil, além da cordilheira dos Andes abrangendo o Peru e Bolívia. No Outono a maior atividade se concentra no litoral do Maranhão e Pará, Sudeste e Noroeste  da Colômbia. No inverno a máxima atividade  fica restrita na região costeira da Colômbia e Panamá. Finalmente na primavera observa-se o máximo se estendendo do Panamá, Colômbia, Venezuela, Sul da Venezuela, Noroeste e centro da Amazônia e Nordeste da Bolívia. 
 
As maiores densidade de raios por ano por quilômetro quadrado sobre a AS são observadas: na Foz do Rio Catatumbo na Venezuela com 371,2 raios ano$^{-1}$ km$^{-2}$ e em Cochabamba na Bolívia com $\simeq$60  raios ano$^{-1}$ km$^{-2}$. Regiões com densidades de raios entre 30-60 raios ano$^{-1}$ km$^{-2}$, correspondem ao Pico das Agulhas Negras na Serra da Mantiqueira entre Minas Gerais e o Rio de Janeiro, Pico da Neblina no Amazonas, parte central da Bacia do Prata, na cordilheira dos Andes do Peru, e extremo Norte da Cordilheira dos Andes sobre a Colômbia. 

Em termos sazonais, os extremos de densidade de raios ficam restritos ao inverno e primavera. Sendo que na primavera temos as maiores densidades correspondentes aos máximos anuais, exceto Pico das Agulhas Negras e parte central da Bacia do Prata. Já no inverno basicamente sobre o extremo Norte da Cordilheira dos Andes na Colômbia e no lago Maracaibo na Venezuela. As regiões do Pico das Agulhas Negras apresenta alta atividade durante o verão e primavera, já  a parte central da Bacia do Prata, apresenta atividade elétrica  no verão, outono e primavera.


Referente as regiões mais eficientes em termos de densidades de raios por tempestades, temos que a região do Lago Maracaibo, na Foz do Rio Catatumbo (pixel com 772 km$^{2}$) se mostra a mais eficiente da AS, atingindo o valor máximo de 11,73 $\times$ 10$^{-2}$ ano$^{-1}$ km$^{-2}$, que representa {114 333} raios ano$^{-1}$, seguida da Bacia do Prata e Serra de Córdoba da Argentina com valores $\simeq$ 5,5 $\times$ 10$^{-2}$ ano$^{-1}$ km$^{-2}$.


\section{TEMPESTADES ELÉTRICAS SEVERAS}

As tempestades elétricas severas, ou, as mais intensas e raras, foram definidas a partir do limiar de 90\% da distribuição de frequência dos índices FT e FTA, que tiveram pelo menos um perfil vertical do PR com chuva válida e $VT_m$ maior do que 60 segundos  definindo um grupo de 9475 tempestades elétricas extremas de FTA com valores entre 29,3--1258 $\times$ 10$^{-4}$ raios min$^{-1}$ km$^{-2}$, e mais 9475 tempestades elétricas extremas de FT com os valores entre 47,2--1283,6 raios min$^{-1}$.  Estes valores são consistentes com os trabalhos de \citeonline{cecil2005}, que encontraram estes valores nos \textit{top} 0,01\% das PFs (categoria 3, 4, 5).


As tempestades elétricas extremas de FTA tem 72\% de fração convectiva e 32\% de fração estratiforme, enquanto as tempestades elétricas extremas de FT 20\% de fração convectiva e 75\%. Estas características indicam que os extremos de FTA correspondem a sistemas novos ou em via de maturação, quando a corrente ascendente é mais intensa, enquanto os sistemas extremos de FT correspondem a sistemas maduros ou em fase de decaimento \cite{learyHouse1979}. 


O estudo da precipitação tridimensional por meio dos CFADs e CCFADs verifica-se que os perfis com raios possuem maiores valores de  $Z_c$ e são mais profundos na atmosfera. Entre 5--7 km de altitude os valores de $Z_c$ para a precipitação com raios atingem valores entre 5 dBZ e 10 dBZ superiores do que para a precipitação sem raios, tanto para os extremos de FTA quanto FT. As regiões sem raios dos eventos extremos de FT são mais estratiformes do que para os extremos de FTA e apresentam forte assinatura da banda brilhante. Os pixeis sem raios da precipitação dos extremos de FTA não há assinatura da banda brilhante nos sistemas sobre o continente, mas há banda brilhante sobre o oceano. Nas regiões com raios, as tempestades elétricas dos extremos de FTA possuem entre 1-3 dBZ a mais do que os valores de $Z_c$ dos extremos de FT, especialmente acima de 5 km de altitude. Do ponto de vista da intensidade convectiva, ambos FTA e FT, são mais vigorosos do equador para os sub-trópicos. 


Com a criação dos CFTDs e CCFTDs identifica-se que durante o desenvolvimento vertical, a intensidade convectiva pode ser mensurada avaliando a redução ($\simeq$7 dBZ) de $Z_c$ devido ao congelamento dos hidrometeoros acima de 0 $^{\circ}$C. Regiões com condições de tempo severo, possuem um caminho de temperatura maior para atingirem o congelamento, devido a alta concentração de gotas pequenas \cite{bigg1953}, o que aumenta a espessura da camada fria de nuvem e consequentemente  expõe os hidrometeoros a um número maior de colisões, contribuindo para uma maior concentração e diversidade de hidrometeoros: cristais de gelo de diferente formas geométricas, agregados e flocos de neve, \textit{graupel} e granizo com diferentes tamanhos e densidades. 

A probabilidade de perfis de $Z_c$ por nível de temperatura (CFTD), apresenta quantis mais largos para os extremos de FTA do que nos extremos de FT. Esse efeito demonstra a maior diversidade de hidrometeoros no ambiente das tempestades elétricas extremas de FTA.    

%Isso implica que nas FTAs temos uma maior camada mista que possibilita maior carregamento ....
%Diferenças entre trópicas e sub-trópicos. Com e sem. 
%Falar das diferenças em termo de temperatura.
%Taxas de variação de $Z_c$ por temperatura.
%Combinando as observações de raios com a análise dos processos microfísicos de crescimentos dos hidrometeoros de nuvem com base na estrutura tridimensional da precipitação, 

A convecção mais intensa sobre a AS consiste em um fenômeno atmosférico capaz de processar a energia potencial convectiva em uma área entre 50-400 km$^{2}$, quando há extremos de FTA, que ocorre nos estágios iniciais das tempestades elétricas. Tempestades elétricas com FT extremo, consistem em um fenômeno  atmosférico com dimensões de $\simeq$50~000 km$^{2}$. As condições de tempo severo que são associada com a intensidade da corrente ascendente, devem ocorrer nas tempestades elétricas extremas de FT, porém quando há novas células convectivas (50-400 km$^{2}$) com FTA extremo, embebidas na extensão das tempestades elétricas maiores. Tempestades elétricas com FT extremo, podem possuir grande número de raios, porém distantes entre si. Nestes casos, temos tempestades grandes com núcleos de raios pouco eficientes e que provavelmente não estão associadas com fortes correntes ascendentes, ou seja, com condições de tempo severo.

Comparando as regiões da Bacia Amazônica (10S-0 e 70W-60W) com a região das tempestades mais severas sobre a AS, na Bacia do Prata (30S-20S e 60W-50W), observa-se que para o 95 \textsuperscript{\underline{o}} percentil do CCFTD, entre 0 $^{\circ}$C e -15 $^{\circ}$C, as tempestades elétricas extremos de FTA na região Amazônica decrescem 5 dBZ enquanto que as tempestades elétricas extremas de FT decrescem 10 dBZ e na Bacia do Prata, FTA decresce também 5 dBZ, enquanto FT decresce apenas 6 dBZ. A máxima taxa de decrescimento de $Z_c$ em função da temperatura (50 \textsuperscript{\underline{o}} percentil) é de $\simeq$-1 dBZ $^{\circ}$C$^{-1}$ para a Amazônia e $\simeq$-0,85 dBZ $^{\circ}$C$^{-1}$ para a Argentina.


As tempestades elétricas com os maiores (95 \textsuperscript{\underline{o}} percentil) valores de FTA, são encontradas sobre uma vasta região da AS, com valores de 52,76 $\times$ 10$^{-4}$ raios min$^{-1}$ km$^{-2}$. Valores de FTA superiores a 84 $\times$ 10$^{-4}$ raios min$^{-1}$ km$^{-2}$ são encontrados ao Sul e Leste da Bacia Amazônica, parte central e Sudeste do Brasil, Sul do Peru, Bolívia, Paraguai, Oeste do Uruguai, Norte e centro da Argentina. Ao considerar o 95 \textsuperscript{\underline{o}} percentil da amostragem de FTA, as tempestades elétricas passam a produzir 148,93 raios min$^{-1}$ km$^{-2}$, associadas a regiões de topografia elevada, principalmente entre o Pantanal Mato-grossense e o Planalto Central Brasileiro, entre as Bacias dos Rios Xingu, Araguaia e Tocantis, região do Planalto Meridional Brasileiro aonde está localizado os planaltos e chapadas da Bacia do Paraná e região Leste da Serra de Córdoba.


Os maiores valores do 95\textsuperscript{\underline{o}} ($\geq$ 92,84 raios min$^{-1}$) e 99\textsuperscript{\underline{o}} (($\geq$ 272,28 raios min$^{-1}$)) percentil do índice FT ficam situados na região Sul da América do Sul.


Os sistemas severos são na verdade as tempestades que tem tanto os valores extremos de FT quanto de FTA, dessa maneira, a intersecção dos dois grupos de extremos indicará a região de máxima severidade. Nesse sentido temos o Planalto Meridional Brasileiro e no semi-árido Argentino a Leste da Cordilheira dos Andes e da Serra de Córdoba.



\section{IMPLICAÇÕES PARA A ELETRIFICAÇÃO DAS NUVENS}



Como os hidrometeoros dos sistemas extremos de FTA crescem em regiões mais frias do que nos extremos de FT,
conforme \citeonline{Takahashi1978}, podemos considerar que os núcleos de raios das tempestades elétricas com extremo de FTA entre 10-0S e 70-60W, possuíram centros de cargas predominantemente negativos, enquanto que os núcleos de raios das tempestades elétricas dos extremos de FT, em que as maiores taxas de crescimento de $Z_c$ ocorreram entre -8\textsuperscript{o}C e 0\textsuperscript{o}C, os centros de cargas foram predominantemente positivos.

%exceto no 95\textsuperscript{\underline{o}} percentil em que o maior acréscimo de $Z_c$ foi em -10\textsuperscript{o}C
Apensar do crescimento do \textit{graupel} e o granizo em temperaturas entre -10\textsuperscript{o}C e 0\textsuperscript{o}C indicar carregamento positivo, as variações do conteúdo de água líquida de nuvem  podem inverter a polaridade das partículas que crescem em temperatura inferior a -10\textsuperscript{o}C \cite{Takahashi1978}. Durante uma tempestade, as correntes ascendentes e o entranhamento na região de crescimento do gelo poderão elevar ou diminuir o conteúdo de água líquida do ambiente.

Também, a taxa de coleta de gotículas de água líquida pelo \textit{graupel} durante o processo de acreção, pode ter mais influência no sinal das cargas dos hidrometeoros do que o conteúdo de água líquida \cite{jayaratne1983,saunders1991effect,brooks1997,Takahashi2002}. 
Apesar do conteúdo de água líquida efetivamente coletado no processo de acresção ser proporcional ao conteúdo de água líquida da nuvem, as velocidades relativas entre as partículas e a eficiência de coleta, além de influenciarem sobre a polaridade, também determinam a estrutura do granizo: poroso, compacto ou esponjoso \cite[p.~335]{mason1971_2ed}.  
 
Mesmo considerando hipoteticamente que entre 10-0S e 70-60W, as tempestades elétricas com extremos de FT possuíram maior probabilidade de centros de cargas principal positivos, a porcentagem de raios nuvem-solo  positivos pode ser menor do que a porcentagem de raios nuvem-solo negativos. Em \citeonline{carey2007}, tempestades elétricas com 25\% dos raios nuvem-solo positivos foram consideradas como tempestades elétricas positivas.

Em \citeonline{fernandes2005} e \citeonline{fernandes2006}, descreve-se que no ambiente amazônico seco e poluído, as nuvens podem ter altura da base mais elevadas e mesmo que os processos de crescimento de hidrometeoros ocorram em regiões de maior altitude e consequentemente de menor temperatura, o centro de carga principal fica mais distante da superfície, causando  o aumento da razão entre os raios intranuvens e nuvem-solo e favorecendo a ocorrência de raios nuvem-solo positivos a partir do centro de carga positivo na base da nuvem, que de acordo com o modelo do tripolo eletrostático proposto em \citeonline{williams1989}, fica mais próximo do solo do que o centro de cargas negativo principal.

%como sugerem os extremos de FTA em relação aos extremos de FT,
%A partir do banco de dados gerado no experimento CHUVA como descrevem \cite{machado2014chuva}, serão possível análises das taxa de crescimento de dBZ~\textsuperscript{o}C, com base em observações por radar, radio-sondagem, conjuntamente com as observações de raios e de campo eletrostático. Desta forma pode-se investigar se o processo microfísico    

Além  dos fatores que podem aumentar a taxa de acresção como: cisalhamento do vento próximo a superfície, aumento da instabilidade condicional e menor espessura entre a base da nuvem e a isoterma de 0\textsuperscript{o}C, que em \citeonline{carey2007,albrecht2011} estiveram associados a ocorrência de tempestades elétricas positivas as quais supostamente teriam centro de carga principal positivo, portanto um tripolo eletrostático invertido, a geometria e densidade volumétrica dos centros de cargas em relação a superfície pode determinar o caminho de menor resistência elétrica para a formação de um raio nuvem-solo positivo ou negativo.



%\apendice
%\include{capitulos/cronogramaProrroga}
%\chapter{ARTIGO SUBMETIDO}

Este artigo foi submetido em 2012, porém recusado, pela revista \textit{Atmospheric Research} edição especial para a \textit{International Conference on Atmospheric Electricity (ICAE)} realizada em 2011 no Rio de Janeiro. Participei da ICAE 2011 apresentando dois trabalhos na forma de pôster.

\includepdf[pages=1, scale=1]{/home/evandro/Dropbox/Tese/submission/atmosres/atmosres-d-12-00171.pdf}
\includepdf[pages=-]{/home/evandro/Dropbox/Tese/submission/ATMOSRES-D-12-00171.pdf}

%\chapter{TRABALHOS APRESENTADOS EM CONFERÊNCIAS}

Foram 5 trabalhos apresentados em 3 conferências internacionais: 2 trabalhos apresentados como primeiro autor, na \textit{International Conference on Atmospheric Electricity (ICAE)} no Rio de Janeiro em 2011;  1 trabalho apresentado como primeiro autor na \textit{16th International Conference on Clouds and Precipitation (ICCP)} realizada em Leipzig, Alemanha em 2012; 2 trabalhos apresentados como primeiro autor, na \textit{International Conference on Atmospheric Electricity (ICAE)} em Norman, OK, Estados Unidos em 2014.

Irei listar apenas os dois trabalhos mais relevante os quais foram apresentados e discutidos mais recentemente na ICAE 2014 e estão sendo aprimorados para a submissão de um artigo para revista. 

\includepdf[pages=-]{/home/evandro/Dropbox/Tese/TrabalhosApresentados/icae2014/AnselmoMorales_oral.pdf}
\includepdf[pages=-]{/home/evandro/Dropbox/Tese/TrabalhosApresentados/icae2014/AnselmoMorales_poster.pdf}









\cleardoublepage
\phantomsection
\renewcommand{\bibname}{REFER\^ENCIAS}
\renewcommand{\arraystretch}{2}
\bibliography{referencias}
%\addcontentsline{toc}{chapter}{Indice Remissívo}
\cleardoublepage
\phantomsection
%\addcontentsline{toc}{chapter}{ÍNDICE REMISSIVO}
\printindex
\end{document}