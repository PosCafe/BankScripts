\documentclass[smaller]{beamer}
\usepackage[brazil]{babel}
\usepackage[utf8]{inputenc}
\usepackage[accumulated]{beamerseminar}
\usepackage{multimedia}
\usepackage{ragged2e}
\usepackage{colortbl}
\usepackage{tikz}
\usetikzlibrary{positioning}
%\usetikzlibrary{shadows}
%\usepgflibrary{arrows}
%\usepackage{tikz}
\usetikzlibrary{shapes,arrows}
\usepackage{verbatim}
%\usepackage[active,tightpage]{preview}
%\PreviewEnvironment{tikzpicture}
%\setlength\PreviewBorder{5pt}%

\setbeamertemplate{caption}[numbered]
\setbeamerfont{caption}{size=\scriptsize}
\setbeamerfont{footnote}{size=\tiny}
\setbeamerfont{section}{size=\scriptsize}
%\usefonttheme[onlysmall]{structuresmallcapsserif}
%structurebold}
\usepackage{tabularx}
\usepackage{textcomp}
\usepackage{color}
%\usepackage{pgf}
%\usepackage{beamerthemesplit}
%\usepackage[font=small,labelfont=small,textfont=small]{caption}
%\usepackage[center,small]{caption}
%\usetheme{default}
\usefonttheme{serif}
\usepackage{hyperref} 
%\usetheme{Copenhagen}
%\usetheme{Antibes}
%\usetheme{Berkeley}
%\usetheme{JuanLesPins}
%\usetheme{Berlin}
%\usetheme{Singapore}
\usetheme{Warsaw}
%\usetheme{CambridgeUS}
%\usetheme{Montpellier}
%\usecolortheme{JuanLesPins}

%\usecolortheme{seagull}
%\usecolortheme{default}
%\usecolortheme[rgb={0,0,0}]{structure}
%\setbeamercolor{title}{bg=black!15}
%\usefonttheme{serif}
%\usefonttheme{structuresmallcapsserif}

%beaver beetle
%\setbeamercolor{grandfather}{bg=red}
%seahorse
%\setbeamercolor{section in toc}{bg=white}
%\usecolortheme[named=fly]{structure}
%\useinnertheme{rounded} 
%\usecolortheme{beetle}
%\usecolortheme{beaver}
\usecolortheme{seahorse}
\useoutertheme[]{split}	
%\useoutertheme[]{smoothbars}
%\useoutertheme[]{shadow}
%\useoutertheme[footline=empty]{split}
%\setbeamertemplate{navigation symbols}{}
\setbeamertemplate{navigation symbols}[vertical]
\setbeamertemplate{footline}[title]
% 	\useoutertheme[footline=empty]{} 
\setbeamertemplate{footline}[default]
\usepackage{graphics}
\title[DCA--IAG--USP]{\textcolor{black}{Morfologia das tempestades el\'{e}tricas na América do Sul}}
\subtitle{}
\date[]{\begin{footnotesize}IAG/USP -- 2015\end{footnotesize}}
\author[IAG]{Evandro M. Anselmo } %\par Adivisor: Carlos A. Morales
\institute{Institute of Astronomy, Geophysics and Atmospheric Science of University of São Paulo (IAG-USP), São Paulo-SP, Brazil. \par}
%\usepackage[pdftex]{graphicx}

\DeclareGraphicsExtensions{.pdf,.png,.jpg,.jpeg,.gif}
\setbeamersize{text margin left=.3cm,text margin right=.3cm}

\pgfdeclareimage[height=3.5cm]{ciclo1}{../img/ciclos/cicloanual19982011total}
\pgfdeclareimage[height=3.5cm]{ciclo2}{../img/ciclos/ciclodiurno19982011total}

\pgfdeclareimage[height=8.5cm]{ciclo3}{../img/ciclos/cicloanual10x1019982011localtime}
\pgfdeclareimage[height=8.5cm]{ciclo4}{../img/ciclos/ciclodiurno10x1019982011localtime}

\pgfdeclareimage[height=5.1cm]{viewTime}{../img/grids/vt_trmm}
\pgfdeclareimage[height=5.1cm]{passagens}{../img/grids/passagens_virs_1998-2011}

\pgfdeclareimage[height=5.1cm]{acumTemp}{../img/grids/densEspacial19982011acumuladoTempestadesPolyfill}
\pgfdeclareimage[height=5.1cm]{acumRaios}{../img/grids/densEspacial_19982011acumuladoFlashPolyfill}

\pgfdeclareimage[height=5.1cm]{txDJF}{../img/TaxaFlash/densEspacial_19982011djfTaxaFlashPolyfill}
\pgfdeclareimage[height=5.1cm]{txMAM}{../img/TaxaFlash/densEspacial_19982011mamTaxaFlashPolyfill}
\pgfdeclareimage[height=5.1cm]{txJJA}{../img/TaxaFlash/densEspacial_19982011jjaTaxaFlashPolyfill}
\pgfdeclareimage[height=5.1cm]{txSON}{../img/TaxaFlash/densEspacial_19982011sonTaxaFlashPolyfill} 
 
\pgfdeclareimage[height=5.1cm]{tempestadesDJF}{../img/DensidadeTempestades/densEspacial19982011djfTempestadesPolyfill} 
\pgfdeclareimage[height=5.1cm]{tempestadesMAM}
{../img/DensidadeTempestades/densEspacial19982011mamTempestadesPolyfill} 
\pgfdeclareimage[height=5.1cm]{tempestadesJJA}{../img/DensidadeTempestades/densEspacial19982011jjaTempestadesPolyfill}
\pgfdeclareimage[height=5.1cm]{tempestadesSON}{../img/DensidadeTempestades/densEspacial19982011sonTempestadesPolyfill}

\pgfdeclareimage[height=8.5cm]{RaiosPorTempestades}{../img/eficiencia/densEspacial_19982011totalEficienciaPolyfill}

\pgfdeclareimage[height=5.1cm]{txRaiosTotal}
{../img/TaxaFlash/densEspacial_19982011totalTaxaFlashPolyfill}
\pgfdeclareimage[height=5.1cm]{txTempestades}
{../img/DensidadeTempestades/densEspacial19982011TotalTempestadesPolyfill}

\pgfdeclareimage[height=8.5cm]{cfad10_semraio_topFTA_percentil}{../img/precipitacao3d/severo/percentil/90th/cfad10_semraio_topFTA_percentil}
\pgfdeclareimage[height=8.5cm]{cfad10_semraio_topFT_percentil}{../img/precipitacao3d/severo/percentil/90th/cfad10_semraio_topFT_percentil}
\pgfdeclareimage[height=8.5cm]{cfad10_comraio_topFTA_percentil}{../img/precipitacao3d/severo/percentil/90th/cfad10_comraio_topFTA_percentil}
\pgfdeclareimage[height=8.5cm]{cfad10_comraio_topFT_percentil}{../img/precipitacao3d/severo/percentil/90th/cfad10_comraio_topFT_percentil}

\pgfdeclareimage[height=8.5cm]{cCumFad_10deg_semraio_topFTApercentil}{../img/precipitacao3d/severo/percentil/90th/cCumFad_10deg_semraio_topFTApercentil}
\pgfdeclareimage[height=8.5cm]{cCumFad_10deg_semraio_topFTpercentil}{../img/precipitacao3d/severo/percentil/90th/cCumFad_10deg_semraio_topFTpercentil}

\pgfdeclareimage[height=8.5cm]{cCumFad_10deg_comraio_topFTApercentil}{../img/precipitacao3d/severo/percentil/90th/cCumFad_10deg_comraio_topFTApercentil}
\pgfdeclareimage[height=8.5cm]{cCumFad_10deg_comraio_topFTpercentil}{../img/precipitacao3d/severo/percentil/90th/cCumFad_10deg_comraio_topFTpercentil}

\pgfdeclareimage[height=8.5cm]{cftd_10deg_comraio_topFTApercentil}{../img/precipitacao3d/severo/percentil/90th/cftd_10deg_comraio_topFTApercentil}
\pgfdeclareimage[height=8.5cm]{ccftd_10deg_comraio_topFTApercentil}{../img/precipitacao3d/severo/percentil/90th/ccftd_10deg_comraio_topFTApercentil}
\pgfdeclareimage[height=8.5cm]{cftd_10deg_comraio_topFTpercentil}{../img/precipitacao3d/severo/percentil/90th/cftd_10deg_comraio_topFTpercentil}
\pgfdeclareimage[height=8.5cm]{ccftd_10deg_comraio_topFTpercentil}{../img/precipitacao3d/severo/percentil/90th/ccftd_10deg_comraio_topFTpercentil}

\pgfdeclareimage[height=4.5cm]{pdf_FTA_FT}{../img/FtaFt/pdf_FTA_FT}
\pgfdeclareimage[height=4.5cm]{cdf_FTA_FT}{../img/FtaFt/cdf_FTA_FT}

\pgfdeclareimage[height=3.9cm]{fracaoChuva_pdf_topFTA}{../img/FtaFt/fracaoChuva_pdf_topFTA}
\pgfdeclareimage[height=3.9cm]{fracaoChuva_pdf_topFT}{../img/FtaFt/fracaoChuva_pdf_topFT}

\pgfdeclareimage[height=4cm]{001_topFTA_10801_0006}{../img/topSevero/topFTA_chuva/001_topFTA_10801_0006}
\pgfdeclareimage[height=4cm]{002_topFTA_58133_0007}{../img/topSevero/topFTA_chuva/002_topFTA_58133_0007}
\pgfdeclareimage[height=4cm]{003_topFTA_28799_0006}{../img/topSevero/topFTA_chuva/003_topFTA_28799_0006}
\pgfdeclareimage[height=4cm]{004_topFTA_29770_0008}{../img/topSevero/topFTA_chuva/004_topFTA_29770_0008}
\pgfdeclareimage[height=4cm]{005_topFTA_52880_0006}{../img/topSevero/topFTA_chuva/005_topFTA_52880_0006}
\pgfdeclareimage[height=4cm]{001_topFT_02837_00010002}{../img/topSevero/topFT_chuva/001_topFT_02837_00010002}
\pgfdeclareimage[height=4cm]{002_topFT_14893_00010002}{../img/topSevero/topFT_chuva/002_topFT_14893_00010002}
\pgfdeclareimage[height=4cm]{004_topFT_03444_0001}{../img/topSevero/topFT_chuva/004_topFT_03444_0001}
\pgfdeclareimage[height=4cm]{005_topFT_56670_0003}{../img/topSevero/topFT_chuva/005_topFT_56670_0003}

\pgfdeclareimage[height=5.0cm]{fabry}{../img/ilustracoes/fabry}
\pgfdeclareimage[height=6.0cm]{takahashi}{../img/ilustracoes/takahashi}

\pgfdeclareimage[height=0.5cm]{barradecor_virs}{../img/topSevero/barradecor_virs}
\pgfdeclareimage[height=0.5cm]{barradecor_chuva}{../img/topSevero/barradecor_chuva}

\pgfdeclareimage[height=7cm]{deriv1}{../img/deriv1}
\pgfdeclareimage[height=7cm]{deriv2}{../img/deriv2}

\pgfdeclareimage[height=8.5cm]{distEspacialValor099thFta}{../img/DistEspacialPercentis/FTA/distEspacialValor099thFta}
\pgfdeclareimage[height=8.5cm]{distEspacialValor099thFt}{../img/DistEspacialPercentis/FT/distEspacialValor099thFt}

\pgfdeclareimage[height=7cm]{contornosAmazonas}{../img/precipitacao3d/deriv_ccftd/Contornos_contornos_cdf_2_1}
\pgfdeclareimage[height=7cm]{derivaAmazonas}{../img/precipitacao3d/deriv_ccftd/deriv_contornos_cdf_2_1}
\pgfdeclareimage[height=7cm]{contornosPrata}{../img/precipitacao3d/deriv_ccftd/Contornos_contornos_cdf_3_3} 
\pgfdeclareimage[height=7cm]{derivaPrata}{../img/precipitacao3d/deriv_ccftd/deriv_contornos_cdf_3_3}
\pgfdeclareimage[height=8cm]{topFTAeFT}{../img/DensidadeTempestades/topFTAeFT/densEspacial_orbita_19982011_90FTAFT_TempestadesPolyfill}

\pgfdeclareimage[height=8.5cm]{CFADgroupclasses}{../img/ilustracoes/resIniciais/CFADgroupclasses}
\pgfdeclareimage[height=7.5cm]{probflashesgroupclasses}{../img/ilustracoes/resIniciais/probflashesgroupclasses}
\pgfdeclareimage[height=8.5cm]{ccfadgroupclasses}{../img/ilustracoes/resIniciais/ccfadgroupclasses}
\pgfdeclareimage[height=8.5cm]{tabela}{../img/ilustracoes/resIniciais/tabela}


\pgfdeclareimage[height=5cm]{HRFCCOMFRV232014}{../img/ilustracoes/HRFCCOMFRV232014}

\pgfdeclareimage[height=7cm]{sensorPackageTraduzido}{../img/TRMM/sensorPackageTraduzido}

\pgfdeclareimage[height=7cm]{131220qlhi}{../img/ilustracoes/131220qlhi}

\pgfdeclareimage[height=2.5cm]{time0_ptbr}{../img/TRMM/time0_ptbr}
\pgfdeclareimage[height=2.5cm]{time100_ptbr}{../img/TRMM/time100_ptbr}
\pgfdeclareimage[height=2.5cm]{time350_ptbr}{../img/TRMM/time350_ptbr}

\tikzstyle{decision} = [diamond, draw, fill=blue!20, text width=9em, text badly centered, node distance=2cm, inner sep=0pt]
\tikzstyle{block} = [rectangle, draw, fill=blue!20, text width=9em, text centered, rounded corners, minimum height=2em]
\tikzstyle{line} = [draw, -latex']
\tikzstyle{cloud} = [draw, ellipse,fill=red!20, node distance=3cm, minimum height=1.5em, text width=9em, text centered]

\begin{document}

\begin{frame}
\titlepage
\end{frame}
 
%\begin{frame}
%\frametitle{Objectives}
%\justifying{
%\begin{itemize}
%\item Evaluate the thunderstorms severity individually based on the TRMM LIS flash  observation and
%\item Identify the most intense convection base on 3D precipitation structure observed by PR TRMM.
%\end{itemize}
%}
%\end{frame}
 
\begin{frame}
\frametitle{Introduction}
\begin{itemize}
\item South America is one of the 3 lightning chimes
\end{itemize}
\centering{
\pgfuseimage{HRFCCOMFRV232014}\\
(http://thunder.msfc.nasa.gov/data/)
}

\end{frame}

\begin{frame}
\frametitle{Introduction}
\begin{itemize}
\item Infrastructure  planning requires knowledge about thunderstorm position, frequency and lightning efficient production
\end{itemize}
\end{frame}

\begin{frame}
\frametitle{Objectives}
\begin{itemize}
\item Create a thunderstorm database based on TRMM measurements
\item Determine the annual cycle and diurnal cycle of thunderstorms in South America
\item Determine the geographical distribution of lightning flashes and thunderstorms
\item Evaluate the thunderstorms severity.
\end{itemize}
\end{frame}

\begin{frame}
\frametitle{Tropical Rainfall Measuring Mission -- TRMM}
\begin{itemize}
\item Launched in November 28th, 1997. Orbit with inclination of 35$^{\circ}$ and altitude of 350 km (August, 2011, the altitude was change for 402.5 km)
\item Sensors: Precipitation Radar
-- PR, TRMM Microwave Imager -- TMI, Visible and Infrared Scanner -- VIRS, Lightning Imaging Sensor -- LIS,  Clouds and the Earth’s Radiant Energy System
-- CERES.
\item Goal: Estimate rainfall over the  tropics and subsequently latent heat flux\footnote{KUMMEROW, C.; BARNES, W.; KOZU, T.; SHIUE, J.; SIMPSON, J. The tropical
rainfall measuring mission (TRMM) sensor package. J. Atmos. Oceanic Technol., v. 15, p. 809--817, 1998}
\end{itemize}
\end{frame} 

\begin{frame}
\frametitle{Tropical Rainfall Measuring Mission -- TRMM}
\centering{
\pgfuseimage{sensorPackageTraduzido}
}

\end{frame}


\begin{frame}
\frametitle{Tropical Rainfall Measuring Mission -- TRMM}
\centering{
TRMM overpasses\\
\pgfuseimage{131220qlhi}
}

\end{frame}

\begin{frame}
\frametitle{Tropical Rainfall Measuring Mission -- TRMM}
\centering{
LIS processing\\ 
\pgfuseimage{time0_ptbr} ~~~\pgfuseimage{time100_ptbr}\\
\pgfuseimage{time350_ptbr}

}

\end{frame}






%Resultados iniciais 
%\begin{frame}
%\frametitle{Initial results: LIS + PR (1998--2010)}
%\centering{
% \pgfuseimage{CFADgroupclasses} 
%}
%\end{frame} 
%
%\begin{frame}
%\frametitle{Initial results: LIS + PR (1998--2010)}
%\centering{
% \pgfuseimage{probflashesgroupclasses} 
%}
%\end{frame} 
 
\begin{frame}
\frametitle{Data and Methodology}
\justifying{

\textbf{TRMM:}
\begin{itemize}
\item TRMM orbital v7 files: LIS, VIRS (1B01), and PR (2A25), between 1998--2011 ($\simeq$30TB of data). 
\item During this period we had 79,932 TRMM orbits, but only 63,613 were over South America (SA) 40$^{\circ}$S--10$^{\circ}$N and 90$^{\circ}$--30$^{\circ}$W. 
\end{itemize}

\textbf{NCEP RII:}
\begin{itemize}
\item NCEP RII reanalysis from 1998-2011: geopotential height and temperature at 17 pressure levels. This data is used to convert TRMM PR altitudes levels in temperature. 
\end{itemize}

\textbf{Defining thunderstorms:}
\begin{itemize}
%\item Subset of the TRMM data for SA region.
%\item Region over South America (SA): 40$^{\circ}$S--10$^{\circ}$N and 90$^{\circ}$--30$^{\circ}$W.
\item Thunderstorms have been defined as clouds with brightness temperature below 258 K at the 1B01 10.8 $\mu$m channel and had at least one LIS lightning flash [Morales and Anagnostou, 2003]\footnote{Morales, C. A., and E. N. Anagnostou, Extending the capabilities of high-frequency rainfall estimation from geostationary-based satellite infrared via a network of long-range lightning observations, J. Hydrometeor, 4, 141--159, 2003.}.
%\item 3D precipitation structure were studied using Contour Frequency of PR reflectivity by Altitude and by Temperature Diagrams.
\end{itemize}
}

%and 154,189 thunderstorms found, only 96,281 thunderstorm had a least one %valid PR profile.
\end{frame}


\begin{frame}
\frametitle{TRMM's thunderstorms (flow chart)}
\begin{center}
\resizebox{8.0cm}{!}{%
\begin{tikzpicture}[node distance = 1.5cm, auto]
    % Place nodes
    \node [decision](search){Locate thunderstorms clusters (T$_b$ $\leq$ 258 K and at least one flash)};    
    \node [cloud,above left = of search](1B01c4){VIRS (1B01) channel 4 (10.8 $\mu$m)};
    \node [cloud,above right = of search](LISfl){LIS flash};  
    \node [cloud,below left = of search](PRtem){PR (2A25): Z$_e$, surface rain, rain type (2A23), geolocation}; 
    \node [cloud,below = of search](LIStem){LIS: flashes, events, groups, view time, geolocation};
    \node [cloud,below right = of search](VIRStem){VIRS (1B01): channel 4, geolocation};   
    \node [block,below = of LIStem](write){Write the TRMM thunderstorm HDF file};
    \path [line] (1B01c4) -- (search);
    \path [line] (LISfl) --  (search);
    \path [line] (search) -- node {read} (PRtem);
    \path [line] (search) -- node {read} (LIStem);
    \path [line] (search) -- node {read} (VIRStem);
    \path [line] (PRtem) -- (write);
    \path [line] (LIStem) -- (write);
    \path [line] (VIRStem) -- (write);
\end{tikzpicture}
}%
\end{center}
%    \node [cloud,below of=write](seach){PR (2A25): Z$_c$, rain type (2A23), geolocation}; 
%    \node [cloud,left of=PR] (LIStem) {LIS: flashes, events, groups, view time, geolocation};
%    \node [cloud, right of=PR] (VIRStem) {VIRS (1B01): channel 4, geolocation};  
%    \node [block,right of=project](write){Write the thunderstorm HDF file: PR (2A25) -- Z$_c$, rain type (2A23), geolocation; LIS -- flashes, events, groups, view time, geolocation; VIRS (1B01) -- channel 4, geolocation}  ;  
\end{frame}

\begin{frame}

\begin{itemize}
\frametitle{TRMM's thunderstorms}
\item 157,592 thunderstorms were found!
\item Only 94,733 had view time ($VT_m$) greater than 1 minute and with at least one PR pixel with valid rain. \\
\bigskip
\pgfuseimage{barradecor_virs}~~
\pgfuseimage{barradecor_chuva}\\
\pgfuseimage{004_topFT_03444_0001}

\end{itemize}

\end{frame}


\begin{frame}
\frametitle{The onset of the South American thunderstorms}

\begin{itemize}
\item Where the thunderstorms are more often in SA?
\item Where are the places with high lightning densities that corresponds with the highest density of thunderstorms?
\item How long does the thunderstorm season last in SA?
\end{itemize}

\end{frame}

\begin{frame}
\frametitle{Onset: TRMM overpasses and LIS view time}
\centering{
(LIS view -- $\mathbf{VT}_{lis}$)~~~~~~~~~~~~~~~~~~~~~~~~~~~~~~~~(VIRS overpasses -- $\mathbf{VT}_{virs}$)\\
\pgfuseimage{viewTime} \pgfuseimage{passagens}
}

\end{frame}

\begin{frame}
\frametitle{Onset: LIS lightning flashes and number of thunderstorms}
\centering{
 (lightning -- $\mathbf{FL}_{lis}$)~~~~~~~~~~~~~~~~~~~~~~~~~~~~~~~~(thunderstorms  -- $\mathbf{P}_{te}$)\\
  \pgfuseimage{acumRaios} \pgfuseimage{acumTemp}
}

\end{frame}


\begin{frame}
\frametitle{Onset: Defining densities ratios}
%{\Large Geographical densities of:}
\Large
\begin{itemize}
\item lightning [km$^{-2}$ year$^{-1}$]

\begin{equation}
\mathbf{DE}_{fl} = \frac{\mathbf{FL}_{lis}}{\mathbf{VT}_{lis} \mathbf{A}_g} 31557600     
\label{defl}
\end{equation}

\item thunderstorms per orbit [km$^{-2}$]
\begin{equation}
\mathbf{DE}_{te} = \frac{\mathbf{P}_{te}}{\mathbf{VT}_{virs} \mathbf{A}_g}    
\label{dete}
\end{equation}

\item lightning per thunderstorms [km$^{-2}$ year$^{-1}$]
\begin{equation}
\mathbf{DE}_{rt} = \frac{\mathbf{FL}_{lis}}{\mathbf{VT}_{lis} \mathbf{A}_g\mathbf{P}_{te}} 31557600 
\label{dert}
\end{equation}
\end{itemize}

\end{frame}


\begin{frame}
\centering{
\Large
Where the thunderstorms are more often in SA?
} 

\end{frame}

\begin{frame}
\frametitle{Onset: lightning $\times$ thunderstorms -- Annual}
\centering{

\pgfuseimage{txRaiosTotal} \pgfuseimage{txTempestades}
}

\end{frame}


\begin{frame}
\frametitle{Onset: lightning $\times$ thunderstorms -- SON}
\centering{
\pgfuseimage{txSON} \pgfuseimage{tempestadesSON}
}

\end{frame}


\begin{frame}
\frametitle{Onset: lightning $\times$ thunderstorms -- DJF}
\centering{
\pgfuseimage{txDJF} \pgfuseimage{tempestadesDJF}
}

\end{frame}

\begin{frame}
\frametitle{Onset: lightning $\times$ thunderstorms -- MAM}
\centering{
\pgfuseimage{txMAM} \pgfuseimage{tempestadesMAM}
}

\end{frame}

\begin{frame}
\frametitle{Onset: lightning $\times$ thunderstorms -- JJA}
\centering{
\pgfuseimage{txJJA} \pgfuseimage{tempestadesJJA}
}

\end{frame}




\begin{frame}
\Large
Where are the places with high lightning densities that corresponds with the highest density of thunderstorms?
\end{frame}

\begin{frame}
\frametitle{Onset: lightning per thunderstorms -- Annual}
\centering{
\pgfuseimage{RaiosPorTempestades}
}

\end{frame}



\begin{frame}
\centering{
\Large
How long does the thunderstorm season last in SA?
}

\end{frame}

\begin{frame}
\frametitle{Onset: Diurnal and Annual Cycle}
\centering{
\pgfuseimage{ciclo1}\pgfuseimage{ciclo2}\\
}

\end{frame} 


\begin{frame}
\frametitle{}
\centering{
Annual cycle\\
\pgfuseimage{ciclo3}
}

\end{frame}


\begin{frame}
\frametitle{}
\centering{
Diurnal cycle\\
\pgfuseimage{ciclo4}
}

\end{frame}

\begin{frame}
\centering{
\Large Thunderstorms Severity
} 

\end{frame}

\begin{frame}
\frametitle{Thunderstorms Severity: Defining severity index}

\begin{itemize}
\item Flash rate (FT) is defined as the ratio of number of lightning flashes ($N_{fl}$) by the mean view time ($VT_m$) in the thunderstorm area extracted.	
\end{itemize}

\begin{equation}
FT = \frac{N_{fl} }{VT_m} 60 ~[\mathrm{min^{-1}}]  
\label{eqFT}  
\end{equation}

\begin{itemize}
\item Flash rate  normalized by thunderstorm area (FTA) is defined as the ratio of number of lightning flashes ($N_{fl}$) by the mean view time ($VT_m$) and thunderstorm area ($A_t$). 
\end{itemize}

\begin{equation}
FTA = \frac{N_{fl}}{VT_m A_t } 60 ~[\mathrm{min^{-1}~km^{-2}}]
\label{eqFTA}
\end{equation}

%\bigskip
%\textbf{Potential severe thunderstorms were systems with FT and FTA in 90th percentile of sampling.} 

\end{frame}

\begin{frame}
\frametitle{Thunderstorms Severity: FTA and FT pdf distribution}
\centering{
\pgfuseimage{pdf_FTA_FT}
}

\end{frame}

\begin{frame}
\frametitle{Thunderstorms Severity: defining the extreme FTA and FT thunderstorms}
\centering{
\begin{itemize}
\item Extreme: 90th percentile
\end{itemize}
\pgfuseimage{cdf_FTA_FT}
\begin{itemize}
\item \textbf{FTA: 29.3 to 1,258.7 $\times$ 10$^{-4}$ fl min$^{-1}$ km$^{-2}$}
\item  \textbf{FT: 47.2 to 1,283.6 fl min$^{-1}$}
\end{itemize}
} 

\end{frame}




\begin{frame}
\frametitle{Thunderstorms Severity: Rain and CV/ST fraction}
\centering{
Thunderstorms with extreme FTA: rain areas \\
\pgfuseimage{fracaoChuva_pdf_topFTA} \\
Thunderstorms with extreme FT: rain areas \\
\pgfuseimage{fracaoChuva_pdf_topFT}
}

\end{frame}

\begin{frame}
\frametitle{Thunderstorms Severity: Top severe FTA index}
\pgfuseimage{barradecor_virs}~~
\pgfuseimage{barradecor_chuva}\\
\pgfuseimage{001_topFTA_10801_0006}\\
\pgfuseimage{002_topFTA_58133_0007}

\end{frame}


%\begin{frame}
%\frametitle{Thunderstorms Severity: Top severe FTA index}
%\pgfuseimage{barradecor_virs}~~
%\pgfuseimage{barradecor_chuva}\\
%\pgfuseimage{005_topFTA_52880_0006}\\
%\pgfuseimage{004_topFTA_29770_0008}
%
%\end{frame}


%\begin{frame}
%\frametitle{Thunderstorms Severity: Top severe FT index}

%\pgfuseimage{barradecor_virs}~~
%\pgfuseimage{barradecor_chuva}\\
%\pgfuseimage{001_topFT_02837_00010002}\\
%\pgfuseimage{002_topFT_14893_00010002}

%\end{frame}

\begin{frame}
\frametitle{Thunderstorms Severity: Top severe FT index}
\pgfuseimage{barradecor_virs}~~
\pgfuseimage{barradecor_chuva}\\
\pgfuseimage{004_topFT_03444_0001}\\
\pgfuseimage{005_topFT_56670_0003}

\end{frame}

  
\begin{frame}
\frametitle{Thunderstorms Severity: CFAD -- FTA (w/o lightning)}
\centering{
\pgfuseimage{cfad10_semraio_topFTA_percentil}}

\end{frame}

\begin{frame}
\frametitle{Thunderstorms Severity: CFAD -- FT (w/o lightning)}
\centering{
\pgfuseimage{cfad10_semraio_topFT_percentil}}

\end{frame}

\begin{frame}
\frametitle{Thunderstorms Severity: CFAD -- FTA (with lightning) }
\centering{
\pgfuseimage{cfad10_comraio_topFTA_percentil}}

\end{frame}

\begin{frame}
\frametitle{Thunderstorms Severity: CFAD - FT (with lightning)}
\centering{
\pgfuseimage{cfad10_comraio_topFT_percentil}}

\end{frame}


\begin{frame}
\frametitle{Thunderstorms Severity: CFADs}
\begin{itemize}
\item FTA pixels are taller than FT.
\item FTA pixels have higher Z$_c$ values than FT, especially above 5 km.
\end{itemize}
\end{frame}

%\begin{frame}
%\frametitle{Size and Brightness Temperature}
%\centering{
%\pgfuseimage{tbareas}}

%\end{frame}

\begin{frame}
\frametitle{Thunderstorms Severity: temperature dependency}

\begin{itemize}
\item It is know that the mixed phase region regulates the electrification charge process. 
\end{itemize}
\centering{
\pgfuseimage{takahashi}
} 

\end{frame}

\begin{frame}
\frametitle{Thunderstorms Severity: temperature dependency}
\begin{itemize}
\item As a consequence, we converted the PR altitude levels into temperature levels based on RII reanalysis.
\item Now we created the Contoured Frequency by Temperature Diagram (CFTD) in addition to the cumulative CFTD (CCFTD).
\end{itemize}

\end{frame}

%\begin{frame}
%\frametitle{3D precipitation analyses -- CFTD and CCFTD }

%\begin{itemize}
%\item For thunderstorms order by FTA index, CFTD show a narrow of probability contour  between 3-5\% (green color) near of surface, mainly in southern part of South America, indicating more rain associate with FTA index.\\
%\item Mainly in mixed region, the FTA group presented higher probability values associated with higher reflectivity values when compared with FT.
%\end{itemize}

%\end{frame}



\begin{frame}
\frametitle{Thunderstorms Severity: CFTD - FTA (with lightning) }
\centering{
\pgfuseimage{cftd_10deg_comraio_topFTApercentil} }

\end{frame}
\begin{frame}
\frametitle{Thunderstorms Severity: CFTD - FT (with lightning)}
\centering{
\pgfuseimage{cftd_10deg_comraio_topFTpercentil}}

\end{frame}


\begin{frame}
\frametitle{Thunderstorms Severity: CCFTD-FTA (with lightning)}
\centering{
\pgfuseimage{ccftd_10deg_comraio_topFTApercentil}}

\end{frame}
\begin{frame}
\frametitle{Thunderstorms Severity: CCFTD-FT (with lightning)}
\centering{
\pgfuseimage{ccftd_10deg_comraio_topFTpercentil}}

\end{frame}


\begin{frame}
\frametitle{Thunderstorms Severity: Rate of Z$_c$ change with temperature}
\centering{
\begin{itemize}
\item Above 0$^{\circ}$ C we might have ice and supercooled water droplets.
\item The rate of change of Z$_c$ with temperature might give an idea about of extend of mixed phase layer.
\item We computed this rate for 30\%, 50\%, 70\% and 95\%.
\item We compare tropical versus sub-tropical regions (Amazon $\times$ La Plata basin)
\end{itemize}
}

\end{frame}

\begin{frame}
\frametitle{Thunderstorms Severity: Z$_c$ change -- La Plata basin (30S-40S and 60W-70W)}
\centering{
\pgfuseimage{contornosPrata} \pgfuseimage{derivaPrata}
}

\end{frame}

\begin{frame}
\frametitle{Thunderstorms Severity: Z$_c$ change -- Amazon (00N-10N and 60W-70W)}
\centering{
\pgfuseimage{contornosAmazonas} \pgfuseimage{derivaAmazonas}
}

\end{frame}


\begin{frame}
\frametitle{Thunderstorms Severity: identifying the efficient and severe regions}
\begin{itemize}
\item For each grid box with 2.5$^{\circ}$ $\times$ 2.5$^{\circ}$ we obtained the FTA and FT pdf.
\item For each grid box, we extracted the 99th percentile value. 
\item Based on 99th percentile FTA and FT values, we build a spatial distribution map that show where are the most efficient (FTA) and severe (FT).
\end{itemize}

\end{frame}

\begin{frame}
\frametitle{Thunderstorms Severity: {\normalsize the most efficient thunderstorms}}
\centering{
\pgfuseimage{distEspacialValor099thFta} 
}

\end{frame}

\begin{frame}
\frametitle{Thunderstorms Severity: the most severe thunderstorms}
\centering{
\pgfuseimage{distEspacialValor099thFt} 
}

\end{frame}


\begin{frame}
\frametitle{Thunderstorms Severity: where are the most efficient and severe thunderstorms}
\centering{
\begin{itemize}
\item Find the thunderstorms that have FTA and FT above 90th percentile.
\item Compute the geographical densities of these thunderstorms (normalized by TRMM overpasses). 
\end{itemize}
}

\end{frame}

\begin{frame}
\frametitle{Thunderstorms Severity: the most severe and efficient thunderstorms}
\centering{
\pgfuseimage{topFTAeFT} }

\end{frame}


\begin{frame}
\frametitle{Conclusions}

\begin{itemize}
\item This study characterized the severe thunderstorms over South America by employing 14 years of TRMM measurements. 
\item The thunderstorms were characterized by integrating TRMM LIS, PR and VIRS measurements.
\item A total of 157,592 electrified clouds have been observed in South America (40S-10N and 90W-30W). 
\item  For the severe thunderstorms database, only 94,733 thunderstorms were used. Those thunderstorms had LIS view time greater and equal to 60 seconds and at least one valid PR raining pixel . 

%\item We defined a new configuration of Yuter and Houze [1995] CFADs that incorporates the temperature levels instead of altitude. Based on these diagrams, CFTD, it is possible to relate Z$_c$ changes with temperature and the width of the distributions in the presence of different hydrometeor types and growth mechanisms.  

\end{itemize}
\end{frame}


\begin{frame}
\frametitle{The onset of the thunderstorms}
\Large{Diurnal cycle:}
\begin{itemize}
\item 40\% of thunderstorm occurred between 13h--17h (LT)
\item In the ocean, we found one maximum at 20h (LT) and another between  4--5 (LT)
\item The nocturnal peak of thunderstorms were found in extreme North of Andes Mountains (Paramillo National Natural Park, Maracaibo lake). 630 systems where observed in 14 years, only between 0h e 00:59h
\item The thunderstorms were more frequent in center of South America (10$^{\circ}$--0$^{\circ}$ S and 70$^{\circ}$--50$^{\circ}$ W  and 20$^{\circ}$--10$^{\circ}$ S and 60$^{\circ}$--50$^{\circ}$ W), between 14h--16h (LT).
\end{itemize}

\end{frame}



\begin{frame}
\frametitle{Onset:}
\Large{Annual cycle:}
\begin{itemize}
\item Thunderstorms season in South America was between October and March with two peaks: January and October.
\item Northeast of South America (0$^{\circ}$--10$^{\circ}$ N and 70$^{\circ}$--50$^{\circ}$ W) the peak of thunderstorm was found in August.
\item In South of South America (40$^{\circ}$--20$^{\circ}$ S and 70$^{\circ}$--60$^{\circ}$ W, Plata basin) was found a shorter thunderstorm season with 2 months.  
\item The longer season was found in Colombia and West of Venezuela with 9 months.
\end{itemize}

\end{frame}

\begin{frame}
\frametitle{Onset: Numbers of thunderstorm associate with seasonal geographical densities of lightning and thunderstorms}
\Large
1$^{\circ}$ Spring = 57,861\\ 
2$^{\circ}$ Summer = 46,077\\
3$^{\circ}$ Autumn = 36,804\\
4$^{\circ}$ Winter = 16,850\\

\end{frame}

\begin{frame}
\frametitle{Onset: Total geographical densities}
Thunderstorms:
\begin{itemize}
\item North and Northwest of South America (Colombia, Panama, Northwest of Amazon): 3.5--4.7 $\times$ 10$^{-4}$ km$^{-2}$
\item Regions with high topography: Northeast of Titicaca Lake in Peru, mountain range in Santa Catarina, Emas National Park in Goias: $\simeq$2.5 $\times$ 10$^{-4}$ km$^{-2}$
\end{itemize}
Lightning:
\begin{itemize}
\item Mouth of the Catatumbo River: 148.1 fl year$^{-1}$ km$^{-2}$ 
\item Cochabamba, Bolivia: $\simeq$60 fl year$^{-1}$ km$^{-2}$
\item The Black Needles peak, Matiqueira mountain range, Neblina peak, La Plata basin: 30--60 fl year$^{-1}$ km$^{-2}$.  
\end{itemize}
Lightning per thunderstorm:
\begin{itemize}
\item 11.73 $\times$ 10$^{-2}$ ano$^{-1}$ km$^{-2}$ (Maracaibo lake). In this grid point (772 km$^{-2}$) each thunderstorm had in average 91 fl year$^{-1}$.  
\end{itemize}

\end{frame}

%Thunderstorm selected by FTA have stronger accretion process than FT
%\item The northern thunderstorm have a enhanced aggregation process and weaker accretion process when compared to the southern thunderstorms.
%\item Those results could explain why we do observed the highest flash rates over the southern part of South America, i.e., because they have a efficient accretion mechanism when compared to the region we found the most thunderstorm that does not produce so much lightning because it has a better aggregation process.

\begin{frame}
\frametitle{Severity of Thunderstorms in South America}
\begin{itemize}
\item \textbf{Two groups of severe systems are found}:
\begin{itemize}
\item FTA 29.3 to 1,258.7 $\times$ 10$^{-4}$ fl min$^{-1}$ km$^{-2}$
\item FT 47.2 to 1,283.6 fl min$^{-1}$ (above category 3 of Cecil et al. [2005] and Zipser et al. [2006])
\end{itemize}
\item \textbf{Cloud top Tb}: FTA thunderstorms are 10 K colder than FT thunderstorms
\item \textbf{Convective $\times$ Stratiform rain fraction}: FTA thunderstorms have 72\% of convective rain fraction and 32\% stratiform and FT systems $\simeq$22\% convective and 65\% stratiform.
\item \textbf{Lightning $\times$ non Lightning pixels (CFAD)}: profiles with lightning pixels have higher Z$_c$ values, and are taller. In the layer between 5 and 7 km height, the Z$_c$ difference can reach 5 to 10 dBZ. At lower levels (2--3km) Z$_c$ is above 50 dBZ compared to 40 dBZ for non-lightning pixels;
\item \textbf{Lightning pixels (CFAD)}: FTA thunderstorms show 1-3 dBZ higher Z$_c$ values than FT systems, specially above 5 km height;
\end{itemize}
\end{frame}

\begin{frame}
\begin{itemize}
\item \textbf{Top 1\% FTA thunderstorms}: More than 148.93 $\times$ 10$^{-4}$ fl min$^{-1}$ km$^{-2}$. Associated with orography in the Pantanal Matogrossense, Central Plateau of Brazil, and the Xingu and Araguaia basin and Tocantins rivers in the Amazon basin, the meridional Plateau of Brazil in Paraná basin and the Andes foothill -- eastward of the Sierra of Cordoba in Argentina.
\item \textbf{Top 1\% severe FT thunderstorms}: More than  272.88 fl min$^{-1}$ and are located in the elevated topography areas like Andes foothill and Sierra of Cordoba in Argentina and also at less pronounced orography at Tocantins central of Brazil, Sierras Gauchas and Catarinenses at South of Brazil to more flat areas that have a water and vegetation contrast like south of Amazon, Rio Branco in Acre, in the boundaries of the states of Acre, Amazon with Peru, central Bolivia, Pantanal Matogrossense, Paraná state, Paraguay and a Plata basin.
\end{itemize}

\end{frame}

\begin{frame}
\frametitle{Severity: Temperature dependence -- CFTD}
\begin{itemize}
\item \textbf{Z$_c$ distribution}: FTA thunderstorms show broader Z$_c$ distribution than FT systems for the same temperature level, specially above 0$^{\circ}$ C and they have 1--3 dBZ higher Z$_c$ values;
\item \textbf{Latitudinal variation}: FT thunderstorm show an enhancement of the collision-coalescence process towards the sub-tropics.
\item  \textbf{Z$_c$ change}: the maximum Z$_c$ decrease with temperature for FTA(FT) thunderstorms is at -2$^{\circ}$ C (0$^{\circ}$ C) and -12$^{\circ}$ C(-8$^{\circ}$ C) for 50\% and 95\% levels profiles respectively in the Amazon and -6$^{\circ}$ C(-2$^{\circ}$ C) and -14$^{\circ}$ C (-6$^{\circ}$ C) for 50\% and 95\% respectively for the La Plata Basin.
\end{itemize}

\end{frame}


\begin{frame}
\frametitle{ACKNOWLEDGEMENTS}
\justifying{
\begin{itemize}
\item This work is part of PhD project that is supported by CNPq grant 140842/2011-0.
\item This work is partially supported by CAPES PROEX program and COELCE.
\item I would like to thank  MSFC\/NASA for providing LIS data set and GSFC\/NASA for providing TRMM dataset.
\item Finally I would like to thank Dra. Rachel Albrecht for the LIS lightning and view time reprocessed dataset.
\end{itemize}
}

\end{frame}

\end{document}