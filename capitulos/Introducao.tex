\chapter{INTRODUÇÃO}


As tempestades elétricas 	são observadas em regiões aonde ocorre levantamento de ar por convecção, pela topografia elevada, propagação de frentes quando combinados com mecanismos de suporte de umidade como aumento da temperatura superficial de oceanos, brisa marítima, brisa de rio e evapotranspiração de florestas.

Observando a climatologia da ocorrência de descargas elétricas atmosféricas sobre a Terra pelos diversos sistemas de detecção de descargas em operação, STARNET, WWLN, WSI, LIS-TRMM, … é evidente que as tempestades elétricas concentram-se sobre os continentes, indicando que a convecção e a topografia são fatores dominantes que contornam a problemática da microfísica da eletrificação atmosférica.

A região tropical da América do Sul, Africa e Continente Marítimo são também conhecidas como chaminés globais de descargas elétricas atmosféricas. Whipple, F.J.W., (1929) já observava que a América do Sul é a chaminé dominante para a manutenção do Circuito Elétrico Atmosférico Global. 

Estando o Brasil em uma das regiões de maior incidência de raios do planeta, os estudos das tempestades elétricas tornar-se necessário para garantir segurança no tráfego aéreo, fluvial, terrestre, nas linhas de transmissão de dados e de energia elétrica e melhor lidar com problemas como enchentes rápidas, chuvas de granizo, tornados e visar uma melhor forma de gerir recursos naturais.

Williams E. R. e Sátori G., (2004) buscaram entender a maior resposta da Curva de Carnegie associada a atividade de tempestades na América do Sul fazendo um estudo comparativo entre as regiões da bacia Amazônica e bacia do Congo. Sobre a maior bacia hidrográfica do Continente Africano, as taxa de raios por Km2 por ano são maiores enquanto que os sistemas precipitantes sobre a bacia Amazônica, observa-se menor densidade de raios porém maior volume de chuva, indicando que as formações estratiformes no continente sul-americano também funcionam como baterias do Circuito Elétrico Global, em que carga negativa é transferida para a Terra por meio das gotas de chuva carregadas (SOULA et al., 2003).

Com o experimento de campo LBA (Large-Scale Biosphere-Atmosphere Experiment in Amazonia) realizado na região de Rondônia entre janeiro e fevereiro de 1999, foi possível identificar alguns fatores importantes que regulam a precipitação na região Amazônica. Além disso o LBA foi importante para validação de dados do satélite TRMM que são amplamente utilizados nesta pesquisa \cite{TRMM LBA papers}.

Silva Dias M. A. F. et al, (2002), fazem uma síntese dos principais resultados e objetivos do LBA, entre estes destaco os estudos de Anagnostou e Morales, (2002), \cite{mais autores tbm} que mostram dois regimes de vento em 700 mb, de Leste e de Oeste, em que observou-se maior precipitação convectiva e atividade elétrica durante o regime de ventos de Leste. Petersen W. A. et al, (2002), investigaram como que esses dois regimes de vento (Leste-Oeste) observados durante o LBA em Rondônia, influenciam no número de descargas elétricas observadas pelo LIS (Ligthning Image Sensor), não apenas para região Amazônica mas para toda a América do Sul durante 4 verões entre 1997 e 2000.

A variação intra-sazonal da atividade elétrica durante o período chuvoso mostrou-se evidente. Petersen W. A. et al, (2002) identificaram regiões de extremos opostos de atividade elétrica que devem estar associados ao mecanismos de manutenção da monção na América do Sul, principalmente com a dinâmica que envolve Zona de Convergência do Atlântico Sul (ZCAS) \cite{ breack fase monson }.   

Esta pesquisa investiga a morfologia das tempestades elétricas sobre a América do Sul, os sistemas que são as baterias do Circuito Elétrico Atmosférico Global. 

Cecil et al, (2005) investigou o subconjunto de dados do TRMM composto por sistemas precipitantes denominado como Precipitation Features (PF), desenvolvido por Nesbitt et al, (2000). As PF classificadas como as mais severas da AS concentraram-se no sul da região tropical, associadas aos mecanismos dinâmicas de formação de Sistemas Convectivos de Meso-escala \cite{quem descreve SCM na AS?}. 

Um estudo sobre a morfologia de sistemas individualmente, para a região Sul- amazônia para os regimes de Leste-Oeste, é mostrado também em Petersen W. A. et al, (2002) com base nas PF. Observou-se que apesar da atividade elétrica indicar extremos opostos durante a estação úmida, o volume de chuva produzido pelos sistemas foram iguais, porém os fluxos de calor latente na coluna atmosférica não. No regime de Leste a precipitação associa-se com vigorosas regiões de mistura, enquanto no de Oeste os sistemas possuem maior área de chuva e são mais estratificados.

Além das PF, identifica-se diversos estudos que definiram clusters de nuvens fazendo o agrupamento regiões com temperatura de brilho mais baixas, observadas por sensores de radiação infravermelha em satélites.  Mapes and House, (1993), utilizaram esta metodologia e também fazem uma síntese de trabalhos que buscaram selecionar clusters de nuvens a partir de limiares de temperatura em infravermelho, como por exemplo Maddox R. A., (1980) que observou a ocorrência de duas regiões, uma com temperatura de brilho <= -32C (241K) e área >= 100000 km2 e outra região menor, no interior da região maior, com temperatura de brilho <= -52C (221K) e área >= 50000 km2 em Sistemas Convectivos de Meso-escala (SCM) nos Estados Unidos.

Morales C. A. e Anagnostou E. M., (2002) desenvolveram um algoritmo hidro-estimador estudando regiões de temperatura de brilho em infravermelho e a precipitação observada pelo radar a bordo do satélite TRMM. Dados da Sferics Timing and Ranging Network (STARNET) foram utilizados e clusters com raios e sem raios foram identificados. Foi observado que as descargas localizadas pela STARNET, em 90% dos casos, estiveram associados a regiões com temperatura de brilho menores do que 258K.

Porém a radiação infravermelha observada por satélites, corresponde apenas a irradiação do topo das nuvens. Nuvens finas, com formação acima da isoterma de 0°C, como por exemplo as nuvem cirrus, podem cobrir grandes extensões e não estar associadas a precipitação nem descargas elétricas.

Houze R. A. Jr, (1993) define SCM, por exemplo linhas de instabilidades observando extensões com chuva contínua observada por radar. Em Mohr and Zipser (1996) SCMs sobre os trópicos foram observados a partir do espalhamento radiativo em micro-ondas (85-GHz PCT), em que regiões contínuas >=2000 Km2 com PCT <= 250 K foram principais critérios na identificação dos sistemas.

Combinando dados do PR e TMI abordo do TRMM, Nesbitt et. al. (2000) desenvolveu uma metodologia para selecionar sistemas precipitantes, os quais foram definidos com Precipitation Features. Desta forma …. 



Em Nesbitt et al (2000), buscou-se 
Afunilando ….

concentram-se nas características morfológica das tempestades elétricas observadas pelo TRMM, identificadas individualmente.

Apesar da atividade elétrica indicar extremos opostos durante a estação úmida 
a identificação de dois regimes de escoamento durante a estação úmida  
chove mais com maiores regiões estratiformes   
Williams E. R. e Sátori G., (2004) mostraram que 


A literatura \cite{} descrevem que a há três chaminés globais de descargas elétricas atmosféricas, o Continente Marítmos/Indonésia, a América do Sul e a África, estando o Brasil no centro de uma das chaminés globais de raios. As tempestades elétricas de vapor d'água ocorrem em nuvens com perfis de velocidade convectivo e com fase fria, geralmente formação cumulunimbus, o que representa problemas para o tráfego aéreo, marítimo, fluvial entre outros como as enchentes rápidas, chuvas de granizo e tornados, sendo então as tempestades  alvo de diversos estudos.


 realizados pela observação desses sistemas sobre a América do Sul.

	As tempestades elétricas 	são observadas com maior frequência nas regiões em que a atmosfera é convectiva e umida. Na região tropical e extratropical, as quais recebem maior potencia de energia solar (quantos W/m2?) e nos locais em que a superfície terrestre aquece e transfere energia na forma de calor sensível para o ar durante o ciclo diurno. Portanto observa-se que as tempestades que produzem as maiores taxas de relâmpagos são iniciadas sobre os continentes. 
	Os mecanismos dinâmicos 
nos trópicos e extra-trópicos dos continentes,   
nos locais em que a superfície terrestre aquece e transfere energia na forma de calor sensível para o ar durante o ciclo diurno. Portanto observa-se que as tempestades que produzem as maiores taxas de relâmpagos são iniciadas sobre os continentes. 
	Os mecanismos dinâmicos 


A grande extensão territorial do Brasil na America do Sul  
m base nos experimentos do LBA, investigar a morfologia dos sistemas na pré monsão, na região amazônica. 

As descargas na região amazônica parecem estar mais associadas com os processos de inibição do que precipitação... Em uma região tropical dominada por processos quentes, as descargas podem indicar inibição de colisão coalescência, desenvolvimento de fase fria e menos precipitação.

mas no período úmido de leste raios relacionam-se mais com a precipitação.

A Rachel já fez uma boa discussão sobre a microfísica dos sistemas da amazônia, períodos seco umido e de transição. As tempesdades foram organizadas em clusters, estudou-se o cliclo de vida, a correncia de raios em areas dematadas e com floresta/outras, e foi explicado a microfisica dos sistemas basicamente com: taxa de raios positivos e negatiovos, ecotops, VIL , CAPE, CINE. Falta explotar a os CFADS para essa região. Como varia a probabilidade de ocorrência por altitude dos perfis de refletividade nos períodos seco de transição e umido.



As tempestades mais eficientes estão mostrando tamanhos diversos. Tanto cluster grandes quanto pequenos podem ser eficientes. Não é uma relação que depende apenas da área, ou da fração convectiva ou estratiforme. Os raios relacionam-se com o ciclo de vida, que no caso é aleatório. Depende do ciclo de vida

Relação exponencial entre max fl no pixel versus fl-rate/km2. Maior a concentração de raios em um único pixel, menos eficiente e mais chuva. Talvez uma reintensificação do sistema maduro. Mas existem duas categorias de sistemas :
1 – os mais eficiente com área ~10^3 e chuva ~10^5
2 – as com maior fl/rate no pixel, menos eficientes e com chuva ~10^9