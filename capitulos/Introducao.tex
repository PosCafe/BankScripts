\chapter{INTRODUÇÃO}

A região tropical das Américas, África e Continente Marítimo são consideradas como as chaminés globais de descargas elétricas atmosféricas -- raios. Em \citeonline{whipple1929} já se observava que no horário de maior atividade de tempestades elétricas sobre a América do Sul (AS), é quando há a maior intensidade do campo elétrico de bom tempo. Portanto a América do Sul é a chaminé dominante no processo de manutenção do circuito elétrico atmosférico global \cite{williams2004}.


%As tempestades elétricas são observadas em regiões aonde ocorre levantamento de ar por convecção, pela topografia elevada, propagação de frentes quando combinados com mecanismos de suporte de umidade como aumento da temperatura superficial de oceanos, brisa marítima, brisa de rio e evapotranspiração de florestas.
%Observando a climatologia da ocorrência de descargas elétricas atmosféricas sobre a Terra pelos diversos sistemas de detecção de descargas em operação, STARNET, WWLN, WSI, LIS-TRMM, … é evidente que as tempestades elétricas concentram-se sobre os continentes, indicando que a convecção e a topografia são fatores dominantes que contornam a problemática da microfísica da eletrificação atmosférica.
%A região tropical da América do Sul, Africa e Continente Marítimo são também conhecidas como chaminés globais de descargas elétricas atmosféricas. Whipple, F.J.W., (1929) já observava que a América do Sul é a chaminé dominante para a manutenção do Circuito Elétrico Atmosférico Global. 


Trabalhos que mostram a densidade global de raios para a região tropical conforme \citeonline{albrecht2009tropical,cecil2014gridded}, observa-se que as maiores densidades de raios do globo ocorrem sobre a América do Sul e África.


Estando os países da América do Sul situados em uma das regiões de maior atividade elétrica atmosférica do globo, saber quando e aonde as tempestades elétricas ocorrem, bem como, quais os locais em que os sistemas são mais eficientes na produção de raios, torna-se fundamental para o planejamento e segurança do transporte aéreo, fluvial e terrestre, das linhas de transmissão de dados e de energia elétrica, etc, setores que quando paralisados devidos aos danos da queda de raios causam prejuízos em todos os setores econômicos. Uma falha no sistema de distribuição de energia elétrica pode cessar a energia elétrica de um bairro, cidade, etc. Em \citeonline{noticiainpe2007}, estima-se prejuízos na ordem de 1 bilhão de dólares anuais em função da densidade de raios observada apenas sobre o Brasil.
 
%e melhor lidar com problemas como enchentes rápidas, chuvas de granizo, tornados e visar uma melhor forma de gerir recursos naturais.
%Williams E. R. e Sátori G., (2004) buscaram entender a maior resposta da Curva de Carnegie associada a atividade de tempestades na América do Sul fazendo um estudo comparativo entre as regiões da bacia Amazônica e bacia do Congo. Sobre a maior bacia hidrográfica do Continente Africano, as taxa de raios por km$^2$ por ano são maiores enquanto que os sistemas precipitantes sobre a bacia Amazônica, observa-se menor densidade de raios porém maior volume de chuva, indicando que as formações estratiformes no continente sul-americano também funcionam como baterias do Circuito Elétrico Global, em que carga negativa é transferida para a Terra por meio das gotas de chuva carregadas (SOULA et al., 2003).

A eficiência de produção de raios de uma tempestade elétrica não depende apenas dos processos de nucleação e colisão coalescência chuva, depende de como ocorrerá o crescimento de cristais de gelo na nuvem, exigindo condições atmosféricas em que a água possa coexistir na fase líquida, gasosa e sólida (região mista) \cite{Takahashi1978,williams1991mixed,korolev2007}. Essas condições são possíveis em regiões atmosféricas com temperaturas entre -5°C e -40°C. Desta forma, o processo de eletrificação dos hidrometeoros até a formação de um raio, depende da capacidade do ar quente e úmido da superfície romper a estabilidade atmosférica e atingir altitude entre 4--15 km, regiões acima da isoterma de 0°C. Portanto, as tempestades elétricas podem indicar condições dinâmicas e termodinâmicas associadas a convecção profunda na atmosfera \cite{doswell2001,zipser2006}.

Além dos raios que atingem o solo -- raios nuvem-solo -- causando danos a sociedade, \citeonline{macgorman1989,carey1998,williams1999}, mostram que condições de tempo severo como: frentes de rajadas com velocidade superior a 92.6 km h$^{-1}$, queda de granizo com diâmetro maior do que 1.9 cm ou tornados, são geralmente precedidas de um salto na taxa de raios das tempestades elétricas governado por raios que não atingem o solo -- raios intra-nuvens -- indicando crescimento de hidrometeoros na região mista.      

Neste sentido, técnicas de seleção de sistemas meteorológicos a partir de dados de sensoriamento remoto, combinadas com o monitoramento da taxa de raios totais dos sistemas são de grande importância na identificação e previsão de curto prazo de tempo severo. 

Diversos estudos definiram sistemas meteorológicos fazendo o agrupamento de regiões na superfície a partir de limiares de temperatura de brilho observadas em satélite. \citeonline{mapes1993}, utilizaram esta metodologia e também fazem uma síntese de trabalhos que buscaram selecionar clusters de nuvens a partir de limiares de temperatura de brilho em infravermelho, como por exemplo \citeonline{Maddox1980} que observou a ocorrência de duas regiões: uma com temperatura de brilho $\leqslant$ -32°C (241K) e área $\geqslant$ 100,000 km$^2$;  outra região menor, no interior da região maior, com temperatura de brilho $\leqslant$ -52°C (221K) e área $\geqslant$ 50,000 km$^2$, associadas com Sistemas Convectivos de Meso-escala (SCM) nos Estados Unidos.

\citeonline{morales2003} desenvolveram um algoritmo hidro-estimador estudando regiões de temperatura de brilho em infravermelho e a precipitação observada pelo radar a bordo do satélite \textit{Tropical Rainfall Measuring Mission} -- TRMM. Dados da \textit{Sferics Timing and Ranging Network} (STARNET) foram utilizados e clusters com raios e sem raios foram identificados. Foi observado que as descargas localizadas pela STARNET, em 90\% dos casos, estiveram associados a regiões com temperatura de brilho menores do que 258 K.


\sigla{name={TRMM},description={\textit{Tropical Rainfall Measuring Mission}}}
\sigla{name={STARNET},description={\textit{Sferics Timing and Ranging Network}}}

%Porém a radiação infravermelha observada por satélites, corresponde apenas a irradiação do topo das nuvens. Nuvens finas, com formação acima da isoterma de 0°C, como por exemplo as nuvem cirrus, podem cobrir grandes extensões e não estar associadas a precipitação nem descargas elétricas.


Em \citeonline{houze1993} Linhas de Instabilidades (LI) foram definidas observando extensões com chuva contínua observada por radar. Em \citeonline{MohrZipser1996} Sistemas Convectivos de Meso-escala (SCM) sobre os trópicos foram observados a partir do espalhamento radiativo em micro-ondas (85-GHz PCT), em que regiões contínuas $\geqslant$ 2000 km$^2$ com PCT $\leqslant$ 250 K foram os principais critérios para a identificação dos sistemas.

\sigla{name={SCM},description={\textit{Sistema Convectivo de Meso-escala}}}
\sigla{name={LI},description={\textit{Linha de Instabilidade}}}

Combinando dados do PR e TMI abordo do TRMM, \citeonline{Nesbitt2000} desenvolveu uma metodologia para selecionar sistemas precipitantes denominados como \textit{Precipitation Features} (PF), em que o principal critério de seleção foi identificar área contínua de chuva na superfície, seja estimada pelas observações de radar ou micro-ondas quando os sistemas estiveram fora da varredura do PR.

\sigla{name={PF},description={\textit{Precipitation Features}}}


\citeonline{cecil2005} mostram que apenas 2.4\% das PFs observadas pelo TRMM em todo globo entre 12/1997--10/2000 possuíram atividade elétrica. Na AS as PFs classificadas com as maiores taxas de raios, concentraram-se na região da Bacia do Prata associadas a Sistemas Convectivos de Meso-escala \cite{Velasco1987,Durkee2009}.   


Com base também nos dados das PFs,  \citeonline{zipser2006} identificou os locais das tempestades mais severas em um estudo global. Apenas 0.1\% da amostragem das PFs em um período de 7 anos foram observadas taxa de raios acima de 32.9 por minuto.  Entre as regiões do globo com as maiores concentrações de PFs que indicaram valores extremos (0.001\%), seja de taxa de raios, seja de mínimas temperaturas de brilho (85 e 37 GHz) ou de máxima altitude com 40 dBZ de fator de refletividade do radar, encontram-se na região Sul da AS que engloba a Bacia do Prata e o extremo Norte da Cordilheira dos Andes que abrange a Colômbia e região do Lago Maracaibo na Venezuela. 
%Conforme aumenta a intensidade das tempestades elétricas, menor será sua probabilidade de ocorrência. Apenas 0.1\% das PFs observou-se taxa de raios acima de 32.9 por minuto. 


Nesta tese, utiliza-se os dados orbitais do sensor de raios (LIS), radiômetro em infravermelho (VIRS) e o radar de precipitação (PR) a bordo do da satélite TRMM entre os anos de 1998 e 2011, para cria um banco de dados de sistemas que possuem atividade elétrica, ou seja, tempestades elétricas, apenas observadas sobre a América do Sul. 

Com este subconjunto de dados do TRMM, é estudada a sazonalidade e o ciclo diurno das tempestades elétricas, bem como a densidade geográfica de raios e de tempestades elétricas, buscando evidenciar regiões ou estações do ano em que as tempestades elétricas possuem processo de eletrificação mais eficientes e então produzem maior número de raios.

A intensidade das tempestades elétricas é estuda com base em seus aspectos morfológicos como área e estrutura tridimensional da precipitação observada pelo PR, além da taxa de raios.





------------------------------------------------------\\
\textit{Acho que essa parte de baixo é Metodologia....}






%A grande extensão territorial do Brasil na America do Sul  
%m base nos experimentos do LBA, investigar a morfologia dos sistemas na pré monsão, na região amazônica. 

%As descargas na região amazônica parecem estar mais associadas com os processos de inibição do que precipitação... Em uma região tropical dominada por processos quentes, as descargas podem indicar inibição de colisão coalescência, desenvolvimento de fase fria e menos precipitação.

%mas no período úmido de leste raios relacionam-se mais com a precipitação.

%...A Rachel já fez uma boa discussão sobre a microfísica dos sistemas da amazônia, períodos seco úmido e de transição. As tempestades foram organizadas em clusters, estudou-se o ciclo de vida, a ocorrência de raios em áreas desmatadas e com floresta/outras, e foi explicado a microfísica dos sistemas basicamente com: taxa de raios positivos e negativos, eco tops, VIL, CAPE, CINE. Falta explotar a os CFADS para essa região. Como varia a probabilidade de ocorrência por altitude dos perfis de refletividade nos períodos seco de transição e úmido.



%Em desenvolvimento ...


%As tempestades mais eficientes estão mostrando tamanhos diversos. Tanto cluster grandes quanto pequenos podem ser eficientes. Não é uma relação que depende apenas da área, ou da fração convectiva ou estratiforme. Os raios relacionam-se com o ciclo de vida, que no caso é aleatório. Depende do ciclo de vida

%Relação exponencial entre max fl no pixel versus fl-rate/km2. Maior a concentração de raios em um único pixel, menos eficiente e mais chuva. Talvez uma reintensificação do sistema maduro. Mas existem duas categorias de sistemas :
%1 – os mais eficiente com área ~10^3 e chuva ~10^5
%2 – as com maior fl/rate no pixel, menos eficientes e com chuva ~10^9



%Com o experimento de campo LBA (Large-Scale Biosphere-Atmosphere Experiment in Amazonia) realizado na região de Rondônia entre janeiro e fevereiro de 1999, foi possível identificar alguns fatores importantes que regulam a precipitação na região Amazônica. Além disso o LBA foi importante para validação de dados do satélite TRMM que são amplamente utilizados nesta pesquisa \cite{silva2002lba,williams2002,albrecht2011}.

%Silva Dias M. A. F. et al, (2002), fazem uma síntese dos principais resultados e objetivos do LBA, entre estes destaco os estudos de Anagnostou e Morales, (2002), \citeonline{Carvalho2002} que mostram dois regimes de vento em 700 mb, de Leste e de Oeste, em que observou-se maior precipitação convectiva e atividade elétrica durante o regime de ventos de Leste. Petersen W. A. et al, (2002), investigaram como que esses dois regimes de vento (Leste-Oeste) observados durante o LBA em Rondônia, influenciam no número de descargas elétricas observadas pelo LIS (Ligthning Image Sensor), não apenas para região Amazônica mas para toda a América do Sul durante 4 verões entre 1997 e 2000.

%A variação intra-sazonal da atividade elétrica durante o período chuvoso mostrou-se evidente. \citeonline{petersen2002trmm},  identificaram regiões de extremos opostos de atividade elétrica que devem estar associados ao mecanismos de manutenção da monção na América do Sul, principalmente com a dinâmica que envolve Zona de Convergência do Atlântico Sul (ZCAS) \cite{CarvalhoJones2002,Carvalho2002}.   


%\subsection{Estrutura tridimensional da precipitação na óptica dos processos microfísicos}

%\label{chuvaEtemperatura}


Em \citeonline{Fabry1995}, é mostrado que processos como a agregação, acreção e colisão coalescência, podem ser estudados em função da espessura da camada de derretimento e flutuações nos valores do fator de refletividade no perfil atmosférico. 

Pois, sendo o fator de refletividade do radar proporcional ao diâmetro das gotas no volume iluminado elevado a 6 potência, os processos de crescimento de flocos de neves, granizo e gotas, são marcados por aumentos abruptos no fator de refletividade do radar. 

E considerando um regime de precipitação estratiforme, o qual é muito mais governado por processos de agregação do que acreção, será observado um aumento acentuado no fator de refletividade do radar em torno da isoterma de 0 $^{\circ}$C associado ao derretimento de flocos de neve. \simbolo{name={$^{\circ}$C},description={Grau Celcius}} Como o índice de refração de micro-ondas no gelo é de $\sim$0,1 e na água líquida de $\sim$0,9, a transição de fase sólida para líquida representa um aumento de 7 dBZ na potência do sinal do radar.


% durante o caminho que a precipitação percorre até a superfície ou temperaturas acima de 0°C.
%Na figura \ref{fabry}, \citeonline{Fabry1995}
%e o trabalho de 
%\begin{figure}[hbp]
%  \centering{
%  \subfloat[\cite{Fabry1995}]{{\includegraphics[scale=0.25]{img/ilustracoes/fabry}} \label{fabry}}
%  \subfloat[\cite{Takahashi2002}]{{\includegraphics[scale=0.35]{img/ilustracoes/takahashi}} \label{taka}}
%  }
%\caption{Fabry Taka}
%\label{fabyTaka} 
%\end{figure} 

Em um ambiente de precipitação convectiva a transição de fase é perturbada por correntes ascendentes e os processos de agregação, acreção e colisão coalescência, os quais são os maiores responsáveis pelo aumento do diâmetro dos hidrometeoros de nuvem, tornam-se mais eficientes. 

A mudança do índice de refração da água não ocorre em torno de 0 $^{\circ}$C, pois no ambiente convectivo teremos água super-resfriada em temperaturas de -15 $^{\circ}$C, o que intensifica o processo de acreção podendo gerar gelo sólido que cai até a superfície.

Portanto, quanto maior a espessura da camada de derretimento, podemos pressupor que, o ambiente terá maior intensidade convectiva, pois terá processos de crescimento de granizo mais ativos.

Consequentemente, a taxa de raios associa-se com a intensidade convectiva devido a acreção\footnote{A acreção é o processo de \textit{rimming} descrito no trabalho de \citeonline{Takahashi1978}.} ser o processo mais eficiente de eletrificação de nuvens, principalmente quando há presença de flocos de neve embebidos na região de fase mista \cite{Takahashi1978,Takahashi2002}. 
