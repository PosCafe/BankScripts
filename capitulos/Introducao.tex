\chapter{INTRODUÇÃO}

Desde \citeonline{whipple1929}, ao associar medidas de campo eletrostático com as observações meteorológicas de superfície dos dias com tempestades elétricas, verifica-se que a  América, África e Continente Marítimo\footnote{Região da Indonésia que abrange o Sudeste da Ásia, Filipinas e Papua-nova Guiné \cite{ramage1968,neale2003}.} são as chaminés de descargas elétricas atmosféricas -- raios -- globais. Em 1929 já se observava que a maior intensificação do campo elétrico de bom tempo está relacionada à atividade de tempestades elétricas sobre a América do Sul (AS). Apesar da Teoria do Circuito Elétrico Atmosférico Global mostrar que a América é a chaminé dominante no processo de manutenção do circuito elétrico atmosférico global, não era possível saber se a maior intensidade do campo eletrostático estava associada com uma maior taxa de raios \cite{dolezalek1972}.


% Portanto, a AS é a chaminé dominante no processo de manutenção do circuito elétrico atmosférico global. 
\sigla{name={AS},description={América do Sul}}
%Conhecer a distribuição geográfica de raios e de tempestades elétricas, bem como das taxas de raios por tempestades elétricas sobre a AS pode auxiliar na compreensão de questões fundamentais  em eletricidade atmosférica como descrito em \cite{dolezalek1972}. Por exemplo em \citeonline{williams2004}, buscou-se 
%Apenas quatro décadas mais tarde, medidas de raios por redes de sensores de VLF em solo começaram a serem desenvolvidas conforme descreve \citeonline{dolezalek1972},

%Mais tarde, foram métodos de identificação de raios a partir de imagens de satélite no visível e infravermelho próximo .

Os estudos de \citeonline{vorpahl1967frequency, sparrow169satellite}, utilizando imagens dos satélites \textit{Orbiting Solar Observatory} (OSO) e de  \citeonline{turman1978}, com satélite do \textit{Defense Meteorological Satellite Program observations} (DMSP), foram importantes para o entendimento da atividade elétrica atmosférica global com base na frequência de ocorrência de raios e não na frequência global de dias de tempestades elétricas como descrito em \citeonline{brooks1925distribution}. Com o uso do sensor de raios \textit{Supplementary Sensor L} (SSL\footnote{Composto por uma matriz de 3 $\times$ 4 (12) fotodiodos de silício (0,4--1,1 $\mu$m, com pico de resposta em 0,8 $\mu$m) capazes de contar os  brilhos transientes  associados a ocorrência de raios a cada segundo. Devido a altitude de 830 km da órbita do DMSP 8531, a matriz de fotodiodos do SSL possuiu um campo de visada aproximadamente 2200 km  $\times$ 3000 km.}) a bordo do DMSP 8531, \citeonline{turman1978} calculou a taxa de raios total -- raios intranuvens e nuvem-solo conjuntamente -- por unidade de área de um complexo de tempestades (4 $\times$ 10$^{-5}$ s$^{-1}$ km$^{-2}$) e a taxa de raios total por unidade de área observada pelo SSL (6 $\times$ 10$^{-8}$ s$^{-1}$ km$^{-2}$), equivalente a uma taxa de raios global de 31 s$^{-1}$, valor menor do que estimado em \citeonline{brooks1925distribution}, de 100 s$^{-1}$.

Em \citeonline{orville1979global}, a taxa de raios global foi calculada e obtidos valores correspondentes com os estimados por \citeonline{brooks1925distribution}.  \citeonline{orville1979global} obtiveram pela primeira vez a taxa de raios global para cada mês do ano, proporcionando uma compreensão sobre a sazonalidade da taxa de raios global. Constatou-se que a taxa de raios total sobre o continente era entre 8-20 vezes maior do que sobre o oceano e que durante o verão do hemisfério Norte a taxa de raios global era 1,4 vezes maior do que durante o verão do hemisfério Sul.

\sigla{name={LIS},description={Sensor imageador de raios}}
\sigla{name={DMSP},description={\textit{Defense Meteorological Satellite Program observations}}}
\sigla{name={OSO},description={\textit{Orbiting Solar Observatory}}}

Considerando observações mais recentes de raios total, como em \cite{christian2003global}, que utilizaram dados \textit{Optical Transient Detector} (OTD) a bordo do satélite MicroLab-1,  estimou-se que a taxa de raios na média anual sobre o oceano era de 5 s$^{-1}$, sobre as regiões continentais entre 31--49 s$^{-1}$ e a  média anual da taxa de raios global de 44$\pm$5 s$^{-1}$. Por meio de mapas da distribuição da densidade geográfica de raios, verificou-se que as maiores extensões em área com as maiores ocorrências de raios por quilômetro quadrado por ano ficavam situadas sobre o continente Americano e Africano, sendo a Bacia do Rio Congo a região mais extensa com as maiores taxas de raios do globo. Atualmente, com as observações do OTD e do \textit{Lightning Imaging Sensor} (LIS) a bordo do satélite \textit{Tropical Rainfall Measuring Mission} (TRMM) estima-se uma taxa de raios global de 46 s$^{-1}$ e o local com a maior densidade de raios anual global é a região oeste do Congo com 160   km$^{-2}$ ano$^{-1}$ \cite{cecil2014gridded}.

\citeonline{williams2004} buscaram entender a maior resposta da Curva de Carnegie associada ao horário de maior atividade de tempestades elétricas sobre a América fazendo um estudo comparativo entre as regiões da Bacia Amazônica e Bacia do Congo. Sobre a bacia hidrográfica congolesa, a taxa de raios por km$^2$ por ano são maiores enquanto que os sistemas precipitantes sobre a Bacia Amazônica possuem menor densidade de raios, porém maior volume de chuva. Com base nas observações de \citeonline{soula2003surface}, \citeonline{williams2004} sugerem que a chuva eletricamente carregada sobre a Bacia Amazônica pode funcionar como um processo de carregamento da superfície terrestre com cargas negativas, ou seja, pode funcionar como bateria do Circuito Elétrico Global.

Apesar da contribuição meridional na média anual da taxa de raios global ser liderada pelo continente Africano, durante o inverno e a primavera austral da América as taxas de raios são maiores sobre a América \cite{christian2003global}. Em \citeonline{albrecht2011b}, a partir de treze anos de dados do LIS, foi constatado que o local da maior densidade anual de raios global estava na região do Lago Maracaibo na Venezuela, sugerindo  uma mudança do máximo de densidade de raios do Congo para o Lago Maracaibo.

Portanto no contexto das medidas globais de raios, a América do Sul encontra-se em um dos locais mais eletricamente ativos do globo. Saber quando e onde as tempestades elétricas ocorrem, bem como, quais os locais em que os sistemas são mais eficientes na produção de raios, torna-se fundamental para o planejamento da infraestrutura dos países sul-americanos, no sentido de garantir segurança no transporte aéreo, fluvial e terrestre, nas linhas de transmissão de dados e de energia elétrica, etc, setores estratégicos que quando paralisados devidos aos danos causados pela queda de raios refletem em prejuízos em cascata em todos os setores econômicos. 

Por exemplo, uma falha no sistema de distribuição de energia elétrica pode cessar a energia elétrica de um bairro, cidade, etc. Pode causar queima de equipamentos eletroeletrônicos devido sobre tensão elétrica, quedas na rede de internet, o que pode paralisar  setores como: educação, pesquisa, comércio e industrias. Também gera grande número de ações judiciais indenizatórias contra as operadoras do sistema de energia, sobrecarregando o sistema judiciário. No Brasil, as empresas prestadoras de serviços de fornecimento de energia elétrica ao consumidor lideram as reclamações nos PROCONs ao lado de empresas de telecomunicações, evidenciando a falta de infraestrutura destes setores \cite{noticiaG1tempestade, noticiaG1levantamento, noticiaFolha, noticiaFolhaRaio, rankinProcon}. Em \citeonline{pinto2005arte, noticiainpe2007}, estimam-se prejuízos na ordem de 1 bilhão de dólares anuais em função da densidade de raios observada apenas sobre o Brasil. 
      
%Desta forma, o processo de eletrificação dos hidrometeoros até a formação de um raio, depende da capacidade do ar quente e úmido da superfície romper a estabilidade atmosférica e atingir altitude entre 4--15 km, regiões acima da isoterma de 0 $^{\circ}$C. Portanto, as tempestades elétricas podem indicar condições dinâmicas e termodinâmicas associadas a convecção profunda na atmosfera \cite{doswell2001,zipser2006}.

Além dos raios que atingem o solo -- raios nuvem-solo -- causando danos à sociedade, as tempestades elétricas indicam condições dinâmicas e termodinâmicas associadas à convecção profunda na atmosfera, pois o processo de eletrificação dos hidrometeoros relacionado com a formação de um raio depende da capacidade do ar quente e úmido da superfície romper a estabilidade atmosférica e atingir altitude entre 4--15 km, regiões acima da isoterma de 0 $^{\circ}$C \cite{doswell2001,zipser2006}. Por exemplo,   \citeonline{macgorman1989,carey1998, gatlin2010, schultz2011} e \citeonline{williams1999} mostram que condições de tempo severo (frentes de rajadas com velocidade superior a 92,6 km h$^{-1}$, queda de granizo com diâmetro maior do que 1,9 cm ou tornados) são geralmente precedidas de um salto na taxa de raios total das tempestades elétricas associado ao intenso crescimento de hidrometeoros na região mista.      
%governado por raios que não atingem o solo -- raios intra-nuvens 

Neste sentido, técnicas de seleção de sistemas meteorológicos a partir de dados de sensoriamento remoto combinadas com as medidas da taxa de raios, são de grande importância para o monitoramento de tempo severo. 

Diversos estudos definiram sistemas meteorológicos fazendo o agrupamento de regiões na superfície a partir de limiares de temperatura de brilho observadas em satélite. \citeonline{Maddox1980} observou a ocorrência de duas regiões: uma com temperatura de brilho $\leqslant$ -32 $^{\circ}$C (241 K) e área $\geqslant$ {100 000} km$^2$;  outra região menor, no interior da região maior, com temperatura de brilho $\leqslant$ -52 $^{\circ}$C (221 K) e área $\geqslant$ {50 000} km$^2$, que estavam associadas com Sistemas Convectivos de Meso-escala (SCM) nos Estados Unidos.  \citeonline{mapes1993} utilizaram esta metodologia e também fazem uma síntese de trabalhos que buscaram selecionar \textit{clusters} de nuvens a partir de limiares de temperatura de brilho em infravermelho. Considerando monitoramento de sistemas sobre a AS, destaco os estudos de \citeonline{machado1998,laurent2002} que foram precursores para a operacionalização do sistema ForTraCC descrito em \citeonline{vila2008}.


Em termos do monitoramento de sistemas juntamente com a taxa de raios, \citeonline{morales2003} desenvolveram o \textit{Sferics
Infrared Rainfall Technique} (SIRT), 
um algoritmo hidroestimador a partir do estudo de regiões com temperatura de brilho em infravermelho do \textit{Geostationary Operational Environmental Satellite} (GOES) coincidentes com medidas de raios da \textit{Sferics Timing and Ranging Network} (STARNET) e a precipitação observada pelo \textit{Precipitation Radar} (PR)  a bordo do satélite TRMM. Os dados da STARNET foram utilizados para definir \textit{clusters} de nuvens com raios e sem raios, em que os raios estiveram associados aos núcleos mais intensos de precipitação. Foi observado que as descargas localizadas pela STARNET, em 90\% dos casos, estiveram associados a regiões com temperatura de brilho em infravermelho menores do que 258 K.

\sigla{name={SIRT},description={\textit{Sferics
Infrared Rainfall Technique}}}
\sigla{name={GOES},description={\textit{Geostationary Operational Environmental Satellite}}}
\sigla{name={PR},description={\textit{Precipitation Radar}}}
\sigla{name={TRMM},description={\textit{Tropical Rainfall Measuring Mission}}}
\sigla{name={STARNET},description={\textit{Sferics Timing and Ranging Network}}}

%Porém a radiação infravermelha observada por satélites, corresponde apenas a irradiação do topo das nuvens. Nuvens finas, com formação acima da isoterma de 0°C, como por exemplo as nuvem cirrus, podem cobrir grandes extensões e não estar associadas a precipitação nem descargas elétricas.


Em \citeonline{houze1993}, Linhas de Instabilidades (LI) foram definidas observando extensões com chuva contínua observada por radar. Em \citeonline{MohrZipser1996}, Sistemas Convectivos de Meso-escala sobre os trópicos foram observados a partir do espalhamento radiativo em micro-ondas (85 GHz  \textit{Polarization Corrected Temperature} (PCT)), em que regiões contínuas $\geqslant$ 2000 km$^2$ com PCT $\leqslant$ 250 K foram os principais critérios para a identificação dos sistemas.

\sigla{name={PCT},description={\textit{Polarization Corrected Temperature}}}
\sigla{name={SCM},description={\textit{Sistema Convectivo de Meso-escala}}}
\sigla{name={LI},description={\textit{Linha de Instabilidade}}}

Combinando dados do PR e o \textit{TRMM Microwave Imager} (TMI) abordo do satélite TRMM, \citeonline{Nesbitt2000} desenvolveram uma metodologia para selecionar sistemas precipitantes denominados como \textit{Precipitation Features} (PFs), em que o principal critério de seleção foi identificar a área contínua de chuva na superfície, seja estimada pelas medidas de refletividade do radar PR ou pelo espalhamento radiativo em micro-ondas registrado pelo TMI, quando os sistemas possuíram extensão maior do que a varredura do PR. A partir desta metodologia, \citeonline{cecil2005} mostraram que apenas 2,4\% das PFs observadas pelo TRMM em todo globo entre 12/1997--10/2000 possuíram atividade elétrica, e para a AS, as PFs classificadas como as mais intensas e consequentemente com as maiores taxas de raios, concentraram-se na região da Bacia do Prata.% associadas a Sistemas Convectivos de Meso-escala \cite{Velasco1987,Durkee2009}.   

\sigla{name={PFs},description={\textit{Precipitation Features}}}
\sigla{name={TMI},description={\textit{TRMM Microwave Imager}}}
\sigla{name={TRMM},description={\textit{Tropical Rainfall Measuring Mission}}}

Mais tarde, \citeonline{zipser2006} utilizou medidas dos sensores do TRMM associadas à intensidade convectiva das PFs, estimando os locais das tempestades elétricas mais severas do globo. Apenas em 0,1\% da amostragem das PFs (período de 7 anos) foram observadas taxa de raios acima de 32,9 min$^{-1}$.  Entre as regiões do globo com as maiores concentrações de PFs que indicaram valores extremos (0,001\%), seja de taxa de raios, seja de mínimas temperaturas de brilho (85 e 37 GHz) ou de máxima altitude com 40 dBZ de fator de refletividade do radar, encontra-se a região Sul da AS que engloba a Bacia do Prata e o extremo Norte da Cordilheira dos Andes que abrange a Colômbia e região do Lago Maracaibo na Venezuela. 
%Conforme aumenta a intensidade das tempestades elétricas, menor será sua probabilidade de ocorrência. Apenas 0.1\% das PFs observou-se taxa de raios acima de 32.9 por minuto. 

Com base em estudos a respeito da severidade dos sistemas como em \citeonline{doswell2001, brooks2003, zipser2006}, em que as medidas  associadas à intensidade convectiva indicaram os locais com condição de tempo severo, o estudo da morfologia da estrutura tridimensional da precipitação observadas pelo PR associado com a taxa de raios observada pelo LIS pode esclarecer informações a respeito da intensidade convectiva, de modo a verificar a relação entre a taxa de raios e a severidade dos sistemas. 


\section{INTENSIDADE CONVECTIVA E A PRECIPITAÇÃO OBSERVADA POR RADAR}
\label{introRadar}

A Refletividade do Radar ($\eta$)
\begin{equation}
\eta = \sum_{i=1, 2, 3 ... }^{\Delta V} \sigma_i,
\label{refletividade}
\end{equation}
é o somatório da seção de retroespalhamento ($\sigma$) dos hidrometeoros ($i$) contidos em um elemento de volume ($\Delta V$) iluminado -- \textit{gate} -- do feixe do radar \cite{battan1973}.

\simbolo{name={$\eta$},description={Refletividade do Radar}}
\simbolo{name={$\Delta V$},description={Elemento de volume ($\Delta V$) iluminado pelo radar}}
\simbolo{name={$\sigma$},description={Seção de retroespalhamento}}

Considerando que o parâmetro de tamanho ($\alpha$), dado pela relação 
\begin{equation}
\alpha = \dfrac{2\pi R_{h} }{\lambda},
\label{parametroTam} 
\end{equation}
em que $R_{h}$ é o raio do hidrometeoro e portanto, $2\pi R_{h}$ representa a área da seção transversal do hidrometeoro precipitável na atmosfera e $\lambda$ o comprimento de onda emitido pelo radar.  Quando $\alpha$ $<<$ 1, o espalhamento da radiação eletromagnética pelas partículas será do tipo Rayleigh. Neste caso, a seção transversal de retroespalhamento $\sigma$ pode ser escrita como 
\begin{equation}
\sigma = \dfrac{\lambda^2 \alpha^6}{\pi} |K|^2,
\label{sigma}
\end{equation}
em que $|K|^2$ corresponde ao índice de refração dos hidrometeoros. Como a equação \ref{parametroTam} depende do raio do hidrometeoros $R_h$, podemos reescrever a equação \ref{sigma} considerando o diâmetro 
\begin{equation}
D_h = 2 R_h.
\label{diametro}
\end{equation}

\simbolo{name={$R_{h}$},description={Raio do hidrometeoro}}
\simbolo{name={$\alpha$},description={Parâmetro de tamanho}}
\simbolo{name={$\pi$},description={Relação entre o perímetro e o diâmetro da circunferência}}
\simbolo{name={$R_{h}$},description={Raio do hidrometeoro}}
\simbolo{name={$\vert K \vert$},description={Grandeza relacionada ao índice de refração}}
\simbolo{name={$D_h$},description={Diâmetro do hidrometeoro}}

Então, substituindo as equações \ref{parametroTam} e \ref{diametro} em \ref{sigma}, obtemos que
\begin{equation}
\sigma = \dfrac{\pi^5 K^2  }{ \lambda^4 } D_h^6.
\label{sigma2}
\end{equation}

Considerando o hidrometeoro como esférico, podemos relacionar o $D_h$ com a sua quantidade de massa, sendo  
\begin{equation}
D_h = \left( \dfrac{6 M_h}{\pi \rho} \right)^{\frac{1}{3}},
\label{dh}
\end{equation}
em que, $\rho$ é a densidade e $M_h$ a massa do hidrometeoro.
\simbolo{name={$M_h$},description={Massa do hidrometeoro}}
\simbolo{name={$\rho$},description={Densidade}}

Então substituindo a equação \ref{dh} em \ref{sigma2}, obtém-se
\begin{equation}
\sigma = \dfrac{36 \pi^3 |K|^2  }{ \lambda^4 \rho^2 M_h^2} .
\label{sigma3}
\end{equation}

Portanto, observe que a Refletividade do Radar $\eta$ (equação  \ref{refletividade}) depende  de uma relação entre quantidade de massa $M_h$, densidade $\rho$ e também do índice de refração $|K|^2$ dos hidrometeoros, conforme mostra a equação \ref{sigma3}. Entretanto, o radar mede apenas a potência do sinal retroespalhado ($P_r$). Logo, pode-se ter uma ideia da concentração de obstáculos espalhadores, porém não podemos ter certeza a respeito da massa, índice de refração e densidade dos alvos. 

\simbolo{name={$P_r$},description={Potência recebida}}

O que se faz para estimar a chuva é considerar que todos os alvos associados ao espalhamento do feixe do radar são gotas de água líquida esféricas. 

Neste caso, podemos combinar as equações \ref{refletividade} e \ref{sigma2}, obtendo $\eta$ em função do Fator de Refletividade  do radar ($Z$), em que 
%\begin{equation}
%P_r = V_{m} \dfrac{P_t G^2 \lambda^2}{2\ln2(4\pi)^3 r^4}   \dfrac{\pi^5 |K|^2  }{ \lambda^4 } \sum_{i=1, 2, 3 ... }^{V_{m}}  D_{h_i}^6.      
%\end{equation} 
%\begin{equation}
%P_r = \dfrac{P_t G^2 \lambda^2 \phi \varphi H}{ 512 (2\ln2)\pi^2 r^2} \dfrac{\pi^5 |K|^2  }{ \lambda^4 }  \sum_{i=1, 2, 3 ... }^{\Delta V}  D_{h_i}^6.   
%\end{equation} 
%\begin{equation}
%P_r = \dfrac{\pi^3 P_t G^2  \phi \varphi H }{ 512 (2\ln2) %\lambda^2 } \dfrac{ |K|^2  }{ r^2 }  \sum_{i=1, 2, 3 ... }^{\Delta %V}  D_{h_i}^6, 
%\end{equation} 
\begin{equation}
\eta =  \dfrac{\pi^5 |K|^2  }{ \lambda^4 } \sum_{i=1, 2, 3 ... }^{\Delta V} D_{h_i}^6,
\end{equation}
sendo 
\begin{equation}
Z =  \sum_{i=1, 2, 3 ... }^{\Delta V}  D_{h_i}^6,
\label{fz}
\end{equation}
o Fator de Refletividade do Radar.

Conforme a equação do radar descrita em \citeonline{battan1973}, cada medida de $P_r$ depende de parâmetros fixos como ganho da antena, potência do sinal transmitido, largura de pulso, efeito de lóbulo, comprimento de onda e ângulo sólido associado ao feixe emitido. Considerando que todas as constantes que envolvem a equação do radar equivalem a $C$ e $r$ sendo a distância entre o radar e o alvo espalhador, temos que a cada \textit{gate} do radar, $P_r$ será  
\begin{equation}
P_r = C \dfrac{|K|^2}{r^2}  Z .
\end{equation}

Desta maneira, a partir da $P_r$ medida pelo radar, sabendo a distância $r$ do alvo e considerando que a chuva é composta de esferas de água líquida ($\vert K_{\mathrm{agua}}\vert^2=0,931$), podemos determinar $Z$, pois  
\begin{equation}
Z = \dfrac{P_r r^2}{C |K|^2}.
\label{zsimples}
\end{equation}
%pois, $|K|^2=0.931$ para água líquida.

Conforme mostra a equação \ref{fz}, $Z$ depende de $D_h$ que está associado com a massa ou o volume de chuva pela equação \ref{dh}.

\simbolo{name={$Z$},description={Fator de refletividade do radar}}

No entanto, ao observar o perfil vertical do Fator de Refletividade do Radar $Z$, verifica-se que o feixe atinge regiões na atmosfera com temperaturas abaixo de 0 $^{\circ}$C. Nestes casos, a potência $P_r$ estará também associada ao espalhamento em gelo de nuvem ao invés de apenas água líquida. Também, mesmo sabendo as distâncias $r$ dos alvos espalhadores, não podemos afirmar sobre a temperatura da atmosfera para cada distância $r$ do radar, se haverá água super-resfriada acima da isoterma de 0~$^{\circ}$C ou haverá granizo caindo abaixo da isoterma de 0~$^{\circ}$C. Então, os dados brutos das observações de radar consideram $|K|^2$ como constante, geralmente $\vert K_{\mathrm{agua}}\vert^2 = 0,931$.

%Portanto, acima da da isoterma de 0 $^{\circ}$C, espera-se uma diminuição abrupta de $Z$ devido a mudança da constante dielétrica da água associada ao seu congelamento, sendo que a equação do radar mantem $|K|_{\mathrm{agua}}^2$ fixo para as medidas acima da isoterma de 0 $^{\circ}$C.

Aplicando $10\log_{10}$, na equação \ref{zsimples}, e assumindo  $\vert K_{\mathrm{agua}}\vert^2 = 0,931$ e $\vert K_{\mathrm{gelo}}\vert^2= 0,197$, para uma mesma medida de $P_r$, as diferenças observadas nos valores de $Z$ em dB irão corresponder a
%dBZ_{\mathrm{gelo}} - dBZ_{\mathrm{agua}} = - 10\log_{10}(K_{\mathrm{gelo}}^2 ) + 10\log_{10}(K_{\mathrm{agua}}^2).
%Substituindo os valores de $\vert K_{\mathrm{agua}}\vert^2$ e $\vert K_{\mathrm{gelo}}\vert^2$, obtém-se que 
\begin{align}
dBZ_{\mathrm{agua}}-dBZ_{\mathrm{gelo}} &=  10\log_{10}(\vert K_{\mathrm{gelo}}\vert^2 ) - 10\log_{10}(\vert K_{\mathrm{agua}}\vert^2)\\
dBZ_{\mathrm{agua}}- dBZ_{\mathrm{gelo}} &= -6,7 dBZ,
\end{align}
mostrando que devido ao índice de refração do gelo ser menor do que o índice de refração da água ($\vert K_{\mathrm{agua}}\vert^2 > \vert K_{\mathrm{gelo}}\vert^2$), ao considerar $\vert K_{\mathrm{agua}}\vert^2$ em regiões que os hidrometeoros estão congelados, haverá uma redução de 6,7 dBZ em relação à considerar $\vert K_{\mathrm{gelo}}\vert^2$.  

Nas observações de $Z$ no perfil vertical, a região ou camada de derretimento do gelo é bastante marcada, pois haverá uma redução de $\simeq$7 dBZ em regiões acima da isoterma de {0~$^{\circ}$C} devido ao congelamento dos hidrometeoros, enquanto que logo abaixo da isoterma de 0~$^{\circ}$C, haverá um aumento de $Z$ devido ao derretimento dos hidrometeoros. Conforme descrevem \citeonline{austin1950}, quando há derretimento de flocos de neve no perfil atmosférico, observa-se um aumento da Refletividade -- $\eta$ -- logo abaixo da isoterma de {0~$^{\circ}$C} denominado como banda brilhante e que este aumento de $\eta$ pode não estar associado apenas com a mudança da constante dielétrica do gelo para água líquida, mas conforme ilustra a figura \ref{Austin},  processos de colisão-coalescência causam um efeito de aumento contínuo da refletividade abaixo do nível de congelamento e o aumento da velocidade terminal das gotas favorece as rupturas e a evaporação, causando redução dos diâmetros das gotas e consequentemente diminuindo $\eta$,  fatores estes que irão influenciar sobre a banda brilhante observada por radar.

\begin{figure}[ht]
\centering
\includegraphics[width=\textwidth]{img/ilustracoes/AustinBemis}
\caption{Efeitos que influenciam sobre a banda brilhante observada por radar (adaptada de \citeonline{austin1950}).}
\label{Austin}
\end{figure}

Estes efeitos sobre o perfil de $Z$ na região de derretimento  foram explorados em \citeonline{Fabry1995} para identificar processos de agregação, acreção e colisão coalescência, observados a partir de flutuações nos valores de $Z$ acima e abaixo do nível de congelamento, pois acima do nível de congelamento, um aumento nos valores de $Z$ conforme sugere a curva da figura \ref{Austin} referente ao efeito de coalescência, pode estar associado a processos de crescimento de cristais de gelo, flocos de neve (agregação) e granizo (acreção), sendo que abaixo do nível de derretimento, as flutuações de $Z$ foram associadas à colisão-coalescência, rupturas e evaporação das gotas. 

Em \citeonline{Fabry1995}, os processos de crescimentos de hidrometeoros também foram estudados a partir da espessura da camada de derretimento, pois está relacionada com o \textit{lapse-rate} da atmosfera \cite[p.~462]{austin1950,mason1971_2ed}. Em uma atmosfera instável, com convecção profunda e precipitação convectiva, a camada de transição de fase de gelo para a água liquida é perturbada por correntes ascendentes. A mudança do índice de refração da água não ocorre apenas logo abaixo de 0 $^{\circ}$C, pois no ambiente convectivo teremos água super-resfriada em temperaturas de -15 $^{\circ}$C, o que intensifica o processo de acreção podendo gerar granizo que cai derretendo até a superfície. Nestes casos espera-se uma camada de derretimento mais espessa, sem a definição da banda brilhante. 
\simbolo{name={$^{\circ}$C},description={Grau Celsius}} 

Considerando um regime de precipitação estratiforme, que é governado por processos de agregação, será observada a banda brilhante e uma camada de derretimento menos espessa, pois os flocos de neve possuem velocidade terminal e densidade inferior  ao granizo/saraiva e o $graupel$, portanto, percorrem um caminho menor até o derretimento \cite{Fabry1995}. 

Estes efeitos no perfil de $Z$ refentes aos processos de formação dos hidrometeoros e que dependem da intensidade convectiva do ambiente são ilustrados de maneira sintética na figura \ref{convestraDerr}. A partir de um sistema de tempestade observado pelo TRMM, podemos analisar a média dos perfis estratiformes (menor intensidade convectiva) e a média dos perfis convectivos (maior intensidade convectiva) separadamente, identificando os principais processos de nuvem descritos nesta sessão.

\newpage
\begin{figure}[htb]
\centering
\includegraphics[width=0.8\textwidth]{img/ilustracoes/tempestade}\\
\includegraphics[width=\textwidth]{img/ilustracoes/perfis}
\caption{Interpretação dos perfis médios de $Z$ observado pelo PR a bordo do TRMM para uma determinada tempestade elétrica. A barra de cores corresponde a temperatura de brilho em infravermelho observada pelo \textit{Visible and Infrared Scanner} (VIRS).}
\label{convestraDerr}
\end{figure}

%\begin{xalignat}{3}
%\mathbf{n} \cdot \mathbf{E} = 0 && &e  && \mathbf{n} \cdot \mathbf{B} = 0.
%\end{xalignat}

%\begin{equation}
%|K|^2 = \left( \dfrac{m^2-1}{m^2+2}\right)^2
%\end{equation}

%Também, sabendo que $Z$ é proporcional ao diâmetro dos hidrometeoros $D_h$ elevado a 6 potência, como mostra a equação \ref{fz}, os processos de crescimento de flocos de neves, granizo e gotas, são marcados por aumentos de $Z$ no perfil de altitude, conforme sugere a curva da figura \ref{Austin}, relacionada ao efeito de coalescência, porém no caso da formação de neve e granizo, o aumento de $Z$ ocorre acima do nível de congelamento.
%, enquanto que abaixo do nível de congelamento.
%, os acréscimos nos valores de $Z$ podem indicar processos de colisão coalescência e os decréscimos de $Z$, a evaporação e rompimento/quebra das gotas. 

%Considerando um regime de precipitação estratiforme, que é governado por processos de agregação, será observado um aumento acentuado no fator de refletividade do radar em logo abaixo da isoterma de 0 $^{\circ}$C associado ao derretimento de flocos de neve. \simbolo{name={$^{\circ}$C},description={Grau Celcius}} 







% durante o caminho que a precipitação percorre até a superfície ou temperaturas acima de 0°C.
%Na figura \ref{fabry}, \citeonline{Fabry1995}
%e o trabalho de 
%\begin{figure}[hbp]
%  \centering{
%  \subfloat[\cite{Fabry1995}]{{\includegraphics[scale=0.25]{img/ilustracoes/fabry}} \label{fabry}}
%  \subfloat[\cite{Takahashi2002}]{{\includegraphics[scale=0.35]{img/ilustracoes/takahashi}} \label{taka}}
%  }
%\caption{Fabry Taka}
%\label{fabyTaka} 
%\end{figure} 

%Consequentemente, a taxa de raios associa-se com a intensidade convectiva devido a acreção\footnote{A acreção é o processo de \textit{rimming} descrito no trabalho de \citeonline{Takahashi1978}.} ser o processo mais eficiente de eletrificação de nuvens, principalmente quando há presença de flocos de neve embebidos na região de fase mista \cite{Takahashi1978,Takahashi2002}. 

\section{TEORIAS DE ELETRIFICAÇÃO DAS NUVENS}
\index{Eletrificação das nuvens}

Os processos de eletrificação das nuvens são intrínsecos ao processo de desenvolvimento da precipitação, especialmente em regiões com temperaturas entre -5 $^{\circ}$C e -40 $^{\circ}$C, portanto, está fortemente relacionado ao crescimento do gelo de nuvem \cite{mason1953}. 

A partir de medidas em superfície, observa-se que os campos eletrostáticos produzidos pelas tempestades elétricas são da ordem de dezenas de milhares de volts por metro, que correspondem a centros de cargas nas nuvens com dezenas de coulombs. 
\citeonline{williams1989} mostra uma síntese de trabalhos com medidas de campo eletrostático de tempestades elétricas no período entre 1752 e 1989 relatando que na maioria das observações a curva de campo elétrico observada correspondia com a perturbação causada por uma estrutura tripolar de cargas nas nuvens, havendo um centro de carga positivo na parte superior, um centro de carga negativo na região central e um centro de carga positivo menos intenso na base da nuvem conforme mostra a figura \ref{fig:tripeletr}.


\begin{figure}[ht]
\centering 
\includegraphics[width=\textwidth]{img/ilustracoes/tripoloeletr}
\caption{Representação do tripolo eletrostático de uma tempestade elétrica. A medida de campo eletrostático na superfície (parte inferior da figura), corresponde aos centros de cargas $Q_{+}$, $Q_{-}$ e $q_{+}$ (adaptada de \citeonline{ogawahandbook}).}
\label{fig:tripeletr}
\end{figure}

Mesmo que o modelo do tripolo eletrostático proposto por \citeonline{williams1989} seja uma teoria condizente com a estrutura de cargas dominante em uma tempestade elétrica, considerando sondagens de campo elétrico no interior das tempestades \citeonline{rust1996} verificam que os centros de cargas podem estar distribuídos de maneira mais complexa. 
\citeonline{stolzenburg1998} criou um modelo conceitual para a estrutura de cargas das tempestades elétricas exposto na figura \ref{fig:multipcentros}, sugerindo que nas regiões onde ocorrem ventos ascendentes pode haver 4 ou mais centros de carga enquanto as regiões com correntes descendentes, 6 ou mais centros de cargas.

\begin{figure}[ht]
\centering 
\includegraphics[width=\textwidth]{img/ilustracoes/nuvem}
\caption{Estrutura elétrica de uma tempestade elétrica idealizada a partir de sondagens de campo eletrostático realizadas no interior de nuvens de tempestades (adaptada de \citeonline{stolzenburg1998}).}
\label{fig:multipcentros}
\end{figure}

%Experimentos relacionados a simulação de nuvens em laboratório mostram que a presença de gelo é fundamental no processo e que a carga adquirida pelos granizos depende da temperatura do ambiente, do conteúdo de água líquida da nuvem, da velocidade de colisão entre os hidrometeoros e dos tamanhos dos cristais de gelo 1978,saunders2008}. 

De modo a explicar os intensos campos elétricos associados às nuvens de tempestades elétricas e o confinamento dos centros de cargas em regiões com temperaturas entre -5~$^{\circ}$C e -40~$^{\circ}$C, as teorias de eletrificação de hidrometeoros podem ser divididas em duas grandes frentes teóricas: Eletrificação por Convecção e a Eletrificação por Precipitação.

%Os processos de eletrificação das nuvens não é completamente entendido ou descrito na literatura devido a sua complexidade, que envolve desde fenômenos em meso-escala, por exemplo a convergência de massas de ar até as propriedades físico-químicas da água.  


\subsection{Eletrificação por Convecção}
\index{Teoria de!Eletrificação por Convecção}

Pressupõe-se que as cargas elétricas são geradas por fontes externas às nuvens, associadas a ionização de moléculas do ar atmosférico por átomos radioativos na superfície terrestre ou por radiação cósmica \cite{wilson1956,grenet1947, vonnegut1962,phillips1967}.

Devido a esta distribuição de íons livres na atmosfera, o campo elétrico de bom tempo atrai os íons positivos para próximo à superfície terrestre e quando há rompimento da estabilidade atmosférica, o movimento ascendente transporta os íons positivos próximos a superfície terrestre para o interior das nuvens, como ilustrado na figura \ref{fig:elec-a}. Conforme a nuvem se desenvolve verticalmente, íons negativos são atraídos pelas cargas positivas introjetadas na nuvem tornando o seu topo  negativamente carregado, como ilustra a figura \ref{fig:elec-b}. Com o acúmulo de íons negativos no topo das nuvens e a atuação de correntes descendentes, ocorre estranhamento lateral das cargas negativas do topo da nuvem concentrando regiões de cargas negativas próximas a base da nuvem nas regiões laterais e intensificando a atração de íons positivos a partir da superfície, como ilustra a figura \ref{fig:elec-c} \cite{vonnegut1962,wagner1981,vonnegut1995}.

%Conforme descrito por \citeonline{vonnegut1995}, o campo elétrico de tempo bom pode concentrar íons positivos na baixa atmosfera. Na ocorrência de térmicas os íons positivos podem ser transportados para o interior das nuvens eletrificandos-as positivamente, conforme pode ser visualizado na figura \ref{fig:elec-a}. Com o crescimento vertical da nuvem e o excesso de cargas positivas,  íons  negativos são atraídos tornando o topo da nuvem negativamente carregado (ver figura \ref{fig:elec-b}).
% Esse mecanismo de eletrificação pode ser produzido pela distribuição de íons livres na atmosfera \cite{wilson1956,phillips1967}  \cite[apud \cite{vonnegut1995}]{grenet,wagner1981}. 

\begin{figure}[ht]
   \centering
   \subfloat[Íons positivos injetados pela convecção.]{\includegraphics[width=5cm]{img/ilustracoes/eleconvectiva-a}\label{fig:elec-a}} 
   \subfloat[Íons negativos são atraídos.]{\includegraphics[width=5cm]{img/ilustracoes/eleconvectiva-b} \label{fig:elec-b}}  
   \subfloat[Retro-alimentação positiva devido ao efeito corona.]{\includegraphics[width=6cm]{img/ilustracoes/eleconvectiva-c} \label{fig:elec-c}}
   \caption{Representação do processo convectivo de eletrização.}
   \label{fig:elec}
\end{figure}

\citeonline{phillips1967} e \citeonline{vonnegut1991} mostram evidências que comprovam a participação da Eletrização por Convecção na eletrificação das nuvens de tempestades elétricas. Porém verifica-se que a disponibilidade de íons da atmosfera não é suficiente para proporcionar centros de cargas tão intensos conforme se observa nas medidas de campo eletrostático. 

\subsection{Eletrificação por Precipitação}
\index{Teoria de!Eletrificação por Precipitação}

A Teoria de Eletrificação por Precipitação sugere que os processos de eletrização das nuvens estão associados com a formação e interação entre os hidrometeoros e as propriedades físicas da água. É o mecanismo mais aceito para explicar a estrutura dos centros de cargas observados nas tempestades elétricas conforme as observações em \citeonline{stolzenburg1998} e \citeonline{williams1989}, pois sugere que uma região de centro de carga é definida a partir da sedimentação de uma população de partículas com quantidade de massa equivalentes, pois passaram por condições de crescimento semelhantes, adquirindo uma mesma polaridade elétrica. 


%Em função dos diferentes níveis de temperatura, pressão de vapor, correntes ascendentes e descendentes, o ambiente de nuvem proporciona o crescimento de gotas com diferentes tamanhos, diversos tipos de cristais de gelo e granizo com diferentes tamanhos. Com a atuação da força gravitacional, as correntes ascendentes, descendentes e a força de arraste, os hidrometeoros adquirem velocidades diferenciadas e colidem entre si.

Durante o processo de crescimento de hidrometeoros de nuvem, poderá haver transferência de cargas entre os mesmos, especialmente quando há colisões em que o tempo de colisão é pequeno, gerando fricção e se houver rupturas de gotas ou de cristais de gelo \cite{Lenard1892, reynolds1957, matthews1964, jonas1968, simpson1909}. Durante o crescimento e as colisões dos hidrometeoros, a eletrificação poderá ocorrer por Processo Indutivo ou Processo Não-indutivo.


\subsubsection{Processo indutivo} 
\index{Processo de eletrificação! indutivo}

Ocorre quando há colisões entre os hidrometeoros e não ocorre coalescência ou acreção portanto, torna-se mais provável na colisão entre o \textit{graupel}\footnote{Granizo com diâmetro menor que 2 mm.} e cristais de gelo.

As colisões ocorrem sobre a influência de um campo elétrico $\mathbf{E}$ já existente na nuvem, provocando a polarização do \textit{graupel} e dos cristais de gelo conforme na figura \ref{fig:ind}, para situação antes da colisão. Depois da colisão conforme mostra a ilustração na figura \ref{fig:ind}, cargas são transferidas, tornando o \textit{graupel} carregado com carga negativa e os cristais de gelo com carga positiva ou falta de elétrons.

\begin{figure}[ht]
   \centering
   \includegraphics[height=10cm]{img/ilustracoes/indutivo}
   \caption{Eletrização dos hidrometeoros por processo indutivo. A ilustração representa a configuração das cargas antes e depois da colisão.}
   \label{fig:ind}
\end{figure}

Porém, \citeonline{macgorman1998} apontam que a intensidade do campo elétrico de bom tempo não possui intensidade suficiente para polarizar as partículas de gelo de nuvem. Então, o processo de colisão indutivo deve ocorrer após um mecanismo de eletrização não-indutivo promover um campo elétrico $\mathbf{E}$ no interior da nuvem. 

\subsubsection{Processo não-indutivo}
\index{Processo de eletrificação! não-indutivo}

Conforme diversos estudos dos mecanismos de eletrificação de nuvem realizados em laboratórios que buscaram recriar as condições atmosféricas relacionadas ao crescimento dos hidrometeoros, foi constatado que o carregamento não-indutivo depende: do tamanho dos hidrometeoros, do conteúdo de água líquida dentro da nuvem, da temperatura e da velocidade de impacto entre os hidrometeoros  \cite{reynolds1957, Takahashi1978, baker1994, Saunders1999, pereyra2000}. Em \citeonline{Takahashi1978} e \citeonline{Takahashi2002}, descreve-se que o mecanismo mais eficiente de eletrização dos hidrometeoros envolve a colisões entre o \textit{graupel}/granizo com neve seca. 

No trabalho de \citeonline{saunders2008} são expostos alguns dos principais resultados de experimentos de laboratório que investigaram a carga, positiva ou negativa, adquirida pelo \textit{graupel} durante o processo de acreção em função da temperatura e o conteúdo de água líquida. Na figura  \ref{fig:saunders}, podemos observar diferentes curvas que marcam a fronteira entre o carregamento positivo e negativo do \textit{graupel}. Apesar de diferenças entres as curvas, que são associadas às diferentes características das câmaras de nuvens que cada autor utilizou, observa-se que em temperaturas mais altas do que -10 $^{\circ}$C, o carregamento é positivo, para temperaturas mais baixas do que -10 $^{\circ}$C o carregamento é negativo, e que o aumento da taxa de acreção pode tornar o carregamento positivo do \textit{graupel} em regiões com temperaturas abaixo de -10 $^{\circ}$C.

 

%\begin{figure}[htp]
%\centering 
%\subfloat[(adaptada de \citeonline{saunders2008}).]%{\includegraphics[width=9cm]{img/ilustracoes/saunders} %\label{fig:saunders}}
%\subfloat[(adaptada de \citeonline{Takahashi2002}).]%{\includegraphics[width=9cm]{img/ilustracoes/takahashi} %\label{fig:saunders}}
%\label{takasaunders}
%\caption{}
%\end{figure}

\begin{figure}[htp]
\centering 
\includegraphics[width=15cm]{img/ilustracoes/saunders}
\caption{Sinal da carga adquirida pelo gelo durante o processo de acreção, conforme diferentes experimentos realizados em laboratório (adaptada de \citeonline{saunders2008}).}
\label{fig:saunders}
\end{figure}

Considerando o estudo de \citeonline{faraday1859}, observa-se que em temperaturas logo abaixo de 0 $^{\circ}$C, partículas de gelo podem possuir uma camada de água líquida super-resfriada com aproximadamente 10 {\AA}. Neste ambiente, segundo \citeonline{fletcher}, as moléculas de água tendem a orientar a parte negativa do seu dipolo elétrico permanente para a superfície da água, em contato com o ar. Esse efeito também pode ser observado em partículas de gelo derretendo em temperaturas logo acima de 0 $^{\circ}$C, conforme sugerido em \citeonline{fletcher}.
\simbolo{name={{\AA}},description={Angstron}}

No trabalho de \citeonline{Baker1987}, observa-se que a camada de água líquida super-resfriada sobre o gelo, também denominada como camada de água quase-líquida, pode crescer/diminuir por difusão de vapor, e que se a taxa de difusão de vapor for maior sobre a camada quase-líquida do que sobre a porção de gelo, a partícula de gelo adquiri carga positiva e quando a taxa de difusão de vapor é maior sobre o gelo do que sobre a camada de água quase-líquida, a carga adquirida pelo gelo é negativa. Também é constatado que as partículas de gelo que possuem esta camada de água quase-líquida, ao colidirem com outras partículas podem trocar massa e carga entre si, sendo que, geralmente as partículas com maior quantidade de água líquida transferem massa para as partículas com menor quantidade de água quase-líquida \cite{baker1994}. 

O efeito termoelétrico no gelo também contribui para a eletrização das partículas de gelo sob a influência de gradientes de temperatura, pois há dissociação da água em frações com dimensões moleculares das partículas de gelo, gerando disponibilidade de íons de H$^{+}$ e OH$^{-}$. Como a mobilidade do íon de hidrogênio é maior do que da hidroxila devido ao peso molecular, nas regiões mais quentes ocorre maior difusão de íons de hidrogênio deixando maiores concentrações de hidroxilas (OH$^{-}$) sobre o gelo, portanto proporcionando o carregamento negativo de determinada porção da partícula de gelo e aumentando a disponibilidade de íons positivos (H$^{+}$) no ambiente de nuvem \cite{latham1961}.  

\section{PROPOSTA}

A condição de tempo severo em \citeonline{brooks2003}, tratada em termos da ocorrência de granizo, rajadas de ventos e ocorrência de tornados, foi diagnosticada a partir de observações e variáveis meteorológicas associadas com a intensidade convectiva. No entanto, os estudos de \citeonline{macgorman1989, carey1998, gatlin2010, schultz2011, williams1999} e \citeonline{cecil2005} buscaram encontrar relações entre a taxa temporal de raios dentro das tempestades elétricas com a ocorrência de tempo severo e observaram que os \textit{lightning jumps} precediam a ocorrência de eventos de tempo severo em algumas situações.

Por outro lado, as tempestades potencialmente severas foram identificadas e estudadas em \citeonline{zipser2006} por meio do banco de dados das FPs desenvolvido em \citeonline{Nesbitt2000}, enquanto que os estudos de \citeonline{albrecht2011b, christian2003global} e \citeonline{cecil2014gridded} concentraram-se em descrever somente a distribuição de densidade geográfica de raios sobre globo. 

A intensidade convectiva das PFs em \citeonline{cecil2005, zipser2006} e \citeonline{Rasmussen2011}, foram investigadas conforme o volume de chuva observado pelo PR ou altura de 40 dBZ ou espalhamento de gelo ou taxa de raios no tempo, porém as tempestades elétricas não foram classificadas em função da sua densidade espaço temporal de raios ou frequência de ocorrência.

Apesar de \citeonline{zipser2006} mostrarem o ciclo diurno e o ciclo anual das PFs potencialmente severas sobre o globo e em \citeonline{christian2003global} e \citeonline{cecil2014gridded} o ciclo diurno e o ciclo anual de ocorrência de raios global, carece na literatura estudos a respeito do ciclo diurno e ciclo anual focado nas tempestades elétricas, de modo a descrever o marco das tempestades elétricas para diferentes regiões da América do Sul.

Nesta pesquisa realiza-se um estudo focado nos sistemas precipitantes que possuem atividade elétrica -- as tempestades elétricas -- apenas sobre a América do Sul a partir das observações orbitais do satélite TRMM, mais especificamente do sensor de raios LIS, do radiômetro \textit{Visible and InfraRed Scanner} (VIRS) e do radar PR, desde o início de 1998 até o final de 2011.%, totalizando 14 anos de observações.% e desta forma criar-se um banco de dados de tempestades elétricas do TRMM.

A intensidade das tempestades elétricas neste trabalho de pesquisa é estudada com base nas observações de raios do LIS e aspectos morfológicos como: dimensões relacionadas a extensão dos sistemas e a estrutura tridimensional da precipitação observada pelo PR. No entanto, a intensidade convectiva das tempestades elétricas com base nas observações do PR é investigada a partir das distribuições de probabilidade de ocorrência dos perfis de refletividade do PR em função da altitude, conforme em \citeonline{Petersen2001} e também a partir das distribuições de probabilidade de ocorrência dos perfis de refletividade do PR em função da temperatura do perfil atmosférico, pois os processos de eletrificação de nuvem dependem essencialmente da temperatura (entre -5 $^{\circ}$C e -40 $^{\circ}$C), conteúdo de água líquida e velocidade vertical \cite{Takahashi1978, saunders2008} e considerando uma grande extensão como é o caso do continente Sul-americano teremos variação das alturas das isotermas com a latitude, estações do ano e passagens de sistemas sinóticos.

Mesmo que a intensidade convectiva possa ser mensurada pela altura de 40 dBZ observada pelo PR \cite{zipser2006, Rasmussen2011}, a condição de tempo severo, deve estar mais relacionada com a intensificação dos processos microfísicos que corroboram com os processos de eletrificação de nuvens associados a convecção intensa, como por exemplo o processo de acreção que envolve a formação do \textit{graupel} e o granizo e que podem ser estimados pelas observações de refletividade de radares meteorológicos, conforme descreve-se em \ref{introRadar} \cite[p.~462]{austin1950,Fabry1995,mason1971_2ed}.

Portanto, diante lacunas identificadas na literatura refente ao estudo de sistemas precipitantes associados a condição de tempo severo e estudos das distribuições de densidade geográficas de raios, de maneira mais específica, esta tese propõe-se a:

\begin{itemize}
\item Construir um banco de dados de tempestades elétricas do TRMM sobre a América do Sul a partir das observações orbitais do sensor de raios LIS, do radiômetro \textit{Visible and InfraRed Scanner} (VIRS) e do radar PR, desde o início de 1998 até o final de 2011.
\item Determinar o marco das tempestades elétricas na AS a partir do ciclo diurno e ciclo anual para diferentes regiões da América do Sul.
\item Construir o mapa de densidade geográfica de tempestades elétricas e o mapa da densidade de raios por tempestade elétrica, evidenciando as regiões e estações do ano em que os processos de eletrificação das tempestades elétricas são mais eficientes na produção de raios.
\item Caracterizar a severidade das tempestades elétricas em função da sua taxa de raios no tempo (raios [min$^{-1}$]) e da sua taxa de raios no tempo por área (raios [min$^{-1}$] [km$^{-2}$]).
\item Identificar processos microfísicos \cite[p.~462]{austin1950,Fabry1995,mason1971_2ed} com base no estudo da morfologia da estrutura tridimensional da precipitação observada pelo PR que indique quais as tempestades elétricas que representam maior intensidade convectiva: as com elevados números de raios durante os $\simeq$90 segundos que o TRMM as observa ou as com elevadas densidades de raios na superfície.
\end{itemize}

%Além da intensidade convectiva dos sistemas estar associada com a atividade elétrica dos mesmo, como mostram  \citeonline{} a altitude em que o PR observa valores de 40 dBZ, é um indicativo da velocidade da corrente ascendente. Porém, as altitudes das isotermas variam com a latitude e os processos de eletrificação dependem essencialmente do temperatura e conteúdo de água líquida super-resfriada.    

%de nuvens dependem 
%dependerá uma maior espessura entre a isoterma de 0 C e      

%estudo busca-se responder a seguinte questão: 

% buscando identificar qual é a taxa de raios que corresponde potencialmente às condições de convecção profunda e consequentemente de tempo severo, conforme cada localidade da extensa região da AS.

%A estrutura tridimensional da precipitação observada pelo PR é estuda com base na probabilidade de ocorrência por altitude, conforme descreve \cite{yuter1995}, e também conforme a probabilidade de ocorrência por níveis de temperatura do perfil atmosférica. As seções \ref{introRadar} e \ref{derretimento} fazem uma revisão de conceitos fundamentais que devem ser considerados na interpretação dos diagramas de probabilidade elaborados associados as observações por radar.
 
%------------------------------------------------------\\
%\textit{Acho que essa parte de baixo é Metodologia... ou discussão dos resultados}

%A grande extensão territorial do Brasil na America do Sul  
%m base nos experimentos do LBA, investigar a morfologia dos sistemas na pré monsão, na região amazônica. 

%As descargas na região amazônica parecem estar mais associadas com os processos de inibição do que precipitação... Em uma região tropical dominada por processos quentes, as descargas podem indicar inibição de colisão coalescência, desenvolvimento de fase fria e menos precipitação.

%mas no período úmido de leste raios relacionam-se mais com a precipitação.

%...A Rachel já fez uma boa discussão sobre a microfísica dos sistemas da amazônia, períodos seco úmido e de transição. As tempestades foram organizadas em clusters, estudou-se o ciclo de vida, a ocorrência de raios em áreas desmatadas e com floresta/outras, e foi explicado a microfísica dos sistemas basicamente com: taxa de raios positivos e negativos, eco tops, VIL, CAPE, CINE. Falta explotar a os CFADS para essa região. Como varia a probabilidade de ocorrência por altitude dos perfis de refletividade nos períodos seco de transição e úmido.

%Em desenvolvimento ...
%As tempestades mais eficientes estão mostrando tamanhos diversos. Tanto cluster grandes quanto pequenos podem ser eficientes. Não é uma relação que depende apenas da área, ou da fração convectiva ou estratiforme. Os raios relacionam-se com o ciclo de vida, que no caso é aleatório. Depende do ciclo de vida

%Relação exponencial entre max fl no pixel versus fl-rate/km2. Maior a concentração de raios em um único pixel, menos eficiente e mais chuva. Talvez uma reintensificação do sistema maduro. Mas existem duas categorias de sistemas :
%1 – os mais eficiente com área ~10^3 e chuva ~10^5
%2 – as com maior fl/rate no pixel, menos eficientes e com chuva ~10^9



%Com o experimento de campo LBA (Large-Scale Biosphere-Atmosphere Experiment in Amazonia) realizado na região de Rondônia entre janeiro e fevereiro de 1999, foi possível identificar alguns fatores importantes que regulam a precipitação na região Amazônica. Além disso o LBA foi importante para validação de dados do satélite TRMM que são amplamente utilizados nesta pesquisa \cite{silva2002lba,williams2002,albrecht2011}.

%Silva Dias M. A. F. et al, (2002), fazem uma síntese dos principais resultados e objetivos do LBA, entre estes destaco os estudos de Anagnostou e Morales, (2002), \citeonline{Carvalho2002} que mostram dois regimes de vento em 700 mb, de Leste e de Oeste, em que observou-se maior precipitação convectiva e atividade elétrica durante o regime de ventos de Leste. Petersen W. A. et al, (2002), investigaram como que esses dois regimes de vento (Leste-Oeste) observados durante o LBA em Rondônia, influenciam no número de descargas elétricas observadas pelo LIS (Ligthning Image Sensor), não apenas para região Amazônica mas para toda a América do Sul durante 4 verões entre 1997 e 2000.

%A variação intra-sazonal da atividade elétrica durante o período chuvoso mostrou-se evidente. \citeonline{petersen2002trmm},  identificaram regiões de extremos opostos de atividade elétrica que devem estar associados ao mecanismos de manutenção da monção na América do Sul, principalmente com a dinâmica que envolve Zona de Convergência do Atlântico Sul (ZCAS) \cite{CarvalhoJones2002,Carvalho2002}.   

%\section{OBJETIVOS... PROPOSTA...}
%\begin{itemize}
%\item Criar um banco de dados de tempestades elétricas do TRMM. 
%\item Criar mapas que identifique a densidade de tempestades elétricas e de raios sobre a América do Sul.
%\item Descrever o ciclo diurno e o ciclo anual das tempestades elétricas do TRMM.
%\item Classificar a intensidade das tempestades elétricas com base na taxa de raios e no estudo da frequência de ocorrência do Fator de Refletividade do radar por temperatura e por altura. 
%\end{itemize}
%...
