\chapter{INTRODUÇÃO}

Desde \citeonline{whipple1929}, ao associar medidas de campo eletrostático com as observações meteorológicas globais de superfície dos dias com tempestades elétricas, verifica-se que a região tropical das Américas, África e Continente Marítimo são as chaminés de descargas elétricas atmosféricas -- raios -- globais. Em 1929 já se observava que a maior intensificação do campo elétrico de bom tempo está relacionada a maior atividade de tempestades elétricas sobre a América do Sul (AS). Portanto, a AS é a chaminé dominante no processo de manutenção do circuito elétrico atmosférico global.
\sigla{name={AS},description={América do Sul}}


Mais tarde em \citeonline{orville1979global}, foram utilizadas imagens de satélite -- \textit{Defense Meteorological Satellite Program
observations} (DMSP) -- para obter mapas da densidade global de raios, promovendo o entendimento dos locais de maior atividade elétrica atmosférica global com base na frequência de ocorrência de raios e não na frequência global de dias de tempestades elétricas como descrito em  \citeonline{brooks1925distribution}.

\sigla{name={LIS},description={Sensor imageador de raios}}
\sigla{name={DMSP},description={\textit{Defense Meteorological Satellite Program
observations}}}

Considerando observações mais recentes de raios total, como em \cite{christian2003global}, que utilizam dados do ODS, estima-se que na média anual, sobre a região oceânica, ocorrem 5 raios s$^{-1}$. Sobre as regiões continentais são observados entre 31 -- 49 raios s$^{-1}$ e a na média anual global 44$\pm$5 raios s$^{-1}$.


As densidades globais de raios em \citeonline{christian2003global,cecil2014gridded}, mostram que as maiores extensões em área com as maiores valores de densidade de raios ficam situadas sobre o continente Americano e Africano. 

 \citeonline{albrecht2009tropical,cecil2014gridded}, observa-se que as maiores densidades de raios do globo ocorrem sobre a América do Sul e África.

Estando os países da América do Sul situados em uma das regiões de maior atividade elétrica atmosférica do globo, saber quando e aonde as tempestades elétricas ocorrem, bem como, quais os locais em que os sistemas são mais eficientes na produção de raios, torna-se fundamental para o planejamento da infraestrutura dos países Sul-americanos, no sentido de garantir segurança no transporte aéreo, fluvial e terrestre, nas linhas de transmissão de dados e de energia elétrica, etc, setores estratégicos que quando paralisados devidos aos danos causados pela queda de raios refletem em prejuízos em cascata em todos os setores econômicos. 

Por exemplo, uma falha no sistema de distribuição de energia elétrica pode cessar a energia elétrica de um bairro, cidade, etc. Pode causar queima de equipamentos eletroeletrônicos devido sobre tensão elétrica, causar quedas na rede de internet, o que pode paralisar  setores como: educação, pesquisa, comércio e industrias. Também gera grande número de ações judiciais indenizatórias contra as operadoras do sistema de energia, sobrecarregando o sistema judiciário. No Brasil, as empresas prestadoras de serviços de  fornecimento de energia elétrica ao consumidor lideram as reclamações nos PROCONs ao lado de empresas de telecomunicações, evidenciando a falta de infraestrutura destes setores. Em \citeonline{pinto2005arte,noticiainpe2007}, estima-se prejuízos na ordem de 1 bilhão de dólares anuais em função da densidade de raios observada apenas sobre o Brasil. 
  
%e melhor lidar com problemas como enchentes rápidas, chuvas 
%de granizo, tornados e visar uma melhor forma de gerir recursos naturais.
%Williams E. R. e Sátori G., (2004) buscaram entender a maior resposta da Curva de Carnegie associada a atividade de tempestades na América do Sul fazendo um estudo comparativo entre as regiões da bacia Amazônica e bacia do Congo. Sobre a maior bacia hidrográfica do Continente Africano, as taxa de raios por km$^2$ por ano são maiores enquanto que os sistemas precipitantes sobre a bacia Amazônica, observa-se menor densidade de raios porém maior volume de chuva, indicando que as formações estratiformes no continente sul-americano também funcionam como baterias do Circuito Elétrico Global, em que carga negativa é transferida para a Terra por meio das gotas de chuva carregadas (SOULA et al., 2003).


No entanto, é possível presenciarmos situações de chuva torrencial porém sem que haja raios, pois a capacidade de gerar raios em uma tempestade elétrica não depende apenas dos processos de nucleação e colisão coalescência, depende de como ocorrerá o crescimento de cristais de gelo na nuvem, exigindo condições atmosféricas em que a água possa coexistir na fase líquida, gasosa e sólida (região mista), que são possíveis em regiões atmosféricas com temperaturas entre 0 $^{\circ}$C e -40 $^{\circ}$C \cite{Takahashi1978,williams1991mixed,korolev2007}.

Desta forma, o processo de eletrificação dos hidrometeoros até a formação de um raio, depende da capacidade do ar quente e úmido da superfície romper a estabilidade atmosférica e atingir altitude entre 4--15 km, regiões acima da isoterma de 0 $^{\circ}$C. Portanto, as tempestades elétricas podem indicar condições dinâmicas e termodinâmicas associadas a convecção profunda na atmosfera \cite{doswell2001,zipser2006}.

Além dos raios que atingem o solo -- raios nuvem-solo -- causando danos a sociedade, trabalhos como em \citeonline{macgorman1989,carey1998,williams1999} mostram que condições de tempo severo como: frentes de rajadas com velocidade superior a 92.6 km h$^{-1}$, queda de granizo com diâmetro maior do que 1.9 cm ou tornados, são geralmente precedidas de um salto na taxa de raios das tempestades elétricas governado por raios que não atingem o solo -- raios intra-nuvens -- indicando intenso crescimento de hidrometeoros na região mista.      

Neste sentido, técnicas de seleção de sistemas meteorológicos a partir de dados de sensoriamento remoto, combinadas com o monitoramento da taxa de raios totais\footnote{O termo raios totais faz referencia a todos os tipos de raios, tanto os raios intranuvens quanto os nuvem-solo} dos sistemas são de grande importância na identificação e previsão de curto prazo de tempo severo. 

Diversos estudos definiram sistemas meteorológicos fazendo o agrupamento de regiões na superfície a partir de limiares de temperatura de brilho observadas em satélite. \citeonline{mapes1993}, utilizaram esta metodologia e também fazem uma síntese de trabalhos que buscaram selecionar clusters de nuvens a partir de limiares de temperatura de brilho em infravermelho, como por exemplo \citeonline{Maddox1980} que observou a ocorrência de duas regiões: uma com temperatura de brilho $\leqslant$ -32°C (241K) e área $\geqslant$ 100,000 km$^2$;  outra região menor, no interior da região maior, com temperatura de brilho $\leqslant$ -52°C (221K) e área $\geqslant$ 50,000 km$^2$, associadas com Sistemas Convectivos de Meso-escala (SCM) nos Estados Unidos.

\citeonline{morales2003} desenvolveram um algoritmo hidro-estimador estudando regiões de temperatura de brilho em infravermelho e a precipitação observada pelo radar a bordo do satélite \textit{Tropical Rainfall Measuring Mission} (TRMM). Dados da \textit{Sferics Timing and Ranging Network} (STARNET) foram utilizados e clusters com raios e sem raios foram identificados. Foi observado que as descargas localizadas pela STARNET, em 90\% dos casos, estiveram associados a regiões com temperatura de brilho menores do que 258 K.

\sigla{name={TRMM},description={\textit{Tropical Rainfall Measuring Mission}}}
\sigla{name={STARNET},description={\textit{Sferics Timing and Ranging Network}}}

%Porém a radiação infravermelha observada por satélites, corresponde apenas a irradiação do topo das nuvens. Nuvens finas, com formação acima da isoterma de 0°C, como por exemplo as nuvem cirrus, podem cobrir grandes extensões e não estar associadas a precipitação nem descargas elétricas.


Em \citeonline{houze1993} Linhas de Instabilidades (LI) foram definidas observando extensões com chuva contínua observada por radar. Em \citeonline{MohrZipser1996} Sistemas Convectivos de Meso-escala (SCM) sobre os trópicos foram observados a partir do espalhamento radiativo em micro-ondas (85-GHz PCT), em que regiões contínuas $\geqslant$ 2000 km$^2$ com PCT $\leqslant$ 250 K foram os principais critérios para a identificação dos sistemas.

\sigla{name={SCM},description={\textit{Sistema Convectivo de Meso-escala}}}
\sigla{name={LI},description={\textit{Linha de Instabilidade}}}

Combinando dados do PR e TMI abordo do TRMM, \citeonline{Nesbitt2000} desenvolveu uma metodologia para selecionar sistemas precipitantes denominados como \textit{Precipitation Features} (PF), em que o principal critério de seleção foi identificar área contínua de chuva na superfície, seja estimada pelas observações de radar ou micro-ondas quando os sistemas estiveram fora da varredura do PR.

\sigla{name={PF},description={\textit{Precipitation Features}}}


\citeonline{cecil2005} mostram que apenas 2.4\% das PFs observadas pelo TRMM em todo globo entre 12/1997--10/2000 possuíram atividade elétrica. Na AS as PFs classificadas com as maiores taxas de raios, concentraram-se na região da Bacia do Prata associadas a Sistemas Convectivos de Meso-escala \cite{Velasco1987,Durkee2009}.   


Com base também nos dados das PFs,  \citeonline{zipser2006} identificou os locais das tempestades mais severas em um estudo global. Apenas 0.1\% da amostragem das PFs em um período de 7 anos foram observadas taxa de raios acima de 32.9 por minuto.  Entre as regiões do globo com as maiores concentrações de PFs que indicaram valores extremos (0.001\%), seja de taxa de raios, seja de mínimas temperaturas de brilho (85 e 37 GHz) ou de máxima altitude com 40 dBZ de fator de refletividade do radar, encontra-se a região Sul da AS que engloba a Bacia do Prata e o extremo Norte da Cordilheira dos Andes que abrange a Colômbia e região do Lago Maracaibo na Venezuela. 
%Conforme aumenta a intensidade das tempestades elétricas, menor será sua probabilidade de ocorrência. Apenas 0.1\% das PFs observou-se taxa de raios acima de 32.9 por minuto. 

Nesta tese, faz-se a identificação de sistemas que possuem atividade elétrica, ou seja, tempestades elétricas, apenas sobre a América do Sul  a partir das observações orbitais do satélite TRMM, mais especificamente do sensor de raios (LIS), radiômetro no infravermelho (VIRS) e o radar de precipitação (PR) a bordo do da satélite entre os anos de 1998 e 2011. Desta forma cria-se um banco de dados de tempestades elétricas do TRMM.

Com este subconjunto de dados do TRMM, é estudada a sazonalidade, o ciclo diurno e ciclo anual das tempestades elétricas, bem como a densidade geográfica de raios e de tempestades elétricas, buscando evidenciar regiões ou estações do ano em que as tempestades elétricas possuem processo de eletrificação mais eficientes sobre a América do Sul.

A intensidade das tempestades elétricas é estudada com base na taxa de raios e aspectos morfológicos como: dimensões relacionadas a sua extensão e a estrutura tridimensional da precipitação observada pelo PR, buscando identificar qual é a taxa de raios que corresponde potencialmente a condições de convecção profunda e consequentemente de tempo severo, conforme cada localidade da extensa região da AS.

A estrutura tridimensional da precipitação observada pelo PR é estuda com base na probabilidade de ocorrência por altitude, conforme descreve \cite{yuter1995}, e também conforme a probabilidade de ocorrência por níveis de temperatura do perfil atmosférica. As seções \ref{introRadar} e \ref{derretimento} fazem uma revisão de conceitos fundamentais que devem ser considerados na interpretação dos diagramas de probabilidade elaborados associados as observações por radar.
 
%------------------------------------------------------\\
%\textit{Acho que essa parte de baixo é Metodologia... ou discussão dos resultados}

%A grande extensão territorial do Brasil na America do Sul  
%m base nos experimentos do LBA, investigar a morfologia dos sistemas na pré monsão, na região amazônica. 

%As descargas na região amazônica parecem estar mais associadas com os processos de inibição do que precipitação... Em uma região tropical dominada por processos quentes, as descargas podem indicar inibição de colisão coalescência, desenvolvimento de fase fria e menos precipitação.

%mas no período úmido de leste raios relacionam-se mais com a precipitação.

%...A Rachel já fez uma boa discussão sobre a microfísica dos sistemas da amazônia, períodos seco úmido e de transição. As tempestades foram organizadas em clusters, estudou-se o ciclo de vida, a ocorrência de raios em áreas desmatadas e com floresta/outras, e foi explicado a microfísica dos sistemas basicamente com: taxa de raios positivos e negativos, eco tops, VIL, CAPE, CINE. Falta explotar a os CFADS para essa região. Como varia a probabilidade de ocorrência por altitude dos perfis de refletividade nos períodos seco de transição e úmido.

%Em desenvolvimento ...
%As tempestades mais eficientes estão mostrando tamanhos diversos. Tanto cluster grandes quanto pequenos podem ser eficientes. Não é uma relação que depende apenas da área, ou da fração convectiva ou estratiforme. Os raios relacionam-se com o ciclo de vida, que no caso é aleatório. Depende do ciclo de vida

%Relação exponencial entre max fl no pixel versus fl-rate/km2. Maior a concentração de raios em um único pixel, menos eficiente e mais chuva. Talvez uma reintensificação do sistema maduro. Mas existem duas categorias de sistemas :
%1 – os mais eficiente com área ~10^3 e chuva ~10^5
%2 – as com maior fl/rate no pixel, menos eficientes e com chuva ~10^9



%Com o experimento de campo LBA (Large-Scale Biosphere-Atmosphere Experiment in Amazonia) realizado na região de Rondônia entre janeiro e fevereiro de 1999, foi possível identificar alguns fatores importantes que regulam a precipitação na região Amazônica. Além disso o LBA foi importante para validação de dados do satélite TRMM que são amplamente utilizados nesta pesquisa \cite{silva2002lba,williams2002,albrecht2011}.

%Silva Dias M. A. F. et al, (2002), fazem uma síntese dos principais resultados e objetivos do LBA, entre estes destaco os estudos de Anagnostou e Morales, (2002), \citeonline{Carvalho2002} que mostram dois regimes de vento em 700 mb, de Leste e de Oeste, em que observou-se maior precipitação convectiva e atividade elétrica durante o regime de ventos de Leste. Petersen W. A. et al, (2002), investigaram como que esses dois regimes de vento (Leste-Oeste) observados durante o LBA em Rondônia, influenciam no número de descargas elétricas observadas pelo LIS (Ligthning Image Sensor), não apenas para região Amazônica mas para toda a América do Sul durante 4 verões entre 1997 e 2000.

%A variação intra-sazonal da atividade elétrica durante o período chuvoso mostrou-se evidente. \citeonline{petersen2002trmm},  identificaram regiões de extremos opostos de atividade elétrica que devem estar associados ao mecanismos de manutenção da monção na América do Sul, principalmente com a dinâmica que envolve Zona de Convergência do Atlântico Sul (ZCAS) \cite{CarvalhoJones2002,Carvalho2002}.   

\section{FUNDAMENTOS DA OBSERVAÇÃO DA PRECIPITAÇÃO POR RADAR}
\label{introRadar}

A medida da precipitação por radar, consiste na emissão de um feixe eletromagnético ($P_t$) e na análise da potência do eco ($P_r$) gerado pelo sinal emitido.

O sistema eletrônico de um radar envia um feixe eletromagnético periódico com frequência e a largura ($H$) de pulso definidos de modo a avaliar o eco de obstáculos em distâncias distintas. Assim pode-se distinguir o eco do sinal emitido para diferentes elementos de volume ($\Delta V$) do feixe do radar.

Um elemento de volume ($\Delta V$) de um feixe de radar está associado com a resolução radial da medida, que depende de H, pois a largura do pulso (H) é proporcional a distância mínima ($\frac{H}{2}$) para que dois alvos sejam distinguíveis.
 
Considerando um feixe eletromagnético com azimute de $\phi$ na vertical e $\varphi$ na horizontal, pode-se considerar que $\Delta V$ a uma distância $r$ do radar terá o volume na forma do cilindro elíptico

\begin{equation}
\Delta V = \pi \dfrac{r\phi}{2} \dfrac{r\varphi}{2} \dfrac{H}{2} .
\end{equation}
 
A potência recebida $P_r$ do sinal espalhado referente a ao volume iluminado $\Delta V$, pode ser expressa como
\begin{equation}
P_r = \dfrac{P_t G^2 \lambda^2 \phi \varphi H}{ 512 (2\ln2)\pi^2 r^2} \sum_{i=1, 2, 3 ... }^{\Delta V} \sigma_i ,
\label{radar1}
\end{equation}
em que, $G$ é o ganho da antena, $\lambda$ o comprimento de onda, $r$ a distância do alvo, $\sigma$ a seção transversal de retro-espalhamento das partículas de nuvem e o fator $2\ln2$ corresponde ao efeito de lóbulo no sinal eletromagnético captado \cite{battan1973}.

A Refletividade do Radar 
\begin{equation}
\eta = \sum_{i=1, 2, 3 ... }^{V_{m}} \sigma_i,
\label{refletividade}
\end{equation}
é a somatória da seção de retro-espalhamento dos hidrometeoros a cada elemento de volume iluminado $\Delta V$ (a cada $gate$) do radar.

Considerando que o parâmetro de tamanho ($\alpha$), que é a relação entre a área da seção transversal do hidrometeoro ($2\pi R_{h}$) precipitável na atmosfera e o comprimento de onda da radiação emitida pelo radar
\begin{equation}
\alpha = \dfrac{2\pi R_{h} }{\lambda},
\label{parametroTam} 
\end{equation}
corresponda a um valor de $\alpha$ $<<$ 1, então o espalhamento do feixe do radar pode ser considerando como Rayleigh. Neste caso, a seção transversal de retro-espalhamento pode ser escrita como 
\begin{equation}
\sigma = \dfrac{\lambda^2 \alpha^6}{\pi} K^2,
\label{sigma}
\end{equation}
em que $K^2$ corresponde ao índice de refração dos hidrometeoros. Como a equação \ref{parametroTam} depende do raio $R_h$, podemos reescrever a equação \ref{sigma} considerando o diâmetro 
\begin{equation}
D_h = 2R_h.
\label{diametro}
\end{equation}

Então, substituindo as equações \ref{parametroTam} e \ref{diametro} em \ref{sigma}, obtemos que
\begin{equation}
\sigma = \dfrac{\pi^5 K^2  }{ \lambda^4 } D_h^6.
\label{sigma2}
\end{equation}

Considerando o hidrometeoro como esférico, podemos relacionar o $D_h$ com a sua quantidade de massa, sendo  
\begin{equation}
D_h = \left( \dfrac{6 M_h}{\pi \rho} \right)^{\frac{1}{3}},
\label{dh}
\end{equation}
em que, $\rho$ é a densidade e $M_h$ a massa do hidrometeoro.

Então substituindo a equação \ref{dh} em \ref{sigma2}, obtém-se
\begin{equation}
\sigma = \dfrac{36 \pi^3 K^2  }{ \lambda^4 \rho^2 M_h^2} .
\label{sigma3}
\end{equation}

Portanto, observe que a Refletividade do Radar (equação  \ref{refletividade}), depende  de uma relação entre quantidade de massa $M_h$, densidade $\rho$ e também do índice de refração $K^2$ dos hidrometeoros, conforme mostra a equação \ref{sigma3}.

Porém o radar mede apenas a potência do sinal retro-espalhado ($P_r$). Pode-se ter uma ideia da concentração de obstáculos espalhadores, porém não podemos ter certeza a respeito da massa, índice de refração e densidade dos obstáculos. 

O que se faz para estimar a chuva é considerar que todos os obstáculos associados ao espalhamento do feixe do radar são gotas de água esféricas. Neste caso, podemos combinar as equações \ref{radar1} e \ref{sigma3}, obtendo
%\begin{equation}
%P_r = V_{m} \dfrac{P_t G^2 \lambda^2}{2\ln2(4\pi)^3 r^4}   \dfrac{\pi^5 K^2  }{ \lambda^4 } \sum_{i=1, 2, 3 ... }^{V_{m}}  D_{h_i}^6.      
%\end{equation} 
%\begin{equation}
%P_r = \dfrac{P_t G^2 \lambda^2 \phi \varphi H}{ 512 (2\ln2)\pi^2 r^2} \dfrac{\pi^5 K^2  }{ \lambda^4 }  \sum_{i=1, 2, 3 ... }^{\Delta V}  D_{h_i}^6.   
%\end{equation} 
\begin{equation}
P_r = \dfrac{\pi^3 P_t G^2  \phi \varphi H }{ 512 (2\ln2) \lambda^2 } \dfrac{ K^2  }{ r^2 }  \sum_{i=1, 2, 3 ... }^{\Delta V}  D_{h_i}^6, 
\end{equation} 
sendo o Fator de Refletividade do Radar
\begin{equation}
Z =  \sum_{i=1, 2, 3 ... }^{\Delta V}  D_{h_i}^6.
\label{fz}
\end{equation}
Observe que a cada medida de $P_r$, algumas variáveis serão sempre constantes 
\begin{equation}
C = \dfrac{\pi^3 P_t G^2  \phi \varphi H }{ 512 (2\ln2) \lambda^2 } .
\end{equation}
Então,
\begin{equation}
P_r = C \dfrac{K^2}{r^2}  Z.
\end{equation}

Desta maneira, a partir da $P_r$ medida pelo radar, sabendo a distância $r$ do alvo e considerando que a chuva é composta de esferas de água líquida, então podemos saber $Z$, pois 
\begin{equation}
Z = \dfrac{P_r r^2}{C K^2},
\label{zsimples}
\end{equation}
e, $K^2=0.931$ para água líquida.


Conforme mostra a equação \ref{fz}, $Z$ depende de $D_h$, portanto conforme a equação \ref{dh}, podemos relacionar o diâmetro $D_h$ dos hidrometeoros com a massa ou o volume de chuva.  

É conveniente converter o Fator de Refletividade do Radar ($Z$), para uma escala em decibel, portanto os valores de eco de radar geralmente são expressos em unidade de decibel do Fator de Refletividade (dBZ). 

Aplicando $10\log_{10}$ na equação \ref{zsimples}, temos que

\begin{align}
10\log_{10}(Z)  &= 10\log_{10} \left( \dfrac{P_r r^2}{C K^2} \right)\\
dBZ &= 10\left[ \log_{10}(P_r r^2) - \log_{10}(C K^2)     \right]\\
      &=  10 \lbrace \log_{10}(P_r)+ 2\log_{10}(r) - [ \log_{10}(C) + \log_{10}(K^2)]  \rbrace,
\end{align}
portanto,
\begin{equation}
dBZ =  10\log_{10}(P_r) + 20\log_{10}(r) - 10\log_{10}(C) - 10\log_{10}(K^2)
\end{equation}

\section{FUSÃO DO GELO DE NUVEM E O CRESCIMENTO DOS HIDROMETEOROS}
\label{derretimento}

Ao observar a precipitação por radar no perfil atmosférico, a varredura do feixe irá atingir regiões de altitude com temperaturas abaixo de 0 $^{\circ}$C. Nestes casos, a potência $P_r$ estará associada ao espalhamento em gelo de nuvem. Para quantificar o gelo precipitando, podemos considerar que 
\begin{equation}
dBZ_{\mathrm{gelo}} =  10\log_{10}(P_r) + 20\log_{10}(r) - 10\log_{10}(C) - 10\log_{10}(K_{\mathrm{gelo}}^2).
\label{zg}
\end{equation}

Portanto quando o feixe espalha em altitudes abaixo da isoterma de 0 $^{\circ}$C, pressupõe-se que $P_r$ estará associada a quantidade de água líquida. Então, para quantificar a água precipitando, podemos considerar que
\begin{equation}
dBZ_{\mathrm{agua}} =  10\log_{10}(P_r) + 20\log_{10}(r) - 10\log_{10}(C) - 10\log_{10}(K_{\mathrm{agua}}^2).
\label{za}
\end{equation}

Subtraindo as equações \ref{zg} e \ref{za}, temos
\begin{equation}
dBZ_{\mathrm{gelo}} - dBZ_{\mathrm{agua}} = - 10\log_{10}(K_{\mathrm{gelo}}^2 )+ 10\log_{10}(K_{\mathrm{agua}}^2).
\end{equation}

Sabendo que $K_{\mathrm{agua}}^2 = 0.931$ e $K_{\mathrm{gelo}}^2= 0.197$, então 
\begin{equation}
dBZ_{\mathrm{gelo}} - dBZ_{\mathrm{agua}} =   6.7dBZ,
\end{equation}
mostrando que devido ao índice de refração do gelo ser menor do que o índice de refração da água ($K_{\mathrm{agua}}^2 > K_{\mathrm{gelo}}^2$), ao considerar $K_{\mathrm{gelo}}^2$, conforme mostra a equação \ref{zg}, haverá um acréscimo de potência de  6.7 dBZ em relação a considerar $K_{\mathrm{agua}}^2$, como mostra a equação \ref{za}.

Porém, na observação tridimensional da precipitação, podemos conhecer as distâncias $r$ dos alvos espalhadores, mas não podemos afirmar sobre a temperatura da atmosfera para cada distância $r$ do radar, bem como se haverá água super-resfriada acima de 0 $^{\circ}$C ou gelo sólido caindo na superfície. Então os dados brutos das observações de radar, consideram $K^2$ como constante, geralmente $K_{\mathrm{agua}}^2 = 0.931$  para toda a estrutura tridimensional da precipitação observada.

Portanto em observações de radar no perfil de altitude, a região ou camada de derretimento (fusão do gelo) é bastante marcada, pois, identifica-se um aumento de $\simeq$7 dBZ no Fator de Refletividade do Radar devido a mudança do índice de refração.

Em \citeonline{Fabry1995}, é mostrado que processos como a agregação, acreção e colisão coalescência, podem ser estudados em função da espessura da camada de derretimento e flutuações nos valores do Fator de Refletividade $Z$ no perfil atmosférico. 

A espessura da camada de derretimento está relacionada com o lapse-rate da atmosfera \cite[p.~462]{mason1971_2ed}. Em uma atmosfera instável, com convecção profunda e precipitação convectiva, a camada de transição de fase de gelo para a água liquida é perturbada por correntes ascendentes. A mudança do índice de refração da água não ocorre apenas logo abaixo de 0 $^{\circ}$C, pois no ambiente convectivo teremos água super-resfriada em temperaturas de -15 $^{\circ}$C, o que intensifica o processo de acreção podendo gerar gelo sólido que cai derretendo até a superfície. Nestes casos espera-se uma camada de derretimento mais espessa.

Considerando um regime de precipitação estratiforme, que é governado por processos de agregação, será observado um aumento acentuado no fator de refletividade do radar logo abaixo da isoterma de 0 $^{\circ}$C associado ao derretimento de flocos de neve, que denomina-se banda brilhante. Neste caso espera-se uma camada de derretimento menos espessa, pois os flocos de neve possuem velocidade terminal e densidade inferior as partículas de gelo compacto (ganizo ou saraiva/$graupel$) portanto derretem mais rapidamente, ou seja, percorrem um caminho menor durante o derretimento. 

\simbolo{name={$^{\circ}$C},description={Grau Celcius}} 

%\begin{xalignat}{3}
%\mathbf{n} \cdot \mathbf{E} = 0 && &e  && \mathbf{n} \cdot \mathbf{B} = 0.
%\end{xalignat}

%\begin{equation}
%K^2 = \left( \dfrac{m^2-1}{m^2+2}\right)^2
%\end{equation}

Também, sabendo que $Z$ é proporcional ao diâmetro dos hidrometeoros $D_h$ elevado a 6 potência, como mostra a equação \ref{fz}, os processos de crescimento de flocos de neves, granizo e gotas, são marcados por aumentos exponenciais no Fator de Refletividade do Radar ($Z$) no perfil de altitude. 

%Considerando um regime de precipitação estratiforme, que é governado por processos de agregação, será observado um aumento acentuado no fator de refletividade do radar em logo abaixo da isoterma de 0 $^{\circ}$C associado ao derretimento de flocos de neve. \simbolo{name={$^{\circ}$C},description={Grau Celcius}} 

Acima da região de derretimento, um aumento abruto nos valores de $Z$ podem indicar processo de crescimento de cristais de gelo e de gelo compacto. Enquanto que abaixo da região de derretimento, os acréscimos nos valores de $Z$ podem indicar processos de colisão coalescência e os decréscimos de $Z$, devido a evaporação e rompimento/quebra das gotas. 


% durante o caminho que a precipitação percorre até a superfície ou temperaturas acima de 0°C.
%Na figura \ref{fabry}, \citeonline{Fabry1995}
%e o trabalho de 
%\begin{figure}[hbp]
%  \centering{
%  \subfloat[\cite{Fabry1995}]{{\includegraphics[scale=0.25]{img/ilustracoes/fabry}} \label{fabry}}
%  \subfloat[\cite{Takahashi2002}]{{\includegraphics[scale=0.35]{img/ilustracoes/takahashi}} \label{taka}}
%  }
%\caption{Fabry Taka}
%\label{fabyTaka} 
%\end{figure} 

%Consequentemente, a taxa de raios associa-se com a intensidade convectiva devido a acreção\footnote{A acreção é o processo de \textit{rimming} descrito no trabalho de \citeonline{Takahashi1978}.} ser o processo mais eficiente de eletrificação de nuvens, principalmente quando há presença de flocos de neve embebidos na região de fase mista \cite{Takahashi1978,Takahashi2002}. 


\section{OBJETIVOS... PROPOSTA...}
\begin{itemize}
\item Criar um banco de dados de tempestades elétricas do TRMM. 
\item Criar mapas que identifique a densidade de tempestades elétricas e de raios sobre a América do Sul.
\item Descrever o ciclo diurno e o ciclo anual das tempestades elétricas do TRMM.
\item Classificar a intensidade das tempestades elétricas com base na taxa de raios e no estudo da frequência de ocorrência do Fator de Refletividade do radar por temperatura e por altura. 
\end{itemize}
...