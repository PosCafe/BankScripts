\chapter{TEMPESTADES ELÉTRICAS SEVERAS}

Conforme descrito em \ref{metodoFtaFt}, as taxas de raios das tempestades elétricas neste trabalho de pesquisa, estão associadas aos índices FTA e FT.  Nesta seção identifica-se qual desses índices podem melhor associar-se com a intensidade convectiva das tempestades elétricas.%, ou seja, os sistemas com as maiores taxas de raios por minuto (FT) ou os sistemas com as maiores taxas de raios por minuto por quilômetro quadrado (FTA) da sua extensão. 

Para o estudo de intensidade dos sistemas, foram selecionados apenas as tempestades elétricas as quais possuíram $VT_m$ maior ou igual a 1 minuto e com pelo menos um pixel do campo de visão do PR contido na área do sistema, totalizando 94,733 tempestades elétricas do TRMM. %Como a intensidade convectiva dos sistemas é avaliada com base, principalmente, na morfologia da estrutura tridimensional da precipitação e na taxa de raios, 

As equações \ref{eqFT} e \ref{eqFTA} foram aplicadas nas 94,733 tempestades elétricas selecionadas, e então estudada as distribuições de probabilidades dos índices FTA e FT (figuras \ref{pdfFTAFT} e \ref{cdfFTAFT}). Conforme mostra a figura \ref{pdfFTAFT}, trata-se de distribuições exponenciais de probabilidade. Os valores de FTA e FT para cada quantil da amostragem de tempestades elétricas, tanto para as mais frequentes quanto as mais raras, podem ser verificados por meio da distribuição cumulativa de probabilidade de FTA e FT mostradas na figura \ref{cdfFTAFT}. 

\begin{figure}[!ht]
  \centering
  \includegraphics[height=9cm]{img/FtaFt/pdf_FTA_FT}      
  \caption{Densidade de probabilidade de FTA e FT.} 
   \label{pdfFTAFT} 
\end{figure}

\begin{figure}[!ht]
  \centering 
  \includegraphics[height=9cm]{img/FtaFt/cdf_FTA_FT} 
  \caption{Densidade de probabilidade cumulativa de FTA e FT.}
  \label{cdfFTAFT}
\end{figure}
%\label{seriesFtaFt}

Os sistemas potencialmente severos são selecionados pelo 90\textsuperscript{\underline{o}} percentil das amostras de probabilidades de FTA e FT, que estão associado aos maiores e mais raros valores ocorridos. Portanto, pressupõe-se que as tempestades elétricas as quais provavelmente causaram chuva de granizo, rajadas de ventos com queda de árvores e construções ou tornados estão associadas aos sistemas com valores extremos de FTA ou FT.

O grupo das tempestades elétricas com FTA extremo, possuíram valores entre {29.3--1258.7 $\times$ 10$^{-4}$} raios por minuto por quilômetro quadrado (minuto$^{-1}$ km$^{-2}$), enquanto as tempestades elétricas com extremos de FT possuíram valores entre {47.2--1283.6} raios por minuto (minuto$^{-1}$). 


% Na figura \ref{percetilFtaFt}, temos a série de FTA e FT ordenada, e a linha tracejada vertical corta o 90\% percentil dos índices. 

%\begin{figure}[!ht]
%  \centering
%  \includegraphics[height=6cm]{img/FtaFt/90thFtaFt}	 
%  \caption{90\textsuperscript{\underline{o}} percentil de FTA e FT.}
%  \label{percetilFtaFt}
%\end{figure}


\section{ÁREA E TEMPERATURA DO TOPO DA NUVEM}

Observa-se que os extremos de FTA e FT correspondem a sistemas com tamanhos bem distintos. Conforme é mostrado na figura \ref{size}, verifica-se que as máximas probabilidades de ocorrência de tempestades elétricas associadas com os extremos de FTA, ocorrem em sistemas com área 3 ordens de grandeza menor do que nos extremos de FT.

\begin{figure}[!ht]
  \centering
  \includegraphics[height=9cm,trim=0 0 215 0,clip]{img/tb/TbAreas}   
  \caption{Densidade de probabilidade de extensão em área das tempestades elétricas com extremos de FTA e FT.}
  \label{size}  
\end{figure}

\begin{figure}[!hb]
  \centering 
  \includegraphics[height=9cm,trim=220 0 0 0,clip]{img/tb/TbAreas}
  \caption{Densidade de probabilidade de temperatura de brilho em infravermelho do topo das nuvens das tempestades elétricas com extremos de FTA e FT.}
  \label{tb}
\end{figure}

%\begin{figure}[!hb]
%  \centering{  
%  \subfloat[Densidade de probabilidade de extensão em área.] { \includegraphics[height=7.5cm,trim=0 0 215 0,clip]{img/tb/TbAreas} \label{size}} \\
%  \subfloat[Densidade de probabilidade de temperatura de brilho em infravermelho.]{ \includegraphics[height=7.5cm,trim=220 0 0 0,clip]{img/tb/TbAreas} \label{tb}} 
%  }
%  \label{t_tb}
%  \caption{Estudo das frequências de ocorrências de tempestades elétricas selecionas pelo 90\textsuperscript{\underline{o}} percentil dos índices de FT e FTA, por extensão em área e por temperatura de brilho de topo das nuvens.}
%\end{figure}

%As tempestades elétricas com valores extremos de FT são maiores em extensão.
Na figura \ref{areaFTAFTA}, pode-se observar que os sistemas com tamanho entre 10$^2$--10$^3$ km$^2$, não ultrapassam 20 raios por minuto (FT). As tempestades elétricas com FT superior a 100 raios por minuto, possuíram tamanho entre 10$^{4}$--10$^{6}$ km$^2$. Note que há um aumento exponencial de FT conforme aumenta a extensão das tempestades elétricas. No entanto FTA diminui exponencialmente com o aumento da área das tempestades elétricas.

%Uma tempestade elétrica com 10$^5$ km$^2$, terá maior número de descargas observadas durante o tempo de visada do LIS do que uma com 10$^2$ km$^2$.

\begin{figure}[!ht]
  \centering
  \includegraphics[height=9cm]{img/FtaFt/area_FTA_FT}   
  \caption{Dispersão entre a áreas das tempestades elétricas e os valores de FTA e FT. As linhas horizontais marcam os valores de FTA (cor preta) e FT (cor azul) referente ao 90\textsuperscript{\underline{o}} percentil das respectivas amostragem.}
  \label{areaFTAFTA}  
\end{figure}

\begin{figure}[!hb]
  \centering 
  \includegraphics[height=9cm]{img/FtaFt/volChuva_FTA_FT}
  \caption{Dispersão entre o volume de chuva das tempestades elétricas e os valores de FTA e FT.  As linhas horizontais marcam os valores de FTA (cor preta) e FT (cor azul) referente ao 90\textsuperscript{\underline{o}} percentil das respectivas amostragem.}
  \label{volchuvaFTAFT}
\end{figure}

%\begin{figure}[!hb]
%  \centering 
%  \subfloat[areas versus taxa de raios]{ \includegraphics[height=7.5cm]{img/FtaFt/area_FTA_FT}\label{areaFTAFTA}} \\
%  \subfloat[volume de chuva]{\includegraphics[height=7.5cm]{img/FtaFt/volChuva_FTA_FT}\label{pdfFt}} 
%  \caption{Dispersão referente aos índices FTA e FT.}
%  \label{areaFtaFt}
%\end{figure} 
%Quando a densidade espacial de descargas aumenta muito em uma região com centenas de quilômetros quadrados, em torno de 100 descargas intranuvens para uma nuven-solo, como por exemplo os maiores valores de $Z=IC/CG$ mostrados por \cite{evandro2009} na região de Campo Grande - MS no Brasil, a capacidade do LIS de identificar brilhos transientes provavelmente fica comprometida devido a resolução horizontal da CCD.

Ao normalizar a taxa de raios no tempo por $A_t$, o número de raios fica diluído na extensão do sistema, evidenciando que os maiores valores de FTA correspondem aos sistemas com as maiores densidades espaciais de raios, cuja a área e o número de raios são menores do que nos sistemas com extremos de FT.

Na figura \ref{volchuvaFTAFT}, observa-se que conforme aumenta FT o volume de chuva das tempestades elétricas também aumenta exponencialmente, de maneira semelhante ao aumento de FT com a área (figura \ref{areaFTAFTA}). Para FTA, há um comportamento inverso. Conforme aumenta FTA, o volume de chuva dos sistemas diminui.  

A frequência de ocorrência das temperaturas de brilho associadas a radiância espectral observada no canal 4 do VIRS para todos os pixeis que definiram as áreas dos sistemas, é mostrada na figura \ref{tb}. Observa-se que o maior valor de probabilidade para a curva das tempestades elétricas com índice extremo de FTA, possui temperatura de topo de nuvens aproximadamente 10 K mais frias do que nas tempestades elétricas com extremos de FT, indicando que a convecção nos sistemas ordenados por FTA é mais profunda na maioria das situações.

\citeonline{morales2003} ao desenvolver a \textit{Sferics Infrared Rainfall Technique} (SIRT), mostram que as regiões com temperatura de brilho inferior a 215 K e com ocorrência de \textit{sferics} foram as regiões categorizadas como de maior precipitação associada.\sigla{name={SIRT},description={\textit{Sferics Infrared Rainfall Technique} }}

Neste trabalho de pesquisa, ao selecionar as tempestades elétricas com índice extremo de FTA, os maiores valores de probabilidade de ocorrência, conforme é mostrado na figura \ref{tb}, concentram-se em temperaturas de brilho abaixo de 215 K.

Os sistemas selecionados pelo 90\textsuperscript{\underline{o}} percentil do índice FT possuem maior extensão em área e maior volume de chuva. São sistemas com vasta extensão estratiforme conforme descrevem \citeonline{Rasmussen2011}. As regiões das tempestades elétricas com precipitação convectiva, as quais tem potencial de gerar chuva de granizo, frentes de rajada e tornados, ocupam área bem menor do que as áreas com  precipitação estratiforme \cite{Jr2007}.

Nas figuras \ref{pdffracaoFTA}, \ref{cdffracaoFTA} e \ref{pdffracaoFTA},  \ref{cdffracaoFTA}, apresenta-se o estudo das probabilidades das frações de chuva para as tempestades elétricas com valores extremos de FTA e FT. As curvas denominadas como convectivo, estratiforme e outros, correspondem a fração de perfis do PR classificados como convectivo, estratiforme e outros em relação a toda a área de chuva. A curva denominada na legenda como chuva total corresponde a fração da área de chuva em relação a área total da tempestade elétrica. A curva denominada como varredura do PR, mostra a fração da área da tempestade elétrica que esteve dentro da varredura do PR.

Avaliando a densidade de probabilidade da fração convectiva e fração estratiforme das tempestades elétricas, figuras \ref{pdffracaoFTA} e \ref{pdffracaoFT}, verifica-se que os extremos de FTA associam-se com tempestades elétricas com 70\% de área convectiva e 30\% de área estratiforme, enquanto que para os extremos de FT possuíram 20\% de fração convectiva e 75\% de fração estratiforme.
%juntamente com a as respectivas distribuições de probabilidade acumulativa,

Como os sistemas extremos de FT possuem maior área do que os extremos de FTA, verifica-se nas figuras \ref{pdffracaoFT} e \ref{cdffracaoFT}, que o PR observou com maior frequência apenas uma porção de 25\% da área das tempestades elétricas com extremo de FT. Quando observa-se as figuras \ref{pdffracaoFTA} e \ref{cdffracaoFTA}, o PR observou com maior frequência 100\% da área das tempestades elétricas com extremo de FTA. Portanto a fração de chuva total dos extremos de FT (figura \ref{pdffracaoFT}) possui um valor de apenas $\simeq$10\% devido a varredura do PR ser menor do que a do VIRS e as tempestades elétricas abrangerem uma extensão que ultrapassa o alcance do PR. 

\begin{figure}[!ht]
  \centering
  \includegraphics[height=9cm]{img/FtaFt/fracaoChuva_pdf_topFTA}   
  \caption{Dispersão entre a áreas das tempestades elétricas e os valores de FTA e FT. As linhas horizontais marcam os valores de FTA (cor preta) e FT (cor azul) referente ao 90\textsuperscript{\underline{o}} percentil das respectivas amostragem.}
  \label{pdffracaoFTA}  
\end{figure}

\begin{figure}[!hb]
  \centering 
  \includegraphics[height=9cm]{img/FtaFt/fracaoChuva_cdf_topFTA}
  \caption{Dispersão entre o volume de chuva das tempestades elétricas e os valores de FTA e FT.  As linhas horizontais marcam os valores de FTA (cor preta) e FT (cor azul) referente ao 90\textsuperscript{\underline{o}} percentil das respectivas amostragem.}
  \label{cdffracaoFTA}
\end{figure}


\begin{figure}[!ht]
  \centering
  \includegraphics[height=9cm]{img/FtaFt/fracaoChuva_pdf_topFT}   
  \caption{Dispersão entre a áreas das tempestades elétricas e os valores de FTA e FT. As linhas horizontais marcam os valores de FTA (cor preta) e FT (cor azul) referente ao 90\textsuperscript{\underline{o}} percentil das respectivas amostragem.}
  \label{pdffracaoFT}  
\end{figure}

\begin{figure}[!hb]
  \centering 
  \includegraphics[height=9cm]{img/FtaFt/fracaoChuva_cdf_topFT}
  \caption{Dispersão entre o volume de chuva das tempestades elétricas e os valores de FTA e FT.  As linhas horizontais marcam os valores de FTA (cor preta) e FT (cor azul) referente ao 90\textsuperscript{\underline{o}} percentil das respectivas amostragem.}
  \label{cdffracaoFT}
\end{figure}









Talvez alguns sistemas com extremos valores de FTA estejam em estágio de maturação e conforme vão se dissipando vão ganhando área de chuva estratiforme e se enquadrando no grupo dos maiores índice de FT. 

%.....
%Para avaliar qual dos índices representaram a maior severidade de tempo, a morfologia da estrutura 3D da precipitação foi estudada por meio dos diagramas CFAD, CCFAD, CFTD e CCFTD. %para os 10\% das amostras de FT e FTA com os maiores valores.
%......

\section{SEVERIDADE COM BASE NA ESTRUTURA 3D DA PRECIPITAÇÃO}

Nesta etapa iremos avaliar a intensidade convectiva com base nos perfis de $Z_c$ do PR, contidos nos sistemas com índices extremos de FTA e FT. Como os sistemas com extremos de FT possuem área na ordem de 10$^5$ km$^2$, o PR observou com maior frequência apenas  30\% da área total destas tempestades elétricas. Pois, geralmente a varredura do PR não contempla toda a sua extensão. Para os sistemas escolhidos pelos extremos de FTA o PR teve maior probabilidade de observar entre 90-100\% da área dos sistemas.

Nas figuras \ref{ftacfadwith}, \ref{ftacfadwithout}, \ref{ftcfadwith} e \ref{ftcfadwithout} foram calculados os CFADs para as tempestades elétricas com índices FTA e FT extremos, distribuídas conforme cada região de 10 por 10 graus na superfície terrestre. Para localizar a caixa de 10 por 10 graus em que cada sistema esteve contido, foi considerado a latitude e longitude do centro geométrico da área definida por cada sistema.

%----------------------------------------
\begin{sidewaysfigure}%[!H]
\centering
\includegraphics[width=19.5cm]{img/precipitacao3d/severo/percentil/90th/cfad10_semraio_topFTA_percentil}
\caption{CFADs para os extremos de FTA. Porção da precipitação sem raios.}
\label{ftacfadwithout}
\end{sidewaysfigure} 


\begin{sidewaysfigure}%[!H]
\centering
\includegraphics[width=19.5cm]{img/precipitacao3d/severo/percentil/90th/cfad10_semraio_topFT_percentil}
\caption{CFADs para os extremos de FT	. Porção da precipitação sem raios.}
\label{ftcfadwithout}
\end{sidewaysfigure} 
%----------------------------------------

%\begin{figure}[!ht]
%  \centering
%  \includegraphics[height=13.5cm]{img/precipitacao3d/severo/percentil/90th/cfad10_semraio_topFTA_percentil}
% \caption{CFADs para os extremos de FTA. Porção da precipitação sem raios.}
% \label{ftacfadwithout}
%\end{figure} 

\begin{sidewaysfigure}%[!H]
  \centering
  \includegraphics[width=19.5cm]{img/precipitacao3d/severo/percentil/90th/cfad10_comraio_topFTA_percentil}
  \caption{CFADs para os extremos de FTA. Porção da precipitação com raios.}
  \label{ftacfadwith}   
\end{sidewaysfigure} 


\begin{sidewaysfigure}%[!H]
  \centering
  \includegraphics[width=19.5cm]{img/precipitacao3d/severo/percentil/90th/cfad10_comraio_topFT_percentil}
  \caption{CFADs para os extremos de FT. Porção da precipitação com raios.}
  \label{ftcfadwith}   
\end{sidewaysfigure} 

%\begin{figure}[!ht]
%  \centering
%  \includegraphics[height=13.5cm]{img/precipitacao3d/severo/percentil/90th/cfad10_comraio_topFTA_percentil}
%  \caption{CFADs para os extremos de FTA. Porção da precipitação com raios.}
%  \label{ftacfadwith}   
%\end{figure} 

As posições geográficas dos eventos do LIS e dos perfis de $Z_c$ válidos do PR, foram projetadas em uma grade regular de 0.05 graus. Os perfis de $Z_c$ projetados em pontos de grade em que tiveram eventos do LIS, definiram as regiões aqui denominadas como precipitação dos núcleos de raios. Os CFADs foram calculados para a porção da chuva com, figuras \ref{ftacfadwith} e \ref{ftcfadwith}, e sem, figuras \ref{ftacfadwithout} e \ref{ftcfadwithout}, atividade elétrica de nuvem.

Note que no canto superior direito de cada CFAD temos alguns valores estatísticos que representam: (\%) \simbolo{name={\%},description={Nos diagramas, CFAD, CCFAD, CFTD e CCFTD, representam: a porcentagem de perfis convectivos, estratiformes e outros, respectivamente}} a porcentagem de perfis convectivos, estratiformes e outros, respectivamente; (P) \simbolo{name={P},description={Nos diagramas, CFAD, CCFAD, CFTD e CCFTD, representam: número de perfis do PR computados}} o número de perfis do PR computados; (L) \simbolo{name={L},description={Nos diagramas, CFAD, CCFAD, CFTD e CCFTD, representam: o número de ocorrência de $Z_c$ no nível de altitude de máxima ocorrência}} o número de ocorrência de $Z_c$ no nível de altitude de máxima ocorrência; (H) \simbolo{name={H},description={Nos diagramas, CFAD, CCFAD, CFTD e CCFTD, representam: o nível de altitude, em quilômetros, aonde ocorreu o máximo de ocorrências de $Z_c$}} o nível de altitude, em quilômetros, aonde ocorreu o máximo de ocorrências de $Z_c$.

Comparando os CFADs da chuva com e sem raios, representados para os extremos de FTA nas figuras \ref{ftacfadwithout} e \ref{ftacfadwith} e para os extremos de FT, nas figuras \ref{ftacfadwithout} e \ref{ftcfadwithout}, é evidente que a porção sem raios é a parte menos severa dos sistemas. Os níveis de contorno de probabilidades dos CFADs da precipitação sem raios possuem suas máximas altitudes aproximadamente 3 quilômetros abaixo das máximas altitudes atingidas pelos contornos dos CFADs da precipitação com raios. A porção sem raios dos sistemas possuíram maior percentual de perfis estratiformes e menores valores de $Z_c$ com os contornos de probabilidades entre 1-10\%, em todos os níveis de altitude.



\begin{figure}[!ht]
  \centering
  \includegraphics[height=13.5cm]{img/precipitacao3d/severo/percentil/90th/cfad10_semraio_topFT_percentil}
 \caption{CFADs para os extremos de FT. Porção da precipitação sem raios.}
 \label{ftcfadwithout}
\end{figure} 

\begin{figure}[!ht]
  \centering
   \adjustbox{trim={0\width} {0.435\height} {0\width} {0\height} , clip}%
   {\includegraphics[width=\textwidth]{img/precipitacao3d/severo/percentil/90th/cCumFad_10deg_semraio_topFTpercentil}}
 \caption{CCFDs para os extremos de FT entre 20S-10N e 90W-30W. Porção da precipitação sem raios.}
 \label{ftccfadwithout}
\end{figure} 

A porção eletricamente ativa possui maior percentual de perfis convectivos e com maiores valores de $Z_c$ associado aos contornos de probabilidade, confirmando a correlação positiva entre descargas elétricas e a produção de precipitação \cite{Petersen1998}.

A convecção é mais ativa nas regiões dos núcleos de raios, aonde a precipitação está associadas com frentes de rajadas, chuvas de granizo e enchentes rápidas. Fora dos núcleos de raios temos a parte da precipitação mais estratiforme, composta por hidrometeoros que não possuem velocidade terminal suficiente para precipitar nos núcleos de raios, e caem mais afastados da região eletricamente ativa.     % Dependendo principalmente das condições de calor umidade e cisalhamento vertical do vento as células 


% \caption{Diagramas de Contorno de Frequência por Altitude (CFADs). Em cada CFAD pode-se verificar: a porcentagem (\%) de perfis convectivos, estratiformes e outros, respectivamente; (P) o numero de perfis do PR computados, (L) o número de ocorrência de refletividade no nível de máxima ocorrência e (H) o nível de máxima ocorrência.}

Se avaliarmos apenas os níveis de contorno com probabilidade entre 2-3.7\% (cor verde), observa-se que os máximos de $Z_c$ associados à chuva da porção sem raios, figuras \ref{ftacfadwithout} e \ref{ftcfadwithout}, não ultrapassaram os 40 dBZ em nenhuma região, enquanto que para a porção de chuvas com raios, figuras \ref{ftacfadwith} e \ref{ftcfadwith}, os valores de $Z_c$ entre 0-5 km de altitude registram valores entre 45-50 dBZ.

A figura \ref{ftcfadwithout} mostra que a precipitação sem raios dos extremos de FT na região tropical, entre 20S-10N e 90W-30W, possui banda brilhante marcada entre 4-5 km de altitude, principalmente nos perfis com probabilidade de ocorrência entre 2-5.3\%, nas cores de contorno em verde e amarelo. 

Podemos observar a banda brilhante dos sistemas com índice extremo de FT na porção sem raios de maneira mais elucidativa por meio dos CCFADs da figura \ref{ftccfadwithout}, os quais evidenciam que entre o 12\textsuperscript{\underline{o}} e o 95\textsuperscript{\underline{o}} percentil da $f_{pdf}(x,y)$ normalizada por altitude, que define cada CFAD entre 20S-10N e 90W-30W na figura \ref{ftcfadwithout}, há uma queda no valor de $Z_c$ logo abaixo de 5 quilômetros de altitude em cada região de 10 por 10 graus. 


Na figura \ref{ftacfadwithout}, que representa a porção sem raios da precipitação tridimensional dos sistemas com índice extremo de FTA, não se observa banda brilhante marcada nos contornos de probabilidade de $Z_c$ por altitude. Há um aumento contínuo de $Z_c$,  conforme os níveis de altitude diminuem, sem a diminuição abrupta de $Z_c$ logo abaixo de 5 quilômetros.   

%observa-se que, entre 20S-10N e 90W-30W,  
%a banda brilhante é evidente apenas nas regiões costeiras e oceânicas, nas caixas entre 0-10N  %e 90-80W, entre 10-0S e 40-30W e entre 20-10S e 70-60W.

As chuvas na superfície associadas com a precipitação sem raios das tempestades elétricas entre 20S-10N e 90W-30W, região tropical, referentes aos extremos de FT têm maiores valores de probabilidades com valores de $Z_c$ mais moderados do que quando compara-se com os extremos de FTA, os quais possuem perfis de $Z_c$ com maior aleatoriedade, mas podem atingir valores em dBZ superiores. Note como os contornos de probabilidade, principalmente entre 0.3-3.7\% representados pelas cores em azul e verde, são mais alargados na chuva sem raios dos extremos de FTA, figura \ref{ftacfadwithout} do que na chuva sem raios dos extremos de FT, figura \ref{ftcfadwithout}. Na chuva sem raios dos sistemas com extremos de FT, os contornos da figura \ref{ftcfadwithout} são mais estreitos, indicando menor aleatoriedade nos valores de $Z_c$ observados.

Na região entre 40-20S e 70-50W que engloba a Bacia do Prata, a banda brilhante foi menos evidente nos contornos de probabilidade associados a estrutura tridimensional da precipitação fora dos núcleos de raios, tanto para os extremos de FT, figura \ref{ftccfadwithout}, quanto para os extremos de FTA, figura \ref{ftacfadwithout}. 

Entre 40-20S e 70-50W, a porção sem raios da chuva dos extremos de FTA mostra que entre 0-5 km de altitude, a probabilidade de valores inferiores de $Z_c$ em relação as porções sem raios dos extremos de FT é maior. Observe como a mediana das amostras de probabilidades, marcada pela linha de contorno na cor preta no 50\textsuperscript{\underline{o}} percentil do CCFAD em cada caixa de 10 por 10 graus nas figuras \ref{ftccfdsubtrop} e \ref{ftaccfdsubtrop}, indica maior taxa de precipitação entre 0-5 km na porção sem raios das tempestades elétricas com índice FT extremo, figura \ref{ftccfdsubtrop}, mesmo que a estatística na parte superior direita de cada CCFAD indique maior percentual de perfis convectivos para a porção sem raios dos extremos de FTA, figura \ref{ftaccfdsubtrop}.


%, evidenciando que a chuva sem raios dos sistemas com as maiores taxas de raios no tempo é mais severa nesta região.

\begin{figure}[!ht]
  \centering  
  \adjustbox{trim={.349\width} {.045\height} {.322\width} {.565\height},clip}%
  {\includegraphics[width=27cm] {img/precipitacao3d/severo/percentil/90th/cCumFad_10deg_semraio_topFTpercentil}}
 \caption{CCFDs para os extremos de FT entre 40-20S e 70-50W. Porção da precipitação sem raios.}
 \label{ftccfdsubtrop}
\end{figure} 

\begin{figure}[!ht]
%  \centering  
%  \adjustbox{trim={.0\width} {.04\height} {0\width} {.565\height},clip}%
  \centering  
  \adjustbox{trim={.349\width} {.045\height} {.322\width} {.565\height},clip}%  
  {\includegraphics[width=27cm] {img/precipitacao3d/severo/percentil/90th/cCumFad_10deg_semraio_topFTApercentil}}
 \caption{CCFDs para os extremos de FTA entre 40-20S e 70-50W. Porção da precipitação sem raios.}
 \label{ftaccfdsubtrop}
\end{figure} 

Porém, a precipitação contida fora dos núcleos de raios dos sistemas extremos selecionados pelo índice FTA, situados entre 40-20S e 70-50W,  e que é explicitada por meio dos CFADs da figura \ref{ftacfadwithout}, revela que a probabilidade entre 0.001-2\%, representados pelas cores de contorno em preto e azul, atingem valores superiores de $Z_c$ do que quando compara-se com os sistemas extremos de FT, na figura \ref{ftcfadwithout}, também entre 40-20S e 70-50W.

Apesar da mediana das probabilidades dos CFADs mostrarem que a precipitação entre 0-5 km de altitude foi mais intensa para os sistemas extremos de FT e localizados entre 40-20S e 70-50W, ao avaliar os contornos de probabilidade cumulativa dos CCFADs na figura \ref{ftaccfdsubtrop}, referente ao estudo da estrutura tridimensional da precipitação fora dos núcleos de raios dos extremos de FTA, observa-se que, acima do 80\textsuperscript{\underline{o}} percentil os valores de $Z_c$ foram superiores em relação aos sistemas com índice extremo de FT, na figura \ref{ftccfdsubtrop}.

Os CFADs referentes as tempestades elétricas selecionadas por FTA possuem contornos de probabilidade em níveis de altitude mais elevados do que os CFADs dos sistemas selecionados por FT, tanto para a porção com raios quanto para a porção sem raios da precipitação dos sistemas.

%A diferença mais notável pode ser observada entre a figura \ref{ftacfadwithout} e \ref{ftcfadwithout} para 0S-10S e 50W-60W, que abrange principalmente o estado do Pará, e parte do Amazônas, Tocantis e Mato Grosso. O CFAD em \ref{ftacfadwithout} define valores de probabilidade em altitude 2 km mais elevada do que em \ref{ftcfadwithout}.

\begin{figure}[!ht]
  \centering
  \includegraphics[height=13.5cm]{img/precipitacao3d/severo/percentil/90th/cfad10_comraio_topFT_percentil}
  \caption{CFADs para os extremos de FT. Porção da precipitação com raios.}
  \label{ftcfadwith}   
\end{figure} 

%Nas regiões entre 10N-0S e 70W-80W e entre 20S-40S e 50W-60W, em que \cite{cecil2005} apontam como região das tempestades mais severas na América do Sul, os CFADs em \ref{ftacfadwith} e \ref{ftacfadwithout} possuem contornos de probabilidade aproximadamente 1 km mais elevado do que em \ref{ftcfadwith} e \ref{ftcfadwithout}.

Como o último nível de altitude dos CFADs deste trabalho é limitado por altitudes com até 10\% de L, a maior definição de probabilidades de ocorrência de $Z_c$ em altitude para as tempestades selecionadas pelo índice FTA, indica que a convecção é mais intensa nos extremos de FTA do que nos extremos de FT. Principalmente quando observamos a morfologia da estrutura tridimensional da precipitação dos núcleos de raios, para os extremos do FTA e FT, expressa nos CFADs das figuras \ref{ftacfadwith} e \ref{ftcfadwith}, aonde os perfis de precipitação são classificados majoritariamente como convectivos.

A precipitação é bem mais frequente próxima da superfície, entre 0-5 km de altitude. Acima da região de mistura, a precipitação é mais rara de ocorrer. Em \cite{liu2008}, é mostrado que a densidade espacial de sistemas com no mínimo 20 dBZ em 2 km de altitude é globalmente maior do que os sistemas que atingem 20 dBZ em níveis superiores de altitude.


%A região de 10 por 10 graus, a qual o valor de H marcado no topo direito de cada CFAD, é menor para a precipitação dos núcleos de raios dos sistemas com extremo de FTA, figura \ref{ftacfadwith}, do que para  a precipitação dos núcleos de raios dos sistemas com extremos de FT, figura \ref{ftcfadwith}, e mesmo assim, o CFAD dos extremos de FTA, figura \ref{ftacfadwith}, possuiu maior altitude nos níveis de contorno de probabilidade de $Z_c$, o índice FTA mostra que a chuva  
%esteve associado com maior severidade de tempo do que FT.
%Pois, mesmo que a refletividade mais ocorrente esteja abaixo da região de mistura, a precipitação também é frequente conforme o aumento da altitude, mostrando que nestas regiões, os sistemas com índice FTA extremo têm maior número de ocorrência de chuva 
%mais chuvas na superfície e também maior precipitação acima de 10 km de altitude.
%com bastante representatividade estatística.
%maior quantidade de hidrometeoros na região de mistura e

Por exemplo na região do Panamá, Colômbia e Equador, entre 10N-0S e 70W-80W, o CFAD da figura \ref{ftacfadwith} possui contornos de probabilidade até 16 km de altitude. Na figura \ref{ftcfadwith}, os níveis de contorno param em 15 km.

A precipitação tridimensional observada nos núcleos de raios, explicitada nos contorno dos CFADs a cada 10 graus na figura \ref{ftacfadwith}, referente ao índice FTA, mostram valores de refletividade entre 1-3 dBZ maiores do que na figura \ref{ftcfadwith}, referente ao índice FT, principalmente quando observa-se os contornos de probabilidade de $Z_c$ acima de 5 km de altitude. Para a precipitação entre 1-2 km de altitude os valores são mais semelhantes entre as tempestades elétricas selecionadas por FTA e FT. 

%Porém, nos sistemas extremos de FTA, figura \ref{ftacfadwith}, há um estreitamento da região de contorno com os maiores valores de probabilidade associada a chuva na superfície, entre 3-5\%. Entre 20S-40S e 40-70W, o estreitamente é maior do que as demais regiões mostrando que as chuvas possuem maior probabilidade de estarem associadas com valores de 45 dBZ em \ref{ftacfadwith}.      

As mais baixas probabilidades de $Z_c$ observadas nos CFADs das figuras \ref{ftacfadwith} e \ref{ftcfadwith}, estão associadas com a estrutura tridimensional da precipitação mais severa. Observe os contornos de probabilidade entre 0.001-0.5\%. Estes níveis de contorno revelam os valores de $Z_c$ da precipitação mais rara entre os sistemas com índice extremo de FTA e FT, os quais provavelmente estiveram associados com enchentes rápidas, alta taxa de raios, chuva de granizo, fortes rajadas de vento e até mesmo ocorrência de tornados em algumas regiões. 
% nas figuras \ref{ftacfadwith} e \ref{ftcfadwith}

Os valores maiores valores de $Z_c$ foram registrados na figura \ref{ftacfadwith} entre 20S-40S e 40W-70W, sobre a Bacia do Rio da Prata, que abrange o Sul do Brasil, Paraguai, Uruguai e Argentina. A dinâmica de formação de Sistemas Convectivos de Meso-escala, como é discutido em \cite{Velasco1987} e \cite{Durkee2009}, somados com efeitos de topografia, como por exemplo na região da Serra de Córdoba na Argentina, a qual \cite{Rasmussen2011} mostram grande ocorrência de convecção profunda, promoveram sistemas em que a estrutura tridimensional da precipitação dos núcleos de raios atingiram valores de $Z_c$ superiores a 45 dBZ entre 10-15 km de altitude e chuvas na superfície com $Z_c$ acima de 55 dBZ, como mostram os contornos de probabilidade entre 0.001-0.5\%.

\subsection{A precipitação dos sistemas severos e o perfil atmosférico de temperatura.}

Os diagramas CCFTD e CFTD, descritos em \ref{chuvaEtemperatura}, são expostos nas figuras \ref{ccftd_fta_com}, \ref{ccftd_ft_com}, \ref{cftd_fta_com} e \ref{cftd_ft_com}, associados as tempestades elétricas com índice extremo de FTA e FT, apenas em suas porções com raios.

A partir dos CCFTDs das figuras \ref{ccftd_fta_com} e \ref{ccftd_ft_com}, iremos avaliar a intensidade convectiva dos sistemas com índice extremo de FTA e FT em determinadas regiões, com base na velocidade de crescimento ou decrescimento dos valores de $Z_{c}$ associados os contornos de probabilidade do 30\textsuperscript{\underline{o}}, 50\textsuperscript{\underline{o}}, 70\textsuperscript{\underline{o}} e 95\textsuperscript{\underline{o}} percentil das amostras de probabilidades expressas nos CFTDs das figuras \ref{cftd_fta_com} e \ref{cftd_ft_com}.	

\begin{figure}
  \centering
  \includegraphics[height=13.5cm]{img/precipitacao3d/severo/percentil/90th/cftd_10deg_comraio_topFTApercentil}
 \caption{CFTDs para os extremos de FTA. Porção da precipitação com raios.}
 \label{cftd_fta_com}
\end{figure} 

\begin{figure}
  \centering
  \includegraphics[height=13.5cm]{img/precipitacao3d/severo/percentil/90th/ccftd_10deg_comraio_topFTApercentil}
  \caption{CCFTDs para os extremos de FTA. Porção da precipitação com raios.}
  \label{ccftd_fta_com}   
\end{figure} 

  %\caption{Diagramas de Contorno de Frequência por Temperatura (CFTD) e Diagramas de Contorno de Frequência Cumulativa por Temperatura (CCFTD). Em cada CFTD e CCFTD, pode-se verificar: a porcentagem (\%) de perfis convectivos, estratiformes e outros, respectivamente; (P) o numero de perfis do PR computados, (L) o número de ocorrência de $Z_{c}$ no nível de temperatura de máxima ocorrência e (T) o nível de temperatura de máxima ocorrência de $Z_{c}$.}
 
%\clearpage
\begin{figure}
  \includegraphics[height=13.5cm]{img/precipitacao3d/severo/percentil/90th/cftd_10deg_comraio_topFTpercentil}
 \caption{CFTDs para os extremos de FT. Porção da precipitação com raios.}
 \label{cftd_ft_com}
\end{figure} 

\begin{figure}
  \includegraphics[height=13.5cm]{img/precipitacao3d/severo/percentil/90th/ccftd_10deg_comraio_topFTpercentil}
  \caption{CCFTDs para os extremos de FT. Porção da precipitação com raios.}
  \label{ccftd_ft_com}   
\end{figure} 

Então, para a região central da Bacia do Rio Amazonas, entre 10-0S e 70-60W, extraí-se as linhas de contorno do CCFTD referentes as probabilidades acumulativas de 30\%, 50\%, 70\% e 95\%. Desta forma obtemos quatro funções $f(x)=y$, \simbolo{name={$f(x)=y$},description={Função de uma variável}} em que $y$ corresponde aos valores de $Z_c$ e $x$ o perfil atmosférico de temperatura. Fazendo a derivada $\dfrac{dy}{dx}$ pode-se avaliar taxa de variação de $Z_c$ por temperatura (dBZ/\textsuperscript{o}C), para diferentes regimes de chuva, das mais frequentes até as mais raras, como mostra a figura \ref{deriv_amazonas}.

\begin{figure}[!ht]
  \centering
  \includegraphics[height=9cm]{img/precipitacao3d/deriv_ccftd/deriv_contornos_cdf_2_1}
  \caption{Taxa de variação de $Z_c$ no perfil de temperatura atmosférico para a região central da Bacia do Rio Amazonas, entre 10-0S e 70-60W.}
  \label{deriv_amazonas}  
\end{figure} 

% nos diferentes quartis do CCFTD dos extremos de FTA,  figura \ref{ccftd_fta_com},  e dos extremos de FT, figura \ref{ccftd_ft_com}. 

Na figura \ref{deriv_amazonas}, observa-se que a taxa de aumento de $Z_c$ em torno de -40 \textsuperscript{o}C e -15 \textsuperscript{o}C, é maior para os extremos de FTA, porém em torno de 0 \textsuperscript{o}C, a taxa de aumento de $Z_c$ é maior para os extremos de FT, mostrando que os hidrometeoros dos sistemas extremos de FTA, crescem em regiões mais frias do que nos extremos de FT.

% agregação e acreção é maior para os sistemas extremos de FTA e na região de derretimento, em torno de 0 \textsuperscript{o}C, o efeito da banda brilhante é mais acentuado para os extremos de FT.  

O aumento do fator de refletividade em torno de 0 \textsuperscript{o}C está associado a mudança do índice de refração da água devido a sua fusão. Já o aumento do fator de refletividade em torno de -40 \textsuperscript{o}C e -15 \textsuperscript{o}C representam o crescimento de hidrometeoros por agregação e acreção \cite{Fabry1995,Takahashi1978}.

Note na figura \ref{deriv_amazonas}, como a precipitação do 95\textsuperscript{\underline{o}} percentil de probabilidade de ocorrência tanto para FTA quanto para FT, é o regime de precipitação mais severa. Pois, há o crescimento de $Z_c$ em torno de -10 \textsuperscript{o}C e -15 \textsuperscript{o}C e não há banda brilhante, indicando precipitação a partir de granizo\footnote{Em \cite{Fabry1995}, este tipo de perfil é discutido como chuva a partir de gelo compacto.}. No 30\textsuperscript{\underline{o}}, 50\textsuperscript{\underline{o}} e 70\textsuperscript{\underline{o}} percentil dos extremos de FT, o efeito da banda brilhante associada ao derretimento é mais evidente do que para os extremos de FTA. 

Quando comparamos a região central da Bacia do Rio Amazonas, com a região central da Bacia do Rio da Prata, entre 30-20S e 60-50W, a microfísica de eletrificação se mostra diferente em cada local. Observa-se que no 50\textsuperscript{\underline{o}} percentil, a taxa de crescimento de $Z_c$ entre -40 \textsuperscript{o}C e -20 \textsuperscript{o}C é maior para a região da Bacia do Prata, figura \ref{deriv_prata}, do que para a região da Bacia Amazônica, figura \ref{deriv_amazonas}, tanto para os sistemas extremos de FTA quando para os sistemas extremos de FT, indicando maior crescimento de flocos de neve na precipitação severa sobre a Bacia do Prata. 


\begin{figure}[!ht]
  \centering
  \includegraphics[height=9cm]{img/precipitacao3d/deriv_ccftd/deriv_contornos_cdf_3_3}
  \caption{Taxa de variação de $Z_c$ no perfil de temperatura atmosférico para a região central da Bacia do Rio da Prata, entre 30-20S e 60-50W.}
  \label{deriv_prata}  
\end{figure} 

Apesar do 95\textsuperscript{\underline{o}} percentil mostrar maiores taxas de dBZ/\textsuperscript{o}C, em -15 \textsuperscript{o}C tanto para FTA quanto FT sobre a Bacia do Rio Amazonas, na figura \ref{deriv_amazonas}, do que sobre a Bacia do Rio da Prata, na figura \ref{deriv_prata}, os contorno de probabilidade acumulativa de 95\% nos CCFTD das figuras \ref{ccftd_fta_com} e \ref{ccftd_ft_com}, em -15 \textsuperscript{o}C, mostram valores de $Z_c$ de aproximadamente 3 dBZ superiores na região da Bacia Platina. Mesmo que o 95\textsuperscript{\underline{o}} percentil mostre maior crescimento de hidrometeoros na região mista sobre a Bacia Amazônica, a precipitação do 95\textsuperscript{\underline{o}} percentil na Bacia do Prata foi mais severa, pois possui maiores valores de $Z_c$.

O aumento abrupto de $Z_c$ associado a fusão da água, entre 30-20S e 60-50W, figura \ref{deriv_prata}, principalmente do 50\textsuperscript{\underline{o}} e 70\textsuperscript{\underline{o}} percentil, ocorrem em -4 \textsuperscript{o}C, enquanto que, entre 10-0S e 70-60W, figura \ref{deriv_amazonas}, o aumento de $Z_c$ ocorre mais próximo de 0 \textsuperscript{o}C, o que indica maior presença de água super-resfriada associada ao processo de derretimento da precipitação entre 30-20S e 60-50W, região da Bacia do Rio da Prata.  

Na região da Bacia do Prata, representada na figura \ref{deriv_prata}, o efeito de banda brilhante também é mais pronunciado para a precipitação com raios dos extremos de FT, o que mostra que as regiões eletricamente ativas da precipitação dos sistemas com índice extremo de FTA é menos estratificada do que nos extremos de FT, em ambas as Bacias Hidrológicas: da Prata e do Amazonas.

\newpage
\section{SEVERIDADE REGIONALIZADA}

Aqui, o estudo da densidade de probabilidade de FTA e FT, conforme mostrado na figura \ref{pdfFTAFT}, foi feito para os sistemas ocorridos em cada região de 2.5 por 2.5 graus de latitude e longitude entre 40N-10S e 90-30W. Verifica-se a distribuição geográfica, dos valores de FTA e FT mais frequentes e mais raros conforme cada localidade.

Buscando identificar quais dos índices, FTA ou FT foi mais sensível para indicar a intensidade convectiva das tempestades elétricas, torna-se interessante verificar quais são as regiões aonde sistemas com os menores valores de FTA e FT são mais frequentes.

Nas figuras \ref{extremosInfFTA} e \ref{extremosInfFT}, temos os valores de FTA e FT  para o 5\textsuperscript{\underline{o}} e 10\textsuperscript{\underline{o}} percentil, das distribuições de probabilidades regionalizadas a cada 2.5 por 2.5 graus.

\begin{figure}
  \subfloat[\textsuperscript{\underline{o}} percentil de FTA]{{\includegraphics[height=6.5cm, trim=0 0 0 0, clip]{img/DistEspacialPercentis/FTA/distEspacialValor005thFta}} \label{5oFta}}\\
  \subfloat[10\textsuperscript{\underline{o}} percentil de FTA]{{\includegraphics[height=6.5cm, trim=0 0 0 0, clip]{img/DistEspacialPercentis/FTA/distEspacialValor010thFta}} \label{10oFta}} 
  \caption{Distribuição espacial dos valores do 5\textsuperscript{\underline{o}} e 10\textsuperscript{\underline{o}} percentil da amostra de probabilidade do índice FTA a cada região de 2.5 por 2.5 graus de latitude e longitude.}
\label{extremosInfFTA}
\end{figure} 

A linha de contorno na cor preta em cada mapa apresentado nesta seção, corresponde ao valor do percentil determinado para a análise regional, porém, é referente a amostragem total exposta na figura \ref{cdfFTAFT}.

No ambiente oceânico e costeiro, as tempestades elétricas mais frequentes devem possuir menores índices de severidade do que no continente, pois na costa e oceano observa-se as maiores probabilidades de ocorrência de chuva quente \cite{Liu2009}. 
%o aquecimento da superfície durante o ciclo diurno é menor e o processo de colisão coalescência é dominante em relação aos processos que envolvem a formação de gelo de nuvem 

Nas figuras \ref{5oFta} e \ref{10oFta}, os contornos com valores de 0.05 $\times$ 10$^{-4}$ e 0.12 $\times$ 10$^{-4}$ raios minutos$^{-1}$
km$^{-2}$ respectivamente, demarcam claramente a divisão entre a convecção oceânica e a continental. O Oceano Verde, conceito associado a convecção durante o regime de ventos de Oeste na estação chuvosa Amazônica, discutido por \citeonline{silva2002lba,williams2002}, é bastante evidente. A região central da Bacia Amazônica possui os  valores de FTA na mesma ordem de magnitude e no mesmo percentil das densidades de probabilidades de FTA regionalizadas das tempestades elétricas oceânicas e costeiras.

Os valores de FT associados ao 5\textsuperscript{\underline{o}} e 10\textsuperscript{\underline{o}} percentil, mostrados nas figuras \ref{5oFt} e \ref{10oFt}, revelam os menores valores de FT no centro do continente, principalmente nas regiões continentais fora da área de atuação da ZCIT e de sistemas transientes subtropicais.

\begin{figure}[!ht]
  \centering{
  \subfloat[5\textsuperscript{\underline{o}} percentil de FT]{{\includegraphics[height=6.5cm, trim=0 0 0 0, clip]{img/DistEspacialPercentis/FT/distEspacialValor005thFt}} \label{5oFt}}\\
  \subfloat[10\textsuperscript{\underline{o}} percentil de FT]{{\includegraphics[height=6.5cm, trim=0 0 0 0, clip]{img/DistEspacialPercentis/FT/distEspacialValor010thFt}} \label{10oFt}}
  }    
  \caption{Distribuição espacial dos valores do 5\textsuperscript{\underline{o}} e 10\textsuperscript{\underline{o}} percentil da amostra de probabilidade do índice FT a cada região de 2.5 por 2.5 graus de latitude e longitude.}
\label{extremosInfFT}
\end{figure} 

Para avaliar a distribuição geográfica dos extremos superiores dos índices FTA e FT, verificou-se os valores do 95\textsuperscript{\underline{o}} e 99\textsuperscript{\underline{o}} percentil das amostragens, os quais são expostos nos mapas das figuras \ref{extremosSupFTA} e \ref{extremosSupFT}.

%: regionalizadas, representados nas cores das figuras \ref{extremosSupFTA} e \ref{extremosSupFT}; e totais (ver figura \ref{seriesFtaFt}), representados pela linha de contorno das figuras \ref{extremosSupFTA} e \ref{extremosSupFT}.

Observa-se na figura \ref{95oFta} que os sistemas com índice FTA  superior à 52.76 $\times$ 10$^{-4}$ raios minutos$^{-1}$
km$^{-2}$, são considerados de severidade extrema, pois correspondem a valores superiores ao valor do 95\textsuperscript{\underline{o}} percentil da amostragem total de FTA da figura \ref{pdfFta}. Porém, em regiões no interior do continente, os valores de FTA do 95\textsuperscript{\underline{o}} percentil das amostragens regionalizadas, atingiram  111.97 $\times$ 10$^{-4}$ raios minutos$^{-1}$ km$^{-2}$.
%, mostrando que nestas regiões, valores de 52.76 $\times$ 10$^{-4}$ raios minutos$^{-1}$ km$^{-2}$ são mais frequentes do que nas regiões em que passa a linha de contorno de 52.76 $\times$ 10$^{-4}$ raios minutos$^{-1}$ km$^{-2}$.
%raios por minutos por quilômetros quadrado

\begin{figure}[!ht]
\centering
{\includegraphics[height=13.5cm, trim=0 0 0 0, clip]{img/DistEspacialPercentis/FTA/distEspacialValor095thFta}} 
\caption{Distribuição espacial dos valores do 95\textsuperscript{\underline{o}} percentil da amostra de probabilidade do índice FTA a cada região de 2.5 por 2.5 graus de latitude e longitude.}
\label{95oFta}
\end{figure} 
  
\begin{figure}[!ht]
\centering  
{\includegraphics[height=13.5cm, trim=0 0 0 0, clip]{img/DistEspacialPercentis/FTA/distEspacialValor099thFta}}
\caption{Distribuição espacial dos valores do  99\textsuperscript{\underline{o}} percentil da amostra de probabilidade do índice FTA a cada região de 2.5 por 2.5 graus de latitude e longitude.}
\label{99oFta}
\end{figure} 


Os valores do 99\textsuperscript{\underline{o}} percentil na figura \ref{99oFta}, mostram que no Leste do estado do Amazonas, no Acre e Tocantis e Sudeste do Peru e Norte da Bolívia, regiões estas que compõem o Oceano Verde, a severidade extrema de FTA possui valores entre 148.93-230.00  $\times$ 10$^{-4}$ raios minutos$^{-1}$ km$^{-2}$,  valores que correspondem aos mais extremos do continente Sul-americano. 

Mesmo que a Floresta Amazônica seja um Oceano Verde para atmosfera durante as fases ativas do SAMS, durante o regime de ventos de Leste na estação chuvosa que associa-se as fases inativas da SAMS e durante a estação de transição seca-úmida (SON), ``o Mar Verde"  ~fica revolto. Mesmo que a Floresta Amazônica dialogue com a precipitação como um oceano, este oceano possui temperatura superficial média na classe das maiores temperaturas superficiais continentais globais e está cercado por um vasto continente. Portanto, tem a capacidades de gerar tempestades elétricas extremamente severas, mostrando que a interação entre a Floresta Amazônica e a atmosfera é bastante diversificada.

As regiões dos maiores valores do 95\textsuperscript{\underline{o}} e 99\textsuperscript{\underline{o}} percentil do índice FTA, os quais são  expostos nas figuras \ref{95oFta} e \ref{99oFta}, são principalmente: a Bacia do Rio da Prata, a região Leste Amazônia e as regiões  do planalto Brasileiro, que se estendem por quase todo o país.

Observa-se que os sistemas mais severos da América do Sul ocorrem associados ao relevo nas regiões entre o Pantanal Mato-grossense e o Planalto Central Brasileiro, entre as Bacias dos Rios: Xingu, Araguaia e Tocantis e também o Planalto Central Brasileiro, entre a Bacia do Rio Paraná e o Planalto Meridional Brasileiro, aonde está localizado os planaltos e chapadas da Bacia do Paraná. Nestas regiões os sistemas severos possuem índice FTA superiores à 80 $\times$ 10$^{-4}$ raios minutos$^{-1}$ km$^{-2}$, como mostram as cores das figuras \ref{95oFta} e \ref{99oFta}. 

Note que para saber aproximadamente o número de raios produzidos pelos sistemas extremos de FTA temos que multiplicar o índice FTA pela área do sistema. Por exemplo, a equação \ref{FTAkm2}, descreve que nas regiões em que os sistemas extremos possuem 100 $\times$ 10$^{-4}$ raios minutos$^{-1}$ km$^{-2}$, um sistemas severo com área de 10$^3$ km$^2$ então possui 10 raios observados pelo LIS em 1 minuto.  

\begin{equation}
100 \times 10^{-4} \left[ \frac{\mathrm{raios}}{\mathrm{minutos}~\mathrm{km}^2} \right]  10^3 [ \mathrm{km}^2 ] = 10 \left[ \frac{\mathrm{raios}}{\mathrm{minutos}}\right]  
\label{FTAkm2}
\end{equation}

Os mapas das figuras \ref{95oFt} e \ref{99oFt}, mostam que nas Bacias: do Rio da Prata principalmente, do Rio Araguaia, Rio Xingu e Rio Tocantis, são locais em que os sistemas possuem os maiores índices de FT tanto no 95\textsuperscript{\underline{o}} quanto no 99\textsuperscript{\underline{o}} percentil.

\begin{figure}[!ht]
\centering
{\includegraphics[height=13.5cm, trim=0 0 0 0, clip]{img/DistEspacialPercentis/FT/distEspacialValor095thFt}} 
\caption{Distribuição espacial dos valores do 95\textsuperscript{\underline{o}} percentil da amostra de probabilidade do índice FT a cada região de 2.5 por 2.5 graus de latitude e longitude.}
\label{95oFt}
\end{figure} 
  
\begin{figure}[!ht]
\centering  
{\includegraphics[height=13.5cm, trim=0 0 0 0, clip]{img/DistEspacialPercentis/FT/distEspacialValor099thFt}}
\caption{Distribuição espacial dos valores do  99\textsuperscript{\underline{o}} percentil da amostra de probabilidade do índice FT a cada região de 2.5 por 2.5 graus de latitude e longitude.}
\label{99oFt}
\end{figure}

Os maiores valores do 95\textsuperscript{\underline{o}} e 99\textsuperscript{\underline{o}} percentil do índice FT, figuras \ref{95oFt} e \ref{99oFt},  ficam situados na região Sul da América do Sul, compatível com a região em que \citeonline{cecil2005}, apontam como o local das tempestades categoria 5, ou seja, das mais severas do globo.