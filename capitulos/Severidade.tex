\chapter{A SEVERIDADE DOS SISTEMAS}

Conforme descrito em \ref{metodoFtaFt}, as taxas de raios das tempestades elétricas neste trabalho de pesquisa, estão associadas aos índices FTA e FT.  Nesta seção identificamos qual desses índices representaram tempestades elétricas com maior intensidade convectiva, ou seja, os sistemas com as maiores taxas de raios por minuto ou os sistemas com as maiores taxas de raios por minuto por quilômetro quadrado de sua extensão. 

Ao aplicar as equações \ref{eqFT} e \ref{eqFTA} na base de dados de 94,711 tempestades elétricas, as quais tiveram pelo menos um pixel da varredura do PR contida na área do sistema e com tempo médio de visada do LIS maior ou igual a 1 minuto, foram estudadas as distribuições de probabilidades dos índices FTA e FT. Conforme mostra a figura \ref{seriesFtaFt}, trata-se de distribuições exponenciais de probabilidade. 

\begin{figure}[!hb]
  \centering 
  \subfloat[FTA] { \includegraphics[height=5cm]{img/FtaFt/pdfFta} \label{pdfFta}} 
  \subfloat[FT]{ \includegraphics[height=5cm]{img/FtaFt/pdfFt} \label{pdfFt}} 
   \caption{Densidade de probabilidade dos valores da série referente aos índices FTA e FT.}
   \label{seriesFtaFt}
\end{figure}

Os sistemas potencialmente severos foram selecionados pelo 90\textsuperscript{\underline{o}} percentil, associado aos máximos valores de FTA e FT. A linha vermelha nas figuras \ref{pdfFta} e \ref{pdfFt} marca o limite cuja os índices são considerados extremos. Os valores de FTA e FT a direita da linha vermelha correspondem ao conjunto dos 9472 sistemas, que correspondem aos 10\% mais raros da amostragem nos 14 anos de observação do TRMM.  

Portanto será investigada a severidade apenas dos sistemas com índices FTA e FT extremos, os quais possuem valores acima de 18.0 $\times$ 10$^{-4}$ raios por minuto por quilômetro quadrado, como mostra a figura \ref{pdfFta}, ou acima de 6.3 raios por minutos, como mostra a figura \ref{pdfFt}. Porém, os máximos valores de FTA e FT das tempestades elétricas foram de 1258.8 $\times$ 10$^{-4}$ raios por minuto por quilômetro quadrado e 1283.6 raios por minuto respectivamente. 


% Na figura \ref{percetilFtaFt}, temos a série de FTA e FT ordenada, e a linha tracejada vertical corta o 90\% percentil dos índices. 

%\begin{figure}[!ht]
%  \centering
%  \includegraphics[height=6cm]{img/FtaFt/90thFtaFt}	 
%  \caption{90\textsuperscript{\underline{o}} percentil de FTA e FT.}
%  \label{percetilFtaFt}
%\end{figure}


\section{TAMANHO E TEMPERATURA DE TOPO DAS NUVENS DE TEMPESTADES ELÉTRICAS POTENCIALMENTE SEVERAS}

Observa-se que os extremos de FT e FTA correspondem a sistemas com tamanhos bem distintos. Conforme é mostrado na figura \ref{size}, verifica-se que as máximas probabilidades de ocorrência de tempestades elétricas associadas com os extremos de FTA, ocorrem em sistemas com área 3 ordens de grandeza menor do que nos extremos de FT.



\begin{figure}[!hb]
  \centering{  
  \subfloat[Densidade de probabilidade de extensão em área.] { \includegraphics[scale=1.1,trim=0 0 215 0,clip]{img/tb/TbAreas} \label{size}} 
  \subfloat[Densidade de probabilidade de temperatura de brilho em infravermelho.]{ \includegraphics[scale=1.1,trim=220 0 0 0,clip]{img/tb/TbAreas} \label{tb}} 
  }
  \label{t_tb}
  \caption{Estudo das frequências de ocorrências de tempestades elétricas selecionas pelo 90\textsuperscript{\underline{o}} percentil dos índices de FT e FTA, por extensão em área e por temperatura de brilho de topo das nuvens.}
\end{figure}


As tempestades elétricas ordenadas pelo índice FT são maiores em extensão por que conforme aumenta a área do sistema, maior a probabilidade de haver raios na região. Uma tempestade elétrica com 10$^5$ km$^2$, provavelmente terá maior número de descargas observadas durante o tempo de visada do LIS do que uma com 10$^2$ km$^2$. 

%Quando a densidade espacial de descargas aumenta muito em uma região com centenas de quilômetros quadrados, em torno de 100 descargas intranuvens para uma nuven-solo, como por exemplo os maiores valores de $Z=IC/CG$ mostrados por \cite{evandro2009} na região de Campo Grande - MS no Brasil, a capacidade do LIS de identificar brilhos transientes provavelmente fica comprometida devido a resolução horizontal da CCD.

Ao normalizar a taxa de raios no tempo por $A_t$, o número de raios fica diluído na extensão do sistema, evidenciando que os maiores valores de FTA correspondem aos sistemas com as maiores densidades espaciais de raios, cuja a extensão em área e o número de raios possuem maior probabilidade de ser menor do que nos sistemas com extremos de FT.

A frequência de ocorrência das temperaturas de brilho associadas a radiância espectral observada no canal 4 do VIRS para todos os pixeis que definiram as áreas dos sistemas, é mostrada na figura \ref{tb}. Observa-se que o maior valor de probabilidade para a curva das tempestades elétricas com índice extremo de FTA, possui temperatura de topo de nuvens aproximadamente 10 K mais frias do que nas tempestades elétricas com extremos de FT, indicando que a convecção nos sistemas ordenados por FTA é mais profunda na maioria das situações.



\citeonline{morales2003} ao desenvolver a \textit{Sferics Infrared Rainfall Technique} (SIRT), mostram que as regiões com temperatura de brilho inferior a 215 K e com ocorrência de \textit{sferics} foram as regiões categorizadas como de maior precipitação associada.\sigla{name={SIRT},description={\textit{Sferics Infrared Rainfall Technique} }}

Neste trabalho de pesquisa, ao selecionar as tempestades elétricas com índice extremo de FTA, os maiores valores de probabilidade de ocorrência, conforme é mostrado na figura \ref{tb}, concentram-se em temperaturas de brilho abaixo de 215 K.

Os sistemas selecionados pelo 90\textsuperscript{\underline{o}} percentil do índice FT possuem maior extensão em área e maior volume de chuva. São sistemas com vasta extensão estratiforme conforme descrevem \citeonline{Rasmussen2011}. As regiões das tempestades elétricas com precipitação convectiva, as quais são capazes de gerar chuva de granizo, frentes de rajada, tornados, enchentes rápidas, ocupam área bem menor do que as áreas estratiformes \cite{Jr2007}.

Avaliando a densidade de probabilidade de fração de chuva total, convectiva e estratiforme das tempestades elétricas, os máximos valores de ocorrência associados aos extremos valores de FTA concentraram-se nas tempestades elétricas com 70\% de área convectiva e 40\% de área estratiforme, enquanto que para os extremos de FT possuíram 20\% de fração convectiva e 75\% de fração estratiforme.
%juntamente com a as respectivas distribuições de probabilidade acumulativa,

Talvez alguns sistemas com extremos valores de FTA estejam em estágio de maturação e conforme vão se dissipando vão ganhando área de chuva estratiforme e se enquadrando no grupo dos maiores índice de FT. 


%.....
%Para avaliar qual dos índices representaram a maior severidade de tempo, a morfologia da estrutura 3D da precipitação foi estudada por meio dos diagramas CFAD, CCFAD, CFTD e CCFTD. %para os 10\% das amostras de FT e FTA com os maiores valores.
%......

\section{ESTRUTURA 3D DA PRECIPITAÇÃO DAS TEMPESTADES ELÉTRICAS SEVERAS}

Nesta etapa iremos avaliar a intensidade convectiva com base nos perfis de $Z_c$ do PR, contidos nos sistemas com índices extremos de FTA e FT. Como os sistemas com extremos de FT possuem área na ordem de 10$^5$ km$^2$, o PR observou com maior frequência apenas  30\% da área total destas tempestades elétricas. Pois, geralmente a varredura do PR não contempla toda a sua extensão. Para os sistemas escolhidos pelos extremos de FTA o PR teve maior probabilidade de observar entre 90-100\% da área dos sistemas.

Nas figuras \ref{ftacfadwith}, \ref{ftacfadwithout}, \ref{ftcfadwith} e \ref{ftcfadwithout} foram calculados os CFADs para as tempestades elétricas com índices FTA e FT extremos, distribuídas conforme cada região de 10 por 10 graus na superfície terrestre. Para localizar a caixa de 10 por 10 graus em que cada sistema esteve contido, foi considerado a latitude e longitude do centro geométrico da área definida por cada sistema.


\begin{figure}[!ht]
  \centering
  \includegraphics[height=13.5cm]{img/precipitacao3d/severo/cfad/cfad10_semraio_topFTA}
 \caption{CFADs para os extremos de FTA. Porção da precipitação sem raios.}
 \label{ftacfadwithout}
\end{figure} 

\begin{figure}[!ht]
  \centering
  \includegraphics[height=13.5cm]{img/precipitacao3d/severo/cfad/cfad10_comraio_topFTA}
  \caption{CFADs para os extremos de FTA. Porção da precipitação com raios.}
  \label{ftacfadwith}   
\end{figure} 

As posições geográficas dos eventos do LIS e dos perfis de $Z_c$ válidos do PR, foram projetadas em uma grade regular de 0.05 graus. Os perfis de $Z_c$ projetados em pontos de grade em que tiveram eventos do LIS, definiram as regiões aqui denominadas como precipitação dos núcleos de raios. Os CFADs foram calculados para a porção da chuva com, figuras \ref{ftacfadwith} e \ref{ftcfadwith}, e sem, figuras \ref{ftacfadwithout} e \ref{ftcfadwithout}, atividade elétrica de nuvem.

Note que no canto superior direito de cada CFAD temos alguns valores estatísticos que representam: (\%) \simbolo{name={\%},description={Nos diagramas, CFAD, CCFAD, CFTD e CCFTD, representam: a porcentagem de perfis convectivos, estratiformes e outros, respectivamente}} a porcentagem de perfis convectivos, estratiformes e outros, respectivamente; (P) \simbolo{name={P},description={Nos diagramas, CFAD, CCFAD, CFTD e CCFTD, representam: número de perfis do PR computados}} o número de perfis do PR computados; (L) \simbolo{name={L},description={Nos diagramas, CFAD, CCFAD, CFTD e CCFTD, representam: o número de ocorrência de $Z_c$ no nível de altitude de máxima ocorrência}} o número de ocorrência de $Z_c$ no nível de altitude de máxima ocorrência; (H) \simbolo{name={H},description={Nos diagramas, CFAD, CCFAD, CFTD e CCFTD, representam: o nível de altitude, em quilômetros, aonde ocorreu o máximo de ocorrências de $Z_c$}} o nível de altitude, em quilômetros, aonde ocorreu o máximo de ocorrências de $Z_c$.

Comparando os CFADs da chuva com e sem raios, representados para os extremos de FTA nas figuras \ref{ftacfadwithout} e \ref{ftacfadwith} e para os extremos de FT, nas figuras \ref{ftacfadwithout} e \ref{ftcfadwithout}, é evidente que a porção sem raios é a parte menos severa dos sistemas. Os níveis de contorno de probabilidades dos CFADs da precipitação sem raios possuem suas máximas altitudes aproximadamente 3 quilômetros abaixo das máximas altitudes atingidas pelos contornos dos CFADs da precipitação com raios. A porção sem raios dos sistemas possuíram maior percentual de perfis estratiformes e menores valores de $Z_c$ com os contornos de probabilidades entre 1-10\%, em todos os níveis de altitude.



\begin{figure}[!ht]
  \centering
  \includegraphics[height=13.5cm]{img/precipitacao3d/severo/cfad/cfad10_semraio_topFT}
 \caption{CFADs para os extremos de FT. Porção da precipitação sem raios.}
 \label{ftcfadwithout}
\end{figure} 

\begin{figure}[!ht]
  \centering
   \adjustbox{trim={0\width} {0.435\height} {0\width} {0\height} , clip}%
   {\includegraphics[width=\textwidth]{img/precipitacao3d/severo/cfad/cCumFad_10deg_semraio_topFT}}
 \caption{CCFDs para os extremos de FT entre 20S-10N e 90W-30W. Porção da precipitação sem raios.}
 \label{ftccfadwithout}
\end{figure} 







A porção eletricamente ativa possui maior percentual de perfis convectivos e com maiores valores de $Z_c$ associado aos contornos de probabilidade, confirmando a correlação positiva entre descargas elétricas e a produção de precipitação \cite{Petersen1998}.

A convecção é mais ativa nas regiões dos núcleos de raios, aonde a precipitação está associadas com frentes de rajadas, chuvas de granizo e enchentes rápidas. Fora dos núcleos de raios temos a parte da precipitação mais estratiforme, composta por hidrometeoros que não possuem velocidade terminal suficiente para precipitar nos núcleos de raios, e caem mais afastados da região eletricamente ativa.     % Dependendo principalmente das condições de calor umidade e cisalhamento vertical do vento as células 


% \caption{Diagramas de Contorno de Frequência por Altitude (CFADs). Em cada CFAD pode-se verificar: a porcentagem (\%) de perfis convectivos, estratiformes e outros, respectivamente; (P) o numero de perfis do PR computados, (L) o número de ocorrência de refletividade no nível de máxima ocorrência e (H) o nível de máxima ocorrência.}

Se avaliarmos apenas os níveis de contorno com probabilidade entre 2-3.7\% (cor verde), observa-se que os máximos de $Z_c$ associados à chuva da porção sem raios, figuras \ref{ftacfadwithout} e \ref{ftcfadwithout}, não ultrapassaram os 40 dBZ em nenhuma região, enquanto que para a porção de chuvas com raios, figuras \ref{ftacfadwith} e \ref{ftcfadwith}, os valores de $Z_c$ entre 0-5 km de altitude registram valores entre 45-50 dBZ.

A figura \ref{ftcfadwithout} mostra que a precipitação sem raios dos extremos de FT na região tropical, entre 20S-10N e 90W-30W, possui banda brilhante marcada entre 4-5 km de altitude, principalmente nos perfis com probabilidade de ocorrência entre 2-5.3\%, nas cores de contorno em verde e amarelo. 

Podemos observar a banda brilhante dos sistemas com índice extremo de FT na porção sem raios de maneira mais elucidativa por meio dos CCFADs da figura \ref{ftccfadwithout}, os quais evidenciam que entre o 12\textsuperscript{\underline{o}} e o 95\textsuperscript{\underline{o}} percentil da $f_{pdf}(x,y)$ normalizada por altitude, que define cada CFAD entre 20S-10N e 90W-30W na figura \ref{ftcfadwithout}, há uma queda no valor de $Z_c$ logo abaixo de 5 quilômetros de altitude em cada região de 10 por 10 graus. 


Na figura \ref{ftacfadwithout}, que representa a porção sem raios da precipitação tridimensional dos sistemas com índice extremo de FTA, não se observa banda brilhante marcada nos contornos de probabilidade de $Z_c$ por altitude. Há um aumento contínuo de $Z_c$,  conforme os níveis de altitude diminuem, sem a diminuição abrupta de $Z_c$ logo abaixo de 5 quilômetros.   

%observa-se que, entre 20S-10N e 90W-30W,  
%a banda brilhante é evidente apenas nas regiões costeiras e oceânicas, nas caixas entre 0-10N  %e 90-80W, entre 10-0S e 40-30W e entre 20-10S e 70-60W.

As chuvas na superfície associadas com a precipitação sem raios das tempestades elétricas entre 20S-10N e 90W-30W, região tropical, referentes aos extremos de FT têm maiores valores de probabilidades com valores de $Z_c$ mais moderados do que quando compara-se com os extremos de FTA, os quais possuem perfis de $Z_c$ com maior aleatoriedade, mas podem atingir valores em dBZ superiores. Note como os contornos de probabilidade, principalmente entre 0.3-3.7\% representados pelas cores em azul e verde, são mais alargados na chuva sem raios dos extremos de FTA, figura \ref{ftacfadwithout} do que na chuva sem raios dos extremos de FT, figura \ref{ftcfadwithout}. Na chuva sem raios dos sistemas com extremos de FT, os contornos da figura \ref{ftcfadwithout} são mais estreitos, indicando menor aleatoriedade nos valores de $Z_c$ observados.

Na região entre 40-20S e 70-50W que engloba a Bacia do Prata, a banda brilhante foi menos evidente nos contornos de probabilidade associados a estrutura tridimensional da precipitação fora dos núcleos de raios, tanto para os extremos de FT, figura \ref{ftccfadwithout}, quanto para os extremos de FTA, figura \ref{ftacfadwithout}. 

Entre 40-20S e 70-50W, a porção sem raios da chuva dos extremos de FTA mostra que entre 0-5 km de altitude, a probabilidade de valores inferiores de $Z_c$ em relação as porções sem raios dos extremos de FT é maior. Observe como a mediana das amostras de probabilidades, marcada pela linha de contorno na cor preta no 50\textsuperscript{\underline{o}} percentil do CCFAD em cada caixa de 10 por 10 graus nas figuras \ref{ftccfdsubtrop} e \ref{ftaccfdsubtrop}, indica maior taxa de precipitação entre 0-5 km na porção sem raios das tempestades elétricas com índice FT extremo, figura \ref{ftccfdsubtrop}, mesmo que a estatística na parte superior direita de cada CCFAD indique maior percentual de perfis convectivos para a porção sem raios dos extremos de FTA, figura \ref{ftaccfdsubtrop}.


%, evidenciando que a chuva sem raios dos sistemas com as maiores taxas de raios no tempo é mais severa nesta região.

\begin{figure}[!ht]
  \centering  
  \adjustbox{trim={.349\width} {.045\height} {.322\width} {.565\height},clip}%
  {\includegraphics[width=27cm] {img/precipitacao3d/severo/cfad/cCumFad_10deg_semraio_topFT}}
 \caption{CCFDs para os extremos de FT entre 40-20S e 70-50W. Porção da precipitação sem raios.}
 \label{ftccfdsubtrop}
\end{figure} 

\begin{figure}[!ht]
%  \centering  
%  \adjustbox{trim={.0\width} {.04\height} {0\width} {.565\height},clip}%
  \centering  
  \adjustbox{trim={.349\width} {.045\height} {.322\width} {.565\height},clip}%  
  {\includegraphics[width=27cm] {img/precipitacao3d/severo/cfad/cCumFad_10deg_semraio_topFTA}}
 \caption{CCFDs para os extremos de FTA entre 40-20S e 70-50W. Porção da precipitação sem raios.}
 \label{ftaccfdsubtrop}
\end{figure} 

Porém, a precipitação contida fora dos núcleos de raios dos sistemas extremos selecionados pelo índice FTA, situados entre 40-20S e 70-50W,  e que é explicitada por meio dos CFADs da figura \ref{ftacfadwithout}, revela que a probabilidade entre 0.001-2\%, representados pelas cores de contorno em preto e azul, atingem valores superiores de $Z_c$ do que quando compara-se com os sistemas extremos de FT, na figura \ref{ftcfadwithout}, também entre 40-20S e 70-50W.

Apesar da mediana das probabilidades dos CFADs mostrarem que a precipitação entre 0-5 km de altitude foi mais intensa para os sistemas extremos de FT e localizados entre 40-20S e 70-50W, ao avaliar os contornos de probabilidade cumulativa dos CCFADs na figura \ref{ftaccfdsubtrop}, referente ao estudo da estrutura tridimensional da precipitação fora dos núcleos de raios dos extremos de FTA, observa-se que, acima do 80\textsuperscript{\underline{o}} percentil os valores de $Z_c$ foram superiores em relação aos sistemas com índice extremo de FT, na figura \ref{ftccfdsubtrop}.

Os CFADs referentes as tempestades elétricas selecionadas por FTA possuem contornos de probabilidade em níveis de altitude mais elevados do que os CFADs dos sistemas selecionados por FT, tanto para a porção com raios quanto para a porção sem raios da precipitação dos sistemas.

%A diferença mais notável pode ser observada entre a figura \ref{ftacfadwithout} e \ref{ftcfadwithout} para 0S-10S e 50W-60W, que abrange principalmente o estado do Pará, e parte do Amazônas, Tocantis e Mato Grosso. O CFAD em \ref{ftacfadwithout} define valores de probabilidade em altitude 2 km mais elevada do que em \ref{ftcfadwithout}.

\begin{figure}[!ht]
  \centering
  \includegraphics[height=13.5cm]{img/precipitacao3d/severo/cfad/cfad10_comraio_topFT}
  \caption{CFADs para os extremos de FT. Porção da precipitação com raios.}
  \label{ftcfadwith}   
\end{figure} 

%Nas regiões entre 10N-0S e 70W-80W e entre 20S-40S e 50W-60W, em que \cite{cecil2005} apontam como região das tempestades mais severas na América do Sul, os CFADs em \ref{ftacfadwith} e \ref{ftacfadwithout} possuem contornos de probabilidade aproximadamente 1 km mais elevado do que em \ref{ftcfadwith} e \ref{ftcfadwithout}.

Como o último nível de altitude dos CFADs deste trabalho é limitado por altitudes com até 10\% de L, a maior definição de probabilidades de ocorrência de $Z_c$ em altitude para as tempestades selecionadas pelo índice FTA, indica que a convecção é mais intensa nos extremos de FTA do que nos extremos de FT. Principalmente quando observamos a morfologia da estrutura tridimensional da precipitação dos núcleos de raios, para os extremos do FTA e FT, expressa nos CFADs das figuras \ref{ftacfadwith} e \ref{ftcfadwith}, aonde os perfis de precipitação são classificados majoritariamente como convectivos.

A precipitação é bem mais frequente próxima da superfície, entre 0-5 km de altitude. Acima da região de mistura, a precipitação é mais rara de ocorrer. Em \cite{liu2008}, é mostrado que a densidade espacial de sistemas com no mínimo 20 dBZ em 2 km de altitude é globalmente maior do que os sistemas que atingem 20 dBZ em níveis superiores de altitude.


%A região de 10 por 10 graus, a qual o valor de H marcado no topo direito de cada CFAD, é menor para a precipitação dos núcleos de raios dos sistemas com extremo de FTA, figura \ref{ftacfadwith}, do que para  a precipitação dos núcleos de raios dos sistemas com extremos de FT, figura \ref{ftcfadwith}, e mesmo assim, o CFAD dos extremos de FTA, figura \ref{ftacfadwith}, possuiu maior altitude nos níveis de contorno de probabilidade de $Z_c$, o índice FTA mostra que a chuva  
%esteve associado com maior severidade de tempo do que FT.
%Pois, mesmo que a refletividade mais ocorrente esteja abaixo da região de mistura, a precipitação também é frequente conforme o aumento da altitude, mostrando que nestas regiões, os sistemas com índice FTA extremo têm maior número de ocorrência de chuva 
%mais chuvas na superfície e também maior precipitação acima de 10 km de altitude.
%com bastante representatividade estatística.
%maior quantidade de hidrometeoros na região de mistura e

Por exemplo na região do Panamá, Colômbia e Equador, entre 10N-0S e 70W-80W, o CFAD da figura \ref{ftacfadwith} possui contornos de probabilidade até 16 km de altitude. Na figura \ref{ftcfadwith}, os níveis de contorno param em 15 km.

A precipitação tridimensional observada nos núcleos de raios, explicitada nos contorno dos CFADs a cada 10 graus na figura \ref{ftacfadwith}, referente ao índice FTA, mostram valores de refletividade entre 1-3 dBZ maiores do que na figura \ref{ftcfadwith}, referente ao índice FT, principalmente quando observa-se os contornos de probabilidade de $Z_c$ acima de 5 km de altitude. Para a precipitação entre 1-2 km de altitude os valores são mais semelhantes entre as tempestades elétricas selecionadas por FTA e FT. 

%Porém, nos sistemas extremos de FTA, figura \ref{ftacfadwith}, há um estreitamento da região de contorno com os maiores valores de probabilidade associada a chuva na superfície, entre 3-5\%. Entre 20S-40S e 40-70W, o estreitamente é maior do que as demais regiões mostrando que as chuvas possuem maior probabilidade de estarem associadas com valores de 45 dBZ em \ref{ftacfadwith}.      

As mais baixas probabilidades de $Z_c$ observadas nos CFADs das figuras \ref{ftacfadwith} e \ref{ftcfadwith}, estão associadas com a estrutura tridimensional da precipitação mais severa. Observe os contornos de probabilidade entre 0.001-0.5\%. Estes níveis de contorno revelam os valores de $Z_c$ da precipitação mais rara entre os sistemas com índice extremo de FTA e FT, os quais provavelmente estiveram associados com enchentes rápidas, alta taxa de raios, chuva de granizo, fortes rajadas de vento e até mesmo ocorrência de tornados em algumas regiões. 
% nas figuras \ref{ftacfadwith} e \ref{ftcfadwith}

Os valores maiores valores de $Z_c$ foram registrados na figura \ref{ftacfadwith} entre 20S-40S e 40W-70W, sobre a Bacia do Rio da Prata, que abrange o Sul do Brasil, Paraguai, Uruguai e Argentina. A dinâmica de formação de Sistemas Convectivos de Meso-escala, como é discutido em \cite{Velasco1987} e \cite{Durkee2009}, somados com efeitos de topografia, como por exemplo na região da Serra de Córdoba na Argentina, a qual \cite{Rasmussen2011} mostram grande ocorrência de convecção profunda, promoveram sistemas em que a estrutura tridimensional da precipitação dos núcleos de raios atingiram valores de $Z_c$ superiores a 45 dBZ entre 10-15 km de altitude e chuvas na superfície com $Z_c$ acima de 55 dBZ, como mostram os contornos de probabilidade entre 0.001-0.5\%.

\subsection{A precipitação dos sistemas severos e o perfil atmosférico de temperatura.}

Os diagramas CCFTD e CFTD, descritos em \ref{chuvaEtemperatura}, são expostos nas figuras \ref{ccftd_fta_com}, \ref{ccftd_ft_com}, \ref{cftd_fta_com} e \ref{cftd_ft_com}, associados as tempestades elétricas com índice extremo de FTA e FT, apenas em suas porções com raios.

A partir dos CCFTDs das figuras \ref{ccftd_fta_com} e \ref{ccftd_ft_com}, iremos avaliar a intensidade convectiva dos sistemas com índice extremo de FTA e FT em determinadas regiões, com base na velocidade de crescimento ou decrescimento dos valores de $Z_{c}$ associados os contornos de probabilidade do 30\textsuperscript{\underline{o}}, 50\textsuperscript{\underline{o}}, 70\textsuperscript{\underline{o}} e 95\textsuperscript{\underline{o}} percentil das amostras de probabilidades expressas nos CFTDs das figuras \ref{cftd_fta_com} e \ref{cftd_ft_com}.	

\begin{figure}[!ht]
  \centering
  \includegraphics[height=13.5cm]{img/precipitacao3d/severo/cftd/cftd_10deg_comraio_topFTA}
 \caption{CFTDs para os extremos de FTA. Porção da precipitação com raios.}
 \label{cftd_fta_com}
\end{figure} 

\begin{figure}[!ht]
  \centering
  \includegraphics[height=13.5cm]{img/precipitacao3d/severo/cftd/ccftd_10deg_comraio_topFTA}
  \caption{CCFTDs para os extremos de FTA. Porção da precipitação com raios.}
  \label{ccftd_fta_com}   
\end{figure} 

  %\caption{Diagramas de Contorno de Frequência por Temperatura (CFTD) e Diagramas de Contorno de Frequência Cumulativa por Temperatura (CCFTD). Em cada CFTD e CCFTD, pode-se verificar: a porcentagem (\%) de perfis convectivos, estratiformes e outros, respectivamente; (P) o numero de perfis do PR computados, (L) o número de ocorrência de $Z_{c}$ no nível de temperatura de máxima ocorrência e (T) o nível de temperatura de máxima ocorrência de $Z_{c}$.}
 

\begin{figure}[!ht]
  \centering
  \includegraphics[height=13.5cm]{img/precipitacao3d/severo/cftd/cftd_10deg_comraio_topFT}
 \caption{CFTDs para os extremos de FT. Porção da precipitação com raios.}
 \label{cftd_ft_com}
\end{figure} 

\begin{figure}[!ht]
  \centering
  \includegraphics[height=13.5cm]{img/precipitacao3d/severo/cftd/ccftd_10deg_comraio_topFT}
  \caption{CCFTDs para os extremos de FT. Porção da precipitação com raios.}
  \label{ccftd_ft_com}   
\end{figure} 

Então, para a região central da Bacia do Rio Amazonas, entre 10-0S e 70-60W, extraí-se as linhas de contorno do CCFTD referentes as probabilidades acumulativas de 30\%, 50\%, 70\% e 95\%. Desta forma obtemos quatro funções $f(x)=y$, \simbolo{name={$f(x)=y$},description={Função de uma variável}} em que $y$ corresponde aos valores de $Z_c$ e $x$ o perfil atmosférico de temperatura. Fazendo a derivada $\dfrac{dy}{dx}$ pode-se avaliar taxa de variação de $Z_c$ por temperatura (dBZ/\textsuperscript{o}C), para diferentes regimes de chuva, das mais frequentes até as mais raras, como mostra a figura \ref{deriv_amazonas}.

\begin{figure}[!ht]
  \centering
  \includegraphics[height=9cm]{img/precipitacao3d/deriv_ccftd/deriv_contornos_cdf_2_1}
  \caption{Taxa de variação de $Z_c$ no perfil de temperatura atmosférico para a região central da Bacia do Rio Amazonas, entre 10-0S e 70-60W.}
  \label{deriv_amazonas}  
\end{figure} 

% nos diferentes quartis do CCFTD dos extremos de FTA,  figura \ref{ccftd_fta_com},  e dos extremos de FT, figura \ref{ccftd_ft_com}. 

Na figura \ref{deriv_amazonas}, observa-se que a taxa de aumento de $Z_c$ em torno de -40 \textsuperscript{o}C e -15 \textsuperscript{o}C, é maior para os extremos de FTA, porém em torno de 0 \textsuperscript{o}C, a taxa de aumento de $Z_c$ é maior para os extremos de FT, mostrando que os hidrometeoros dos sistemas extremos de FTA, crescem em regiões mais frias do que nos extremos de FT.

% agregação e acreção é maior para os sistemas extremos de FTA e na região de derretimento, em torno de 0 \textsuperscript{o}C, o efeito da banda brilhante é mais acentuado para os extremos de FT.  

O aumento do fator de refletividade em torno de 0 \textsuperscript{o}C está associado a mudança do índice de refração da água devido a sua fusão. Já o aumento do fator de refletividade em torno de -40 \textsuperscript{o}C e -15 \textsuperscript{o}C representam o crescimento de hidrometeoros por agregação e acreção \cite{Fabry1995,Takahashi1978}.

Note na figura \ref{deriv_amazonas}, como a precipitação do 95\textsuperscript{\underline{o}} percentil de probabilidade de ocorrência tanto para FTA quanto para FT, é o regime de precipitação mais severa. Pois, há o crescimento de $Z_c$ em torno de -10 \textsuperscript{o}C e -15 \textsuperscript{o}C e não há banda brilhante, indicando precipitação a partir de granizo\footnote{Em \cite{Fabry1995}, este tipo de perfil é discutido como chuva a partir de gelo compacto.}. No 30\textsuperscript{\underline{o}}, 50\textsuperscript{\underline{o}} e 70\textsuperscript{\underline{o}} percentil dos extremos de FT, o efeito da banda brilhante associada ao derretimento é mais evidente do que para os extremos de FTA. 

Quando comparamos a região central da Bacia do Rio Amazonas, com a região central da Bacia do Rio da Prata, entre 30-20S e 60-50W, a microfísica de eletrificação se mostra diferente em cada local. Observa-se que no 50\textsuperscript{\underline{o}} percentil, a taxa de crescimento de $Z_c$ entre -40 \textsuperscript{o}C e -20 \textsuperscript{o}C é maior para a região da Bacia do Prata, figura \ref{deriv_prata}, do que para a região da Bacia Amazônica, figura \ref{deriv_amazonas}, tanto para os sistemas extremos de FTA quando para os sistemas extremos de FT, indicando maior crescimento de flocos de neve na precipitação severa sobre a Bacia do Prata. 


\begin{figure}[!ht]
  \centering
  \includegraphics[height=9cm]{img/precipitacao3d/deriv_ccftd/deriv_contornos_cdf_3_3}
  \caption{Taxa de variação de $Z_c$ no perfil de temperatura atmosférico para a região central da Bacia do Rio da Prata, entre 30-20S e 60-50W.}
  \label{deriv_prata}  
\end{figure} 

Apesar do 95\textsuperscript{\underline{o}} percentil mostrar maiores taxas de dBZ/\textsuperscript{o}C, em -15 \textsuperscript{o}C tanto para FTA quanto FT sobre a Bacia do Rio Amazonas, na figura \ref{deriv_amazonas}, do que sobre a Bacia do Rio da Prata, na figura \ref{deriv_prata}, os contorno de probabilidade acumulativa de 95\% nos CCFTD das figuras \ref{ccftd_fta_com} e \ref{ccftd_ft_com}, em -15 \textsuperscript{o}C, mostram valores de $Z_c$ de aproximadamente 3 dBZ superiores na região da Bacia Platina. Mesmo que o 95\textsuperscript{\underline{o}} percentil mostre maior crescimento de hidrometeoros na região mista sobre a Bacia Amazônica, a precipitação do 95\textsuperscript{\underline{o}} percentil na Bacia do Prata foi mais severa, pois possui maiores valores de $Z_c$.

O aumento abrupto de $Z_c$ associado a fusão da água, entre 30-20S e 60-50W, figura \ref{deriv_prata}, principalmente do 50\textsuperscript{\underline{o}} e 70\textsuperscript{\underline{o}} percentil, ocorrem em -4 \textsuperscript{o}C, enquanto que, entre 10-0S e 70-60W, figura \ref{deriv_amazonas}, o aumento de $Z_c$ ocorre mais próximo de 0 \textsuperscript{o}C, o que indica maior presença de água super-resfriada associada ao processo de derretimento da precipitação entre 30-20S e 60-50W, região da Bacia do Rio da Prata.  

Na região da Bacia do Prata, representada na figura \ref{deriv_prata}, o efeito de banda brilhante também é mais pronunciado para a precipitação com raios dos extremos de FT, o que mostra que as regiões eletricamente ativas da precipitação dos sistemas com índice extremo de FTA é menos estratificada do que nos extremos de FT, em ambas as Bacias Hidrológicas: da Prata e do Amazonas.

\newpage
\section{SEVERIDADE REGIONALIZADA}


\begin{figure}[!ht]
  \centering{
  \subfloat[5\textsuperscript{\underline{o}} percentil de FTA]{{\includegraphics[height=6.5cm, trim=0 0 0 0, clip]{img/DistEspacialPercentis/FTA/distEspacialValor005thFta}} \label{5oFta}}
  \subfloat[10\textsuperscript{\underline{o}} percentil de FTA]{{\includegraphics[height=6.5cm, trim=0 0 0 0, clip]{img/DistEspacialPercentis/FTA/distEspacialValor010thFta}} \label{10oFta}}
  }    
  \caption{Distribuição espacial dos valores do 5\textsuperscript{\underline{o}} e 10\textsuperscript{\underline{o}} percentil da amostra de probabilidade do índice FTA a cada região de 2.5 por 2.5 graus.}
\label{extremosInfFTA}
\end{figure} 





\begin{figure}[!ht]
  \centering{
  \subfloat[5\textsuperscript{\underline{o}} percentil de FT]{{\includegraphics[height=6.5cm, trim=0 0 0 0, clip]{img/DistEspacialPercentis/FT/distEspacialValor005thFt}} \label{5oFta}}
  \subfloat[10\textsuperscript{\underline{o}} percentil de FT]{{\includegraphics[height=6.5cm, trim=0 0 0 0, clip]{img/DistEspacialPercentis/FT/distEspacialValor010thFt}} \label{10oFta}}
  }    
  \caption{Distribuição espacial dos valores do 5\textsuperscript{\underline{o}} e 10\textsuperscript{\underline{o}} percentil da amostra de probabilidade do índice FT a cada região de 2.5 por 2.5 graus.}
\label{extremosInfFT}
\end{figure} 





\begin{figure}[!ht]
  \centering{
  \subfloat[95\textsuperscript{\underline{o}} percentil de FTA]{{\includegraphics[height=6.5cm, trim=0 0 0 0, clip]{img/DistEspacialPercentis/FTA/distEspacialValor095thFta}} \label{95oFta}}
  \subfloat[99\textsuperscript{\underline{o}} percentil de FTA]{{\includegraphics[height=6.5cm, trim=0 0 0 0, clip]{img/DistEspacialPercentis/FTA/distEspacialValor099thFta}} \label{99oFta}}
  }    
  \caption{Distribuição espacial dos valores do 95\textsuperscript{\underline{o}} e 99\textsuperscript{\underline{o}} percentil da amostra de probabilidade do índice FT a cada região de 2.5 por 2.5 graus.}
\label{extremosSupFTA}
\end{figure} 







\begin{figure}[!ht]
  \centering{
  \subfloat[95\textsuperscript{\underline{o}} percentil de FT]{{\includegraphics[height=6.5cm, trim=0 0 0 0, clip]{img/DistEspacialPercentis/FT/distEspacialValor095thFt}} \label{95oFt}}
  \subfloat[99\textsuperscript{\underline{o}} percentil de FT]{{\includegraphics[height=6.5cm, trim=0 0 0 0, clip]{img/DistEspacialPercentis/FT/distEspacialValor099thFt}} \label{99oFt}}
  }    
  \caption{Distribuição espacial dos valores do 95\textsuperscript{\underline{o}} e 99\textsuperscript{\underline{o}} percentil da amostra de probabilidade do índice FT a cada região de 2.5 por 2.5 graus.}
\label{extremosSupFT}
\end{figure} 

