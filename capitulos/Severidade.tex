\chapter{TEMPESTADES ELÉTRICAS SEVERAS}

Conforme descrito em \ref{metodoFtaFt}, as taxas de raios das tempestades elétricas neste trabalho de pesquisa estão associadas aos índices FTA e FT. Nesta seção identifica-se qual desses índices podem melhor associar-se com a intensidade convectiva das tempestades elétricas.
%, ou seja, os sistemas com as maiores taxas de raios por minuto (FT) ou os sistemas com as maiores taxas de raios por minuto por quilômetro quadrado (FTA) da sua extensão. 

Para o estudo de intensidade dos sistemas, foram selecionados apenas as tempestades elétricas as quais possuíram $VT_m$ maior ou igual a 1 minuto e com pelo menos um pixel do campo de visão do PR contido na área do sistema, totalizando 94,733 tempestades elétricas do TRMM. %Como a intensidade convectiva dos sistemas é avaliada com base, principalmente, na morfologia da estrutura tridimensional da precipitação e na taxa de raios, 

As equações \ref{eqFT} e \ref{eqFTA} foram aplicadas nas 94,733 tempestades elétricas selecionadas, e então estudada as distribuições de probabilidades dos índices FTA e FT (figuras \ref{pdfFTAFT} e \ref{cdfFTAFT}). Conforme mostra a figura \ref{pdfFTAFT}, trata-se de distribuições exponenciais de probabilidade. Os valores de FTA e FT para cada quantil da amostragem de tempestades elétricas, tanto para as mais frequentes quanto as mais raras, podem ser verificados por meio da distribuição cumulativa de probabilidade de FTA e FT mostradas na figura \ref{cdfFTAFT}. 

\begin{figure}[!ht]
  \centering
  \includegraphics[height=9cm]{img/FtaFt/pdf_FTA_FT}      
  \caption{Densidade de probabilidade de FTA e FT.} 
   \label{pdfFTAFT} 
\end{figure}

\begin{figure}[!hb]
  \centering 
  \includegraphics[height=9cm]{img/FtaFt/cdf_FTA_FT} 
  \caption{Densidade de probabilidade cumulativa de FTA e FT.}
  \label{cdfFTAFT}
\end{figure}
%\label{seriesFtaFt}

Os sistemas potencialmente severos são selecionados pelo 90\textsuperscript{\underline{o}} percentil das amostras de probabilidades de FTA e FT, que estão associado aos maiores e mais raros valores ocorridos. Portanto, pressupõe-se que as tempestades elétricas as quais provavelmente causaram chuva de granizo, rajadas de ventos com queda de árvores e construções ou tornados estão associadas aos sistemas com valores extremos de FTA ou FT.

O grupo das tempestades elétricas com FTA extremo, possuíram valores entre {29.3--1258.7 $\times$ 10$^{-4}$} raios por minuto por quilômetro quadrado (minuto$^{-1}$ km$^{-2}$), enquanto as tempestades elétricas com extremos de FT possuíram valores entre {47.2--1283.6} raios por minuto (minuto$^{-1}$). 

Observe que o valor mínimo de FT foi de 0.6 raios por minuto. Em \citeonline{cecil2005}, considera-se que, a mínima taxa de raios no tempo para as PFs é de 0.7 raios por minutos. Porém a resolução espacial da projeção do tempo de visada do LIS utilizada nesta tese possui resolução de 0.25$^{\circ}$ $\times$ 0.25$^{\circ}$  \cite{albrecht2009tropical,albrecht2011b}. Ao considerar a velocidade e altura da órbita do satélite, o tempo de observação do LIS em um ponto de 0.25$^{\circ}$ $\times$ 0.25$^{\circ}$ na superfície terrestre pode atingir até 102 segundos na região zenital. Então, as tempestades elétricas que possuíram apenas 1 raio e $VT_m$ de $\simeq$100 segundos, tiveram o mínimo valor de FT de 0.6 raios min$^{-1}$, sendo esta, a mínima taxa de raios no tempo detectável em uma tempestades elétricas do TRMM desta pesquisa.

% Na figura \ref{percetilFtaFt}, temos a série de FTA e FT ordenada, e a linha tracejada vertical corta o 90\% percentil dos índices. 

%\begin{figure}[!ht]
%  \centering
%  \includegraphics[height=6cm]{img/FtaFt/90thFtaFt}	 
%  \caption{90\textsuperscript{\underline{o}} percentil de FTA e FT.}
%  \label{percetilFtaFt}
%\end{figure}


\section{ÁREA E TEMPERATURA DO TOPO DA NUVEM}

Observa-se que os extremos de FTA e FT correspondem a sistemas com tamanhos bem distintos. Conforme é mostrado na figura \ref{size}, verifica-se que as máximas probabilidades de ocorrência de tempestades elétricas associadas com os extremos de FTA, ocorrem em sistemas com área 3 ordens de grandeza menor do que nos extremos de FT.

\begin{figure}[!ht]
  \centering
  \includegraphics[height=9cm,trim=0 0 215 0,clip]{img/tb/TbAreas}   
  \caption{Densidade de probabilidade de extensão em área das tempestades elétricas com extremos de FTA e FT.}
  \label{size}  
\end{figure}


%\begin{figure}[!hb]
%  \centering{  
%  \subfloat[Densidade de probabilidade de extensão em área.] { \includegraphics[height=7.5cm,trim=0 0 215 0,clip]{img/tb/TbAreas} \label{size}} \\
%  \subfloat[Densidade de probabilidade de temperatura de brilho em infravermelho.]{ \includegraphics[height=7.5cm,trim=220 0 0 0,clip]{img/tb/TbAreas} \label{tb}} 
%  }
%  \label{t_tb}
%  \caption{Estudo das frequências de ocorrências de tempestades elétricas selecionas pelo 90\textsuperscript{\underline{o}} percentil dos índices de FT e FTA, por extensão em área e por temperatura de brilho de topo das nuvens.}
%\end{figure}

%As tempestades elétricas com valores extremos de FT são maiores em extensão.
Na figura \ref{areaFTAFTA}, pode-se observar que os sistemas com tamanho entre 10$^2$--10$^3$ km$^2$, não ultrapassam 20 raios min$^{-1}$ de FT. As tempestades elétricas com FT superior a 100 raios por minuto, possuíram tamanho entre 10$^{4}$--10$^{6}$ km$^2$. Note que há tendencia de aumento exponencial de FT conforme aumenta a extensão das tempestades elétricas, no  entanto, FTA tende a diminui exponencialmente com o aumento da área das tempestades elétricas.

%Uma tempestade elétrica com 10$^5$ km$^2$, terá maior número de descargas observadas durante o tempo de visada do LIS do que uma com 10$^2$ km$^2$.

\begin{figure}[!ht]
  \centering
  \includegraphics[height=9cm]{img/FtaFt/area_FTA_FT}   
  \caption{Dispersão entre as áreas das tempestades elétricas e os valores de FTA e FT. As linhas horizontais marcam os valores de FTA (cor preta) e FT (cor azul) referente ao 90\textsuperscript{\underline{o}} percentil das respectivas amostragem.}
  \label{areaFTAFTA}  
\end{figure}

\begin{figure}[!hb]
  \centering 
  \includegraphics[height=9cm]{img/FtaFt/volChuva_FTA_FT}
  \caption{Dispersão entre o volume de chuva das tempestades elétricas e os valores de FTA e FT.  As linhas horizontais marcam os valores de FTA (cor preta) e FT (cor azul) referente ao 90\textsuperscript{\underline{o}} percentil das respectivas amostragem.}
  \label{volchuvaFTAFT}
\end{figure}

%\begin{figure}[!hb]
%  \centering 
%  \subfloat[areas versus taxa de raios]{ \includegraphics[height=7.5cm]{img/FtaFt/area_FTA_FT}\label{areaFTAFTA}} \\
%  \subfloat[volume de chuva]{\includegraphics[height=7.5cm]{img/FtaFt/volChuva_FTA_FT}\label{pdfFt}} 
%  \caption{Dispersão referente aos índices FTA e FT.}
%  \label{areaFtaFt}
%\end{figure} 
%Quando a densidade espacial de descargas aumenta muito em uma região com centenas de quilômetros quadrados, em torno de 100 descargas intranuvens para uma nuven-solo, como por exemplo os maiores valores de $Z=IC/CG$ mostrados por \cite{evandro2009} na região de Campo Grande - MS no Brasil, a capacidade do LIS de identificar brilhos transientes provavelmente fica comprometida devido a resolução horizontal da CCD.

Ao normalizar a taxa de raios no tempo por $A_t$, o número de raios fica diluído na extensão do sistema, evidenciando que os maiores valores de FTA correspondem aos sistemas com as maiores densidades espaciais de raios, cuja a área e o número de raios são menores do que nos sistemas com extremos de FT.

Na figura \ref{volchuvaFTAFT}, observa-se que conforme aumenta FT o volume de chuva das tempestades elétricas também aumenta exponencialmente, de maneira semelhante ao aumento de FT com a área (figura \ref{areaFTAFTA}). Para FTA, há um comportamento inverso. Conforme aumenta FTA, o volume de chuva dos sistemas diminui.  

A frequência de ocorrência dos pixeis de temperaturas de brilho, associados a radiância espectral em infravermelho observada pelo VIRS, os quais definiram as áreas das tempestades elétricas, é mostrado na figura \ref{tb}. Observa-se que o maior valor de probabilidade para a curva das tempestades elétricas com índice extremo de FTA, possui temperatura de topo de nuvens aproximadamente 10 K mais frias do que nas tempestades elétricas com extremos de FT, indicando que a convecção nos sistemas extremos de FTA é mais profunda na atmosfera na maioria das observações.

\begin{figure}[!ht]
  \centering 
  \includegraphics[height=9cm,trim=220 0 0 0,clip]{img/tb/TbAreas}
  \caption{Densidade de probabilidade de temperatura de brilho em infravermelho (VIRS 10.8$\mu$m) do topo das nuvens das tempestades elétricas com extremos de FTA e FT.}
  \label{tb}
\end{figure}

\citeonline{morales2003} ao desenvolver a \textit{Sferics Infrared Rainfall Technique} (SIRT), mostram que as regiões com temperatura de brilho inferior a 215 K e com ocorrência de \textit{sferics} foram as regiões categorizadas como de maior precipitação associada. Nesta tese, ao selecionar as tempestades elétricas com índice extremo de FTA, os maiores valores de probabilidade de ocorrência, conforme é mostrado na figura \ref{tb}, concentram-se em temperaturas de brilho abaixo de 215 K.

\sigla{name={SIRT},description={\textit{Sferics Infrared Rainfall Technique} }}

Nas figuras \ref{pdffracaoFTA}, \ref{cdffracaoFTA}, \ref{pdffracaoFT} e  \ref{cdffracaoFT}, apresenta-se o estudo das probabilidades das frações de chuva para as tempestades elétricas com valores extremos de FTA e FT. As curvas denominadas como convectivo, estratiforme e outros, correspondem a fração de área de chuva associada aos perfis do PR classificados como convectivo, estratiforme e outros em relação a área total de chuva do sistema. A curva denominada na legenda como chuva total corresponde a fração da área de chuva em relação a área total ($A_t$) da tempestade elétrica. A curva denominada como varredura do PR, mostra a fração da $A_t$ que esteve dentro do alcance da varredura do PR.

\begin{figure}[!ht]
  \centering
  \includegraphics[height=9cm]{img/FtaFt/fracaoChuva_pdf_topFTA}   
  \caption{Densidade de probabilidade das frações de áreas de chuva das tempestades elétricas com extremos de FTA, que foram classificadas (2A23) como convectiva (vermelha), estratiforme (bege) e outros (verde), em relação a toda a área de chuva observada pelo PR; das frações de chuva total (preta), que corresponde as frações das áreas de chuva observadas pelo PR em relação a $A_t$ das tempestades elétricas e também das frações das áreas das tempestades elétricas contidas na varredura do PR (azul).}
  \label{pdffracaoFTA}  
\end{figure}

\begin{figure}[!ht]
  \centering 
  \includegraphics[height=9cm]{img/FtaFt/fracaoChuva_cdf_topFTA}
  \caption{Densidade de probabilidade cumulativa das frações de áreas de chuva das tempestades elétricas com valores extremos de FTA.}
  \label{cdffracaoFTA}
\end{figure}


\begin{figure}[!ht]
  \centering
  \includegraphics[height=9cm]{img/FtaFt/fracaoChuva_pdf_topFT}   
  \caption{Densidade de probabilidade das frações de áreas de chuva das tempestades elétricas com extremos de FT, que foram classificadas (2A23) como convectiva (vermelha), estratiforme (bege) e outros (verde), em relação a toda a área de chuva observada pelo PR; das frações de chuva total (preta), que corresponde as frações das áreas de chuva observadas pelo PR em relação a $A_t$ das tempestades elétricas e também das frações das áreas das tempestades elétricas contidas na varredura do PR (azul).}
  \label{pdffracaoFT}  
\end{figure}

\begin{figure}[!ht]
  \centering 
  \includegraphics[height=9cm]{img/FtaFt/fracaoChuva_cdf_topFT}
  \caption{Densidade de probabilidade cumulativa das frações de áreas de chuva das tempestades elétricas com valores extremos de FT.}
  \label{cdffracaoFT}
\end{figure}

Avaliando a densidade de probabilidade da fração convectiva e fração estratiforme das tempestades elétricas, figuras \ref{pdffracaoFTA} e \ref{pdffracaoFT}, verifica-se que para os extremos de FTA as tempestades elétricas são mais frequentemente observadas com 70\% de área convectiva e 30\% de área estratiforme, enquanto que para os extremos de FT, 20\% de fração convectiva e 75\% de fração estratiforme.
%juntamente com a as respectivas distribuições de probabilidade acumulativa,

Verifica-se nas figuras \ref{pdffracaoFT} e \ref{cdffracaoFT} que para a maioria dos sistemas com FT extremo, o PR conseguiu observar 25\% da $A_t$ das tempestades elétricas. Nas figuras \ref{pdffracaoFTA} e \ref{cdffracaoFTA} das frações de chuva dos sistemas com FTA extremo, o PR observou com maior frequência 100\% da área das tempestades elétricas. Portanto a fração de chuva total dos extremos de FT (figura \ref{pdffracaoFT}) possui um valor de apenas $\simeq$10\% devido a varredura do PR ser menor do que a do VIRS e as tempestades elétricas com os maiores valores de FT abrangerem uma extensão que ultrapassa o alcance do PR. 


Os sistemas selecionados pelo 90\textsuperscript{\underline{o}} percentil do índice FT possuem maior extensão em área e maior volume de chuva. São sistemas com vasta extensão estratiforme conforme descrevem \citeonline{Rasmussen2011}. As regiões das tempestades elétricas com precipitação convectiva, as quais tem potencial de gerar chuva de granizo, frentes de rajada e tornados, ocupam área bem menor do que as áreas com  precipitação estratiforme \cite{Jr2007}.

A maior fração convectiva e menor tamanho das tempestades elétricas com FTA extremos sugerem sistemas em estágio de maturação. Conforme as tempestades elétricas vão entrando em estágio maduro e dissipativo, vão ganhando área de chuva estratiforme e podem começar a se enquadrar no grupo dos extremos de FT. 

%.....
%Para avaliar qual dos índices representaram a maior severidade de tempo, a morfologia da estrutura 3D da precipitação foi estudada por meio dos diagramas CFAD, CCFAD, CFTD e CCFTD. %para os 10\% das amostras de FT e FTA com os maiores valores.
%......

\section{SEVERIDADE COM BASE NA ESTRUTURA 3D DA PRECIPITAÇÃO}

Nesta etapa iremos avaliar a intensidade convectiva com base nos perfis de $Z_c$ do PR, contidos nos sistemas com índices extremos de FTA e FT. 

Nas figuras \ref{ftacfadwithout}, \ref{ftcfadwithout},  \ref{ftacfadwith} e \ref{ftcfadwith} foram calculados os CFADs das tempestades elétricas com índices FTA e FT extremos, para cada região de 10 por 10 graus na superfície sobre a AS. Para localizar a caixa de 10 por 10 graus em que cada sistema esteve contido, foi considerado a latitude e longitude do centro geométrico da área definida por cada sistema. Desta forma, foram obtidas as amostragens de probabilidade de FTA e FT, da mesma maneira que mostrado na figura \ref{pdfFTAFT} ou \ref{cdfFTAFT}, porém para cada região de 10$^{\circ}$ $\times$ 10$^{\circ}$. Desta maneira, o estudo da precipitação tridimensional é feito apenas para o 90\textsuperscript{\underline{o}} percentil das amostragem de FTA e FT regionalizada a cada 10$^{\circ}$ $\times$ 10$^{\circ}$.


\begin{figure}[hb]
  \centering{  
  \subfloat[]{\includegraphics[height=1.0cm]{img/grids/nucleosRaios/colorbar_virs}\label{barravirs}}\\
  \subfloat[]{\includegraphics[height=6.cm,trim=0 47cm 0 0,clip]{img/grids/nucleosRaios/001_topFTA_25471_0003} \label{nr1}} 
  \subfloat[]{\includegraphics[height=6.cm,trim=0 47cm 0 0,clip]{img/grids/nucleosRaios/002_topFTA_36502_0001} \label{nr2}} \\
  \subfloat[]{\includegraphics[height=6.cm,trim=0 47cm 0 0,clip]{img/grids/nucleosRaios/003_topFTA_03444_0001} \label{nr3}} 
  \subfloat[]{\includegraphics[height=6.cm,trim=0 47cm 0 0,clip]{img/grids/nucleosRaios/004_topFTA_05694_0005} \label{nr4}} 
  }
  \caption{Núcleos de raios das tempestades elétricas. Os pontos na cor  verde são os eventos e os símbolos de positivo na cor preta são os raios. Os pixeis em vermelho são as regiões dos núcleos de raios, definidas a partir da projeção dos eventos em uma grade regular de 0.05$^{\circ}$.} %(0.05$^{\circ}$ $\times$ 0.05$^{\circ}$)
\label{nucleosRaios}
\end{figure}



%----------------------------------------
\begin{sidewaysfigure}%[!H]
\centering
\includegraphics[width=19.5cm]{img/precipitacao3d/severo/percentil/90th/cfad10_semraio_topFTA_percentil}
\caption{CFADs para os extremos de FTA. Porção da precipitação sem raios.}
\label{ftacfadwithout}
\end{sidewaysfigure} 


\begin{sidewaysfigure}%[!H]
\centering
\includegraphics[width=19.5cm]{img/precipitacao3d/severo/percentil/90th/cfad10_semraio_topFT_percentil}
\caption{CFADs para os extremos de FT	. Porção da precipitação sem raios.}
\label{ftcfadwithout}
\end{sidewaysfigure} 
%----------------------------------------

%\begin{figure}[!ht]
%  \centering
%  \includegraphics[height=13.5cm]{img/precipitacao3d/severo/percentil/90th/cfad10_semraio_topFTA_percentil}
% \caption{CFADs para os extremos de FTA. Porção da precipitação sem raios.}
% \label{ftacfadwithout}
%\end{figure} 

\begin{sidewaysfigure}%[!H]
  \centering
  \includegraphics[width=19.5cm]{img/precipitacao3d/severo/percentil/90th/cfad10_comraio_topFTA_percentil}
  \caption{CFADs para os extremos de FTA. Porção da precipitação com raios.}
  \label{ftacfadwith}   
\end{sidewaysfigure} 


\begin{sidewaysfigure}%[!H]
  \centering
  \includegraphics[width=19.5cm]{img/precipitacao3d/severo/percentil/90th/cfad10_comraio_topFT_percentil}
  \caption{CFADs para os extremos de FT. Porção da precipitação com raios.}
  \label{ftcfadwith}   
\end{sidewaysfigure} 

%\begin{figure}[!ht]
%  \centering
%  \includegraphics[height=13.5cm]{img/precipitacao3d/severo/percentil/90th/cfad10_comraio_topFTA_percentil}
%  \caption{CFADs para os extremos de FTA. Porção da precipitação com raios.}
%  \label{ftacfadwith}   
%\end{figure} 

	

As posições geográficas dos eventos do LIS e dos perfis de $Z_c$ válidos do PR, foram projetadas em uma grade regular de 0.05 graus. Os perfis de $Z_c$ projetados em pontos de grade em que tiveram eventos do LIS, definiram as regiões aqui denominadas como precipitação dos núcleos de raios. A figura \ref{nucleosRaios}, mostra as regiões dos núcleos de raios das tempestades elétricas. Observe que na parte superior e inferior das figuras \ref{nr1}, \ref{nr2}, \ref{nr3} e \ref{nr4}, há informações referentes a data e hora em que o sistema foi observado, número de raios/eventos (FL/EV), fração do sistema observado pelo PR, área do sistema (A), semi-eixo maior (a), menor (b), distância focal (2c) e excentricidade (e) de uma elipse ajustada as dimensões do sistema. A barra de cores na figura \ref{barravirs}, corresponde as temperaturas de brilho ($<$258K) associada a radiância espectral em infravermelho observada pelo VIRS.


Após definidas as regiões eletricamente ativas de cada sistema com FTA ou FT extremo, os CFADs foram calculados para a precipitação dos núcleos de raios das tempestades elétricas, figuras \ref{ftacfadwith} e \ref{ftcfadwith}, e para a precipitação fora dos núcleos de raios, figuras \ref{ftacfadwithout} e \ref{ftcfadwithout}.

Note que no canto superior direito de cada CFAD temos alguns valores estatísticos que representam: (\%)  a porcentagem de perfis convectivos, estratiformes e outros, respectivamente; (P) o número de perfis do PR computados; (L) o número de ocorrência de $Z_c$ no nível de altitude de máxima ocorrência; (H) o nível de altitude, em quilômetros, aonde ocorreu o máximo de ocorrências de $Z_c$; (N) o número de tempestades elétricas computadas.

\simbolo{name={\%},description={Nos diagramas, CFAD, CCFAD, CFTD e CCFTD, representam: a porcentagem de perfis convectivos, estratiformes e outros, respectivamente}}
\simbolo{name={P},description={Nos diagramas, CFAD, CCFAD, CFTD e CCFTD, representam: número de perfis do PR computados}}
\simbolo{name={L},description={Nos diagramas, CFAD, CCFAD, CFTD e CCFTD, representam: o número de ocorrência de $Z_c$ no nível de altitude de máxima ocorrência}}
\simbolo{name={H},description={Nos diagramas, CFAD, CCFAD, CFTD e CCFTD, representam: o nível de altitude, em quilômetros, aonde ocorreu o máximo de ocorrências de $Z_c$}}
\simbolo{name={N},description={Nos diagramas, CFAD, CCFAD, CFTD e CCFTD, representam: o número de tempestades elétricas computadas}}

Comparando os CFADs da chuva com e sem raios, representados para os extremos de FTA nas figuras \ref{ftacfadwith} e \ref{ftacfadwithout} e para os extremos de FT, nas figuras \ref{ftcfadwith} e \ref{ftcfadwithout}, é evidente que a fração das tempestades elétricas sem raios é a parte de menor velocidade vertical e a fração eletricamente ativa é a região com maior velocidade vertical. Os níveis de contorno de probabilidades dos CFADs da precipitação sem raios possuem suas máximas altitudes aproximadamente 3 quilômetros abaixo das máximas altitudes atingidas pelos contornos dos CFADs da precipitação com raios. A fração da precipitação sem raios dos sistemas é composta predominantemente por perfis estratiformes enquanto que a fração com raios  os perfis convectivos são predominantes.


%\begin{figure}[!ht]
%  \centering
%  \includegraphics[height=13.5cm]{img/precipitacao3d/severo/percentil/90th/cfad10_semraio_topFT_percentil}%
% \caption{CFADs para os extremos de FT. Porção da precipitação sem raios.}
% \label{ftcfadwithout}
%\end{figure} 

%\begin{figure}[!ht]
%  \centering
%   \adjustbox{trim={0\width} {0.435\height} {0\width} {0\height} , clip}%
%   {\includegraphics[width=\textwidth]{img/precipitacao3d/severo/percentil/90th/cCumFad_10deg_semraio_topFTpercentil}}
% \caption{CCFDs para os extremos de FT entre 20S-10N e 90W-30W. Porção da precipitação sem raios.}
% \label{ftccfadwithout}
%\end{figure} 

Se avaliarmos apenas os níveis de contorno com probabilidade entre 2-3.7\% (cor verde), os valores de $Z_c$ entre 0-2 km de altitude da porção sem raios, figuras \ref{ftacfadwithout} e \ref{ftcfadwithout}, não ultrapassaram 40 dBZ em nenhuma região da AS, enquanto que para a porção de chuvas com raios, figuras \ref{ftacfadwith} e \ref{ftcfadwith}, os valores de $Z_c$ atingem 45-50 dBZ. Entre 5-7 km os contornos de probabilidade dos CFADs da precipitação com raios mostram valores de $Z_c$ entre 5-10 dBZ maiores do que na precipitação sem raios, pois a eletrificação das nuvens depende do crescimento dos hidrometeoros na região de fase mista. Havendo maior velocidade vertical, descargas elétricas e maior volume de água na região de fase mista, consequentemente, espera-se maiores volumes de chuvas na superfície (0-2 km) sobre as regiões dos núcleos de raios \cite{Petersen1998}. 


%A convecção é mais ativa nas regiões dos núcleos de raios, aonde a precipitação está associadas com frentes de rajadas, chuvas de granizo e enchentes rápidas. Fora dos núcleos de raios temos a parte da precipitação mais estratiforme, composta por hidrometeoros que não possuem velocidade terminal suficiente para precipitar nos núcleos de raios, e caem mais afastados da região eletricamente ativa.     % Dependendo principalmente das condições de calor umidade e cisalhamento vertical do vento as células 


% \caption{Diagramas de Contorno de Frequência por Altitude (CFADs). Em cada CFAD pode-se verificar: a porcentagem (\%) de perfis convectivos, estratiformes e outros, respectivamente; (P) o numero de perfis do PR computados, (L) o número de ocorrência de refletividade no nível de máxima ocorrência e (H) o nível de máxima ocorrência.}

A figura \ref{ftcfadwithout} mostra que a precipitação sem raios dos extremos de FT na região tropical, entre 20S-10N e 90W-30W, possui banda brilhante marcada entre 4-5 km de altitude, principalmente nos perfis com probabilidade de ocorrência entre 2-5.3\%, nas cores de contorno em verde e amarelo. 

Podemos observar a banda brilhante entre 20S-10N e 90W-30W  dos sistemas com  extremo de FT na porção sem raios de maneira mais elucidativa por meio dos CCFADs da figura \ref{ftccfadwithout}, os quais evidenciam que entre o 12\textsuperscript{\underline{o}} e o 95\textsuperscript{\underline{o}} percentil da amostragem de probabilidade de $Z_c$ por altitude, há uma queda no valor de $Z_c$ logo abaixo de 5 quilômetros de altitude em cada região de 10 por 10 graus. 

No entanto, para a região entre 20S-10N e 90W-30W, ao avaliar os CFADs da figura \ref{ftacfadwithout} ou CCFADs da figura\ref{ftaccfdsubtrop}, que representam a porção sem raios da precipitação tridimensional dos sistemas com índice extremo de FTA, não se observa banda brilhante marcada nos contornos de probabilidade de $Z_c$ por altitude. Há um aumento contínuo de $Z_c$ conforme os níveis de altitude diminuem, sem a diminuição abrupta de $Z_c$ logo abaixo de 5 quilômetros, o que  indica maior velocidade vertical para o grupo dos sistemas extremos de FTA do que para o grupo dos sistemas extremos de FT. 

%observa-se que, entre 20S-10N e 90W-30W,  
%a banda brilhante é evidente apenas nas regiões costeiras e oceânicas, nas caixas entre 0-10N  %e 90-80W, entre 10-0S e 40-30W e entre 20-10S e 70-60W.

A precipitação sem raios das tempestades elétricas entre 20S-10N e 90W-30W referente aos extremos de FT, possuem a mediana das distribuições cumulativas de probabilidades ($f_{cdf}(x,y)$) com valores de $Z_c$ inferiores do que quando compara-se com os extremos de FTA, os quais possuem perfis de $Z_c$ com maior aleatoriedade, mas atingem valores de $Z_c$ superiores. Observe as diferenças entre os CCFADs das figuras \ref{ftaccfdsubtrop} e figura \ref{ftccfadwithout}. Note como entre 20S-10N e 90W-30W os contornos de probabilidade cumulativa, são mais alargados para a precipitação sem raios dos extremos de FTA do que para a precipitação sem raios dos extremos de FT, indicando menor aleatoriedade dos perfis de $Z_c$ para os extremos de FT.

\begin{sidewaysfigure}%[!H]
  \centering
  \includegraphics[width=19.5cm]{img/precipitacao3d/severo/percentil/90th/cCumFad_10deg_semraio_topFTApercentil}
  \caption{CCFDs para os extremos de FTA. Porção da precipitação sem raios.}
  \label{ftaccfdsubtrop}   
\end{sidewaysfigure} 

\begin{sidewaysfigure}%[!H]
  \centering
  \includegraphics[width=19.5cm]{img/precipitacao3d/severo/percentil/90th/cCumFad_10deg_semraio_topFTpercentil}
  \caption{CCFDs para os extremos de FT. Porção da precipitação sem raios.}
  \label{ftccfadwithout}   
\end{sidewaysfigure} 

%\begin{figure}[!ht]
%  \centering  
%  \adjustbox{trim={.0\width} {.04\height} {0\width} {.565\height},clip}%
%  \centering  
%  \adjustbox{trim={.349\width} {.045\height} {.322\width} {.565\height},clip}%  
%  {\includegraphics[width=27cm] {img/precipitacao3d/severo/percentil/90th/cCumFad_10deg_semraio_topFTApercentil}}
% \caption{CCFDs para os extremos de FTA entre 40-20S e 70-50W. Porção da precipitação sem raios.}
% \label{ftaccfdsubtrop}
%\end{figure} 



Na região entre 40-20S e 70-50W que engloba a Bacia do Prata, a banda brilhante foi menos evidente nos contornos de probabilidade associados a estrutura tridimensional da precipitação fora dos núcleos de raios, tanto para os extremos de FT, figura \ref{ftccfadwithout}, quanto para os extremos de FTA, figura \ref{ftaccfdsubtrop}. 

Observe como a linha de contorno na cor preta no 50\textsuperscript{\underline{o}} percentil do CCFAD entre 40-20S e 70-50W nas figuras \ref{ftaccfdsubtrop} e \ref{ftccfadwithout}, indicam maiores valores de $Z_c$ para a fração sem raios das tempestades elétricas com índice FT extremo, mesmo que a estatística na parte superior direita de cada CCFAD indique maior percentual de perfis convectivos para a porção sem raios dos extremos de FTA. Porém, observe que acima do 80\textsuperscript{\underline{o}} percentil de ocorrência de perfis de $Z_c$ das tempestades elétricas com FTA extremo (figura \ref{ftaccfdsubtrop}), os valores de $Z_c$ passam a ser maiores do que para as tempestades elétricas com FT extremo. 

%, evidenciando que a chuva sem raios dos sistemas com as maiores taxas de raios no tempo é mais severa nesta região.

%\begin{figure}[!ht]
%  \centering  
%  \adjustbox{trim={.349\width} {.045\height} {.%322\width} {.565\height},clip}%
%  {\includegraphics[width=27cm] {img/precipitacao3d/severo/percentil/90th/cCumFad_10deg_semraio_topFTpercentil}}
% \caption{CCFDs para os extremos de FT entre 40-20S e 70-50W. Porção da precipitação sem raios.}
% \label{ftccfdsubtrop}
%\end{figure} 

Os CFADs referentes as tempestades elétricas com FTA extremo possuem contornos de probabilidade em níveis de altitude mais elevados do que os CFADs dos sistemas com FT extremo, tanto para a fração com raios quanto para a fração sem raios da precipitação dos sistemas. Como o último nível de altitude dos CFADs deste trabalho é limitado por altitudes com até 10\% de L, a maior definição de probabilidades de ocorrência de $Z_c$ em altitude indica que a convecção é mais intensa nas tempestades elétricas com FTA extremo do que nas tempestades elétricas de FT extremos.

%A diferença mais notável pode ser observada entre a figura \ref{ftacfadwithout} e \ref{ftcfadwithout} para 0S-10S e 50W-60W, que abrange principalmente o estado do Pará, e parte do Amazônas, Tocantis e Mato Grosso. O CFAD em \ref{ftacfadwithout} define valores de probabilidade em altitude 2 km mais elevada do que em \ref{ftcfadwithout}.


%Nas regiões entre 10N-0S e 70W-80W e entre 20S-40S e 50W-60W, em que \cite{cecil2005} apontam como região das tempestades mais severas na América do Sul, os CFADs em \ref{ftacfadwith} e \ref{ftacfadwithout} possuem contornos de probabilidade aproximadamente 1 km mais elevado do que em \ref{ftcfadwith} e \ref{ftcfadwithout}.



A precipitação é bem mais frequente próxima da superfície, entre 0-3 km de altitude. Acima da região de mistura, a precipitação é mais rara de ocorrer. Em \cite{liu2008}, é mostrado que a densidade espacial de sistemas com no mínimo 20 dBZ em 2 km de altitude é globalmente maior do que os sistemas que atingem 20 dBZ em níveis superiores de altitude.


%A região de 10 por 10 graus, a qual o valor de H marcado no topo direito de cada CFAD, é menor para a precipitação dos núcleos de raios dos sistemas com extremo de FTA, figura \ref{ftacfadwith}, do que para  a precipitação dos núcleos de raios dos sistemas com extremos de FT, figura \ref{ftcfadwith}, e mesmo assim, o CFAD dos extremos de FTA, figura \ref{ftacfadwith}, possuiu maior altitude nos níveis de contorno de probabilidade de $Z_c$, o índice FTA mostra que a chuva  
%esteve associado com maior severidade de tempo do que FT.
%Pois, mesmo que a refletividade mais ocorrente esteja abaixo da região de mistura, a precipitação também é frequente conforme o aumento da altitude, mostrando que nestas regiões, os sistemas com índice FTA extremo têm maior número de ocorrência de chuva 
%mais chuvas na superfície e também maior precipitação acima de 10 km de altitude.
%com bastante representatividade estatística.
%maior quantidade de hidrometeoros na região de mistura e

%Por exemplo na região do Panamá, Colômbia e Equador, entre 10N-0S e 70W-80W, o CFAD da figura \ref{ftacfadwith} possui contornos de probabilidade até 16 km de altitude. Na figura \ref{ftcfadwith}, os níveis de contorno param em 15 km.

A precipitação dos núcleos de raios representada nos CFADs a cada 10 por 10 graus de latitude e longitude na figura \ref{ftacfadwith}, dos sistemas extremos de FTA, mostram valores de refletividade entre 1-3 dBZ maiores do que nos CFADs dos sistemas extremos de FT na figura \ref{ftcfadwith}, principalmente quando observa-se os contornos de probabilidade de $Z_c$ acima de 5 km de altitude. Para a precipitação entre 0-2 km de altitude os valores são mais semelhantes entre as tempestades elétricas selecionadas por FTA e FT. 

%Porém, nos sistemas extremos de FTA, figura \ref{ftacfadwith}, há um estreitamento da região de contorno com os maiores valores de probabilidade associada a chuva na superfície, entre 3-5\%. Entre 20S-40S e 40-70W, o estreitamente é maior do que as demais regiões mostrando que as chuvas possuem maior probabilidade de estarem associadas com valores de 45 dBZ em \ref{ftacfadwith}.      


Os contornos de probabilidade entre 0.001-0.5\% das figuras \ref{ftacfadwith} e \ref{ftcfadwith}, revelam os valores dos perfis de $Z_c$ mais raros e mais intensos ocorridos nos sistemas com índice extremo de FTA e FT, os quais provavelmente estiveram associados a condições de tempo severo, ou seja, com enchentes rápidas, alta taxa de raios, chuva de granizo, fortes rajadas de vento e tornados. 
% nas figuras \ref{ftacfadwith} e \ref{ftcfadwith}

Os maiores valores de $Z_c$ foram registrados na análise da precipitação 3D das tempestades elétricas com FTA extremos (figura \ref{ftacfadwith}), entre 20S-40S e 40W-70W sobre a Bacia do Rio da Prata, que abrange o Sul do Brasil, Paraguai, Uruguai e Argentina. A dinâmica de formação de Sistemas Convectivos de Meso-escala, como é discutido em \cite{Velasco1987} e \cite{Durkee2009}, somados com efeitos de topografia, como por exemplo na região da Serra de Córdoba na Argentina, a qual \cite{Rasmussen2011} mostram grande ocorrência de convecção profunda, promoveram sistemas em que a estrutura tridimensional da precipitação dos núcleos de raios atingiram valores de $Z_c$ superiores a 45 dBZ entre 10-15 km de altitude e chuvas na superfície com $Z_c$ acima de 55 dBZ, como mostram os contornos de probabilidade entre 0.001-0.5\%.

%Os processos de eletrificação dos extremos de FTA pode estar associada com a presença de granizo, cristais de gelo e gotas água super-resfriada, enquanto que para os extremos de FT, 

\subsection{A precipitação 3D e o perfil atmosférico de temperatura.}

O estudo dos perfis de $Z_c$ por nível de temperatura é realizado por meio dos diagramas CCFTD e CFTD, os quais são expostos nas figuras \ref{ccftd_fta_com}, \ref{ccftd_ft_com}, \ref{cftd_fta_com} e \ref{cftd_ft_com}, associados as tempestades elétricas com índice extremo de FTA e FT, apenas referente a precipitação dos núcleos de raios.

A partir dos CCFTDs das figuras \ref{ccftd_fta_com} e \ref{ccftd_ft_com}, iremos avaliar a intensidade convectiva dos sistemas com índice extremo de FTA e FT em determinadas regiões, não apenas por meio dos valores de $Z_c$, mas com base na velocidade de crescimento ou decrescimento dos valores de $Z_{c}$ associados os contornos de probabilidade do 30\textsuperscript{\underline{o}}, 50\textsuperscript{\underline{o}}, 70\textsuperscript{\underline{o}} e 95\textsuperscript{\underline{o}} percentil das amostragens de probabilidades expressas nos CFTDs das figuras \ref{cftd_fta_com} e \ref{cftd_ft_com}.	




\begin{sidewaysfigure}%[!H]
\centering
\includegraphics[width=19.5cm]{img/precipitacao3d/severo/percentil/90th/cftd_10deg_comraio_topFTApercentil}
\caption{CFTDs para os extremos de FTA. Porção da precipitação com raios.}
\label{cftd_fta_com}
\end{sidewaysfigure} 

\begin{sidewaysfigure}%[!H]
\centering
\includegraphics[width=19.5cm]{img/precipitacao3d/severo/percentil/90th/ccftd_10deg_comraio_topFTApercentil}
\caption{CCFTDs para os extremos de FTA. Porção da precipitação com raios.}
\label{ccftd_fta_com}
\end{sidewaysfigure} 

\begin{sidewaysfigure}%[!H]
\centering
\includegraphics[width=19.5cm]{img/precipitacao3d/severo/percentil/90th/cftd_10deg_comraio_topFTpercentil}
\caption{CFTDs para os extremos de FT. Porção da precipitação com raios.}
\label{cftd_ft_com}
\end{sidewaysfigure} 

\begin{sidewaysfigure}%[!H]
\centering
\includegraphics[width=19.5cm]{img/precipitacao3d/severo/percentil/90th/ccftd_10deg_comraio_topFTpercentil}
\caption{CCFTDs para os extremos de FT. Porção da precipitação com raios.}
\label{ccftd_ft_com}
\end{sidewaysfigure} 

Portanto, a partir das amostragens de probabilidade do fator de refletividade $Z_c$ por nível de temperatura que define os diagramas CCFTD e CFTD conforme é descrito em \ref{metodologiaPrec3D},  extraí-se as linhas de contorno do CCFTD referentes as probabilidades cumulativas de 30\%, 50\%, 70\% e 95\%. Desta forma, obteve-se quatro funções 
\begin{equation}
f(x)=y ,
\end{equation} 
em que $y$ corresponde aos valores de $Z_c$ e $x$ os valores de temperatura. Fazendo a derivada 
\begin{equation}
f'(x_0)= - \dfrac{y_1 - y_0}{x_1 - x_0},
\label{devivadaContorno}
\end{equation}
então, pode-se avaliar a taxa de variação do fator de refletividade do radar ($Z_c$) por temperatura (dBZ~\textsuperscript{o}C$^{-1}$) para diferentes regimes de precipitação, das mais frequentes até as mais raras. O sinal de menos na equação \ref{devivadaContorno} deve-se ao fato da temperatura diminuir conforme se afasta da origem e queremos mostrar taxas positivas associadas ao aumento de $Z_c$.
\simbolo{name={$f(x)=y$},description={Função de uma variável}} 

\begin{figure}[!ht]
  \centering
  \subfloat[Linhas de contorno dos CCFTDs.]{\includegraphics[height=9cm]{img/precipitacao3d/deriv_ccftd/Contornos_contornos_cdf_2_1} \label{contornosAmazonas}}
  \subfloat[Derivadas das linhas de contorno dos CCFTDs.]{\includegraphics[height=9cm]{img/precipitacao3d/deriv_ccftd/deriv_contornos_cdf_2_1} \label{derivaAmazonas}}
  \caption{Região central da Bacia do Rio Amazonas, entre 10-0S e 70-60W.}
  \label{deriv_amazonas}  
\end{figure} 
%Taxa de variação de $Z_c$ no perfil de temperatura atmosférico para a 
% nos diferentes quartis do CCFTD dos extremos de FTA,  figura \ref{ccftd_fta_com},  e dos extremos de FT, figura \ref{ccftd_ft_com}. 

Para a região central da Bacia do Rio Amazonas, entre 10-0S e 70-60W, os valores de $Z_c$ com os contornos de probabilidade do 30\textsuperscript{\underline{o}}, 50\textsuperscript{\underline{o}}, 70\textsuperscript{\underline{o}} e 95\textsuperscript{\underline{o}} percentil referente as tempestades elétricas com FTA extremo (figura \ref{ccftd_fta_com}), possuem valores entre 1-3 dBZ maiores do que os valores de $Z_c$, nos mesmos percentis, referentes as tempestades elétricas com FT extremo (figura \ref{ccftd_ft_com}), principalmente acima da isoterma de 0 \textsuperscript{o}C, o que indica maior concentração ou maior diâmetro dos hidrometeoros na região fria\footnote{Com temperaturas abaixo de  0\textsuperscript{o}C} dos núcleos de raios associados aos sistemas extremos de FTA. Na figura \ref{contornosAmazonas} pode-se verificar os contornos de probabilidade do 30\textsuperscript{\underline{o}}, 50\textsuperscript{\underline{o}}, 70\textsuperscript{\underline{o}} e 95\textsuperscript{\underline{o}} percentil das amostragens de perfis de $Z_c$  para as tempestades elétricas com extremos de FTA e FT, conjuntamente.

No entanto, entre 10-0S e 70-60W, verifica-se também na figura \ref{derivaAmazonas}, que a taxa de aumento de $Z_c$ referente ao 70\textsuperscript{\underline{o}} e 95\textsuperscript{\underline{o}} percentil, entre -40 \textsuperscript{o}C e -15 \textsuperscript{o}C dos extremos de FTA é maior do que nos extremos de FT. Para os percentis de 30\textsuperscript{\underline{o}} e 50\textsuperscript{\underline{o}}, apesar da intensidade de $Z_c$ ser maior entre -40 \textsuperscript{o}C e -15 \textsuperscript{o}C para os extremos de FTA (ver figura \ref{contornosAmazonas}), as taxas de crescimento em dBZ~\textsuperscript{o}C$^{-1}$ possuem valores semelhantes (ver figura \ref{derivaAmazonas}). Porém, para todos os quantis entre -15 \textsuperscript{o}C e 0 \textsuperscript{o}C da figura \ref{deriv_amazonas}, observa-se que os valores de $Z_c$ para os extremos de FT, sofreram os maiores acréscimos entre -8 \textsuperscript{o}C e 0 \textsuperscript{o}C, enquanto os valores de $Z_c$ para os extremos de FTA sofreram os maiores acréscimos entre -20 \textsuperscript{o}C e -10 \textsuperscript{o}C. % mostrando que os hidrometeoros dos sistemas extremos de FTA cresceram mais em regiões mais frias do que para os extremos de FT.

Como os hidrometeoros dos sistemas extremos de FTA crescem em regiões mais frias do que nos extremos de FT,
conforme \citeonline{Takahashi1978}, podemos considerar que os núcleos de raios das tempestades elétricas com extremo de FTA entre 10-0S e 70-60W, possuíram centros de cargas predominantemente negativos, enquanto que os núcleos de raios das tempestades elétricas dos extremos de FT, em que as maiores taxas de crescimento de $Z_c$ ocorreram entre  -8\textsuperscript{o}C e 0\textsuperscript{o}C, os centros de cargas foram predominantemente positivos.

%exceto no 95\textsuperscript{\underline{o}} percentil em que o maior acréscimo de $Z_c$ foi em -10\textsuperscript{o}C
Apensar do crescimento do \textit{graupel} e o granizo em temperaturas entre -10\textsuperscript{o}C e 0\textsuperscript{o}C indicar carregamento positivo, as variações do conteúdo de água líquida de nuvem  podem inverter a polaridade das partículas que crescem em temperatura inferior a -10\textsuperscript{o}C \cite{Takahashi1978}. Durante uma tempestade, as correntes ascendentes e o entranhamento na região de crescimento do gelo poderão elevar ou diminuir o conteúdo de água líquida do ambiente.

Também, a taxa de coleta de gotículas de água líquida pelo \textit{graupel} durante o processo de acreção, pode ter mais influência no sinal das cargas dos hidrometeoros do que o conteúdo de água líquida \cite{jayaratne1983,saunders1991effect,brooks1997,Takahashi2002}. 
Apesar do conteúdo de água líquida efetivamente coletado no processo de acresção ser proporcional ao conteúdo de água líquida da nuvem, as velocidades relativas entre as partículas e a eficiência de coleta, além de influenciarem sobre a polaridade, também determinam a estrutura do granizo: poroso, compacto ou esponjoso \cite[p.~335]{mason1971_2ed}.  
 
Mesmo considerando hipoteticamente que entre 10-0S e 70-60W, as tempestades elétricas com extremos de FT possuíram maior probabilidade de centros de cargas principal positivo, a porcentagem de raios nuvem-solo  positivos pode ser menor do que a porcentagem de raios nuvem-solo  negativos. Em \citeonline{carey2007}, tempestades elétricas com 25\% dos raios nuvem-solo positivos foram consideradas como tempestades elétricas positivas.

Em \citeonline{fernandes2005} e \citeonline{fernandes2006}, descreve-se que no ambiente amazônico seco e poluído, as nuvens podem ter altura da base mais elevadas e mesmo que os processos de crescimento de hidrometeoros ocorram em regiões de maior altitude e consequentemente de menor temperatura, o centro de carga principal fica mais distante da superfície, causando  o aumento da razão entre os raios intranuvens e nuvem-solo e favorecendo a ocorrência de raios nuvem-solo positivos a partir do centro de carga positivo na base da nuvem, que de acordo com o modelo do tripolo eletrostático proposto em \citeonline{williams1989}, fica mais próximo do solo do que o centro de cargas negativo principal.

%como sugerem os extremos de FTA em relação aos extremos de FT,
%A partir do banco de dados gerado no experimento CHUVA como descrevem \cite{machado2014chuva}, serão possível análises das taxa de crescimento de dBZ~\textsuperscript{o}C, com base em observações por radar, radio-sondagem, conjuntamente com as observações de raios e de campo eletrostático. Desta forma pode-se investigar se o processo microfísico    

Portanto, além da microfísica de eletrificação descrita em \citeonline{Takahashi1978,jayaratne1983,Takahashi2002,saunders2008} e fatores como: o cisalhamento do vento próximo a superfície, aumento da instabilidade condicional e menor espessura entre a base da nuvem e a isoterma de 0\textsuperscript{o}C, que em \citeonline{carey2007,albrecht2011} estiveram associados a ocorrência de tempestades elétricas positivas as quais supostamente teriam centro de carga principal positivo, portanto um tripolo eletrostático invertido, a geometria e densidade volumétrica dos centros de cargas em relação a superfície pode determinar o caminho de menor resistência elétrica para a formação de um raio nuvem-solo positivo ou negativo.

Os maiores valores de $Z_c$ em temperaturas inferiores a -10\textsuperscript{o}C para diferentes quantis da amostragem da probabilidade de $Z_c$ por temperatura dos extremos de FTA, indicam que a velocidade vertical no ambiente das tempestades elétricas com índice FTA extremo são maiores do que para os sistemas com índice FT extremo, pois os maiores valores de FTA mostram perfis de $Z_c$ com maior quantidade de gelo de nuvem, portanto, espera-se correntes ascendentes mais intensas nas tempestades elétricas selecionadas pelos extremos de FTA na região dos núcleos de raios para que seja possível elevar maior quantidade de água em regiões mais frias da atmosfera do que nos casos das tempestades elétricas com FT extremo. 

O fato das taxas de crescimento em dBZ~\textsuperscript{o}C$^{-1}$ para a precipitação dos núcleos de raios das tempestades elétricas com FTA extremo serem maiores  em regiões mais frias do que observa-se para as tempestades elétricas com extremos de FT, sugere maior interação entre o \textit{graupel}, granizo compacto ou poroso e cristais gelo. Conforme \cite{Takahashi1978}, a interação entre cristais de gelo e o \textit{graupel} é o processo de maior eficiência na transferência de cargas entre os hidrometeoros. 

Nos sistemas com FT extremos, a acresção ocorreu preferencialmente em temperaturas superiores a -10\textsuperscript{o}C, ambiente  supostamente com maior quantidade de gotículas de água superesfriadas, sugerindo formação de granizo esponjoso\footnote{Granizo em que parte das gotículas de água super-resfriadas coletadas não congelam. Também denominado como granizo molhado.}, processo de crescimento de hidrometeoros pouco eficiente na eletrização \cite{jayaratne1983}. 

%Portanto, mesmo que a estrutura 3D da precipitação dos extremos de FT esteja associada a sistemas com maior número de raios, a precipitação dos sistemas com as maiores densidades de raios por área (extremos de FTA), sugere processos de eletrificação mais eficientes.


%para as tempestades elétricas  como mostra a figura %\ref{contornosAmazonas},
% e a menor taxa de crescimento de dBZ~\textsuperscript{o}C$^{-1}$ %em 0\textsuperscript{o}C para as tempestades elétricas com FTA %extremo em relação aos extremos de FT, indicam maior intensidade %convectiva nos sistemas com FTA extremo entre 10-0S e 70-60W, %pois além de evidenciar mostra maior quantidade de
% gelo de nuvem e precipitação a partir de gelo sólido. 

Observe na figura \ref{deriv_amazonas}, que logo abaixo de 10\textsuperscript{o}C, próximo da superfície, os valores de $Z_c$ decrescem mais para FTA do que para FT. O decrescimento de $Z_c$ próximo da superfície representa processo de evaporação ou quebra das gotas de chuva, o que indica vento mais intenso na superfície para as tempestades elétricas com FTA extremo. 

%O aumento do fator de refletividade em torno de 0 \textsuperscript{o}C está associado a mudança do índice de refração da água devido a sua fusão. Já o aumento do fator de refletividade em torno de -40 \textsuperscript{o}C e -15 \textsuperscript{o}C representam o crescimento de hidrometeoros por agregação e acreção \cite{Fabry1995,Takahashi1978}.

%Note na figura \ref{deriv_amazonas}, como a precipitação do 95\textsuperscript{\underline{o}} percentil de probabilidade de ocorrência tanto para FTA quanto para FT, é o regime de precipitação mais severa. Pois, há o crescimento de $Z_c$ em torno de -10 \textsuperscript{o}C e -15 \textsuperscript{o}C e não há banda brilhante, indicando precipitação a partir de granizo\footnote{Em \cite{Fabry1995}, este tipo de perfil é discutido como chuva a partir de gelo compacto.}. No 30\textsuperscript{\underline{o}}, 50\textsuperscript{\underline{o}} e 70\textsuperscript{\underline{o}} percentil dos extremos de FT, o efeito da banda brilhante associada ao derretimento é mais evidente do que para os extremos de FTA. 

Quando comparamos a região central da Bacia do Rio Amazonas, figura \ref{deriv_amazonas}, com a região central da Bacia do Rio da Prata, entre 30-20S e 60-50W na figura \ref{deriv_prata}, a microfísica de eletrificação se mostra diferente em cada local. Observa-se que no 50\textsuperscript{\underline{o}} percentil, a taxa de crescimento em dBZ~\textsuperscript{o}C$^{-1}$ entre -40 \textsuperscript{o}C e -20 \textsuperscript{o}C é maior para a região da Bacia do Prata, do que para a região da Bacia Amazônica, tanto para os sistemas extremos de FTA quando para os sistemas extremos de FT, indicando maior crescimento de flocos de neve na estrutura 3D da precipitação severa sobre a Bacia do Prata do que sobre a Bacia Amazônica. 

Comparando as linhas de contorno entre as figuras \ref{contornosAmazonas} e \ref{contornosPrata}, observa-se que em todos os quantis analisados, tanto para a chuva dos extremos de FTA quanto FT, mostram valores de $Z_c$ mais intensos para as tempestades elétricas na região que abrange parte central da bacia do Prata.  


\begin{figure}[!ht]
\centering
\subfloat[Linhas de contorno dos CCFTDs.]{\includegraphics[height=9cm]{img/precipitacao3d/deriv_ccftd/Contornos_contornos_cdf_3_3} \label{contornosPrata}}
\subfloat[Derivadas das linhas de contorno dos CCFTDs.]{\includegraphics[height=9cm]{img/precipitacao3d/deriv_ccftd/deriv_contornos_cdf_3_3} \label{derivaPrata}}
\caption{Região central da Bacia do Rio da Prata, entre 30-20S e 60-50W.}
\label{deriv_prata}
\end{figure}

Apesar do 95\textsuperscript{\underline{o}} percentil mostrar maiores taxas de de crescimento em dBZ por \textsuperscript{o}C, em -15 \textsuperscript{o}C tanto para FTA quanto FT sobre a Bacia do Rio Amazonas, do que sobre a Bacia do Rio da Prata,  os contorno de probabilidade acumulativa de 95\% nos CCFTD das figuras \ref{ccftd_fta_com} e \ref{ccftd_ft_com}, em -15 \textsuperscript{o}C, mostram valores de $Z_c$ de aproximadamente 3 dBZ superiores na região da Bacia Platina. Mesmo que o 95\textsuperscript{\underline{o}} percentil mostre maior crescimento de hidrometeoros na região mista sobre a Bacia Amazônica, a precipitação do 95\textsuperscript{\underline{o}} percentil na Bacia do Prata foi mais severa, pois possui maiores valores de $Z_c$.

O aumento abrupto de $Z_c$ associado a fusão da água, entre 30-20S e 60-50W, figura \ref{deriv_prata}, principalmente do 50\textsuperscript{\underline{o}} e 70\textsuperscript{\underline{o}} percentil, ocorrem em -4 \textsuperscript{o}C, enquanto que, entre 10-0S e 70-60W, figura \ref{deriv_amazonas}, o aumento de $Z_c$ ocorre mais próximo de 0 \textsuperscript{o}C, o que indica maior presença de água super-resfriada associada ao processo de derretimento da precipitação entre 30-20S e 60-50W, região da Bacia do Rio da Prata.  

Na região da Bacia do Prata, representada na figura \ref{deriv_prata}, o efeito de banda brilhante também é mais pronunciado para a precipitação com raios dos extremos de FT, o que mostra que as regiões eletricamente ativas da precipitação dos sistemas com índice extremo de FTA é menos estratificada do que nos extremos de FT, em ambas as Bacias Hidrológicas: da Prata e do Amazonas.

\newpage
\section{SEVERIDADE REGIONALIZADA}

Aqui, o estudo da densidade de probabilidade de FTA e FT, conforme mostrado na figura \ref{pdfFTAFT}, foi feito para os sistemas ocorridos em cada região de 2.5 por 2.5 graus de latitude e longitude entre 40N-10S e 90-30W. Verifica-se a distribuição geográfica, dos valores de FTA e FT mais frequentes e mais raros conforme cada localidade.


A linha de contorno na cor preta em cada mapa apresentado nesta seção, corresponde ao valor do percentil determinado para a análise regional, porém, é referente a amostragem total exposta na figura \ref{cdfFTAFT}.

No ambiente oceânico e costeiro, as tempestades elétricas mais frequentes devem possuir menores índices de severidade do que no continente, pois na costa e oceano observa-se as maiores probabilidades de ocorrência de chuva quente \cite{Liu2009}. 
%o aquecimento da superfície durante o ciclo diurno é menor e o processo de colisão coalescência é dominante em relação aos processos que envolvem a formação de gelo de nuvem 

Nas figuras \ref{5oFta} e \ref{10oFta}, os contornos com valores de 0.05 $\times$ 10$^{-4}$ e 0.12 $\times$ 10$^{-4}$ raios minutos$^{-1}$
km$^{-2}$ respectivamente, demarcam claramente a divisão entre a convecção oceânica e a continental. O Oceano Verde, conceito associado a convecção durante o regime de ventos de Oeste na estação chuvosa Amazônica, discutido por \citeonline{silva2002lba,williams2002}, é bastante evidente. A região central da Bacia Amazônica possui os  valores de FTA na mesma ordem de magnitude e no mesmo percentil das densidades de probabilidades de FTA regionalizadas das tempestades elétricas oceânicas e costeiras.

Os valores de FT associados ao 5\textsuperscript{\underline{o}} e 10\textsuperscript{\underline{o}} percentil, mostrados nas figuras \ref{5oFt} e \ref{10oFt}, revelam os menores valores de FT no centro do continente, principalmente nas regiões continentais fora da área de atuação da ZCIT e de sistemas transientes subtropicais.

\begin{figure}[!ht]
  \centering{
  \subfloat[5\textsuperscript{\underline{o}} percentil de FT]{{\includegraphics[height=6.5cm, trim=0 0 0 0, clip]{img/DistEspacialPercentis/FT/distEspacialValor005thFt}} \label{5oFt}}\\
  \subfloat[10\textsuperscript{\underline{o}} percentil de FT]{{\includegraphics[height=6.5cm, trim=0 0 0 0, clip]{img/DistEspacialPercentis/FT/distEspacialValor010thFt}} \label{10oFt}}
  }    
  \caption{Distribuição espacial dos valores do 5\textsuperscript{\underline{o}} e 10\textsuperscript{\underline{o}} percentil da amostra de probabilidade do índice FT a cada região de 2.5 por 2.5 graus de latitude e longitude.}
\label{extremosInfFT}
\end{figure} 

Para avaliar a distribuição geográfica dos extremos superiores dos índices FTA e FT, verificou-se os valores do 95\textsuperscript{\underline{o}} e 99\textsuperscript{\underline{o}} percentil das amostragens, os quais são expostos nos mapas das figuras \ref{extremosSupFTA} e \ref{extremosSupFT}.

%: regionalizadas, representados nas cores das figuras \ref{extremosSupFTA} e \ref{extremosSupFT}; e totais (ver figura \ref{seriesFtaFt}), representados pela linha de contorno das figuras \ref{extremosSupFTA} e \ref{extremosSupFT}.

Observa-se na figura \ref{95oFta} que os sistemas com índice FTA  superior à 52.76 $\times$ 10$^{-4}$ raios minutos$^{-1}$
km$^{-2}$, são considerados de severidade extrema, pois correspondem a valores superiores ao valor do 95\textsuperscript{\underline{o}} percentil da amostragem total de FTA da figura \ref{pdfFta}. Porém, em regiões no interior do continente, os valores de FTA do 95\textsuperscript{\underline{o}} percentil das amostragens regionalizadas, atingiram  111.97 $\times$ 10$^{-4}$ raios minutos$^{-1}$ km$^{-2}$.
%, mostrando que nestas regiões, valores de 52.76 $\times$ 10$^{-4}$ raios minutos$^{-1}$ km$^{-2}$ são mais frequentes do que nas regiões em que passa a linha de contorno de 52.76 $\times$ 10$^{-4}$ raios minutos$^{-1}$ km$^{-2}$.
%raios por minutos por quilômetros quadrado

\begin{figure}[!ht]
\centering
{\includegraphics[height=13.5cm, trim=0 0 0 0, clip]{img/DistEspacialPercentis/FTA/distEspacialValor095thFta}} 
\caption{Distribuição espacial dos valores do 95\textsuperscript{\underline{o}} percentil da amostra de probabilidade do índice FTA a cada região de 2.5 por 2.5 graus de latitude e longitude.}
\label{95oFta}
\end{figure} 
  
\begin{figure}[!ht]
\centering  
{\includegraphics[height=13.5cm, trim=0 0 0 0, clip]{img/DistEspacialPercentis/FTA/distEspacialValor099thFta}}
\caption{Distribuição espacial dos valores do  99\textsuperscript{\underline{o}} percentil da amostra de probabilidade do índice FTA a cada região de 2.5 por 2.5 graus de latitude e longitude.}
\label{99oFta}
\end{figure} 


Os valores do 99\textsuperscript{\underline{o}} percentil na figura \ref{99oFta}, mostram que no Leste do estado do Amazonas, no Acre e Tocantis e Sudeste do Peru e Norte da Bolívia, regiões estas que compõem o Oceano Verde, a severidade extrema de FTA possui valores entre 148.93-230.00  $\times$ 10$^{-4}$ raios minutos$^{-1}$ km$^{-2}$,  valores que correspondem aos mais extremos do continente Sul-americano. 

Mesmo que a Floresta Amazônica seja um Oceano Verde para atmosfera durante as fases ativas do SAMS, durante o regime de ventos de Leste na estação chuvosa que associa-se as fases inativas da SAMS e durante a estação de transição seca-úmida (SON), ``o Mar Verde"  ~fica revolto. Mesmo que a Floresta Amazônica dialogue com a precipitação como um oceano, este oceano possui temperatura superficial média na classe das maiores temperaturas superficiais continentais globais e está cercado por um vasto continente. Portanto, tem a capacidades de gerar tempestades elétricas extremamente severas, mostrando que a interação entre a Floresta Amazônica e a atmosfera é bastante diversificada.

As regiões dos maiores valores do 95\textsuperscript{\underline{o}} e 99\textsuperscript{\underline{o}} percentil do índice FTA, os quais são  expostos nas figuras \ref{95oFta} e \ref{99oFta}, são principalmente: a Bacia do Rio da Prata, a região Leste Amazônia e as regiões  do planalto Brasileiro, que se estendem por quase todo o país.

Observa-se que os sistemas mais severos da América do Sul ocorrem associados ao relevo nas regiões entre o Pantanal Mato-grossense e o Planalto Central Brasileiro, entre as Bacias dos Rios: Xingu, Araguaia e Tocantis e também o Planalto Central Brasileiro, entre a Bacia do Rio Paraná e o Planalto Meridional Brasileiro, aonde está localizado os planaltos e chapadas da Bacia do Paraná. Nestas regiões os sistemas severos possuem índice FTA superiores à 80 $\times$ 10$^{-4}$ raios minutos$^{-1}$ km$^{-2}$, como mostram as cores das figuras \ref{95oFta} e \ref{99oFta}. 

Note que para saber aproximadamente o número de raios produzidos pelos sistemas extremos de FTA temos que multiplicar o índice FTA pela área do sistema. Por exemplo, a equação \ref{FTAkm2}, descreve que nas regiões em que os sistemas extremos possuem 100 $\times$ 10$^{-4}$ raios minutos$^{-1}$ km$^{-2}$, um sistemas severo com área de 10$^3$ km$^2$ então possui 10 raios observados pelo LIS em 1 minuto.  

\begin{equation}
100 \times 10^{-4} \left[ \frac{\mathrm{raios}}{\mathrm{minutos}~\mathrm{km}^2} \right]  10^3 [ \mathrm{km}^2 ] = 10 \left[ \frac{\mathrm{raios}}{\mathrm{minutos}}\right]  
\label{FTAkm2}
\end{equation}

Os mapas das figuras \ref{95oFt} e \ref{99oFt}, mostam que nas Bacias: do Rio da Prata principalmente, do Rio Araguaia, Rio Xingu e Rio Tocantis, são locais em que os sistemas possuem os maiores índices de FT tanto no 95\textsuperscript{\underline{o}} quanto no 99\textsuperscript{\underline{o}} percentil.

\begin{figure}[!ht]
\centering
{\includegraphics[height=13.5cm, trim=0 0 0 0, clip]{img/DistEspacialPercentis/FT/distEspacialValor095thFt}} 
\caption{Distribuição espacial dos valores do 95\textsuperscript{\underline{o}} percentil da amostra de probabilidade do índice FT a cada região de 2.5 por 2.5 graus de latitude e longitude.}
\label{95oFt}
\end{figure} 
  
\begin{figure}[!ht]
\centering  
{\includegraphics[height=13.5cm, trim=0 0 0 0, clip]{img/DistEspacialPercentis/FT/distEspacialValor099thFt}}
\caption{Distribuição espacial dos valores do  99\textsuperscript{\underline{o}} percentil da amostra de probabilidade do índice FT a cada região de 2.5 por 2.5 graus de latitude e longitude.}
\label{99oFt}
\end{figure}

Os maiores valores do 95\textsuperscript{\underline{o}} e 99\textsuperscript{\underline{o}} percentil do índice FT, figuras \ref{95oFt} e \ref{99oFt},  ficam situados na região Sul da América do Sul, compatível com a região em que \citeonline{cecil2005}, apontam como o local das tempestades categoria 5, ou seja, das mais severas do globo.