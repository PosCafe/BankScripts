\chapter{TEMPESTADES ELÉTRICAS SEVERAS}

Conforme descrito em \ref{metodoFtaFt}, as taxas de raios das tempestades elétricas neste trabalho de pesquisa estão associadas aos índices FTA e FT. Nesta seção identifica-se qual desses índices podem melhor associar-se com a intensidade convectiva das tempestades elétricas.
%, ou seja, os sistemas com as maiores taxas de raios por minuto (FT) ou os sistemas com as maiores taxas de raios por minuto por quilômetro quadrado (FTA) da sua extensão. 

Como a intensidade convectiva será avaliada a partir da taxa de raios (FTA e FT) e análise da estrutura tridimensional da precipitação observada pelo PR, para o estudo de severidade das tempestades elétricas, foram selecionados apenas os sistemas que possuíram $VT_m$ maior ou igual a 1 minuto e com pelo menos um pixel do campo de visão do PR contido na área do sistema com chuva válida, totalizando {94~733} tempestades elétricas do TRMM. 

%Como a intensidade convectiva dos sistemas é avaliada com base, principalmente, na morfologia da estrutura tridimensional da precipitação e na taxa de raios, 

As equações \ref{eqFT} e \ref{eqFTA} foram aplicadas nas {94~733} tempestades elétricas selecionadas, e então estudada as distribuições de probabilidades dos índices FTA e FT (figuras \ref{pdfFTAFT} e \ref{cdfFTAFT}). Conforme mostra a figura \ref{pdfFTAFT}, trata-se de distribuições exponenciais de probabilidade. Os valores de FTA e FT para cada quantil da amostragem de tempestades elétricas, tanto para as mais frequentes quanto as mais raras, podem ser verificados por meio da distribuição cumulativa de probabilidade de FTA e FT mostradas na figura \ref{cdfFTAFT}. 

Na mediana da amostragem de FTA,  as tempestades elétricas tiveram 2,3 $\times$ 10$^{-4}$ raios min$^{-1}$ km$^{-2}$  e na mediana da amostragem de FT, foram 3,4 raios min$^{-1}$. O maior número de tempestades elétricas possuem os menores valores FTA e FT, conforme é mostrado na figura \ref{pdfFTAFT}. Note que o primeiro degrau referente a curva da distribuição cumulativa de FT na figura \ref{cdfFTAFT}, corresponde a 28,9\% da amostragem das {94~733} tempestades elétricas do TRMM e o primeiro degrau da distribuição cumulativa de FTA (figura \ref{cdfFTAFT}) corresponde a 9,2\% de {94~733}. 

\begin{figure}[!ht]
  \centering
  \includegraphics[height=9.5cm]{img/FtaFt/pdf_FTA_FT}      
  \caption{Densidade de probabilidade de FTA e FT.} 
   \label{pdfFTAFT} 
\end{figure}

\begin{figure}[!hb]
  \centering 
  \includegraphics[height=9.5cm]{img/FtaFt/cdf_FTA_FT} 
  \caption{Densidade de probabilidade cumulativa de FTA e FT.}
  \label{cdfFTAFT}
\end{figure}
%\label{seriesFtaFt}

Os sistemas potencialmente severos são selecionados pelo 90\textsuperscript{\underline{o}} percentil das amostragens de probabilidades de FTA e FT, que correspondem aos 10\% mais intensos e mais raros valores de FTA e FT ocorridos. Portanto, pressupõe-se que as tempestades elétricas as quais provavelmente causaram chuva de granizo, rajadas de ventos com queda de árvores e construções ou tornados devem estar associadas aos sistemas com valores extremos de FTA ou FT.

Desta forma, temos dois grupo de tempestades elétricas do TRMM extremas, um composto por 9475 tempestades elétricas com valores extremos de FTA e outro com 9475 tempestades elétricas com valores extremos de FT. Os valores extremos de FTA correspondem a tempestades elétricas com {29,3--1258,7 $\times$ 10$^{-4}$} raios por minuto por quilômetro quadrado ([min$^{-1}$] [km$^{-2}$]), enquanto as tempestades elétricas com FT extremos possuíram valores entre {47,2--1283,6} raios por minuto ([min$^{-1}$]). Observe que, os extremos do índice FT, correspondem as taxas de raios no tempo das PFs de categoria três, quatro e cinco, as quais produziram entre 30,9--1351 raios min$^{-1}$, que totalizaram 5727 PFs observadas sobre o globo em 3 anos de dados do TRMM \cite{cecil2005}. Neste trabalho de pesquisa, que considera apenas a região da AS, porém 14 anos de dados do TRMM, os extremos de FT ({47,2--1283,6} raios min$^{-1}$) representam um total de 9475 tempestades elétricas do TRMM.      

%, 0,09\% da base de dados globais de três anos das PFs
%Note também que o grupo das tempestades elétricas severas desta tese\footnote{90\textsuperscript{\underline{o}} percentil das amostras de probabilidades de FTA e FT}, é formado por um total de 9473 sistemas apenas sobre a América do Sul.
% As tempestades elétricas severas, correspondem ao total de 9473 

As figuras \ref{topFTA} e \ref{topFT} ilustram as medidas do TRMM para 3 tempestades elétricas com FTA e FT extremos e que possuíram pelo menos 50\% da sua área dentro da visada do PR. 



\begin{figure}[!ht]
\begin{flushleft}
  \includegraphics[height=1.1cm]{img/topSevero/barradecor_virs}~~~~~~~~~~~
  \includegraphics[height=1.1cm]{img/topSevero/barradecor_chuva}
\end{flushleft}
  \centering   
  \subfloat[]{\includegraphics[width=\textwidth]{img/topSevero/topFTA_chuva/001_topFTA_10801_0006} \label{fta1}} \\
  \subfloat[]{\includegraphics[width=\textwidth]{img/topSevero/topFTA_chuva/002_topFTA_58133_0007} \label{fta2}} \\
%  \subfloat[]{\includegraphics[width=\textwidth]{img/topSevero/topFTA_chuva/003_topFTA_28799_0006} \label{fta3}} \\
  \subfloat[]{\includegraphics[width=\textwidth]{img/topSevero/topFTA_chuva/005_topFTA_52880_0006} \label{fta4}} \\
  \caption{Visualização das tempestades elétricas com os maiores valores de FTA: $T_b$ do topo da nuvem, $Z_c$ a 2 km de altitude e perfis médios de $Z_c$ da precipitação convectiva (cor vermelha), estratiforme (azul) e total (preta).} %(0.05$^{\circ}$ $\times$ 0.05$^{\circ}$)
\label{topFTA}
\end{figure}

\begin{figure}[!ht]
 \begin{flushleft}
  \includegraphics[height=1.1cm]{img/topSevero/barradecor_virs}~~~~~~~~~~~
  \includegraphics[height=1.1cm]{img/topSevero/barradecor_chuva}
 \end{flushleft}
  \centering            
  \subfloat[]{\includegraphics[width=\textwidth]{img/topSevero/topFT_chuva/001_topFT_02837_00010002} \label{ft1}} \\
 % \subfloat[]{\includegraphics[width=0.8\textwidth]{img/topSevero/topFT_chuva/002_topFT_14893_00010002} \label{ft2}} \\
  \subfloat[]{\includegraphics[width=\textwidth]{img/topSevero/topFT_chuva/003_topFT_05694_0005} \label{ft3}} \\
  \subfloat[]{\includegraphics[width=\textwidth]{img/topSevero/topFT_chuva/004_topFT_03444_0001} \label{ft4}} 
 
  \caption{Visualização das tempestades elétricas com os maiores valores de FT: $T_b$ do topo da nuvem, $Z_c$ a 2 km de altitude e perfis médios de $Z_c$ da precipitação convectiva (cor vermelha), estratiforme (azul) e total (preta).} %(0.05$^{\circ}$ $\times$ 0.05$^{\circ}$)
\label{topFT}
\end{figure}

Na parte superior e inferior dos painéis representados pelas figuras \ref{topFTA} e \ref{topFT}, há informações referentes a data e hora em que o sistema foi observado, número de raios/eventos (FL/EV), fração do sistema observado pelo PR, valores dos índices FTA e FT, área do sistema (A), semi-eixo maior (a), menor (b), distância focal (2c) e excentricidade (e) de uma elipse ajustada as dimensões do sistema. As barras de cores  correspondem aos valores de $T_b$ $\leq$ 258 K que definem a área das tempestades elétricas e ao lado os valores de $Z_c$ referente a precipitação a 2 km de altitude. À direita dos mapas que representam o topo das nuvens de tempestades elétricas e a chuva em 2 km, são plotados os perfis médios de $Z_c$ por altitude, classificados como convectivos e estratiformes, nas cores vermelha e azul respectivamente e o perfil médio de $Z_c$ considerando todos os perfis: convectivo, estratiforme e outros, que é representado pela curva na cor preta.

Observe que o valor mínimo de FT foi de 0,6 raios por minuto, diferente de \citeonline{cecil2005}, que considerou a mínima taxa de raios no tempo para as PFs de 0,7 raios por minutos. Porém a resolução espacial da projeção do tempo de visada do LIS utilizada nesta tese possui resolução de 0,25$^{\circ}$ $\times$ 0,25$^{\circ}$  \cite{albrecht2009tropical,albrecht2011b}. Ao considerar a velocidade e altura da órbita do satélite, o tempo de observação do LIS em um ponto de 0,25$^{\circ}$ $\times$ 0,25$^{\circ}$ na superfície terrestre pode atingir até 102 segundos na região zenital. Então, as tempestades elétricas que possuíram apenas 1 raio e $VT_m$ de $\simeq$100 segundos, tiveram o mínimo valor de FT de 0,6 raios min$^{-1}$, sendo esta, a taxa mínima de raios no tempo detectável em uma tempestade elétrica do TRMM neste estudo.

% Na figura \ref{percetilFtaFt}, temos a série de FTA e FT ordenada, e a linha tracejada vertical corta o 90\% percentil dos índices. 

%\begin{figure}[!ht]
%  \centering
%  \includegraphics[height=6cm]{img/FtaFt/90thFtaFt}	 
%  \caption{90\textsuperscript{\underline{o}} percentil de FTA e FT.}
%  \label{percetilFtaFt}
%\end{figure}

\section{ÁREA E TEMPERATURA DO TOPO DA NUVEM}


%\begin{figure}[!hb]
%  \centering{  
%  \subfloat[Densidade de probabilidade de extensão em área.] { \includegraphics[height=7.5cm,trim=0 0 215 0,clip]{img/tb/TbAreas} \label{size}} \\
%  \subfloat[Densidade de probabilidade de temperatura de brilho em infravermelho.]{ \includegraphics[height=7.5cm,trim=220 0 0 0,clip]{img/tb/TbAreas} \label{tb}} 
%  }
%  \label{t_tb}
%  \caption{Estudo das frequências de ocorrências de tempestades elétricas selecionas pelo 90\textsuperscript{\underline{o}} percentil dos índices de FT e FTA, por extensão em área e por temperatura de brilho de topo das nuvens.}
%\end{figure}

%As tempestades elétricas com valores extremos de FT são maiores em extensão.

Na figura \ref{areaFTAFTA}, pode-se observar a dispersão entre os valore de FTA e FT em função da área das 94~733 tempestades elétricas selecionadas para o estudo da severidade neste capítulo. As linhas horizontais na figura \ref{areaFTAFTA} demarcam o 90\textsuperscript{\underline{o}} percentil das amostras de probabilidades de FTA (cor preta) e FT (cor azul). Verifica-se que os sistemas com tamanho entre 10$^2$--10$^3$ km$^2$, não ultrapassam 20 raios min$^{-1}$ de FT e as tempestades elétricas com FT superior a 100 raios por minuto, possuíram tamanho entre 10$^{4}$--10$^{6}$ km$^2$ e existe uma tendencia de aumento exponencial de FT conforme aumenta a área das tempestades elétricas, enquanto que FTA tende a diminui exponencialmente com o aumento da área das tempestades elétricas.

%Uma tempestade elétrica com 10$^5$ km$^2$, terá maior número de descargas observadas durante o tempo de visada do LIS do que uma com 10$^2$ km$^2$.


%\begin{figure}[!hb]
%  \centering 
%  \subfloat[areas versus taxa de raios]{ \includegraphics[height=7.5cm]{img/FtaFt/area_FTA_FT}\label{areaFTAFTA}} \\
%  \subfloat[volume de chuva]{\includegraphics[height=7.5cm]{img/FtaFt/volChuva_FTA_FT}\label{pdfFt}} 
%  \caption{Dispersão referente aos índices FTA e FT.}
%  \label{areaFtaFt}
%\end{figure} 
%Quando a densidade espacial de descargas aumenta muito em uma região com centenas de quilômetros quadrados, em torno de 100 descargas intranuvens para uma nuven-solo, como por exemplo os maiores valores de $Z=IC/CG$ mostrados por \cite{evandro2009} na região de Campo Grande - MS no Brasil, a capacidade do LIS de identificar brilhos transientes provavelmente fica comprometida devido a resolução horizontal da CCD.

Ao normalizar a taxa de raios no tempo por $A_t$, o número de raios fica diluído na extensão do sistema, evidenciando que os maiores valores de FTA correspondem aos sistemas com as maiores densidades espaciais de raios, cuja a área e o número de raios são menores do que nos sistemas com extremos de FT.

Na figura \ref{volchuvaFTAFT}, referente dispersão entre os valore de FTA e FT em função do volume de chuva, observa-se que conforme aumenta FT o volume de chuva das tempestades elétricas também aumenta exponencialmente, de maneira semelhante ao aumento de FT com a área (figura \ref{areaFTAFTA}). Para FTA, há um comportamento inverso. Conforme aumenta FTA, o volume de chuva dos sistemas diminui. 

Conforme mostra as linhas horizontais na figura \ref{volchuvaFTAFT} que demarcam o 90\textsuperscript{\underline{o}} percentil das amostras de probabilidades de FTA (cor preta) e FT (cor azul), pode-se observar que as tempestades elétricas que causam os maiores volumes de chuva estão associadas aos extremos de FT, porém os critérios de severidade de tempo\footnote{Frentes de rajadas com velocidade superior a 92,6 km h$^{-1}$, queda de granizo com diâmetro maior do que 1,9 cm ou tornados} conforme descrevem \cite{carey1998, williams1999, zipser2006}, não associam-se com o volume de chuva.  No entanto, para populações localizadas em regiões de várzea de rios, como por exemplo a população ribeirinha da Amazônia, ou em regiões serranas como Teresópolis, Nova Friburgo, Petrópolis no Rio de Janeiro, grandes volumes de chuva podem representar condições de tempo severo, pois causam deslizamentos de terra, alagamentos, perdas de vidas, etc, que provocam danos enormes à sociedade \cite{INPEchuvaSevera, WikichuvaSevera, BBC}.

%Apesar de que populações localizadas em regiões de várzea de rios, como por exemplo a população ribeirinha da Amazônia, ou em regiões serranas como Teresópolis, Nova Friburgo, Petrópolis no Rio de Janeiro, grandes volumes de chuva associadas a vastas extensões de precipitação estratiforme, devido a atuação de sistemas como frentes ou ZCAS, podem causar deslizamentos de terra, alagamentos, etc,  que podem causar dados a sociedade até maiores do que a passagem de um tornado.

\begin{figure}[!ht]
  \centering
  \includegraphics[height=9.5cm]{img/FtaFt/area_FTA_FT}   
  \caption{Dispersão entre as áreas das tempestades elétricas e os valores de FTA e FT. As linhas horizontais marcam os valores de FTA (cor preta) e FT (cor azul) referente ao 90\textsuperscript{\underline{o}} percentil das respectivas amostragem.}
  \label{areaFTAFTA}  
\end{figure}

\begin{figure}[!hb]
  \centering 
  \includegraphics[height=9.5cm]{img/FtaFt/volChuva_FTA_FT}
  \caption{Dispersão entre o volume de chuva das tempestades elétricas e os valores de FTA e FT.  As linhas horizontais marcam os valores de FTA (cor preta) e FT (cor azul) referente ao 90\textsuperscript{\underline{o}} percentil das respectivas amostragem.}
  \label{volchuvaFTAFT}
\end{figure}


Na figura \ref{size} é apresentada a distribuição de probabilidade de ocorrência de área das tempestades elétricas apenas para os valores de FTA e FT acima do 90\textsuperscript{\underline{o}} percentil de ocorrência. Observa-se que os valores extremos de FTA e FT correspondem a sistemas com tamanhos bem distintos. Por exemplo, verifica-se que as máximas probabilidades de tempestades elétricas dos extremos de FTA, ocorrem em sistemas com área 3 ordem de grandeza menor do que nos extremos de FT.


\begin{figure}[!ht]
  \centering
  \includegraphics[height=9.5cm]{img/tb/areasSeveras}   
  \caption{Densidade de probabilidade de extensão em área das 9475 tempestades elétricas com valores extremos de FTA  e das 9475 tempestades elétricas com valores extremos de FT.}
  \label{size}  
\end{figure}

Considerando também apenas os extremos de FTA e FT, é mostrado na figura \ref{tb}, a frequência de ocorrência dos pixeis de temperaturas de brilho ($T_b$ $\leq$ 258 K) que definiram as áreas das tempestades elétricas. Observa-se que o maior valor de probabilidade para a curva das tempestades elétricas com índice extremo de FTA possui temperatura de topo de nuvens aproximadamente 10 K mais frias do que nas tempestades elétricas com extremos de FT, indicando que a convecção nos sistemas extremos de FTA é mais profunda na atmosfera na maioria das observações.

\citeonline{morales2003} ao desenvolver o algoritmo  SIRT, mostram que as regiões com temperatura de brilho inferior a 215 K e com ocorrência de \textit{sferics} foram as regiões categorizadas como de maior precipitação associada. Nesta tese, ao selecionar as tempestades elétricas com índice extremo de FTA, os maiores valores de probabilidade de ocorrência, conforme é mostrado na figura \ref{tb}, concentram-se em temperaturas de brilho abaixo de 215 K.

\sigla{name={SIRT},description={\textit{Sferics Infrared Rainfall Technique} }}



\begin{figure}[!ht]
  \centering 
  \includegraphics[height=9.5cm]{img/tb/TbSeveras}
  \caption{Densidade de probabilidade de temperatura de brilho em infravermelho (VIRS 10,8$\mu$m) do topo das nuvens das 9475 tempestades elétricas com valores extremos de FTA  e das 9475 tempestades elétricas com valores extremos de FT.}
  \label{tb}
\end{figure}


Nas figuras \ref{pdffracaoFTA}, \ref{cdffracaoFTA}, \ref{pdffracaoFT} e  \ref{cdffracaoFT}, apresenta-se o estudo das probabilidades das frações de chuva para as tempestades elétricas com valores extremos de FTA e FT. As curvas denominadas como convectivo, estratiforme e outros, correspondem a fração de área de chuva associada aos perfis do PR classificados como convectivo, estratiforme e outros em relação a área total de chuva do sistema. A curva denominada na legenda como chuva total corresponde a fração da área de chuva em relação a área total ($A_t$) da tempestade elétrica que é definida pela $T_b$ $\leq$ 258 K. A curva denominada como varredura do PR, mostra a fração da $A_t$ que esteve dentro do alcance da varredura do PR.

\begin{figure}[!ht]
  \centering
  \includegraphics[height=9cm]{img/FtaFt/fracaoChuva_pdf_topFTA}   
  \caption{Densidade de probabilidade das frações de áreas de chuva das tempestades elétricas extremas de FTA, que foram classificadas (2A23) como convectiva (vermelha), estratiforme (bege) e outros (verde), em relação a toda a área de chuva observada pelo PR; das frações de chuva total (preta), que corresponde as frações das áreas de chuva observadas pelo PR em relação a $A_t$ das tempestades elétricas e também das frações das áreas das tempestades elétricas contidas na varredura do PR (azul).}
  \label{pdffracaoFTA}  
\end{figure}

\begin{figure}[!ht]
  \centering 
  \includegraphics[height=9cm]{img/FtaFt/fracaoChuva_cdf_topFTA}
  \caption{Densidade de probabilidade cumulativa das frações de áreas de chuva das tempestades elétricas extremas de FTA.}
  \label{cdffracaoFTA}
\end{figure}

\begin{figure}[!ht]
  \centering
  \includegraphics[height=9cm]{img/FtaFt/fracaoChuva_pdf_topFT}   
  \caption{Densidade de probabilidade das frações de áreas de chuva das tempestades elétricas extremas de FT, que foram classificadas (2A23) como convectiva (vermelha), estratiforme (bege) e outros (verde), em relação a toda a área de chuva observada pelo PR; das frações de chuva total (preta), que corresponde as frações das áreas de chuva observadas pelo PR em relação a $A_t$ das tempestades elétricas e também das frações das áreas das tempestades elétricas contidas na varredura do PR (azul).}
  \label{pdffracaoFT}  
\end{figure}

\begin{figure}[!ht]
  \centering 
  \includegraphics[height=9cm]{img/FtaFt/fracaoChuva_cdf_topFT}
  \caption{Densidade de probabilidade cumulativa das frações de áreas de chuva das tempestades elétricas extremas de FT.}
  \label{cdffracaoFT}
\end{figure}


Verifica-se na curva denominada como varredura do PR na figura \ref{pdffracaoFT} que para a maioria dos sistemas com FT extremo, o PR conseguiu observar apenas 30\% da $A_t$ das tempestades elétricas. Na curva denominada como varredura do PR da figura \ref{pdffracaoFTA} referente as tempestades elétricas com FTA extremo, o PR observou com a maior probabilidade 100\% de $A_t$, efeito que se deve ao diferente tamanho dos sistemas extremos de FTA em relação aos extremos de FT, como mostra a figura \ref{size}. A mediana da amostragem de fração de chuva total das tempestades elétricas extremas de FTA (figura \ref{pdffracaoFTA}) mostra o valore de 62\%, enquanto a mediana da amostragem de fração de chuva total das tempestades elétricas extremas de FT (figura \ref{pdffracaoFT}), foi de apenas 5\%, pois a varredura do PR é menor do que a do VIRS e as tempestades elétricas com os maiores valores de FT abrangem uma extensão que ultrapassa o alcance do PR. 

Apesar da diferença de amostragem espacial do PR para com os FTA e FT, a amostragem estatística é suficiente para mostrar que na mediana, FTA tem 72\% de fração convectiva e 32\% estratiforme, conforme verifica-se nos valores de fracão de chuva associado ao 50\textsuperscript{\underline{o}} percentil das distribuições cumulativas de probabilidade da figura \ref{cdffracaoFTA}. Por outro lado, o 50\textsuperscript{\underline{o}} percentil das distribuições cumulativas da figura \ref{cdffracaoFT}, mostram que os extremos de FT tem 22\% de fração convectiva e 65\% de fração estratiforme. 

Estes resultados corroboram com as observações de \citeonline{Rasmussen2011}, em que os extensos sistemas convectivos estiveram associados com regiões de vasta extensão estratiforme. Anteriormente \citeonline{learyHouse1979}, observaram que sistemas convectivos em desenvolvimentos tem a maior área convectiva, enquanto os em dissipação e mais fracos possuem maior área estratiforme. A distribuição de probabilidade de $T_b$ da figura \ref{tb}, mostra que os extremos de FTA tem topos mais frios, o que se espera de nuvens em desenvolvimento. Portanto, a maior fração convectiva e menor tamanho das tempestades elétricas com FTA extremos sugerem sistemas em via de maturar ou sistemas novos que conforme as vão entrando em estágio maduro e dissipativo, vão ganhando área de chuva estratiforme e podem começar a se enquadrar no grupo dos extremos de FT. 

%Avaliando a densidade de probabilidade da fração convectiva e fração estratiforme das tempestades elétricas, figuras \ref{pdffracaoFTA} e \ref{pdffracaoFT}, verifica-se que para os extremos de FTA as tempestades elétricas são mais frequentemente observadas com 70\% de área convectiva e 30\% de área estratiforme, enquanto que para os extremos de FT, 20\% de fração convectiva e 75\% de fração estratiforme.
%juntamente com a as respectivas distribuições de probabilidade acumulativa,
%Os sistemas selecionados pelo 90\textsuperscript{\underline{o}} percentil do índice FT possuem maior extensão em área e maior volume de chuva. São sistemas com vasta extensão estratiforme conforme descrevem \citeonline{Rasmussen2011}. As regiões das tempestades elétricas com precipitação convectiva, as quais tem potencial de gerar chuva de granizo, frentes de rajada e tornados, ocupam área bem menor do que as áreas com  precipitação estratiforme \cite{Jr2007}.


%.....
%Para avaliar qual dos índices representaram a maior severidade de tempo, a morfologia da estrutura 3D da precipitação foi estudada por meio dos diagramas CFAD, CCFAD, CFTD e CCFTD. %para os 10\% das amostras de FT e FTA com os maiores valores.
%......

\section{SEVERIDADE COM BASE NA ESTRUTURA 3D DA PRECIPITAÇÃO}

%Nesta etapa iremos avaliar a intensidade convectiva com base nos perfis de $Z_c$ do PR
%, contidos nos sistemas com índices extremos de FTA e FT. 

%Nas figuras ,  
Nesta etapa iremos avaliar a intensidade convectiva com base nos perfis de $Z_c$ do PR. Para tanto, foram calculados os CFADs das tempestades elétricas extremas de FTA e FT, para suas frações com raios (figuras \ref{ftacfadwith} e \ref{ftcfadwith}) e sem raios (figuras \ref{ftacfadwithout}, \ref{ftcfadwithout}), para os extremos ocorridos em cada região de 10 por 10 graus de latitude e longitude na superfície sobre a AS. Para localizar a caixa (10$^{\circ}$ $\times$ 10$^{\circ}$) em que cada sistema esteve contido, foi calculada a latitude e longitude do centro geométrico da área definida por cada sistema. Novas amostragens de probabilidade de FTA e FT, foram obtidas da mesma maneira que mostrado na figura \ref{pdfFTAFT} ou \ref{cdfFTAFT}, porém para cada região de 10$^{\circ}$ $\times$ 10$^{\circ}$. O estudo da precipitação tridimensional foi feito apenas para as tempestades elétricas com valores maiores ou igual ao valor do 90\textsuperscript{\underline{o}} percentil das amostragem de FTA e FT de cada região de 10$^{\circ}$ $\times$ 10$^{\circ}$.

As posições geográficas dos eventos do LIS e dos perfis de $Z_c$ válidos do PR, foram projetadas em uma grade regular de 0,05 graus. Os perfis de $Z_c$ projetados em pontos de grade em que tiveram eventos do LIS, definiram as regiões aqui denominadas como precipitação dos núcleos de raios. A figura \ref{nucleosRaios}, ilustra as regiões dos núcleos de raios das tempestades elétricas.


\begin{figure}[!htb]
  \centering{  
  \subfloat[]{\includegraphics[height=1.0cm]{img/grids/nucleosRaios/colorbar_virs}\label{barravirs}}\\
  \subfloat[]{\includegraphics[height=6.cm,trim=0 47cm 0 0,clip]{img/grids/nucleosRaios/001_topFTA_25471_0003} \label{nr1}} 
  \subfloat[]{\includegraphics[height=6.cm,trim=0 47cm 0 0,clip]{img/grids/nucleosRaios/002_topFTA_36502_0001} \label{nr2}} \\
  \subfloat[]{\includegraphics[height=6.cm,trim=0 47cm 0 0,clip]{img/grids/nucleosRaios/003_topFTA_03444_0001} \label{nr3}} 
  \subfloat[]{\includegraphics[height=6.cm,trim=0 47cm 0 0,clip]{img/grids/nucleosRaios/004_topFTA_05694_0005} \label{nr4}} 
  }
  \caption{Núcleos de raios das tempestades elétricas. Os pontos ``." na cor  verde são os eventos e os símbolos de positivo ``+" na cor preta são os raios. Os pixeis em vermelho são as regiões dos núcleos de raios, definidas a partir da projeção dos eventos em uma grade regular de 0,05$^{\circ}$.} %(0.05$^{\circ}$ $\times$ 0.05$^{\circ}$)
\label{nucleosRaios}
\end{figure}


% Observe que na parte superior e inferior das figuras \ref{nr1}, \ref{nr2}, \ref{nr3} e \ref{nr4}, há informações referentes a data e hora em que o sistema foi observado, número de raios/eventos (FL/EV), fração do sistema observado pelo PR, área do sistema (A), semi-eixo maior (a), menor (b), distância focal (2c) e excentricidade (e) de uma elipse ajustada as dimensões do sistema. A barra de cores na figura \ref{barravirs}, corresponde as temperaturas de brilho ($<$258K) associada a radiância espectral em infravermelho observada pelo VIRS.

Após definidas as regiões eletricamente ativas de cada sistema com FTA ou FT extremo, os CFADs foram calculados para a precipitação dos núcleos de raios das tempestades elétricas, figuras \ref{ftacfadwith} e \ref{ftcfadwith}, e para a precipitação fora dos núcleos de raios, figuras \ref{ftacfadwithout} e \ref{ftcfadwithout}.

%----------------------------------------
\begin{sidewaysfigure}
\centering
\includegraphics[width=19.5cm]{img/precipitacao3d/severo/percentil/90th/cfad10_semraio_topFTA_percentil}
\caption{CFADs para os extremos de FTA. Porção da precipitação sem raios.}
\label{ftacfadwithout}
\end{sidewaysfigure} 
\begin{sidewaysfigure}
\centering
\includegraphics[width=19.5cm]{img/precipitacao3d/severo/percentil/90th/cfad10_semraio_topFT_percentil}
\caption{CFADs para os extremos de FT. Porção da precipitação sem raios.}
\label{ftcfadwithout}
\end{sidewaysfigure} 
\begin{sidewaysfigure}
  \centering
  \includegraphics[width=19.5cm]{img/precipitacao3d/severo/percentil/90th/cfad10_comraio_topFTA_percentil}
  \caption{CFADs para os extremos de FTA. Porção da precipitação com raios.}
  \label{ftacfadwith}   
\end{sidewaysfigure} 
\begin{sidewaysfigure}
  \centering
  \includegraphics[width=19.5cm]{img/precipitacao3d/severo/percentil/90th/cfad10_comraio_topFT_percentil}
  \caption{CFADs para os extremos de FT. Porção da precipitação com raios.}
  \label{ftcfadwith}   
\end{sidewaysfigure} 

Note que no canto superior direito de cada CFAD temos alguns valores estatísticos que representam: (\%)  a porcentagem de perfis convectivos, estratiformes e outros, respectivamente; (P) o número de perfis do PR computados; (L) o número de ocorrência de $Z_c$ no nível de altitude de máxima ocorrência; (H) o nível de altitude, em quilômetros, aonde ocorreu o máximo de ocorrências de $Z_c$; (N) o número de tempestades elétricas computadas.

\simbolo{name={\%},description={Nos diagramas, CFAD, CCFAD, CFTD e CCFTD, representam: a porcentagem de perfis convectivos, estratiformes e outros, respectivamente}}
\simbolo{name={P},description={Nos diagramas, CFAD, CCFAD, CFTD e CCFTD, representam: número de perfis do PR computados}}
\simbolo{name={L},description={Nos diagramas, CFAD, CCFAD, CFTD e CCFTD, representam: o número de ocorrência de $Z_c$ no nível de altitude de máxima ocorrência}}
\simbolo{name={H},description={Nos diagramas, CFAD, CCFAD, CFTD e CCFTD, representam: o nível de altitude, em quilômetros, aonde ocorreu o máximo de ocorrências de $Z_c$}}
\simbolo{name={N},description={Nos diagramas, CFAD, CCFAD, CFTD e CCFTD, representam: o número de tempestades elétricas computadas}}

Comparando os CFADs da chuva com e sem raios, representados para os extremos de FTA nas figuras \ref{ftacfadwith} e \ref{ftacfadwithout} e para os extremos de FT, nas figuras \ref{ftcfadwith} e \ref{ftcfadwithout}, é evidente que a fração das tempestades elétricas sem raios é a parte de menor velocidade vertical e a fração eletricamente ativa é a região com maior velocidade vertical. Os níveis de contorno de probabilidades dos CFADs da precipitação sem raios possuem suas máximas altitudes aproximadamente 3 quilômetros abaixo das máximas altitudes atingidas pelos contornos dos CFADs da precipitação com raios. A fração da precipitação sem raios dos sistemas é composta predominantemente por perfis estratiformes enquanto que a fração com raios os perfis convectivos são predominantes.


%\begin{figure}[!ht]
%  \centering
%  \includegraphics[height=13.5cm]{img/precipitacao3d/severo/percentil/90th/cfad10_semraio_topFT_percentil}%
% \caption{CFADs para os extremos de FT. Porção da precipitação sem raios.}
% \label{ftcfadwithout}
%\end{figure} 

%\begin{figure}[!ht]
%  \centering
%   \adjustbox{trim={0\width} {0.435\height} {0\width} {0\height} , clip}%
%   {\includegraphics[width=\textwidth]{img/precipitacao3d/severo/percentil/90th/cCumFad_10deg_semraio_topFTpercentil}}
% \caption{CCFDs para os extremos de FT entre 20S-10N e 90W-30W. Porção da precipitação sem raios.}
% \label{ftccfadwithout}
%\end{figure} 

Se avaliarmos apenas os níveis de contorno com probabilidade entre 2-3,7\% (cor verde), os valores de $Z_c$ entre 0-2 km de altitude da porção sem raios, figuras \ref{ftacfadwithout} e \ref{ftcfadwithout}, não ultrapassaram 40 dBZ em nenhuma região da AS, enquanto que para a porção de chuvas com raios, figuras \ref{ftacfadwith} e \ref{ftcfadwith}, os valores de $Z_c$ atingem 45-50 dBZ. Entre 5-7 km os contornos de probabilidade dos CFADs da precipitação com raios mostram valores de $Z_c$ entre 5-10 dBZ maiores do que na precipitação sem raios, pois a eletrificação das nuvens depende do crescimento dos hidrometeoros na região de fase mista. Havendo maior velocidade vertical, descargas elétricas e maior volume de água na região de fase mista, consequentemente, espera-se maiores volumes de chuvas na superfície (0-2 km) sobre as regiões dos núcleos de raios \cite{Petersen1998}. 


%A convecção é mais ativa nas regiões dos núcleos de raios, aonde a precipitação está associadas com frentes de rajadas, chuvas de granizo e enchentes rápidas. Fora dos núcleos de raios temos a parte da precipitação mais estratiforme, composta por hidrometeoros que não possuem velocidade terminal suficiente para precipitar nos núcleos de raios, e caem mais afastados da região eletricamente ativa.     % Dependendo principalmente das condições de calor umidade e cisalhamento vertical do vento as células 


% \caption{Diagramas de Contorno de Frequência por Altitude (CFADs). Em cada CFAD pode-se verificar: a porcentagem (\%) de perfis convectivos, estratiformes e outros, respectivamente; (P) o numero de perfis do PR computados, (L) o número de ocorrência de refletividade no nível de máxima ocorrência e (H) o nível de máxima ocorrência.}

A figura \ref{ftcfadwithout} mostra que a precipitação sem raios dos extremos de FT na região tropical, entre 20S-10N e 90W-30W, possui banda brilhante marcada entre 4-5 km de altitude, principalmente nos perfis com probabilidade de ocorrência entre 2-5,3\%, nas cores de contorno em verde e amarelo, o que claramente demostra o predomínio de áreas estratiformes, caracterizadas processo de formação de flocos de neve e agregados, que são pouco eficiência no carregamento das nuvens de tempestades elétricas. 

Podemos observar a banda brilhante entre 20S-10N e 90W-30W  dos sistemas com  extremo de FT na porção sem raios de maneira mais elucidativa por meio dos CCFADs da figura \ref{ftccfadwithout}, os quais evidenciam que entre o 12\textsuperscript{\underline{o}} e o 95\textsuperscript{\underline{o}} percentil da amostragem de probabilidade de $Z_c$ por altitude, há uma queda no valor de $Z_c$ logo abaixo de 5 quilômetros de altitude em cada região de 10 por 10 graus. 

No entanto, para a região entre 20S-10N e 90W-30W, ao avaliar os CFADs da figura \ref{ftacfadwithout} ou CCFADs da figura \ref{ftaccfdsubtrop}, que representam a porção sem raios da precipitação tridimensional dos sistemas com índice extremo de FTA, não se observa banda brilhante marcada nos contornos de probabilidade de $Z_c$ por altitude. Conforme os níveis de altitude diminuem, não se observa uma diminuição abrupta de $Z_c$ logo abaixo de 5 quilômetros, devido a evaporação, o que indica menor presença de neve derretendo.

%o que indica menor presença de neve derretendo e processos de colisão-coalescência mais intensos para o grupo dos sistemas extremos de FTA do que para o grupo dos sistemas extremos de FT. 
%observa-se que, entre 20S-10N e 90W-30W,  
%a banda brilhante é evidente apenas nas regiões costeiras e oceânicas, nas caixas entre 0-10N  %e 90-80W, entre 10-0S e 40-30W e entre 20-10S e 70-60W.

A precipitação sem raios das tempestades elétricas entre 20S-10N e 90W-30W referente aos extremos de FT, possuem a mediana das distribuições cumulativas de probabilidades com valores de $Z_c$ inferiores do que quando compara-se com os extremos de FTA, os quais possuem perfis de $Z_c$ com maior aleatoriedade, mas atingem valores de $Z_c$ superiores. Observe as diferenças entre os CCFADs das figuras \ref{ftaccfdsubtrop} e figura \ref{ftccfadwithout}. Note como entre 20S-10N e 90W-30W os contornos de probabilidade cumulativa são mais alargados para a precipitação sem raios dos extremos de FTA do que para a precipitação sem raios dos extremos de FT, indicando menor aleatoriedade dos perfis de $Z_c$ para os extremos de FT.

\begin{sidewaysfigure}%[!H]
  \centering
  \includegraphics[width=19.5cm]{img/precipitacao3d/severo/percentil/90th/cCumFad_10deg_semraio_topFTApercentil}
  \caption{CCFDs para os extremos de FTA. Porção da precipitação sem raios.}
  \label{ftaccfdsubtrop}   
\end{sidewaysfigure} 

\begin{sidewaysfigure}%[!H]
  \centering
  \includegraphics[width=19.5cm]{img/precipitacao3d/severo/percentil/90th/cCumFad_10deg_semraio_topFTpercentil}
  \caption{CCFDs para os extremos de FT. Porção da precipitação sem raios.}
  \label{ftccfadwithout}   
\end{sidewaysfigure} 

%\begin{figure}[!ht]
%  \centering  
%  \adjustbox{trim={.0\width} {.04\height} {0\width} {.565\height},clip}%
%  \centering  
%  \adjustbox{trim={.349\width} {.045\height} {.322\width} {.565\height},clip}%  
%  {\includegraphics[width=27cm] {img/precipitacao3d/severo/percentil/90th/cCumFad_10deg_semraio_topFTApercentil}}
% \caption{CCFDs para os extremos de FTA entre 40-20S e 70-50W. Porção da precipitação sem raios.}
% \label{ftaccfdsubtrop}
%\end{figure} 

Na região entre 40-20S e 70-50W que engloba a Bacia do Prata, a banda brilhante não é tão pronunciada na área sem raios, tanto para os extremos de FTA (figuras \ref{ftacfadwithout} e \ref{ftaccfdsubtrop}) quanto para os extremos de FT (figuras \ref{ftacfadwithout} e \ref{ftccfadwithout}), como é observado nas regiões mais tropicais, entre 20N--10S e 80--30W. Provavelmente, as condições de instabilidade baroclínica associada a passagens de sistemas transientes, e SCM conforme descrevem \citeonline{Durkee2009}, entre 40-20S e 70-50W, promovem condições de cisalhamento vertical do vento que intensifica a velocidade vertical nas frações sem raios das tempestades elétricas subtropicais, o que favorece a formação de mais cristais de gelo, agregados e o \textit{graupel}.

Observe como a linha de contorno na cor preta no 50\textsuperscript{\underline{o}} percentil do CCFAD entre 40-20S e 70-50W nas figuras \ref{ftaccfdsubtrop} e \ref{ftccfadwithout}, indicam maiores valores de $Z_c$ para a fração sem raios das tempestades elétricas com índice FT extremo, mesmo que a estatística na parte superior direita de cada CCFAD indique maior percentual de perfis convectivos para a porção sem raios dos extremos de FTA. Porém, observe que acima do 80\textsuperscript{\underline{o}} percentil de ocorrência de perfis de $Z_c$ das tempestades elétricas com FTA extremo (figura \ref{ftaccfdsubtrop}), os valores de $Z_c$ passam a ser maiores do que para as tempestades elétricas com FT extremo. 

%, evidenciando que a chuva sem raios dos sistemas com as maiores taxas de raios no tempo é mais severa nesta região.

%\begin{figure}[!ht]
%  \centering  
%  \adjustbox{trim={.349\width} {.045\height} {.%322\width} {.565\height},clip}%
%  {\includegraphics[width=27cm] {img/precipitacao3d/severo/percentil/90th/cCumFad_10deg_semraio_topFTpercentil}}
% \caption{CCFDs para os extremos de FT entre 40-20S e 70-50W. Porção da precipitação sem raios.}
% \label{ftccfdsubtrop}
%\end{figure} 

Os CFADs referentes as tempestades elétricas com FTA extremo possuem contornos de probabilidade em níveis de altitude mais elevados do que os CFADs dos sistemas com FT extremo, tanto para a fração com raios quanto para a fração sem raios da precipitação dos sistemas. Como o último nível de altitude dos CFADs deste trabalho é limitado por altitudes com até 10\% de L, a maior definição de probabilidades de ocorrência de $Z_c$ em altitude indica que a convecção é mais intensa nas tempestades elétricas com FTA extremo do que nas tempestades elétricas de FT extremos.

%A diferença mais notável pode ser observada entre a figura \ref{ftacfadwithout} e \ref{ftcfadwithout} para 0S-10S e 50W-60W, que abrange principalmente o estado do Pará, e parte do Amazônas, Tocantis e Mato Grosso. O CFAD em \ref{95oFta}ef{ftacfadwithout} define valores de probabilidade em altitude 2 km mais elevada do que em \ref{ftcfadwithout}.


%Nas regiões entre 10N-0S e 70W-80W e entre 20S-40S e 50W-60W, em que \cite{cecil2005} apontam como região das tempestades mais severas na América do Sul, os CFADs em \ref{ftacfadwith} e \ref{ftacfadwithout} possuem contornos de probabilidade aproximadamente 1 km mais elevado do que em \ref{ftcfadwith} e \ref{ftcfadwithout}.



%A precipitação é bem mais frequente próxima da superfície, entre 0-3 km de altitude. Acima da região de fase mista a precipitação é mais rara de ocorrer. Em \cite{liu2008}, é mostrado que a densidade espacial de sistemas com no mínimo 20 dBZ em 2 km de altitude é globalmente maior do que os sistemas que atingem 20 dBZ em níveis superiores de altitude.


%A região de 10 por 10 graus, a qual o valor de H marcado no topo direito de cada CFAD, é menor para a precipitação dos núcleos de raios dos sistemas com extremo de FTA, figura \ref{ftacfadwith}, do que para  a precipitação dos núcleos de raios dos sistemas com extremos de FT, figura \ref{ftcfadwith}, e mesmo assim, o CFAD dos extremos de FTA, figura \ref{ftacfadwith}, possuiu maior altitude nos níveis de contorno de probabilidade de $Z_c$, o índice FTA mostra que a chuva  
%esteve associado com maior severidade de tempo do que FT.
%Pois, mesmo que a refletividade mais ocorrente esteja abaixo da região de mistura, a precipitação também é frequente conforme o aumento da altitude, mostrando que nestas regiões, os sistemas com índice FTA extremo têm maior número de ocorrência de chuva 
%mais chuvas na superfície e também maior precipitação acima de 10 km de altitude.
%com bastante representatividade estatística.
%maior quantidade de hidrometeoros na região de mistura e

%Por exemplo na região do Panamá, Colômbia e Equador, entre 10N-0S e 70W-80W, o CFAD da figura \ref{ftacfadwith} possui contornos de probabilidade até 16 km de altitude. Na figura \ref{ftcfadwith}, os níveis de contorno param em 15 km.

Nos CFADs da precipitação dos núcleos de raios na figura \ref{ftacfadwith}, dos sistemas extremos de FTA, mostram valores de refletividade entre 1-3 dBZ maiores do que em relação aos sistemas extremos de FT da figura \ref{ftcfadwith}, principalmente quando observa-se os contornos de probabilidade de $Z_c$ acima de 5 km de altitude. Para a precipitação entre 0-2 km de altitude os valores são mais semelhantes entre as tempestades elétricas selecionadas por FTA e FT. 

%Porém, nos sistemas extremos de FTA, figura \ref{ftacfadwith}, há um estreitamento da região de contorno com os maiores valores de probabilidade associada a chuva na superfície, entre 3-5\%. Entre 20S-40S e 40-70W, o estreitamente é maior do que as demais regiões mostrando que as chuvas possuem maior probabilidade de estarem associadas com valores de 45 dBZ em \ref{ftacfadwith}.      

Os contornos de probabilidade entre 0,001-0,5\% das figuras \ref{ftacfadwith} e \ref{ftcfadwith}, revelam os valores dos perfis de $Z_c$ mais raros e mais intensos ocorridos nos sistemas com índice extremo de FTA e FT, os quais provavelmente estiveram associados a condições de tempo severo, ou seja, com enchentes rápidas, alta taxa de raios, chuva de granizo, fortes rajadas de vento e tornados. 
% nas figuras \ref{ftacfadwith} e \ref{ftcfadwith}

Os maiores valores de $Z_c$ foram registrados na análise da precipitação 3D das tempestades elétricas com FTA extremos (figura \ref{ftacfadwith}), entre 20S-40S e 40W-70W sobre a Bacia do Rio da Prata, que abrange o Sul do Brasil, Paraguai, Uruguai e Argentina, em que a precipitação dos núcleos de raios atingiram valores de $Z_c$ superior a 45 dBZ entre 10-15 km de altitude e $Z_c$ acima de 55 dBZ na superfície, como mostram os contornos de probabilidade entre 0,001-0,5\% da figura \ref{ftacfadwith}. Este efeito pode estar associado com a dinâmica de formação de Sistemas Convectivos de Meso-escala, conforme descrevem \citeonline{Velasco1987} e \citeonline{Durkee2009} somados aos efeitos de topografia, como por exemplo na região da Serra de Córdoba na Argentina, em que \citeonline{Rasmussen2011} observaram grande ocorrência de convecção profunda.


\subsection{A precipitação 3D e o perfil atmosférico de temperatura.}

O estudo dos perfis de $Z_c$ por nível de temperatura é realizado por meio dos diagramas CCFTD e CFTD, expostos nas figuras \ref{ccftd_fta_com}, \ref{ccftd_ft_com}, \ref{cftd_fta_com} e \ref{cftd_ft_com}, associados as tempestades elétricas com índice extremo de FTA e FT de cada região de 10$^{\circ}$ $\times$ 10$^{\circ}$ de latitude e longitude, apenas referente a precipitação dos núcleos de raios.

Iremos avaliar a intensidade convectiva dos sistemas com índice  FTA e FT extremo em determinadas regiões, não apenas por meio dos valores de $Z_c$, mas com base na taxa de variação de $Z_{c}$ em função da temperatura, a partir dos contornos dos CCFTDs das figuras \ref{ccftd_fta_com} e \ref{ccftd_ft_com}, que representam os perfis de $Z_c$ em função da temperatura para o 30\textsuperscript{\underline{o}}, 50\textsuperscript{\underline{o}}, 70\textsuperscript{\underline{o}} e 95\textsuperscript{\underline{o}} percentil das amostragens de probabilidades expressas nos CFTDs das figuras \ref{cftd_fta_com} e \ref{cftd_ft_com}.	

\begin{sidewaysfigure}%[!H]
\centering
\includegraphics[width=19.5cm]{img/precipitacao3d/severo/percentil/90th/cftd_10deg_comraio_topFTApercentil}
\caption{CFTDs para os extremos de FTA. Porção da precipitação com raios.}
\label{cftd_fta_com}
\end{sidewaysfigure} 

\begin{sidewaysfigure}%[!H]
\centering
\includegraphics[width=19.5cm]{img/precipitacao3d/severo/percentil/90th/ccftd_10deg_comraio_topFTApercentil}
\caption{CCFTDs para os extremos de FTA. Porção da precipitação com raios.}
\label{ccftd_fta_com}
\end{sidewaysfigure} 

\begin{sidewaysfigure}%[!H]
\centering
\includegraphics[width=19.5cm]{img/precipitacao3d/severo/percentil/90th/cftd_10deg_comraio_topFTpercentil}
\caption{CFTDs para os extremos de FT. Porção da precipitação com raios.}
\label{cftd_ft_com}
\end{sidewaysfigure} 

\begin{sidewaysfigure}%[!H]
\centering
\includegraphics[width=19.5cm]{img/precipitacao3d/severo/percentil/90th/ccftd_10deg_comraio_topFTpercentil}
\caption{CCFTDs para os extremos de FT. Porção da precipitação com raios.}
\label{ccftd_ft_com}
\end{sidewaysfigure} 

Portanto, a partir das amostragens de probabilidade do fator de refletividade $Z_c$ por nível de temperatura que define os diagramas CCFTD e CFTD conforme é descrito em \ref{metodologiaPrec3D},  extraí-se as linhas de contorno do CCFTD referentes as probabilidades cumulativas de 30\%, 50\%, 70\% e 95\%. Desta forma, obteve-se quatro funções 
\begin{equation}
f(x)=y ,
\end{equation} 
em que $y$ corresponde aos valores de $Z_c$ e $x$ os valores de temperatura. Fazendo a derivada 
\begin{equation}
f'(x_0)= - \dfrac{y_1 - y_0}{x_1 - x_0},
\label{devivadaContorno}
\end{equation}
então, pode-se avaliar a taxa de variação do fator de refletividade do radar ($Z_c$) por nível de temperatura (dBZ~\textsuperscript{o}C$^{-1}$) para diferentes regimes de precipitação, das mais frequentes até as mais raras. O sinal de menos na equação \ref{devivadaContorno} deve-se ao fato da temperatura diminuir com a altitude e queremos mostrar taxas positivas associadas ao aumento de $Z_c$ e taxas negativas quando há diminuição de $Z_c$.
\simbolo{name={$f(x)=y$},description={Função de uma variável}} 

\begin{figure}[!ht]
  \centering
  \subfloat[Perfis verticais de $Z_c$ de acordo com diferentes percentis do CCFTD.]{\includegraphics[height=9cm]{img/precipitacao3d/deriv_ccftd/Contornos_contornos_cdf_2_1} \label{contornosAmazonas}}
  \subfloat[Derivadas dos perfis verticais de $Z_c$ de diferentes percentis do CCFTD.]{\includegraphics[height=9cm]{img/precipitacao3d/deriv_ccftd/deriv_contornos_cdf_2_1} \label{derivaAmazonas}}
  \caption{Região central da Bacia do Rio Amazonas, entre 10-0S e 70-60W.}
  \label{deriv_amazonas}  
\end{figure} 
%Taxa de variação de $Z_c$ no perfil de temperatura atmosférico para a 
% nos diferentes quartis do CCFTD dos extremos de FTA,  figura \ref{ccftd_fta_com},  e dos extremos de FT, figura \ref{ccftd_ft_com}. 

Para a região central da Bacia do Rio Amazonas, entre 10-0S e 70-60W, os valores de $Z_c$ com os contornos de probabilidade do 30\textsuperscript{\underline{o}}, 50\textsuperscript{\underline{o}}, 70\textsuperscript{\underline{o}} e 95\textsuperscript{\underline{o}} percentil referente as tempestades elétricas com FTA extremo (figura \ref{ccftd_fta_com}), possuem valores entre 1-3 dBZ maiores do que os valores de $Z_c$ referentes as tempestades elétricas com FT extremo (figura \ref{ccftd_ft_com}), principalmente acima da isoterma de 0 \textsuperscript{o}C, o que indica maior concentração ou densidade ou diâmetro dos hidrometeoros na região fria\footnote{Com temperaturas abaixo de  0\textsuperscript{o}C} dos núcleos de raios associados aos sistemas extremos de FTA. Este efeito pode ser observado de forma mais elucidativa na figura \ref{contornosAmazonas}, em que os perfis verticais de $Z_c$ referentes aos contornos de probabilidade cumulativa do 30\textsuperscript{\underline{o}}, 50\textsuperscript{\underline{o}}, 70\textsuperscript{\underline{o}} e 95\textsuperscript{\underline{o}} percentil das amostragens de perfis de $Z_c$ em função da temperatura  para as tempestades elétricas com extremos de FTA e FT são plotados conjuntamente.

Adicionalmente, entre 10-0S e 70-60W, na figura \ref{derivaAmazonas}, observa-se que a taxa de diminuição de $Z_c$ referente ao 70\textsuperscript{\underline{o}} e 95\textsuperscript{\underline{o}} percentil, entre -40 \textsuperscript{o}C e -15 \textsuperscript{o}C dos extremos de FTA é maior do que nos extremos de FT. Para os percentis de 30\textsuperscript{\underline{o}} e 50\textsuperscript{\underline{o}}, apesar da intensidade de $Z_c$ ser maior entre -40 \textsuperscript{o}C e -15 \textsuperscript{o}C para os extremos de FTA (ver figura \ref{contornosAmazonas}), as taxas de diminuição em dBZ~\textsuperscript{o}C$^{-1}$ possuem valores semelhantes (ver figura \ref{derivaAmazonas}). Porém, para todos os quantis entre -15 \textsuperscript{o}C e 0 \textsuperscript{o}C da figura \ref{deriv_amazonas}, observa-se que os valores de $Z_c$ para os extremos de FT, sofreram os maiores decréscimos entre -8 \textsuperscript{o}C e 0 \textsuperscript{o}C, enquanto os valores de $Z_c$ para os extremos de FTA sofreram os maiores decréscimos entre -20 \textsuperscript{o}C e -10 \textsuperscript{o}C. % mostrando que os hidrometeoros dos sistemas extremos de FTA cresceram mais em regiões mais frias do que para os extremos de FT.

Quando não se observa uma diminuição abrupta de 7 dBZ abaixo de 0 $^{\circ}$C, isso significa que entre 0 $^{\circ}$C e -20 $^{\circ}$C existem partículas de gelo com densidades diversificadas, bem como água super-resfriada. Os CCFTDs das figuras \ref{ccftd_fta_com} e \ref{ccftd_fta_com}, mostram que entre 0 $^{\circ}$C e -20 $^{\circ}$C a precipitação das tempestades elétricas com FTA extremos apresentam maior amplitude  $Z_c$, e quantis de probabilidade de $Z_c$ por temperatura mais alargados, mostrando os perfis são mais aleatórios do que para a precipitação dos extremos de FT, o que está associado a formação de partículas maiores e mais densas. Se as tempestades elétricas com FTA extremos são sistemas novos, as correntes ascendentes são mais intensas do que nos extremos de FT, então, há de se esperar mais gotículas pequenas nos sistemas extremos de FTA e menos gotas grandes do que em relação aos extremos de FT.

De acordo com \citeonline{bigg1953}, as gotas maiores se congelam em temperaturas mais elevadas do que as gotas menores. Portanto, as gotas menores percorrem um caminho maior em altitude para atingirem temperaturas de congelamento, o que aumenta a probabilidade de colisões entre os hidrometeoros e consequentemente aumenta a eficiência de formação do granizo, pois a espessura de interação entre o \textit{graupel} e gotas pequenas super-resfriadas será maior do que em relação a nuvens de tempestades elétricas com fracas correntes ascendentes e maior fração estratiforme (gotas maiores), que é evidenciado pela taxa de diminuição de $Z_c$ por nível de temperatura na figura \ref{deriv_amazonas}, cuja as tempestades elétricas de FTA extremo tem maior taxa de diminuição entre -20$^{\circ}$C e -10$^{\circ}$C e as tempestades elétricas de FT extremo tem as maiores taxas diminuições (dBZ $^{\circ}$C) entre 0 $^{\circ}$C e -8 $^{\circ}$C. Em termos de eletrificação, a colisão entre o \textit{graupel}/granizo  com cristais de gelo é muito mais eficiente do que a colisão entre agregados \cite{Takahashi1978, jayaratne1983}.


Os maiores valores de $Z_c$ em temperaturas inferiores a -10\textsuperscript{o}C para diferentes quantis da amostragem da probabilidade de $Z_c$ por temperatura dos extremos de FTA, indicam que a velocidade vertical no ambiente das tempestades elétricas com índice FTA extremo são maiores do que para os sistemas com índice FT extremo, pois os maiores valores de FTA mostram perfis de $Z_c$ com maior quantidade de gelo de nuvem, portanto, espera-se correntes ascendentes mais intensas nas tempestades elétricas selecionadas pelos extremos de FTA na região dos núcleos de raios para que seja possível elevar maior quantidade de água em regiões mais frias da atmosfera do que nos casos das tempestades elétricas com FT extremo. Este resultado reforça que as tempestades elétricas do grupos dos eventos extremos de FTA, são sistemas novos ou em via de maturação, quando as correntes ascendentes são mais intensas, enquanto o grupo dos eventos extremos de FT são sistemas maduros ou em dissipação.

O fato das taxas de crescimento em dBZ~\textsuperscript{o}C$^{-1}$ para a precipitação dos núcleos de raios das tempestades elétricas com FTA extremo serem maiores  em regiões mais frias do que observa-se para as tempestades elétricas com extremos de FT, sugere que nos extremos de FTA há maior interação entre o \textit{graupel} e neve seca, com possível formação de granizo compacto ou poroso, e menor interação entre o \textit{graupel} e neve molhada do que em relação aos extremos de FT. Conforme \cite{Takahashi1978}, a interação entre cristais de gelo e o \textit{graupel} é o processo de maior eficiência na transferência de cargas entre os hidrometeoros. 

Nos sistemas com FT extremos entre 10-0S e 70-60W, a acresção ocorreu preferencialmente em temperaturas superiores a -10\textsuperscript{o}C, ambiente supostamente com maior quantidade de gotículas de água superesfriadas do que em regiões com temperaturas entre -20\textsuperscript{o}C e -10\textsuperscript{o}C, sugerindo maior formação de granizo esponjoso\footnote{Granizo em que parte das gotículas de água super-resfriadas coletadas não congelam. Também denominado como granizo molhado ou granizo macio.}, processo de crescimento de hidrometeoro pouco eficiente para a eletrização \cite{jayaratne1983}. 

%Portanto, mesmo que a estrutura 3D da precipitação dos extremos de FT esteja associada a sistemas com maior número de raios, a precipitação dos sistemas com as maiores densidades de raios por área (extremos de FTA), sugere processos de eletrificação mais eficientes.

%para as tempestades elétricas  como mostra a figura %\ref{contornosAmazonas},
% e a menor taxa de crescimento de dBZ~\textsuperscript{o}C$^{-1}$ %em 0\textsuperscript{o}C para as tempestades elétricas com FTA %extremo em relação aos extremos de FT, indicam maior intensidade %convectiva nos sistemas com FTA extremo entre 10-0S e 70-60W, %pois além de evidenciar mostra maior quantidade de
% gelo de nuvem e precipitação a partir de gelo sólido. 

Observe na figura \ref{deriv_amazonas}, que logo abaixo de 10\textsuperscript{o}C próximo da superfície, os valores de $Z_c$ aumentam mais para FTA do que para FT. O aumento de $Z_c$ próximo da superfície representa processo de evaporação ou quebra das gotas de chuva maiores, o que pode indicar rajadas de ventos mais intensas na superfície para as tempestades elétricas com FTA extremo. 

%O aumento do fator de refletividade em torno de 0 \textsuperscript{o}C está associado a mudança do índice de refração da água devido a sua fusão. Já o aumento do fator de refletividade em torno de -40 \textsuperscript{o}C e -15 \textsuperscript{o}C representam o crescimento de hidrometeoros por agregação e acreção \cite{Fabry1995,Takahashi1978}.

%Note na figura \ref{deriv_amazonas}, como a precipitação do 95\textsuperscript{\underline{o}} percentil de probabilidade de ocorrência tanto para FTA quanto para FT, é o regime de precipitação mais severa. Pois, há o crescimento de $Z_c$ em torno de -10 \textsuperscript{o}C e -15 \textsuperscript{o}C e não há banda brilhante, indicando precipitação a partir de granizo\footnote{Em \cite{Fabry1995}, este tipo de perfil é discutido como chuva a partir de gelo compacto.}. No 30\textsuperscript{\underline{o}}, 50\textsuperscript{\underline{o}} e 70\textsuperscript{\underline{o}} percentil dos extremos de FT, o efeito da banda brilhante associada ao derretimento é mais evidente do que para os extremos de FTA. 

Quando comparamos a região central da Bacia do Amazonas, figura \ref{deriv_amazonas}, com a região entre 30-20S e 60-50W que abrange parte central da Bacia do Rio da Prata, figura \ref{deriv_prata}, a microfísica de eletrificação se mostra diferente em cada local. Observa-se que no 50\textsuperscript{\underline{o}} percentil, a taxa de diminuição em dBZ~\textsuperscript{o}C$^{-1}$ entre -40 \textsuperscript{o}C e -20 \textsuperscript{o}C é maior para a região da Bacia do Prata, do que para a região da Bacia Amazônica, tanto para os sistemas extremos de FTA quanto para os sistemas extremos de FT. Na atmosfera, entre -40 \textsuperscript{o}C e -20 \textsuperscript{o}C as condições de temperatura e supersaturação em relação ao gelo são favoráveis para o crescimento de cristais de gelo, mostrando que o processo de agregação é mais intenso sobre a Bacia do Prata do que sobre a Bacia Amazônica. Também,  os extremos de FTA e FT (figuras \ref{derivaAmazonas} e \ref{derivaPrata}) apresentam valores de $Z_c$





\begin{figure}[!ht]
\centering
\subfloat[Perfis verticais de $Z_c$ de acordo com diferentes percentis do CCFTD.]{\includegraphics[height=9cm]{img/precipitacao3d/deriv_ccftd/Contornos_contornos_cdf_3_3} \label{contornosPrata}}
\subfloat[Derivadas dos perfis verticais de $Z_c$ de diferentes percentis do CCFTD.]{\includegraphics[height=9cm]{img/precipitacao3d/deriv_ccftd/deriv_contornos_cdf_3_3} \label{derivaPrata}}
\caption{Quadrante que abrange a região central da Bacia do Rio da Prata, entre 30-20S e 60-50W.}
\label{deriv_prata}
\end{figure}

Conforme \citeonline[p.~263]{mason1971_2ed}, as diferentes formas geométricas dos cristais de gelo irão depender da temperatura e supersaturação do ambiente atmosférico em relação ao gelo. Portanto, espera-se maior quantidade e diversidade de cristais de gelo associada a precipitação tridimensional dos núcleos de raios  da região da Bacia do Prata.

Comparando as variações de $Z_c$ em função da temperatura referente ao 95\textsuperscript{\underline{o}} percentil entre as figuras \ref{derivaAmazonas} e \ref{derivaPrata}, observa-se que para ambos os extremos de FTA ou FT, o decrescimento de $Z_c$ em torno de -15 \textsuperscript{o}C é maior para a região da Bacia do Rio Amazonas, o que indica maior formação de granizo. Provavelmente, para os perfis de $Z_c$ mais raros e intensos entre 10-0S e 70-60W a taxa de acreção é maior do que entre 30-20S e 60-50W. 

Porém, para os perfis de $Z_c$ associado as linhas de contorno das figuras \ref{contornosAmazonas} e \ref{contornosPrata}, observa-se que para todos os quantis analisados, tanto para a chuva dos extremos de FTA quanto FT, os valores de $Z_c$ são mais intensos nas tempestades elétricas da região que abrange a parte central da Bacia do Prata. Então pode-se afirmar que a intensidade convectiva dos núcleos de raios entre 30-20S e 60-50W é maior do que entre 10-0S e 70-60W. 

%Apesar do 95\textsuperscript{\underline{o}} percentil mostrar maiores taxas de de crescimento em dBZ por \textsuperscript{o}C, em -15 \textsuperscript{o}C tanto para FTA quanto FT sobre a Bacia do Rio Amazonas, do que sobre a Bacia do Rio da Prata,  os contorno de probabilidade acumulativa de 95\% nos CCFTD das figuras \ref{ccftd_fta_com} e \ref{ccftd_ft_com}, em -15 \textsuperscript{o}C, mostram valores de $Z_c$ de aproximadamente 3 dBZ superiores na região da Bacia Platina. 
%Mesmo que o 95\textsuperscript{\underline{o}} percentil mostre maior crescimento de hidrometeoros na região mista sobre a Bacia Amazônica, a precipitação do 95\textsuperscript{\underline{o}} percentil na Bacia do Prata foi mais severa, pois possui maiores valores de $Z_c$.

\section{SEVERIDADE REGIONALIZADA}


O estudo da densidade de probabilidade de FTA e FT, conforme mostrado na figura \ref{pdfFTAFT}, foi feito para as tempestades elétricas ocorridas em cada região de 2,5 por 2,5 graus de latitude e longitude entre 40N-10S e 90-30W. Desta forma, verifica-se quais são os valores extremos de FTA e FT em cada região de 2,5$^{\circ}$ $\times$ 2,5$^{\circ}$ sobre a América do Sul. 

%Os mapas mostrados nesta seção são semelhantes aos mapas globais de valores extremos regionalizados das medidas relacionadas a intensidade convectiva das PFs mostrados em \citeonline{zipser2006}. %Porém nesta pesquisa, a intensidade convectiva é investigada com base nos índices FTA e FT. 

%O Oceano Verde, conceito associado a convecção durante o regime de ventos de Oeste na estação chuvosa Amazônica, discutido por \citeonline{silva2002lba,williams2002}, é bastante evidente. A região central da Bacia Amazônica possui os  valores de FTA na mesma ordem de magnitude e no mesmo percentil das densidades de probabilidades de FTA regionalizadas das tempestades elétricas oceânicas e costeiras.

%Os valores de FT associados ao 5\textsuperscript{\underline{o}} e 10\textsuperscript{\underline{o}} percentil, mostrados nas figuras \ref{5oFt} e \ref{10oFt}, revelam os menores valores de FT no centro do continente, principalmente nas regiões continentais fora da área de atuação da ZCIT e de sistemas transientes subtropicais.

Os mapas das figuras \ref{95oFta}, \ref{99oFta}, \ref{95oFt}  e \ref{99oFt}, mostram os valores de FTA e FT do 95\textsuperscript{\underline{o}} e 99\textsuperscript{\underline{o}} percentil de ocorrência. Pontos da grade de 2,5$^{\circ}$ $\times$ 2,5$^{\circ}$ com menos do que 100 tempestades elétricas ocorridas nos 14 anos de dados foram preenchidos na cor branca. A barra de cores dos mapas correspondem aos valores extremos regionais, enquanto a linha de contorno na cor preta marca o extremo em relação a todas as tempestades observadas sobre a AS, conforme a amostragem das tempestades elétricas exposta na figura \ref{cdfFTAFT}.

\begin{figure}[!ht]
\centering
{\includegraphics[height=14cm, trim=0 0 0 0, clip]{img/DistEspacialPercentis/FTA/distEspacialValor095thFta}} 
\caption{Distribuição espacial dos valores do 95\textsuperscript{\underline{o}} percentil da amostra de probabilidade do índice FTA a cada região de 2,5 por 2,5 graus de latitude e longitude.}
\label{95oFta}
\end{figure}

Observa-se na figura \ref{95oFta}, que o valor do 95\textsuperscript{\underline{o}} percentil da amostragem total de FTA corresponde a 52,76 $\times$ 10$^{-4}$ raios min$^{-1}$
km$^{-2}$. Porém, em algumas regiões os valores de FTA do 95\textsuperscript{\underline{o}} percentil das amostragens regionalizadas, atingiram  111 $\times$ 10$^{-4}$ raios min$^{-1}$ km$^{-2}$. Portanto, eventos considerados extremos em relação a todo o continente Sul-americano podem ser bem mais frequentes nas regiões em que o 95\textsuperscript{\underline{o}} percentil da amostragem regional correspondeu a valores superiores a 52,76 $\times$ 10$^{-4}$ raios min$^{-1}$ km$^{-2}$.
  
\begin{figure}[!ht]
\centering  
{\includegraphics[height=14cm, trim=0 0 0 0, clip]{img/DistEspacialPercentis/FTA/distEspacialValor099thFta}}
\caption{Distribuição espacial dos valores do  99\textsuperscript{\underline{o}} percentil da amostra de probabilidade do índice FTA a cada região de 2,5 por 2,5 graus de latitude e longitude.}
\label{99oFta}
\end{figure} 

As regiões dos maiores valores do 95\textsuperscript{\underline{o}} e 99\textsuperscript{\underline{o}} percentil do índice FTA, os quais são expostos nas figuras \ref{95oFta} e \ref{99oFta}, são principalmente:  região ao Oeste da Cordilheira dos Andes Central até sua parte Meridional, a Bacia do Rio da Prata, região Leste da Amazônia e as regiões do planalto Brasileiro, entre o Pantanal Mato-grossense e o Planalto Central Brasileiro, entre as Bacias dos Rios Xingu, Araguaia e Tocantis e na região do Planalto Meridional Brasileiro aonde está localizado os planaltos e chapadas da Bacia do Paraná. Nestas regiões os sistemas com índice FTA extremo possuem valores superiores à 80 $\times$ 10$^{-4}$ raios min$^{-1}$ km$^{-2}$, conforme as cores das figuras \ref{95oFta} e \ref{99oFta}. 

% que se estendem por quase todo o país. Observa-se que os sistemas com FTA mais intensos da América do Sul ocorrem associados ao relevo nas regiões entre

Note que para saber aproximadamente o número de raios produzidos pelos sistemas extremos de FTA temos que multiplicar o índice FTA pela área do sistema. Por exemplo, a equação \ref{FTAkm2}, demostra que nas regiões em que os sistemas extremos possuem 100 $\times$ 10$^{-4}$ raios min$^{-1}$ km$^{-2}$, um sistemas severo com área de 10$^3$ km$^2$ então possui 10 raios observados pelo LIS em 1 minuto.  

\begin{equation}
100 \times 10^{-4} \left[ \frac{\mathrm{raios}}{\mathrm{min}~\mathrm{km}^2} \right]  10^3 [ \mathrm{km}^2 ] = 10 \left[ \frac{\mathrm{raios}}{\mathrm{min}}\right]  
\label{FTAkm2}
\end{equation}

Os mapas das figuras \ref{95oFt} e \ref{99oFt}, mostram que nas Bacias: do Rio da Prata principalmente, do Rio Araguaia, Rio Xingu e Rio Tocantis, são locais em que os sistemas possuem os maiores índices de FT tanto no 95\textsuperscript{\underline{o}} quanto no 99\textsuperscript{\underline{o}} percentil.

\begin{figure}[!ht]
\centering
{\includegraphics[height=14cm, trim=0 0 0 0, clip]{img/DistEspacialPercentis/FT/distEspacialValor095thFt}} 
\caption{Distribuição espacial dos valores do 95\textsuperscript{\underline{o}} percentil da amostra de probabilidade do índice FT a cada região de 2,5 por 2,5 graus de latitude e longitude.}
\label{95oFt}
\end{figure} 
  
\begin{figure}[!ht]
\centering  
{\includegraphics[height=14cm, trim=0 0 0 0, clip]{img/DistEspacialPercentis/FT/distEspacialValor099thFt}}
\caption{Distribuição espacial dos valores do  99\textsuperscript{\underline{o}} percentil da amostra de probabilidade do índice FT a cada região de 2,5 por 2,5 graus de latitude e longitude.}
\label{99oFt}
\end{figure}

Os maiores valores do 95\textsuperscript{\underline{o}} e 99\textsuperscript{\underline{o}} percentil do índice FT, figuras \ref{95oFt} e \ref{99oFt},  ficam situados na região Sul da América do Sul, compatível com a região em que \citeonline{cecil2005} apontam como o local das PFs com as maiores taxas de raios por minuto da AS.

No entanto, os mapas dos extremos regionalizados do índice FTA mostram localidades com potencial de tempo severo em regiões mais adentro do continente Sul-americano, que se estende conforme a topografia do Planalto Central e Meridional brasileiro, no Leste da Bacia Amazônia e ao Oeste da Cordilheira dos Andes. 

Os maiores valores de extremos de FTA mostrados nas figuras \ref{95oFta} e \ref{99oFta} correspondem com as regiões apresentadas em \citeonline{zipser2006}, com os maiores valores extremos de máxima altura com refletividade superior a 40 dBZ observadas pelo PR, enquanto os maiores valores de extremos de FT correspondem com os locais das PFs com maiores taxas de raios no tempo.

Observe que entre 40--20S e 70--60W, os mapas desta seção revelam informações a respeito da intensidade convectiva (índices FTA e FT) das tempestades elétricas extremas sobre a AS em regiões do semi-árido Argentino, as quais foram desconsideradas em \citeonline[fig. 6a]{zipser2006} por apresentarem baixo número de amostragem de PFs, menos do que 1000 PFs. Como podemos observar na seção \ref{cicloanualsecao}, entre 40--20S e 70--60W, observou-se a mais curta estação de tempestades elétricas da AS, porém, durante a estação de tempestades elétricas identifica-se os maiores índices extremos de FT e FTA da AS. Conforme aponta \citeonline{brooks2003}, esta região do semi-árido Argentino ao leste da Cordilheira dos Andes possui o maior número de dias no ano com energia potencial convectiva (CAPE) $\geq$ 2000 J kg$^{-1}$. 

\sigla{name='CAPE',description={Energia potencial convectiva, do inglês, \textit{Convective Available Potential Energy}}} 

Ao comparar os valores extremos entre o 95\textsuperscript{\underline{o}} e 99\textsuperscript{\underline{o}} percentil nas regiões continentais, a intensidade de ambos os índices FTA e FT aumenta em mais de 250\%. Conforme o percentil de ocorrência de FTA e FT aumenta a partir do 95\textsuperscript{\underline{o}} percentil, as regiões de 2,5 por 2,5 graus de latitude e longitude em que estes sistemas ocorrem são mais restritas e a intensidade convectiva (FTA e FT) aumenta aceleradamente em relação aos sistemas mais frequentes.
