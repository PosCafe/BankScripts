\chapter{MARCO DAS TEMPESTADES ELÉTRICAS NA AMÉRICA DO SUL}
%\chapter{Marco das Tempestades Elétricas na América do Sul }

O Marco das tempestades elétricas descreve os locais e quando estes sistemas ocorrem. Determina-se a sazonalidade, o ciclo diurno e a distribuição espacial de raios e de sistemas sobre a região da América do Sul. 


\section{CICLO DIURNO E CICLO ANUAL}

Utilizando a base de dados de tempestades elétricas construída nesta pesquisa, foi estudada a frequência de ocorrências dos sistemas no decorrer das horas do dia, figura \ref{ciclodiurnototal}, e meses do ano, figura \ref{cicloanualtotal}. Deste modo, obtivemos na figura \ref{diurnoanual}, o ciclo diurno e anual das tempestades elétricas por meio da distribuição de probabilidade de ocorrências.
% entre 10$^{\circ}$ Norte -- 40$^{\circ}$ Sul e 30$^{\circ}$  -- 90$^{\circ}$ Oeste.

\begin{figure}[!hb]
  \centering{
  \subfloat[Ciclo diruno.]{{\includegraphics[scale=0.95]{img/ciclos/ciclodiurno19982011total}} \label{ciclodiurnototal}}
  \subfloat[Ciclo anual.]{{\includegraphics[scale=0.95]{img/ciclos/cicloanual19982011total}} \label{cicloanualtotal}}
  }
\caption{Ciclo diurno e anual das tempestades elétricas observadas em hora local. Os valores de probabilidade foram normalizados pelo total dos 154,189 sistemas identificados.}
\label{diurnoanual} 
\end{figure} 

A figura \ref{ciclodiurnototal},  mostra que, entre 14h e 15h as tempestades elétricas são mais prováveis, indicando que o aquecimento da superfície do continente e o aumento da camada limite planetária no decorrer do dia são ingredientes que podem aumentar a probabilidade de ocorrência em até 4,6 vezes em relação aos horários de menor fluxo de calor sensível para a atmosfera. Enquanto o TRMM observou 2312 tempestades elétricas às 9h (hora local), às 14h foram observadas 13,877.

No ciclo anual, conforme mostra a figura \ref{cicloanualtotal}, observa-se que a estação de tempestades elétricas na América do Sul possui dois picos, um em outubro e outro em março, porém contempla os meses de outubro, novembro, dezembro, janeiro, fevereiro e março. A maior probabilidade de ocorrência esteve associada ao mês de outubro, que concentrou 16,961 tempestades elétricas observadas em 14 anos. % Entre janeiro e março as tempestades elétricas tiveram probabilidade de 0.3\% menor do que no início da estação, em outubro.   


O ciclo diurno também foi estudado para cada região de 10 por 10 graus, como mostra a figura \ref{diurno}. 



\begin{figure}[!hb]
\centering{\includegraphics[scale=1.5]{img/ciclos/ciclodiurno10x1019982011localtime}}  
\caption{Ciclo diurno em hora local para as tempestades elétricas observadas em cada região de 10 por 10 graus. Os valores de probabilidade são mostrados em porcentagem e foram normalizados pelo total de 154,189 sistemas observados.}
\label{diurno}
\end{figure}


Mesmo que em uma análise geral mostre a importância do aquecimento superficial do continente para a ocorrência de tempestades elétricas, sistemas noturnos sobre a Colômbia e Venezuela são bastante frequentes.

Na figura \ref{diurno}, entre 0$^{\circ}$--10$^{\circ}$ Norte e 80$^{\circ}$--70$^{\circ}$ Oeste, às 0h em hora local, temos o maior valor de probabilidade (0.4\%) de tempestades elétricas noturnas da América do Sul, o que representou um número de 617 sistemas observados em 14 anos, apenas entre 0h e 00:59h.


A circulação de vale e montanha associada com a topografia elevada na Colômbia, principalmente a região do Parque Nacional Natural Paramillo, e o Lago Maracaibo na Venezuela, combinados com a atuação da Zona de Convergência Intertropical (ZCIT), promovem condições para o desenvolvimento de tempestades elétricas noturnas de maneira mais eficiente do que as demais regiões. \sigla{name={ZCIT},description={Zona de Convergência Intertropical}}

  
Na região das águas mais protegidas do Oceano Pacífico, pela rica biodiversidade que regula todos os Oceanos, entre 0$^{\circ}$--10$^{\circ}$ Norte e 90$^{\circ}$--80$^{\circ}$ Sul abrangendo o Parque Nacional da Ilha do Coco na Costa Rica e parte das ilhas Galápagos no Equador, foi a região oceânica com a maior probabilidade de ocorrência de tempestades elétricas. Esta possui um ciclo diurno duplo de tempestades elétricas. Elas ocorrem às 4h, em hora local, e as 14h. A maior probabilidade de ocorrência (0.15\%) foi observada às 4h, que correspondeu à 231 sistemas.

No Pacífico Sul, as tempestades elétricas são mais raras do que as demais regiões devido a atuação permanente da subsidência da Célula de Hadley, que modula a Alta Subtropical do Pacífico Sul, responsável também por regiões como o Deserto do Atacama e parte do semi-árido Argentino.

Na região do Atlântico Subtropical, a probabilidade de tempestades elétricas é maior do que no Atlântico Norte. A passagem de sistemas transientes entre 40$^{\circ}$--30$^{\circ}$ Sul e 50$^{\circ}$--30$^{\circ}$ Oeste e 30$^{\circ}$--20$^{\circ}$ Sul e 40$^{\circ}$--30$^{\circ}$ Oeste, gera maior número de tempestades elétricas oceânicas do que com a atuação da ITCZ no Atlântico Tropical. Observa-se também que nas regiões oceânicas o ciclo diurno das tempestades elétricas indica maior atividade noturna e não às 14-15h igual no continente.

%No Atlântico Tropical Norte, entre 0$^{\circ}$--10$^{\circ}$ Norte e 50$^{\circ}$--30$^{\circ}$ Oeste, o pico de tempestades elétricas deve estar associado com a atividade da ZCIT.

O pico de atividade de tempestades elétricas durante o ciclo diurno, figura \ref{diurno}, ocorreu entre 10$^{\circ}$--0$^{\circ}$ Sul e 70$^{\circ}$--50$^{\circ}$ Oeste e 20$^{\circ}$--10$^{\circ}$ Sul e 60$^{\circ}$--50$^{\circ}$ Oeste. Em cada uma destas três caixas, observou-se a probabilidade de aproximadamente 0.8\% entre as 14h e 15h, mostrando que em toda esta região, o TRMM observou uma média de 3 tempestades elétricas a cada 2 dias para apenas estas duas horas do dia.


Entre 30$^{\circ}$--20$^{\circ}$ Sul e 60$^{\circ}$--50$^{\circ}$ Oeste, na figura \ref{diurno}, região de grande atividade de Sistemas Convectivos de Meso-escala (MCS) conforme descrevem \citeonline{Durkee2009}, encontra-se um máximo durante a tarde e os sistemas noturnos tiveram probabilidade de ocorrência similar as demais regiões continentais, exceto na Colômbia e Venezuela, mostrando que os MCS ao Sul da América do Sul com ciclo de vida maior do que 9h ou com formação noturna, são raros, mesmo neste banco de dados composto apenas por tempestades elétricas. Na figura \ref{anual}, observa-se que durante a estação de tempestades elétricas, há três máximos de atividade: em outubro, janeiro e março. O máximo é observado em janeiro com 1234 sistemas identificados na região.	\sigla{name={MCS},description={Sistemas Convectivos de Meso-escala}}

\begin{figure}[!ht]
\centering{\includegraphics[scale=1.5]{img/ciclos/cicloanual10x1019982011localtime}}  
\caption{Ciclo anual em hora local para as tempestades elétricas observadas em cada região de 10 por 10 graus. Os valores de probabilidade são mostrados em porcentagem e foram normalizados pelo total de 154,189 sistemas observados.}
\label{anual}
\end{figure}

 
No ciclo anual mostrado na figura \ref{anual}, observa-se uma clara diferença sazonal na ocorrência de tempestades elétricas entre os dois Hemisférios. Sobre o Hemisfério Norte as tempestades ocorrem principalmente entre os meses de abril e agosto, enquanto no Hemisfério Sul entre setembro e março, apesar das características individuais de cada região como por exemplo dois ou três picos de atividade durante a estação.

O deslocamento da ZCIT durante o verão setentrional, inverte a estação de tempestades elétricas entre 10$^{\circ}$ Sul e 10$^{\circ}$ Norte, diminuindo o número de sistemas no inverno austral.

Na região de clima semi-árido na Argentina, entre 40$^{\circ}$--20$^{\circ}$ Sul e 70$^{\circ}$--60$^{\circ}$ Norte, figura \ref{anual}, local das tempestades mais severas e convecção mais profunda da América do Sul como apontam \citeonline{cecil2005, Romatschke2010}, foi encontrada a estação de tempestades elétricas mais curta e bem definida. Entre Maio e Agosto a probabilidade de ocorrência de sistemas é praticamente 0\%, enquanto a estação de tempestades elétricas se define entre dezembro e janeiro.  


Na figura \ref{anual}, entre 20$^{\circ}$--40$^{\circ}$ Sul e 60$^{\circ}$--70$^{\circ}$ Oeste, local das tempestades mais severas e convecção mais profunda da América do Sul como aponta \cite{cecil2005, Romatschke2010}, possui uma estação de tempestade curta e muito bem definida. Entre maio e agosto praticamente não há raios, pois é no verão que o jato de baixos níveis, que traz umidade da Amazônia, se intensifica e dispara os processos de eletrificação nesta região Sul da AS.



Portanto a probabilidade de ocorrência de sistemas observados pelo TRMM, depende da quantidade de tempo em que o satélite ficou observando o continente. Fazendo o acumulado do view time do LIS para os 14 anos, como mostra a figura \ref{vt}, observa-se que algumas regiões ao sul da AS, o satélite permanece 10 dias a mais do que em regiões ao norte.

As probabilidades de ocorrência mostradas nas figuras \ref{anual} e \ref{diurno} descrevem bem comportamento dos ciclos diurno e anual, porém os valores de probabilidades são tendenciosos nas regiões em que o satélite passou mais tempo observando.



Na figura \ref{tempestadesTotal}, temos na esquerda o acumulado de todos os pontos do VIRS que definiram a área dos sistemas e na parte direita temos o acumulado de todos os flashes contidos nos sistemas identificados. Ambas as concentrações, seja de área de tempestades elétricas ou de ocorrência de flashes foram normalizados pela área da grade e view time. O view time apresentado em dias na figura \ref{view} foi convertido para ano, e as taxas calculadas em concentração por $km^{-2}ano^{-1}$.

Observa-se (figura \ref{tempestadesTotal}) um máximo em área de ocorrência sobre a Colômbia e região central da Floresta Amazônica. Mesmo que os sistemas com maior número de raios observados pelo TRMM estejam mais concentrados no Sul da América do Sul, as tempestades elétricas são observadas com maior frequência à Noroeste da AS. 

As regiões com taxa de flashes superior a 36$km^{-2}ano^{-1}$ na Amazônia, possuem acima de 150 tempestades elétricas $km^{-2}ano^{-1}$. Na região da Argentina e Paraguai, as mesmas taxas de flashes são atingidas, mas com uma taxa de sistemas em torno de 60$km^{-2}ano^{-1}$.  

Na figura \ref{DensidadeTempestadesSazonal}, o view time foi acumulado para o período de cada estação do ano (DJF, MAM, JJA, SON), para os 14 anos de dados, e a taxa de tempestades elétricas e de flashes foram calculadas em concentração por $km^{-2}dia^{-1}$ para cada estação. 

Apesar da figura \ref{diurnoanual} mostrar que em outubro, novembro e dezembro temos maior probabilidade de ocorrência de tempestades elétricas sobre o continente, a figura \ref{tempestadesSON} mostra qual a distribuição espacial desses sistemas e qual a taxa de raios que os sistemas produziram. 


\begin{figure}
\centering{\includegraphics[scale=0.88]{img/DensidadeTempestades/densEspacial19982011TotalTempestadesPolyfill}}  
\caption{Densidade espacial total de tempestades elétricas.}
\label{tempestadesTotal}
\end{figure}

\begin{figure}
  \centering{
  \subfloat[DJF]{{\includegraphics[scale=0.88]{img/DensidadeTempestades/densEspacial19982011djfTempestadesPolyfill}} \label{tempestadesDJF}}
  \subfloat[MAM]{{\includegraphics[scale=0.88]{img/DensidadeTempestades/densEspacial19982011mamTempestadesPolyfill}} \label{tempestadesMAM}}

  \subfloat[JJA]{{\includegraphics[scale=0.88]{img/DensidadeTempestades/densEspacial19982011jjaTempestadesPolyfill}} \label{tempestadesJJA}}
  \subfloat[SON]{{\includegraphics[scale=0.88]{img/DensidadeTempestades/densEspacial19982011sonTempestadesPolyfill}} \label{tempestadesSON}}
  
  }    
  \caption{Densidade espacial sazonal das tempestades elétricas}
\label{DensidadeTempestadesSazonal}
\end{figure} 

\begin{figure}
\centering{\includegraphics[scale=0.88]{img/TaxaFlash/densEspacial_19982011totalTaxaFlashPolyfill}}  
\caption{Densidade espacial total de raios}
\label{tempestadesTotal}
\end{figure}

\begin{figure}
  \centering{
  \subfloat[DJF]{{\includegraphics[scale=0.88]{img/TaxaFlash/densEspacial_19982011djfTaxaFlashPolyfill}} \label{tempestadesDJF}}
  \subfloat[MAM]{{\includegraphics[scale=0.88]{img/TaxaFlash/densEspacial_19982011mamTaxaFlashPolyfill}} \label{tempestadesMAM}}

  \subfloat[JJA]{{\includegraphics[scale=0.88]{img/TaxaFlash/densEspacial_19982011jjaTaxaFlashPolyfill}} \label{tempestadesJJA}}
  \subfloat[SON]{{\includegraphics[scale=0.88]{img/TaxaFlash/densEspacial_19982011sonTaxaFlashPolyfill}} \label{tempestadesSON}}
  }    
  \caption{Densidade espacial sazonal de raios}
\label{TaxaFlash}
\end{figure} 

\begin{figure}
\centering{\includegraphics[scale=0.88]{img/TaxaFlashTempestade/densEspacial19982011totalEficienciaPolyfill}}  
\caption{Eficiencia de tempestade}
\label{eficiencia}
\end{figure}