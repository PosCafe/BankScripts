\chapter{MARCO DAS TEMPESTADES ELÉTRICAS NA AMÉRICA DO SUL}

O Marco das tempestades elétricas descreve os locais e quando estes sistemas ocorrem na América do Sul. Para tanto, determina-se a sazonalidade, o ciclo diurno, o ciclo anual e a densidade geográfica de raios e das tempestades elétricas.

\section{CICLO DIURNO}

Utilizando a base de dados de tempestades elétricas construída nesta pesquisa, determinou-se a frequência de ocorrências dos sistemas no decorrer das horas do dia, figura \ref{ciclodiurnototal}. Deste modo, obtivemos o ciclo diurno das tempestades elétricas por meio da distribuição de probabilidade de ocorrências.

\begin{figure}[!hb]
  \centering
  {{\includegraphics[height=7.5cm]{img/ciclos/ciclodiurno19982011total}}}
\caption{Ciclo diurno das tempestades elétricas observadas em hora local. Os valores de probabilidade foram normalizados pelo total dos 157,592 sistemas identificados.}
\label{ciclodiurnototal}
\end{figure} 

Observa-se que 40\%  das tempestades elétricas observadas pelo TRMM ocorrem entre 13h e 17h, indicando que o aquecimento da superfície do continente e o aumento da camada limite planetária no decorrer do dia são ingredientes que podem aumentar a probabilidade de ocorrência em  relação aos horários de menor fluxo de calor sensível para a atmosfera. Por exemplo, às 9h a probabilidade de tempestade elétrica é de 1.6\% e às 14h é de 8.8\%, portanto às 14h a probabilidade de ocorrência de tempestade tempestade elétrica é 5.4 vezes maior do que às 9h.

% Enquanto o TRMM observou 2312 tempestades elétricas às 9h (hora local), às 14h foram observadas 13,877.

O ciclo diurno também foi estudado para cada região de 10 por 10 graus, como mostra a figura \ref{diurno}. 

\begin{figure}[!hb]
\centering
{\includegraphics[height=13.5cm]{img/ciclos/ciclodiurno10x1019982011localtime}}  
\caption{Ciclo diurno em hora local para as tempestades elétricas observadas em cada região de 10 por 10 graus. Os valores de probabilidade são mostrados em porcentagem e foram normalizados pelo total de 157,592 sistemas observados.}
\label{diurno}
\end{figure}

Pode-se observar que existe um predomínio de ocorrências de tempestades elétricas entre 13h e 17h sobre o continente. Sobre o oceano observa-se  uma distribuição bimodal, com pico no começo da noite e durante a madruga.

%, que correspondem ao ciclo diurno das \textit{precipitation features} em \citeonline{nesbitt2003diurnal}. 

Sobre os oceanos, os processos de formação de nuvens e consequentemente de formação de tempestades elétricas se mostram mais ativo no horário em que a temperatura superficial e a probabilidade de ocorrência de sistemas sobre o continente  diminui. Neste horário a superfície do oceano pode estar com temperaturas maiores do que as temperaturas sobre a superfície do continente, aumentando a convergência sobre o oceano. A atividade convectiva intensa entre 13-17h sobre o continente também aumenta a cobertura de nuvens do tipo cirrus sobre o oceano inibindo a formação de nuvens \cite{nesbitt2003diurnal}.

% associado ao efeito radiativo de espalhamento e aquecimento de camadas mais elevadas da superfície    

%Mesmo que em uma análise geral mostre a importância do aquecimento superficial do continente para a ocorrência de tempestades elétricas, sistemas noturnos sobre a Colômbia e Venezuela são bastante frequentes. 

Entre 0$^{\circ}$--10$^{\circ}$ Norte e 80$^{\circ}$--70$^{\circ}$ Oeste e às 0h, observou-se a maior probabilidade ($\simeq$0.4\%) de tempestades elétricas noturnas da América do Sul, o que representou um número de 630 sistemas observados em 14 anos, apenas entre 0h e 00:59h. A circulação de vale e montanha associada com a topografia elevada na Colômbia, principalmente a região do Parque Nacional Natural Paramillo, e o Lago Maracaibo na Venezuela, e a atuação da Zona de Convergência Intertropical (ZCIT), promovem condições para o desenvolvimento de tempestades elétricas noturnas de maneira mais eficiente do que as demais regiões \cite{burgesser2012}.\sigla{name={ZCIT},description={Zona de Convergência Intertropical}}

Entre 0$^{\circ}$--10$^{\circ}$ Norte e 90$^{\circ}$--80$^{\circ}$ Oeste, abrangendo o Panamá e parte Sul da Costa Rica, e a região do Oceano Pacífico que engloba o Parque Nacional da Ilha do Coco na Costa Rica e parte das ilhas Galápagos no Equador,  a  região oceânica e costeira com a maior probabilidade de ocorrência de tempestades elétricas. Observe os valores de densidade de tempestades elétricas neste quadrante geográfico, na próxima seção, em \ref{secaoDensidades} na figura \ref{densidadeTempestade}. O ciclo diurno das tempestades elétricas nesta região revela uma distribuição bimodal, com um pico às 4h e outro às 14h. O pico das 14h, provavelmente está associado as tempestades elétricas da região do Panamá, Costa Rica e suas respectivas regiões costeiras adjacentes, as quais sofrem maior aquecimento superficial durante o dia, enquanto o maior pico que  ocorreu às 4h provavelmente corresponde com as trocas de energia na forma de calor entre o oceano e atmosfera.

Entre 30$^{\circ}$--10$^{\circ}$ Sul e 90$^{\circ}$--80$^{\circ}$ Oeste e entre 30$^{\circ}$--20$^{\circ}$ Sul e 80$^{\circ}$--70$^{\circ}$ Oeste, região do Pacífico, as tempestades elétricas são mais raras do que as demais regiões devido a atuação permanente da subsidência da Célula de Hadley que modula a Alta Subtropical do Pacífico Sul, responsável também por regiões como o Deserto do Atacama e parte do semi-árido Argentino \cite{reboita2010regimes}.

Na região do Atlântico Subtropical, a probabilidade de tempestades elétricas é maior do que no Atlântico Norte. A passagem de sistemas transientes entre 40$^{\circ}$--30$^{\circ}$ Sul e 50$^{\circ}$--30$^{\circ}$ Oeste e 30$^{\circ}$--20$^{\circ}$ Sul e 40$^{\circ}$--30$^{\circ}$ Oeste, gera maior número de tempestades elétricas oceânicas do que com a atuação da ITCZ no Atlântico Tropical. Observa-se também que nas regiões oceânicas o ciclo diurno das tempestades elétricas indica maior atividade noturna.


A maior atividade horária de tempestades elétricas, ocorreu entre 10$^{\circ}$--0$^{\circ}$ Sul e 70$^{\circ}$--50$^{\circ}$ Oeste e 20$^{\circ}$--10$^{\circ}$ Sul e 60$^{\circ}$--50$^{\circ}$ Oeste. Em cada uma destas três regiões observou-se a probabilidade de aproximadamente 0.8\% entre as 14h e 16h, mostrando que em toda esta área o TRMM observou 3 tempestades elétricas a cada 2 dias, apenas durante estas duas horas.


Entre 30$^{\circ}$--20$^{\circ}$ Sul e 60$^{\circ}$--50$^{\circ}$ Oeste,  região de grande atividade de Sistemas Convectivos de Meso-escala (MCS) conforme descrevem \citeonline{Durkee2009}, encontra-se um máximo durante a tarde e os sistemas noturnos tiveram probabilidade de ocorrência 2.7 vezes menor do que os valores encontrados sobre os vales na Colômbia e Venezuela, mostrando que a ocorrência dos MCS ao Sul da América do Sul com ciclo de vida maior do que 9h ou com formação noturna, não possuem probabilidade de ocorrência que destaca-se em relação as demais regiões continentais, mesmo neste banco de dados composto apenas por tempestades elétricas. 
\sigla{name={MCS},description={Sistemas Convectivos de Meso-escala}}

\section{CICLO ANUAL}
\label{cicloanualsecao}

Quando se analisa a sazonalidade, observa-se que a estação de tempestades elétricas na América do Sul se configura entre outubro e março e possui dois picos: janeiro, durante o verão austral; e outubro, período de transição entre a estação seca e chuvosa. 


\begin{figure}[!h]
\centering
{\includegraphics[height=7.5cm]{img/ciclos/cicloanual19982011total}}
\caption{Ciclo anual das tempestades elétricas observadas em hora local. Os valores de probabilidade foram normalizados pelo total dos 157,592 sistemas identificados.}
\label{cicloanualtotal}
\end{figure} 

A maior probabilidade foi de 10.9\%  no mês de outubro, conforme mostra a figura \ref{cicloanualtotal}. De outubro até março foram observadas 60.2\% das tempestades elétricas. Em junho observou-se a mínima probabilidade de tempestades elétricas com valor de 5.1\%. Portanto, entre o período de máximo e mínimo anual o número de ocorrência de tempestades elétricas reduz aproximadamente pela metade.


%O ciclo anual representado na figura \ref{cicloanualtotal}, também foi obtido por meio da distribuição de probabilidade de ocorrências, porém, no decorrer dos meses do ano.

O ciclo anual das tempestades elétricas também foi estudado para cada região de 10 por 10 graus de latitude e longitude. O valor de 70 por cento da máxima probabilidade de tempestade elétrica em cada região foi definido como limiar para considerar que a ocorrência de tempestades elétricas aumentou o suficiente para definir uma estação e este valor é representado pela linha horizontal que corta cada gráfico em cada ponto da grade de 10$^{\circ}$ $\times$ 10$^{\circ}$ da figura \ref{anual}. 
%Desta forma, definimos quais os meses que fazem parte da estação das tempestades elétricas em cada região de 10 por 10 graus.
%Foram selecionados os meses em que a probabilidade de ocorrência de tempestade elétrica foi superior à 70 por cento da máxima probabilidade observada em cada região.

\begin{figure}[!h]
\centering{\includegraphics[height=13.5cm]{img/ciclos/cicloanual10x1019982011localtime}}  
\caption{Ciclo anual em hora local para as tempestades elétricas observadas em cada região de 10 por 10 graus de latitude e longitude. Os valores de probabilidade são mostrados em porcentagem e foram normalizados pelo total de 157,592 sistemas observados. As linhas horizontais cortam o valor de 0.7 do máximo de probabilidade, utilizado como limiar para definir as estações de tempestades elétricas.}
\label{anual}
\end{figure}

A tabela \ref{caracEstacao} mostra os meses de duração das estações de tempestades elétricas de acordo com cada região conforme mostra a figura \ref{anual}. Considerando o ciclo anual em cada ponto da grade de 10$^{\circ}$ $\times$ 10$^{\circ}$, observa-se em média uma estação de tempestades elétricas com duração de 4.5 meses. 


\begin{table}[!ht]
\caption{Principais características do ciclo anual de probabilidade de ocorrência de tempestades elétricas observadas entre 1998-2011, em cada região de 10 por 10 graus de latitude longitude.}
\label{caracEstacao}
\centering
\small
\newcommand{\grayline}{\rowcolor[gray]{.88}}
\renewcommand {\tabularxcolumn }[1]{ >{\arraybackslash }m{#1}}
\newcolumntype{W}{>{\centering\arraybackslash}X}
\begin{tabularx}{\textwidth}{p{0.6cm} p{3.5cm} W W W W} %{|p{10cm}|X|X|X|X|X|X|X|X| }
\hline\hline 
\grayline  & Localização & Número de sistemas & Estação (meses) & Duração (meses) & Máximo\\[1.5pt]
\hline
1&0$^{\circ}$--10$^{\circ}$N, 90$^{\circ}$--80$^{\circ}$O& 4173  & Abr--Set &6& Abr\\[1.5pt]\grayline
2&0$^{\circ}$--10$^{\circ}$N, 80$^{\circ}$--70$^{\circ}$O& 14,232 & Mar--Nov &9& Out\\[1.5pt]
3&0$^{\circ}$--10$^{\circ}$N, 70$^{\circ}$--60$^{\circ}$O& 11,946 & Jul--Nov &5& Ago--Set\\[1.5pt]\grayline
4&0$^{\circ}$--10$^{\circ}$N, 60$^{\circ}$--50$^{\circ}$O&  4895 & Jul--Set &3& Ago\\[1.5pt]
5&0$^{\circ}$--10$^{\circ}$N, 50$^{\circ}$--40$^{\circ}$O& 645 & Mai--Jun, Set--Nov &5& Set\\[1.5pt] \grayline
6&0$^{\circ}$--10$^{\circ}$N, 40$^{\circ}$--30$^{\circ}$O& 824 & Out--Dez &3& Dez\\[1.5pt]

7&10$^{\circ}$--0$^{\circ}$S, 90$^{\circ}$--80$^{\circ}$O& 225 & Mar &1& Mar\\[1.5pt]\grayline
8&10$^{\circ}$--0$^{\circ}$S, 80$^{\circ}$--70$^{\circ}$O& 10,014 & Set--Dez &4& Out\\[1.5pt]
9&10$^{\circ}$--0$^{\circ}$S, 70$^{\circ}$--60$^{\circ}$O& 12,605 & Ago--Nov &4& Out\\[1.5pt]\grayline
10&10$^{\circ}$--0$^{\circ}$S, 60$^{\circ}$--50$^{\circ}$O& 12,590 & Set--Dez &4& Out\\[1.5pt]
11&10$^{\circ}$--0$^{\circ}$S, 50$^{\circ}$--40$^{\circ}$O& 7863 & Jan--Abr, Nov--Dez &6&  Jan\\[1.5pt]\grayline
12&10$^{\circ}$--0$^{\circ}$S, 40$^{\circ}$--30$^{\circ}$O& 1363 & Fev--Abr &3&  Mar\\[1.5pt]

13&20$^{\circ}$--10$^{\circ}$S, 90$^{\circ}$--80$^{\circ}$O& 1  &   --0--  & --0-- & --0-- \\[1.5pt]\grayline
14&20$^{\circ}$--10$^{\circ}$S, 80$^{\circ}$--70$^{\circ}$O& 4344 & Jan--Fev,  Set--Dez  &6& Out\\[1.5pt]
15&20$^{\circ}$--10$^{\circ}$S, 70$^{\circ}$--60$^{\circ}$O& 8895 & Jan--Mar, Out--Dez &6& Out\\[1.5pt]\grayline
16&20$^{\circ}$--10$^{\circ}$S, 60$^{\circ}$--50$^{\circ}$O& 10,973 & Jan--Mar,  Out--Dez &6& Out\\[1.5pt]
17&20$^{\circ}$--10$^{\circ}$S, 50$^{\circ}$--40$^{\circ}$O& 8524 & Jan--Mar, Out--Dez &6&  Jan\\[1.5pt]\grayline
18&20$^{\circ}$--10$^{\circ}$S, 40$^{\circ}$--30$^{\circ}$O& 625  & Jan--Mar &3&  Mar\\[1.5pt]

19&30$^{\circ}$--20$^{\circ}$S, 90$^{\circ}$--80$^{\circ}$O& 32 & --0-- & --0-- &  --0-- \\[1.5pt]\grayline
20&30$^{\circ}$--20$^{\circ}$S, 80$^{\circ}$--70$^{\circ}$O& 32 & --0-- &--0--&  --0--\\[1.5pt]
21&30$^{\circ}$--20$^{\circ}$S, 70$^{\circ}$--60$^{\circ}$O& 5607 & Dez--Mar &4& Jan\\[1.5pt]\grayline
22&30$^{\circ}$--20$^{\circ}$S, 60$^{\circ}$--50$^{\circ}$O& 8885  & Dez--Mar &4& Jan\\[1.5pt]
23&30$^{\circ}$--20$^{\circ}$S, 50$^{\circ}$--40$^{\circ}$O& 6121 & Dez--Mar &4& Jan\\[1.5pt]\grayline
24&30$^{\circ}$--20$^{\circ}$S, 40$^{\circ}$--30$^{\circ}$O& 1884 & Dez--Mai&4&  Mar\\[1.5pt]

25&40$^{\circ}$--30$^{\circ}$S, 90$^{\circ}$--80$^{\circ}$O& 258 & Jun &1&  Jun \\[1.5pt]\grayline
26&40$^{\circ}$--30$^{\circ}$S, 80$^{\circ}$--70$^{\circ}$O& 366 & Jan--Mar, Mai--Jun &5& Jan \\[1.5pt]
27&40$^{\circ}$--30$^{\circ}$S, 70$^{\circ}$--60$^{\circ}$O& 7652 & Dez--Jan &2&  Jan\\[1.5pt]\grayline
28&40$^{\circ}$--30$^{\circ}$S, 60$^{\circ}$--50$^{\circ}$O& 5440 & Dez--Mar  &4&  Jan\\[1.5pt]
29&40$^{\circ}$--30$^{\circ}$S, 50$^{\circ}$--40$^{\circ}$O& 2949 & Jan--Set &9&  Abr\\[1.5pt]\grayline
30&40$^{\circ}$--30$^{\circ}$S, 40$^{\circ}$--30$^{\circ}$O& 2301 & Abr--Jun  &3& Mai\\[1.5pt]


\hline 
\end{tabularx}
\end{table}

 
%No ciclo anual mostrado na figura \ref{anual}, observa-se uma clara diferença sazonal na ocorrência de tempestades elétricas entre os dois Hemisférios. Sobre o Hemisfério Norte as tempestades ocorrem principalmente entre os meses de abril e agosto, enquanto no Hemisfério Sul entre setembro e março, apesar das características individuais de cada região como por exemplo dois ou três picos de atividade durante a estação.
%O deslocamento da ZCIT durante o verão setentrional, inverte a estação de tempestades elétricas entre 10$^{\circ}$ Sul e 10$^{\circ}$ Norte, diminuindo o número de sistemas no inverno austral.

Na região referente as linhas 21 e 27 da tabela \ref{caracEstacao} (40$^{\circ}$--20$^{\circ}$ Sul e 70$^{\circ}$--60$^{\circ}$ Oeste),  entre o clima semi-árido na Argentina e parte da Bacia do Prata,  local das tempestades mais severas e de maior probabilidade de ocorrência de núcleos de convecção profunda da AS como apontam \citeonline{cecil2005,zipser2006,Romatschke2010}, foi encontrada a estação de tempestades elétricas com a maior amplitude entre o máximo de ocorrências em janeiro e o mínimo em junho. A probabilidade de tempestades elétricas em junho foi aproximadamente 10 vezes menor do que em janeiro.

Sobre a Colômbia e parte Oeste da Venezuela que abrange o lago Maracaibo, região referente a linha 2 da tabela \ref{caracEstacao}, foi a região em que o TRMM observou o maior número de tempestades elétricas (14,232). Nesta região a estação de tempestades elétricas dura  9 meses, com dois picos: abril e outubro.


As máximas probabilidades de ocorrência de tempestades elétricas durante o ciclo anual, não ocorrem em fase com os máximos anuais de precipitação em algumas regiões da AS, portanto, a definição de uma estação de tempestades elétricas não implica na definição de uma estação chuvosa. 

Durante o final do verão setentrional, entre julho e setembro, a estação chuvosa começa a se deslocar do Hemisfério Norte para o Hemisfério Sul, de Noroeste da AS para Sudeste da AS intensificando-se progressivamente até atingir os maiores acumulados de chuva entre dezembro e janeiro \cite{grimm2003nino,reboita2010regimes,Marengo2012,shi-atlas,bombardi2008variabilidade,cusdodioTese}.
 
Na parte central da AS, linhas  8, 9, 10, 14, 15, 16 da tabela \ref{caracEstacao} (20$^{\circ}$--0$^{\circ}$ Sul e 80$^{\circ}$--50$^{\circ}$ Oeste), as tempestades elétricas ocorrem com maior frequência em outubro, entre a estação seca e chuvosa. Porém os máximos sazonais de precipitação nesta região ocorrem defasados aproximadamente em 5 meses do máximo de tempestades elétricas, entre os meses de fevereiro e abril \cite{grimm2003nino,reboita2010regimes,shi-atlas,bombardi2008variabilidade,cusdodioTese}.

%Assim como as estações chuvosas nas diferentes localidades da América do Sul, as estações de tempestades elétricas se configuram conforme o Sistema de Monção da América do Sul (SMAS) \cite{Zhou1998}.  
%O aumento da temperatura da superfície na primavera austral e o baixo acumulado de chuva e baixa nebulosidade durante o período pré-monção promovem      
%Em \citeonline{Petersen2001}, o estudo realizado referente a estrutura tridimensional da precipitação observada pelo TRMM sobre a região Central da Amazônia, mostrou que a convecção mais profunda ocorre também na transição do período seco para o chuvoso, exatamente quando começa a reversão sazonal do vento em baixos níveis associado ao SMAS conforme apontam \citeonline{Zhou1998}.

Na região Sul da AS, à Leste da Cordilheira dos Andes referentes as linhas  21 e 27 da tabela \ref{caracEstacao}, a estação de tempestades elétricas ocorre entre dezembro e janeiro, em fase com a estação chuvosa. Aqui, a probabilidade de tempestades elétricas aumenta quando o Jato de Baixos Níveis da América do Sul (JBNAS) intensifica o transporte de umidade entre a Bacia Amazônica e a Bacia do Prata durante a atuação do SMAS \cite{marengo2004}.   

A região Sul e Sudeste da América do Sul, referentes as linhas 17, 22, 23 e 28 da tabela \ref{caracEstacao}, a estação de tempestades elétricas ocorre também em fase com o período de máxima configuração do SMAS, entre dezembro e janeiro. \citeonline{petersen2002trmm}, mostram que mudanças na circulação atmosférica durante o período chuvoso associadas as fases ativa e de pausa da SMAS influenciam na densidade de raios sobre a AS, indicando que, nas regiões em que a estação de tempestades elétricas coincide com o período de atuação da ZCAS, a variabilidade intra-sazonal do SMAS  torna-se importante para a ocorrência de tempestades elétricas\cite{CarvalhoJones2002,carvalho2004south}.

Apesar da convergência em grande escala associada a ZCAS, sua atuação contínua durante meses poderia diminuir a instabilidade atmosférica devido a chuvas contínuas e baixa incidência de radiação de onda curta na superfície, o que abaixaria a temperatura da superfície diminuindo o \textit{lapse-rate}. As pausas do SMAS podem ser importantes para o aumento do \textit{lapse rate} da atmosfera e quando há novamente uma fase ativa do SMAS, a atmosfera possui energia e umidade suficiente para eventos de tempestades elétricas.


%Conforme descrito em \citeonline{albrecht2008eletrificaccao},  
%Durante o regime de Oeste, fase de pausa da SMAS, \citeonline{petersen2002trmm} observaram aumento entre 500-700\% da densidade de raios em relação ao regime de Leste, principalmente: no Sul e costa Sul do Brasil; ao Norte do Uruguai; Norte da Argentina, Sul da Bolívia e Oeste do Paraguai; e porção Noroeste da Argentina fronteiriça com o Uruguai. Durante o regime de Leste, \citeonline{petersen2002trmm} observaram aumento entre 50-200\% densidade de raios na região Central da Bacia do Prata, entre o Sudoeste da Argentina e Norte do Paraguai e região Sul da Argentina\cite{jones2002active,albrecht2008eletrificaccao}.

%Durante a fase ativa, apesar da atuação da Zona de Convergência do Atlântico Sul (ZCAS), o aquecimento da superfície diminui devido o aumenta da nebulosidade e chuvas contínuas, não proporcionando condições termodinâmicas propícias para convecção explosiva e sim grandes extensões de nebulosidade com chuvas estratiformes \cite{jones2002active,albrecht2008eletrificaccao,albrecht2011}.

\sigla{name={SMAS},description={Sistema de Monção da América do Sul}}
\sigla{name={AS},description={América do Sul}}

Na região da costa Nordeste da AS, referente as linhas 12 e 18 da tabela \ref{caracEstacao}, a estação de tempestades elétricas possui máxima atividade em março e mínimo em torno de agosto. Nestas regiões o máximo sazonal de precipitação ocorre em fase com o máximo sazonal de tempestades elétricas, porém, durante o período pós-monção na AS \cite{grimm2003nino,reboita2010regimes,shi-atlas,bombardi2008variabilidade,cusdodioTese}.

As regiões oceânicas, referentes as linhas, 1, 5, 7, 25, 29 e 30 da tabela \ref{caracEstacao}, o pico de atividade de tempestades elétricas não ocorre entre outubro e março como mostra o ciclo anual total da figura \ref{cicloanualtotal}, mas entre março e setembro durante o outono e inverno austral.  Durante o inverno, a diferença entre a temperatura da superfície do continente e a temperatura da superfície do oceano é menor, o que favorece a convergência  sobre o oceano e ativando os processos de eletrificação fora do período de maior insolação. Conforme o cilo anual da precipitação na AS mostrado em \citeonline{cusdodioTese}, observa-se que o máximo sazonal de precipitação observada pelo PR nestas regiões oceânicas ocorrem em fase com os máximos sazonais de tempestades elétricas desta pesquisa. 

A região do Atlântico Tropical refente a linha 6 da tabela \ref{caracEstacao}, possui o máximo sazonal de ocorrência de tempestades elétricas em dezembro, enquanto o máximo sazonal de precipitação ocorre durante o outono austral \cite{cusdodioTese}.


%Durante abril e maio, o SMAS vai se desconfigurando e o máximo de chuva começa a retornar para o Hemisfério Norte caminhando de Sudeste para o Nordeste do Brasil e subindo pelo lado Leste da Bacia Amazônica. Neste retorno é que ocorrem os máximos de precipitação em toda a região da Bacia Amazônica, porém o máximo de ocorrência de tempestades elétrica ocorreu na vinda da estação chuvosa para o Hemisfério Sul.

\section{DENSIDADES GEOGRÁFICAS}
\label{secaoDensidades}

Considerando o método descrito em \ref{metodoPass}, referente ao cálculo da densidade de tempestades elétricas e da densidade de raios, foram obtidos os mapas da densidade total de tempestades elétricas e de raios, figuras \ref{densidadeTempestade} e \ref{densidadeRaios}, assim como os mapas da densidade sazonal de raios e de tempestades elétricas que são mostrados nas  figuras \ref{densidadeSazonalRaios} e \ref{DensidadeTempestadesSazonal}. Note que a densidade de raios, tanto a total quanto a sazonal, as quais são apresentadas nesta seção, não correspondem a amostragem total de raios observados pelo LIS corrigida pela eficiência como é mostrado em trabalhos como  \citeonline{albrecht2009tropical,cecil2014gridded}, mas correspondem ao subconjunto dos raios do LIS, os quais estiveram contidos dentro da área das tempestades elétricas, conforme descreve a metodologia de identificação das tempestades elétricas em 
\ref{identificaTempestades}.

%Nesta seção será possível avaliar se as regiões aonde ocorrem o maior/menor número de sistemas correspondem com as regiões com maior/menor número de raios.
%Na figura \ref{densidadeTempestade}, observa-se que a convergência de umidade e liberação calor latente (evapotranspiração) e sensível na região central da Bacia Amazônica promove a máxima ocorrência de tempestades elétricas, principalmente sobre: a Colômbia, Venezuela, Panamá  e região central da Bacia Amazônica abrangendo a parte brasileira, colombiana, venezuelana e peruana.


\begin{figure}[!hb]
 \centering
 {\includegraphics[height=13.5cm]{img/DensidadeTempestades/densEspacial19982011TotalTempestadesPolyfill}}
\caption{Densidade total de tempestades elétricas. As cores correspondem ao número de tempestades elétricas por passagens do VIRS multiplicado por 10$^{-4}$ por quilômetro quadrado (10$^{-4}$ $\times$ [km$^{-2}$]) em cada pronto da grade de 0.25$^{\circ}$ $\times$ 0.25$^{\circ}$. }
 \label{densidadeTempestade}
\end{figure}



%Se integrarmos todos os valores de cada ponto da grade de  0.25$^{\circ}$ $\times$ 0.25$^{\circ}$ da figura \ref{DensidadeTempestadesSazonal} em função de toda a área da região de estudo, o TRMM observou 6.2 tempestades elétricas por órbita.

Das 79,932 órbitas do TRMM entre 1998--2011, apenas 63,613 passaram sobre a região da AS definida nesta pesquisa. Foram identificadas 157,592 tempestades elétricas, portanto, o TRMM observou aproximadamente 5 sistemas a cada 2 órbitas sobre a região da AS. O mapa na figura \ref{densidadeTempestade}, mostra quantas vezes foram observadas tempestades elétricas nos sobrevoos do VIRS sobre cada ponto da grade de 0.25$^{\circ}$ $\times$ 0.25$^{\circ}$. Note que uma tempestades elétrica é um fenômeno que cobre entre dezenas e milhares de pontos de grade de 0.25$^{\circ}$ $\times$ 0.25$^{\circ}$, não sendo um fenômeno pontual como o raio é considerado.

Os maiores valores da escala de densidade de tempestades elétricas na figura \ref{densidadeTempestade}, estiveram entre a 2.5--4.7 $\times$ 10$^{-4}$ km$^{-2}$, indicando que nestes pontos de grade observou-se entre 1--4 tempestades elétricas a cada 10 passagens do VIRS dependendo da área do ponto de grade (entre 772--388 km$^2$).  

As regiões de maiores densidades de tempestades elétricas situam-se na parte Norte e Nordeste da AS. Observa-se duas extensas regiões com as máximas densidades: uma na Colômbia associado ao extremo Norte da Cordilheira dos Andes, outra ao Norte/Noroeste da Floresta Amazônica, abrangendo a parte brasileira, colombiana, venezuelana e peruana.  

A costa Oeste da Colômbia e Panamá, destaca-se como a região de maior densidade de tempestades elétricas costeiras, pois o escoamento atmosférico predominantemente de Leste devido a convergência dos Alísios -- ZCIT -- ao encontrar o extremo Norte da Cordilheira dos Andes entre 0--10$^{\circ}$ Norte, sofre pertubações que disparam tempestades elétricas que continuam a se propagar em sentido Oeste para o Pacífico tropical.  

\begin{figure}[!ht]
 \centering
  {\includegraphics[height=13.5cm]{img/TaxaFlash/densEspacial_19982011totalTaxaFlashPolyfill}}
  \caption{Densidade total de raios. As cores representam o número de raios por ano por quilômetro quadrado ([ano$^{-1}$] [km$^{-2}$]) em cada ponto da grade de 0.25$^{\circ}$ $\times$ 0.25$^{\circ}$. }
  \label{densidadeRaios}
%\caption{Densidade espacial de tempestades elétricas e raios observados entre 1998 e 2011.}
%\label{tempesRaios}
\end{figure}

%A atuação da ITCZ em contato com a Cordilheira dos Andes no extremo Norte da AS, entre 0--10$^{\circ}$ Norte, e a convergência de umidade e liberação calor latente (evapotranspiração) e sensível na região central da Bacia Amazônica,   são os principais propulsores de tempestades elétricas da AS.  %que associa-se intimamente com o SMAS,

A maior extensão em área das máximas densidade de tempestades elétricas sobre a AS foi observada na região da Floresta Amazônica, principalmente a Oeste e Sudoeste do Pico da Neblina, região da cabeceira do Rio Negro, de tríplice fronteira entre Brasil, Colômbia e Venezuela. É notável que a topografia da região Amazônica aumenta a densidade de tempestades elétricas como é o caso da região do Pico da Neblina, principalmente entre a Venezuela e o Brasil, porém esta vasta região contínua com valores de densidade de tempestades elétricas superior a  2.5 $\times$ 10$^{-4}$ km$^{-2}$, sugere que os efeitos da circulação atmosférica de grande escala e processos termodinâmicos conforme aponta \citeonline{albrecht2011}, são os principais agentes de instabilidade atmosférica que promovem o maior número de tempestades elétricas da AS. 

No entanto, ao analisar a densidade de raios juntamente com a densidade de tempestades elétricas, observa-se que na região central do estado do Amazonas, há pontos da grade de 0.25$^{\circ}$ $\times$ 0.25$^{\circ}$ com valores entre 30--40 raios por ano por quilômetro quadrado (ano$^{-1}$ km$^{-2}$), como mostra a figura \ref{densidadeRaios},  e densidades de sistemas, figura \ref{densidadeTempestade}, entre 2.9--3.5 $\times$ 10$^{-4}$ tempestades elétricas por quilômetro quadrado (km$^{-2}$). Na região Norte da Argentina, Sudoeste e Sul do Paraguai, valores de densidade de raios de mesma magnitude são observados, porém com uma densidade de sistemas entre 1.0--1.8 $\times$ 10$^{-4}$ km$^{-2}$. Portanto, podemos afirmar que nas regiões de máxima densidade de raios sobre a Bacia do Prata, as tempestades elétricas produziram entre 195--323\% mais raios por quilômetro quadrado por ano do que em relação as tempestades elétricas observadas sobre as regiões de máximas densidades de raios sobre a Bacia Amazônica. 

Região de topografia elevada como à Noroeste do Lago Titicaca no Peru e algumas regiões do Planalto Brasileiro como sobre a Serra Catarinense e o Parque Nacional das Emas ao Sudoeste de Goias foram também regiões com as maiores densidades de tempestades elétricas da AS.

No estado do Pará, região do deságue do Rio Amazonas no Oceano Atlântico, sobre o Parque Estadual Charapucu e região de deságue do Rio Tocantis no Oceano Atlântico, próximo a cidade de Belém, foram observados também valores de densidades de tempestades elétricas e de densidade de raios entre os maiores registrados sobre AS e não possuem topografia elevada. O efeito de Brisa de Rio e Brisa Marítima combinados com a circulação de grande escala predominantemente de Leste favorecem a dinâmica para a formação de  Linhas de Instabilidade  \cite{kousky1980,cohen1995Li,Clenia2010}.


Nas figuras \ref{densidadeSazonalRaios} e \ref{DensidadeTempestadesSazonal}, a densidade espacial de raios e de tempestades elétricas, foi calculada para os períodos associados a cada estação do ano: dezembro, janeiro e fevereiro (DJF); março, abril e maio (MAM); junho, julho e agosto (JJA) e setembro, outubro e novembro (SON). A tabela \ref{EstacaoQtd} mostra o acumulado de sistemas observados em cada estação do ano.
\sigla{name={DJF},description={Dezembro, janeiro e fevereiro}} 
\sigla{name={MAM},description={Março, abril e maio}}
\sigla{name={JJA},description={Junho, julho e agosto}}
\sigla{name={SON},description={Setembro, outubro e novembro}}


\begin{figure}[!ht]
  \centering{
  \subfloat[DJF]{{\includegraphics[height=6.5cm, trim=0 7 0 0, clip]{img/TaxaFlash/densEspacial_19982011djfTaxaFlashPolyfill}} \label{txDJF}}
  \subfloat[MAM]{{\includegraphics[height=6.5cm, trim=0 7 0 0, clip]{img/TaxaFlash/densEspacial_19982011mamTaxaFlashPolyfill}} \label{txMAM}}

  \subfloat[JJA]{{\includegraphics[height=6.5cm, trim=0 7 0 0, clip]{img/TaxaFlash/densEspacial_19982011jjaTaxaFlashPolyfill}} \label{txJJA}}
  \subfloat[SON]{{\includegraphics[height=6.5cm, trim=0 7 0 0, clip]{img/TaxaFlash/densEspacial_19982011sonTaxaFlashPolyfill}} \label{txSON}}
  }    
  \caption{Densidade sazonal de raios. As cores representam o número de raios por ano por quilômetro quadrado ([ano$^{-1}$] [km$^{-2}$]) em cada ponto da grade de 0.25$^{\circ}$ $\times$ 0.25$^{\circ}$. } 
\label{densidadeSazonalRaios}
\end{figure} 

Na primavera austral (SON), é quando o continente Sul-americano recebe a maior incidência de raios. Ao avaliar a densidade de raios tanto sobre a Bacia do Prata quanto sobre a Bacia Amazônica, observa-se que as maiores densidades de raios ocorreram em SON. Os valores de densidade de raios no mapa da figura \ref{txSON} sobre ambas as bacias hidrográficas  mostram valores entre 4--11 raios mes$^{-1}$ km$^{-2}$ (50--130 raios ano$^{-1}$ km$^{-2}$). Portanto, mesmo que haja maior número de tempestades elétricas sobre a Bacia do Prata durante o verão, as tempestades elétricas de primavera foram as responsáveis pelas maiores densidades de raios durante o ano.

No estudo da sazonabilidade das tempestades elétricas mostrado na seção anterior \ref{cicloanualsecao}, na figura \ref{anual}, entre 30$^{\circ}$--20$^{\circ}$ Sul e 60$^{\circ}$--50$^{\circ}$ Oeste, observa-se um máximo local de probabilidade (0.5\% que representou 789 sistemas) de tempestades elétricas em outubro e que não correspondeu ao período da estação de tempestades elétricas desta região. Portanto, em algumas regiões como é o caso da região central da Bacia do Prata,  o pico da estação de tempestades elétricas não ocorre em fase com a estação em que se observa a máxima densidade de raios. Neste caso as tempestades elétricas responsáveis pelas maiores densidades anuais de raios da região precederam a estação chuvosa, e a estação de tempestades elétricas ocorreu em fase com a estação chuvosa.



%Neste período destaca-se a taxa de raios sobre o Lago Maracaibo durante SON, na Venezuela, que no acumulado dos 14 anos atingiu o valor de 30 raios por mês de observação por quilômetro quadrado em cada ponto da gade de 0.25 graus. Em \citeonline{albrecht2009tropical}, a região do Lago Maracaibo foi apontada como o máximo global das observações do TRMM. 

\begin{table}[!h]
\caption{Total de tempestades elétricas observadas entre 1998-2011, para cada período de três meses associados as estações do ano.}
\label{EstacaoQtd}
\centering
\small
\newcommand{\grayline}{\rowcolor[gray]{.88}}
\renewcommand {\tabularxcolumn }[1]{ >{\arraybackslash }m{#1}}
\newcolumntype{W}{>{\centering\arraybackslash}X}
\begin{tabularx}{\textwidth}{W W} %{|p{10cm}|X|X|X|X|X|X|X|X| }
\hline  \hline 
Estação & Número de sistemas \\[1.5pt]  
 \hline
\grayline Verão~~~~ -- DJF & 46,077 \\[1.5pt]
Outono~~~ -- MAM & 36,804\\[1.5pt]
\grayline Inverno~~ -- JJA  & 16,850\\[1.5pt] 
Primavera -- SON & 57,861\\[1.5pt]
\hline 
\end{tabularx}
\end{table}

%Durante DJF a circulação do JBN trazendo umidade da Amazônia é predominante. É quando é ativado os processos de eletrificações de nuvens entre 40$^{\circ}$--20$^{\circ}$ Sul e 70$^{\circ}$--60$^{\circ}$ Norte.

\begin{figure}[!ht]
  \centering{
  \subfloat[DJF]{{\includegraphics[height=6.5cm, trim=0 7 0 0, clip]{img/DensidadeTempestades/densEspacial19982011djfTempestadesPolyfill}} \label{tempestadesDJF}}
  \subfloat[MAM]{{\includegraphics[height=6.5cm, trim=0 7 0 0, clip]{img/DensidadeTempestades/densEspacial19982011mamTempestadesPolyfill}} \label{tempestadesMAM}}

  \subfloat[JJA]{{\includegraphics[height=6.5cm, trim=0 7 0 0, clip]{img/DensidadeTempestades/densEspacial19982011jjaTempestadesPolyfill}} \label{tempestadesJJA}}
  \subfloat[SON]{{\includegraphics[height=6.5cm, trim=0 7 0 0, clip]{img/DensidadeTempestades/densEspacial19982011sonTempestadesPolyfill}} \label{tempestadesSON}}
  
  }    
  \caption{Densidade espacial sazonal das tempestades elétricas.}
\label{DensidadeTempestadesSazonal}
\end{figure} 


Durante DJF, as mais extensas regiões com as maiores densidades de raios concentram-se: sobre a Argentina; Paraguai; Mato Grosso do Sul; Sul de Mato Grosso; Sudeste brasileiro, principalmente sobre toda a extensão da Serra do Mar entre Santa Catarina e Rio de Janeiro e entre Minas Gerais e o Rio de Janeiro aonde localiza-se o Parque Nacional Itatiaia e o Pico das Agulhas Negras; interior de São Paulo; Sul e Sudoeste de Minas Gerais; região central de Tocantis; Norte do Maranhão; e Norte do Pará região da cidade de Belém.

%Como mostra a figura \ref{txSON}, é evidente que a transição entre o período seco e o período de Monção na AS promovem as condições mais favoráveis  
%Neste período, \citeonline{petersen2002trmm} mostram que, mesmo que se tenha observado diminuição na taxa de raios e redução da intensidade convectiva durante o regime de vento de Oeste durante o experimento TRMM--LBA, (\textit{Large-scale Biosphere Atmosphere Experiment in Amazonia})   em outras regiões da AS, há um aumento da taxa de raios.

Em MAM, período pós-monção na AS, observamos as tempestades elétricas bastante concentradas na região Norte e Nordeste da AS, como mostra a figura \ref{tempestadesMAM}. Durante este período observou-se pontos de grade no mapa da figura \ref{txMAM}, sobre as regiões de Sertão e Agreste Nordestino, no estado da Paraíba entre as cidade de Corema e Patos, também no Rio Grande do Norte sobre a região da bacia do Rio Piranha-Açu, com valores em torno de 4 raios mês$^{-1}$ km$^{-2}$ (50 raios ano$^{-1}$ km$^{-2}$).

%Neste período, principalmente nas regiões das cidade de Belém, Norte do Maranhão, Piauí, Rio Grande do Norte e Paraíba,  ocorrem
%: os máximos de chuva, os máximos de densidade de raios e os máximos de densidade de tempestades elétricas. Esse sincronismo não é comum.


No geral, ao comparar as figuras \ref{densidadeSazonalRaios} e \ref{DensidadeTempestadesSazonal} observa-se que as regiões de máxima densidade sazonal de raios não são as regiões de máxima densidade sazonal de tempestades elétricas. As máximas densidades de raios geralmente ficam situados em regiões deslocadas das máximas densidades de sistemas. Por exemplo, a maior área continua da AS com densidade anual de raios superior a 25 raios ano$^{-1}$ km$^{-2}$, como mostra a figura \ref{densidadeRaios}, ocorre na região Sul da AS. Tanto na figura \ref{densidadeTempestade} quanto na figura \ref{DensidadeTempestadesSazonal}, podemos observar um  gradiente de sistemas nesta região que marca a transição entre o clima Desértico no Deserto do Atacama e Semi-árido da Argentina para o Clima Subtropical Úmido, promovendo um ambiente de transição seco para úmido permanente para os sistemas que iniciam-se principalmente na região da Serra de Córdoba na Argentina e se propagam para Noroeste.

Nas regiões das cidade de Belém e Norte do Maranhão as regiões de máximas densidades anuais de tempestades elétricas coincidem com as regiões de máxima densidades anuais de raios, porém observou-se maiores densidades de raios em DJF, figura \ref{txDJF}, e maiores densidades de tempestades elétricas em MAM, figura \ref{tempestadesMAM}. 

Sobre a região do Lago Maracaibo na Venezuela, a maior densidade sazonal de tempestades elétricas foi observada em SON, figura \ref{tempestadesSON}, em fase com a máxima densidade sazonal de raios que ocorreu em SON, figura \ref{txSON}. Aqui o aumento da densidade de raios está relacionado ao aumento sazonal do número de sistemas. Durante a primavera austral -- SON --  as tempestades elétricas na foz do Rio Catatumbo produziram 31 raios mes$^{-1}$ km$^{-2}$ (371.2 raios ano$^{-1}$ km$^{-2}$). Em \citeonline{albrecht2009tropical}, a região do Lago Maracaibo foi apontada como o máximo global anual das observações do TRMM. 

%, reforçando a hipótese de \citeonline{williams2002}, em que se espera maior atividade elétrica de nuvem em um ambiente de transição entre seco e úmido.

A partir do estudo das densidades anuais de tempestades elétricas e densidades anuais de raios, foi calculada a densidade de raios por tempestades elétricas como mostra a figura \ref{eficiencia}. Os valores desta grandeza (raios [ano$^{-1}$] [km$^{-2}$] por tempestade), associam-se com eficiência de produção de raios por tempestade elétrica em cada região de 0.25$^{\circ}$ $\times$ 0.25$^{\circ}$.

A região da Bacia do Paraguai-Parana-Prata foi a maior extensão contínua com as maiores densidades de raios por tempestade elétrica. Porém destacam-se regiões como: a Serra do Mar no Sudeste, abrangendo o Vale do Ribeira em São Paulo, Pico das Agulhas Negras em Minas Gerais, região serrana do Rio de Janeiro; parte Sul do Tocantis; parte Leste e Norte do Pará e Leste do estado do Amazonas. Estas regiões podem estar associadas com tempo severo, locais cuja a topografia ou a circulação local intensificam os sistemas.

Os maiores valores da escala de cores da figura \ref{eficiencia} que representa a densidade anual de raios por tempestades elétricas, foram observados sobre a região do Lago Maracaibo. Conforme o mapa da figura \ref{densidadeRaios}, sabemos que o ponto de grade com a máxima densidade de raios sobre a região do Lago Maracaibo (148.1 ano$^{-1}$ km$^{-2}$), mostra que sobre a extensão de 772 km$^{-2}$ (área do ponto de grade de  0.25$^{\circ}$ $\times$ 0.25$^{\circ}$), ocorreu 114,333 raios ano$^{-1}$. O máximo valor de  11.73 $\times$ 10$^{-2}$ ano$^{-1}$ km$^{-2}$ da figura \ref{eficiencia} mostra que, sobre a área do ponto de grade de máxima densidade de raios por tempestade, cada tempestade elétrica observada contribuiu em média com 91 raios ano$^{-1}$, portanto, 1256 tempestades elétricas foram responsáveis por toda a produção de raios do ponto de grade.  

%O valor máximo de  11.73 $\times$ 10$^{-2}$ ano$^{-1}$ km$^{-2}$ mostra que, sobre a área de um ponto de grade nesta região ($\simeq$772 km$^{2}$), ocorreram aproximadamente 91 raios ano$^{-1}$ por tempestade elétrica. 

%Conforme \citeonline{cecil2005}, as \textit{precipitation features} com taxa de raios superiores a 297 raios por minuto sobre a região da bacia do Prata possuíram na ordem de 10$^5$ km$^2$ de extensão. Considerando o mapa da figura \ref{eficiencia}, com valores 2.5--7.5 10$^{-2}$ ano$^{-1}$ km$^{-2}$
%Na região do  Parque Nacional Natural Paramillo na Colômbia e no Lago Maracaibo na Venezuela, a taxa de raios por em cada área de tempestade de 0.25 graus mostra valores com a mesma magnitude de regiões na Bacia do Prata, mesmo que o número de raios e de sistemas produzidos ao Norte sejam maiores.

Regiões no pico da Cordilheira dos Andes são bastante eficientes, principalmente na região da cidade de Cochabamba na Bolívia. Alguns pontos de grade mostraram $\simeq$42 raios ano$^{-1}$ por tempestade elétrica.

\begin{figure}[!h]
\centering{\includegraphics[height=13.5cm]{{img/eficiencia/densEspacial_19982011totalEficienciaPolyfill}}}  
\caption{Densidade de raios por tempestade elétrica. As cores representam o número de raios por tempestades elétricas multiplicado por 10$^{-2}$ por ano por quilômetro quadrado (10$^{-2}$ [ano$^{-1}$] [km$^{-2}$]) em cada ponto da grade de 0.25$^{\circ}$ $\times$ 0.25$^{\circ}$.}
\label{eficiencia}
\end{figure}