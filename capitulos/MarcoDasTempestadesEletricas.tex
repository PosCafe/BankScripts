\chapter{MARCO DAS TEMPESTADES ELÉTRICAS NA AMÉRICA DO SUL}


O Marco das tempestades elétricas descreve os locais e quando estes sistemas ocorrem na América do Sul. Determina-se a sazonalidade, o ciclo diurno, a distribuição espacial de raios e das tempestades elétricas. 

\section{CICLO DIURNO E CICLO ANUAL}

Utilizando a base de dados de tempestades elétricas construída nesta pesquisa, foi estudada a frequência de ocorrências dos sistemas no decorrer das horas do dia, figura \ref{ciclodiurnototal}, e meses do ano, figura \ref{cicloanualtotal}. Deste modo, obtivemos na figura \ref{diurnoanual}, o ciclo diurno e anual das tempestades elétricas por meio da distribuição de probabilidade de ocorrências.


\begin{figure}[!hb]
  \centering{
  \subfloat[Ciclo diruno.]{{\includegraphics[scale=0.95]{img/ciclos/ciclodiurno19982011total}} \label{ciclodiurnototal}}
  \subfloat[Ciclo anual.]{{\includegraphics[scale=0.95]{img/ciclos/cicloanual19982011total}} \label{cicloanualtotal}}
  }
\caption{Ciclo diurno e anual das tempestades elétricas observadas em hora local. Os valores de probabilidade foram normalizados pelo total dos 154,189 sistemas identificados.}
\label{diurnoanual} 
\end{figure} 

A figura \ref{ciclodiurnototal},  mostra que, entre 14h e 15h as tempestades elétricas são mais prováveis, indicando que o aquecimento da superfície do continente e o aumento da camada limite planetária no decorrer do dia são ingredientes que podem aumentar a probabilidade de ocorrência em até 4,6 vezes em relação aos horários de menor fluxo de calor sensível para a atmosfera. Enquanto o TRMM observou 2312 tempestades elétricas às 9h (hora local), às 14h foram observadas 13,877.

No ciclo anual, conforme mostra a figura \ref{cicloanualtotal}, observa-se que a estação de tempestades elétricas na América do Sul possui dois picos, um em outubro e outro em março, porém contempla os meses de outubro, novembro, dezembro, janeiro, fevereiro e março. A maior probabilidade de ocorrência esteve associada ao mês de outubro, que concentrou 16,961 tempestades elétricas observadas em 14 anos. % Entre janeiro e março as tempestades elétricas tiveram probabilidade de 0.3\% menor do que no início da estação, em outubro.   


O ciclo diurno também foi estudado para cada região de 10 por 10 graus, como mostra a figura \ref{diurno}. 



\begin{figure}[!hb]
\centering{\includegraphics[scale=1.5]{img/ciclos/ciclodiurno10x1019982011localtime}}  
\caption{Ciclo diurno em hora local para as tempestades elétricas observadas em cada região de 10 por 10 graus. Os valores de probabilidade são mostrados em porcentagem e foram normalizados pelo total de 154,189 sistemas observados.}
\label{diurno}
\end{figure}


Mesmo que em uma análise geral mostre a importância do aquecimento superficial do continente para a ocorrência de tempestades elétricas, sistemas noturnos sobre a Colômbia e Venezuela são bastante frequentes. Na figura \ref{diurno}, entre 0$^{\circ}$--10$^{\circ}$ Norte e 80$^{\circ}$--70$^{\circ}$ Oeste, às 0h em hora local, temos o maior valor de probabilidade (0.4\%) de tempestades elétricas noturnas da América do Sul, o que representou um número de 617 sistemas observados em 14 anos, apenas entre 0h e 00:59h.


A circulação de vale e montanha associada com a topografia elevada na Colômbia, principalmente a região do Parque Nacional Natural Paramillo, e o Lago Maracaibo na Venezuela, combinados com a atuação da Zona de Convergência Intertropical (ZCIT), promovem condições para o desenvolvimento de tempestades elétricas noturnas de maneira mais eficiente do que as demais regiões. \sigla{name={ZCIT},description={Zona de Convergência Intertropical}}

No Oceano Pacífico, entre 0$^{\circ}$--10$^{\circ}$ Norte e 90$^{\circ}$--80$^{\circ}$ Sul abrangendo o Parque Nacional da Ilha do Coco na Costa Rica e parte das ilhas Galápagos no Equador, foi a região oceânica com a maior probabilidade de ocorrência de tempestades elétricas. Esta possui um ciclo diurno duplo de tempestades elétricas. Elas ocorrem às 4h, em hora local, e as 14h. A maior probabilidade de ocorrência (0.15\%) foi observada às 4h, que correspondeu à 231 sistemas.

No Pacífico Sul, as tempestades elétricas são mais raras do que as demais regiões devido a atuação permanente da subsidência da Célula de Hadley, que modula a Alta Subtropical do Pacífico Sul, responsável também por regiões como o Deserto do Atacama e parte do semi-árido Argentino.

Na região do Atlântico Subtropical, a probabilidade de tempestades elétricas é maior do que no Atlântico Norte. A passagem de sistemas transientes entre 40$^{\circ}$--30$^{\circ}$ Sul e 50$^{\circ}$--30$^{\circ}$ Oeste e 30$^{\circ}$--20$^{\circ}$ Sul e 40$^{\circ}$--30$^{\circ}$ Oeste, gera maior número de tempestades elétricas oceânicas do que com a atuação da ITCZ no Atlântico Tropical. Observa-se também que nas regiões oceânicas o ciclo diurno das tempestades elétricas indica maior atividade noturna e não às 14-15h igual no continente.


O pico de atividade de tempestades elétricas durante o ciclo diurno, figura \ref{diurno}, ocorreu entre 10$^{\circ}$--0$^{\circ}$ Sul e 70$^{\circ}$--50$^{\circ}$ Oeste e 20$^{\circ}$--10$^{\circ}$ Sul e 60$^{\circ}$--50$^{\circ}$ Oeste. Em cada uma destas três caixas, observou-se a probabilidade de aproximadamente 0.8\% entre as 14h e 15h, mostrando que em toda esta região, o TRMM observou uma média de 3 tempestades elétricas a cada 2 dias, apenas nessas duas horas.


Entre 30$^{\circ}$--20$^{\circ}$ Sul e 60$^{\circ}$--50$^{\circ}$ Oeste, na figura \ref{diurno}, região de grande atividade de Sistemas Convectivos de Meso-escala (MCS) conforme descrevem \citeonline{Durkee2009}, encontra-se um máximo durante a tarde e os sistemas noturnos tiveram probabilidade de ocorrência 2.7 vezes menor do que os valores encontrados sobre os vales na Colômbia e Venezuela, mostrando que a ocorrência dos MCS ao Sul da América do Sul com ciclo de vida maior do que 9h ou com formação noturna, não possuem probabilidade de ocorrência que destaca-se em relação as demais regiões continentais, mesmo neste banco de dados composto apenas por tempestades elétricas. Na figura \ref{anual}, há um máximo de atividade em outubro que antecede a estação de tempestades elétricas entre dezembro e março. O máximo é observado em janeiro com 1234 sistemas identificados na região.\sigla{name={MCS},description={Sistemas Convectivos de Meso-escala}}


A tabela \ref{caracEstacao} mostra os meses de duração das estações de tempestades elétricas com base no estudo mostrado na figura \ref{anual}. Os períodos em que a probabilidade de ocorrência de sistemas foram superiores à 0.7 do máximo observado na região, foram considerados como os períodos das estações de tempestades elétricas. %Regiões como as linhas 19 e 20 da tabela \ref{caracEstacao} mostram que houveram apenas 32 tempestades elétricas em 14 anos, portanto 


\begin{figure}[!ht]
\centering{\includegraphics[scale=1.5]{img/ciclos/cicloanual10x1019982011localtime}}  
\caption{Ciclo anual em hora local para as tempestades elétricas observadas em cada região de 10 por 10 graus. Os valores de probabilidade são mostrados em porcentagem e foram normalizados pelo total de 154,189 sistemas observados. As linhas horizontais cortam o valor de 0.7 do máximo de probabilidade, utilizado como limiar para definir o início e fim das estações de tempestades elétricas.}
\label{anual}
\end{figure}



\begin{table}[!ht]
\caption{Principais características do ciclo anual de probabilidade de ocorrência de tempestades elétricas observadas entre 1998-2011, em cada região de 10 por 10 graus.}
\label{caracEstacao}
\centering
\small
\newcommand{\grayline}{\rowcolor[gray]{.88}}
\renewcommand {\tabularxcolumn }[1]{ >{\arraybackslash }m{#1}}
\newcolumntype{W}{>{\centering\arraybackslash}X}
\begin{tabularx}{\textwidth}{p{0.6cm} p{3.5cm} W W W W} %{|p{10cm}|X|X|X|X|X|X|X|X| }
\hline\hline 
\grayline  & Localização & Número de sistemas & Estação (meses) & Duração (meses) & Máximo\\[1.5pt]
\hline
1&0$^{\circ}$--10$^{\circ}$N, 90$^{\circ}$--80$^{\circ}$O& 4159  & Abr--Set &6& Abr\\[1.5pt]\grayline
2&0$^{\circ}$--10$^{\circ}$N, 80$^{\circ}$--70$^{\circ}$O& 14,047 & Mar--Nov &9& Out\\[1.5pt]
3&0$^{\circ}$--10$^{\circ}$N, 70$^{\circ}$--60$^{\circ}$O& 11,787 & Jul--Out &4& Set\\[1.5pt]\grayline
4&0$^{\circ}$--10$^{\circ}$N, 60$^{\circ}$--50$^{\circ}$O&  4868 & Jul--Set &3& Ago\\[1.5pt]
5&0$^{\circ}$--10$^{\circ}$N, 50$^{\circ}$--40$^{\circ}$O& 645 & Mai--Jun, Set--Nov &5& Set\\[1.5pt] \grayline
6&0$^{\circ}$--10$^{\circ}$N, 40$^{\circ}$--30$^{\circ}$O& 821 & Out--Dez &3& Dez\\[1.5pt]

7&10$^{\circ}$--0$^{\circ}$S, 90$^{\circ}$--80$^{\circ}$O& 217 & Mar &1& Mar\\[1.5pt]\grayline
8&10$^{\circ}$--0$^{\circ}$S, 80$^{\circ}$--70$^{\circ}$O& 9721 & Set--Dez &4& Out\\[1.5pt]
9&10$^{\circ}$--0$^{\circ}$S, 70$^{\circ}$--60$^{\circ}$O& 12,168 & Ago--Nov &4& Out\\[1.5pt]\grayline
10&10$^{\circ}$--0$^{\circ}$S, 60$^{\circ}$--50$^{\circ}$O& 12,231 & Set--Dez &4& Out\\[1.5pt]
11&10$^{\circ}$--0$^{\circ}$S, 50$^{\circ}$--40$^{\circ}$O& 7731 & Jan--Abr, Nov--Dez &6&  Jan\\[1.5pt]\grayline
12&10$^{\circ}$--0$^{\circ}$S, 40$^{\circ}$--30$^{\circ}$O& 1349 & Jan--Abr &4&  Mar\\[1.5pt]

13&20$^{\circ}$--10$^{\circ}$S, 90$^{\circ}$--80$^{\circ}$O& 1  &   --0--  & 1 & Mai \\[1.5pt]\grayline
14&20$^{\circ}$--10$^{\circ}$S, 80$^{\circ}$--70$^{\circ}$O& 4254 & Jan--Fev,  Set--Dez  &6& Out\\[1.5pt]
15&20$^{\circ}$--10$^{\circ}$S, 70$^{\circ}$--60$^{\circ}$O& 8585 & Jan--Mar, Set--Dez &7& Out\\[1.5pt]\grayline
16&20$^{\circ}$--10$^{\circ}$S, 60$^{\circ}$--50$^{\circ}$O& 10,414 & Jan--Mar,  Out--Dez &6& Out\\[1.5pt]
17&20$^{\circ}$--10$^{\circ}$S, 50$^{\circ}$--40$^{\circ}$O& 8201 & Jan--Mar, Out--Dez &6&  Jan\\[1.5pt]\grayline
18&20$^{\circ}$--10$^{\circ}$S, 40$^{\circ}$--30$^{\circ}$O& 611 & Jan--Mar &3&  Mar\\[1.5pt]

19&30$^{\circ}$--20$^{\circ}$S, 90$^{\circ}$--80$^{\circ}$O& 32 & Mai--Jun &2&  Mai\\[1.5pt]\grayline
20&30$^{\circ}$--20$^{\circ}$S, 80$^{\circ}$--70$^{\circ}$O& 32 & Fev, Mai, Jul--Ago &4&  Fev, Mai,  Jul\\[1.5pt]
21&30$^{\circ}$--20$^{\circ}$S, 70$^{\circ}$--60$^{\circ}$O& 5558 & Dez--Mar &4& Jan\\[1.5pt]\grayline
22&30$^{\circ}$--20$^{\circ}$S, 60$^{\circ}$--50$^{\circ}$O& 8676 & Dez--Mar &4& Jan\\[1.5pt]
23&30$^{\circ}$--20$^{\circ}$S, 50$^{\circ}$--40$^{\circ}$O& 5996 & Dez--Mar &4& Jan\\[1.5pt]\grayline
24&30$^{\circ}$--20$^{\circ}$S, 40$^{\circ}$--30$^{\circ}$O& 1849 & Fev--Mar, Mai, Dez &4&  Mar\\[1.5pt]

25&40$^{\circ}$--30$^{\circ}$S, 90$^{\circ}$--80$^{\circ}$O& 258 & Jun &1&  Jun \\[1.5pt]\grayline
26&40$^{\circ}$--30$^{\circ}$S, 80$^{\circ}$--70$^{\circ}$O& 370 & Jan--Mar, Mai--Jun &5&  Jan\\[1.5pt]
27&40$^{\circ}$--30$^{\circ}$S, 70$^{\circ}$--60$^{\circ}$O& 7638 & Dez--Jan &2&  Jan\\[1.5pt]\grayline
28&40$^{\circ}$--30$^{\circ}$S, 60$^{\circ}$--50$^{\circ}$O& 5403 & Dez--Mar &4&  Jan\\[1.5pt]
29&40$^{\circ}$--30$^{\circ}$S, 50$^{\circ}$--40$^{\circ}$O& 2966 & Jan--Set &9&  Abr\\[1.5pt]\grayline
30&40$^{\circ}$--30$^{\circ}$S, 40$^{\circ}$--30$^{\circ}$O& 2288 & Abr--Jun  &3& Mai\\[1.5pt]


\hline 
\end{tabularx}
\end{table}

 
%No ciclo anual mostrado na figura \ref{anual}, observa-se uma clara diferença sazonal na ocorrência de tempestades elétricas entre os dois Hemisférios. Sobre o Hemisfério Norte as tempestades ocorrem principalmente entre os meses de abril e agosto, enquanto no Hemisfério Sul entre setembro e março, apesar das características individuais de cada região como por exemplo dois ou três picos de atividade durante a estação.
%O deslocamento da ZCIT durante o verão setentrional, inverte a estação de tempestades elétricas entre 10$^{\circ}$ Sul e 10$^{\circ}$ Norte, diminuindo o número de sistemas no inverno austral.

Na região entre o clima semi-árido na Argentina e parte da Bacia do Prata, entre 40$^{\circ}$--20$^{\circ}$ Sul e 70$^{\circ}$--60$^{\circ}$ Norte, figura \ref{anual}, local das tempestades mais severas e convecção mais profunda da América do Sul como apontam \citeonline{cecil2005, Romatschke2010}, foi encontrada uma estação de tempestades elétricas bastante definida entre dezembro e janeiro, sendo que entre maio e agosto, a probabilidade de ocorrência de sistemas é praticamente 0\%.%, pois é no verão que o jato de baixos níveis, que traz umidade da Amazônia, se intensifica e dispara os processos de eletrificação nesta região.

As estações de tempestades elétricas se configuram conforme o Sistema de Monção da América do Sul (SAMS)   \sigla{name={SAMS},description={Sistema de Monção da América do Sul}}. Na região central da América do Sul, observa-se que com o aumento da temperatura da superfície entre julho e setembro, o máximo de precipitação começa a se deslocar do Hemisfério Norte para o Hemisfério Sul e desta forma iniciando a estação chuvosa meridional pela região Oeste da Bacia Amazônica \cite{Zhou1998,grimm2003nino,reboita2010regimes,Marengo2012}.

Entre 10$^{\circ}$--0$^{\circ}$ Sul e 80$^{\circ}$--60$^{\circ}$ Norte,  na figura \ref{anual}, observa-se que o pico da estação de tempestades elétricas ocorreu em outubro, nos primeiros passos da estação chuvosa da América do Sul. Porém o máximo de precipitação nesta região ocorre depois de 4 ou 5 meses. 

Em \citeonline{Petersen2001}, o estudo realizado referente a estrutura tridimensional da precipitação observada pelo TRMM sobre a região Central da Amazônia, mostrou que a convecção mais profunda ocorre também na transição do período seco para o chuvoso, exatamente quando começa a reversão sazonal do vento em baixos níveis associado ao SAMS conforme apontam \citeonline{Zhou1998}.

Com o início do verão austral, o máximo de precipitação caminha até a região Centro Oeste e Sudeste do Brasil. Em janeiro, o SAMS se configura mais ativamente com a atuação da Zona de Convergência do Atlântico Sul (SACZ) \sigla{name={SACZ},description={Zona de Convergência do Atlântico Sul}} e intensificação do Jato de Baixos Níveis (JBN). \sigla{name={JBN},description={Jato de Baixos Níveis}} A atuação do JBN, principalmente nas regiões abaixo de 20$^{\circ}$ Sul, ativa a estação chuvosa e de tempestades elétricas em sincronismo. 

Durante abril e maio, o SAMS vai se desconfigurando e o máximo de chuva começa a retornar para o Hemisfério Norte caminhando de Sudeste para o Nordeste do Brasil e subindo pelo lado Leste da Bacia Amazônica. Neste retorno é que ocorrem os máximos de precipitação em toda a região da Bacia Amazônica, porém o máximo de ocorrência de tempestades elétrica ocorreu na vinda da estação chuvosa para o Hemisfério Sul.

Na região Nordeste do Brasil, entre 10$^{\circ}$--0$^{\circ}$ Sul e 40$^{\circ}$--30$^{\circ}$ Norte, o máximo de chuva ocorre juntamente com o máximo de ocorrência de tempestades elétricas, depois da atuação da SACZ no continente.

%As probabilidades de ocorrência mostradas nas figuras \ref{anual} e \ref{diurno} descrevem bem comportamento dos ciclos diurno e anual, porém os valores de probabilidades são tendenciosos nas regiões em que o satélite passou mais tempo observando, conforme é descrito em \ref{metodoPass}. Pois as densidades de probabilidade em cada região de 10 por 10 graus foram obtidas apenas considerando as latitudes e longitudes médias dos sistemas e a hora, minuto, segundo, dia, mês e ano da observação.

\section{DENSIDADES ESPACIAIS}

Considerando o método descrito em \ref{metodoPass}, referente ao cálculo da densidade espacial de tempestades elétricas, equação \ref{dete}, e densidade espacial de raios, equação \ref{defl}, nesta seção será possível avaliar se as regiões aonde ocorrem o maior número de sistemas, correspondem as regiões com maior número de raios.

Na figura \ref{tempestadestotal}, observa-se que as regiões de máxima ocorrência de tempestades elétricas estão situadas sobre a Colômbia e região central da Bacia Amazônica, abrangendo a parte brasileira, colombiana, venezuelana e peruana.



\begin{figure}[!ht]
 \centering{
  \subfloat[Densidade espacial total de tempestades elétricas. Os valores correspondem ao número de sistemas por ano por quilômetro quadrado em cada pronto da grade de 0.25 graus.]{\includegraphics[scale=0.88]{img/DensidadeTempestades/densEspacial19982011TotalTempestadesPolyfill} \label{tempestadestotal}}
  \subfloat[Densidade espacial total de raios. Os valores correspondem ao número de raios por ano por quilômetro quadrado em cada pronto da grade de 0.25 graus.]{\includegraphics[scale=0.88]{img/TaxaFlash/densEspacial_19982011totalTaxaFlashPolyfill}\label{raiosTotal}}
  }
\caption{Densidade espacial de tempestades elétricas e raios observados entre 1998 e 2011.}
\label{tempesRaios}
\end{figure}

Mesmo que os sistemas com as maiores taxas de raios no tempo observados pelo TRMM, estejam mais concentrados no Sul da América do Sul conforme mostram \citeonline{cecil2005,zipser2006}, as tempestades elétricas são bem mais frequentes à Noroeste da AS.

A atuação da ITCZ combinada com a convergência de umidade e liberação de calor latente e sensível na Floresta Amazônica, além que regular o SAMS, são os principais propulsores de tempestades elétricas da América do Sul. %que associa-se intimamente com o SAMS,

No entanto, os mecanismos de eletrificação são bem mais eficientes nas tempestades elétricas no Sul da AS, sobe a Bacia do Prata. Na figura \ref{tempesRaios}, observa-se que na Amazônia, as regiões com taxa de raios superiores 30 (ano$^{-1}$ km$^{-2}$), possuíram taxa de sistemas acima de 120 (ano$^{-1}$ km$^{-2}$), enquanto na região da Argentina e Paraguai, as mesmas taxas de raios são atingidas com uma taxa de sistemas em torno de 40 (ano$^{-1}$ km$^{-2}$).

Nas figuras \ref{TaxaFlash} e \ref{DensidadeTempestadesSazonal}, a densidade espacial de raios e de tempestades elétricas, foi calculada para os períodos associados a cada estação do ano: dezembro, janeiro e fevereiro (DJF), março, abril e maio (MAM), junho, julho e agosto (JJA) e setembro, outubro e novembro (SON). A tabela \ref{EstacaoQtd} mostra o acumulado de sistemas observados em cada estação do ano.
\sigla{name={DJF},description={Dezembro, janeiro e fevereiro}} 
\sigla{name={MAM},description={Março, abril e maio}}
\sigla{name={JJA},description={Junho, julho e agosto}}
\sigla{name={SON},description={Setembro, outubro e novembro}}


\begin{figure}[!ht]
  \centering{
  \subfloat[DJF]{{\includegraphics[scale=0.88, trim=0 7 0 0, clip]{img/TaxaFlash/densEspacial_19982011djfTaxaFlashPolyfill}} \label{txDJF}}
  \subfloat[MAM]{{\includegraphics[scale=0.88, trim=0 7 0 0, clip]{img/TaxaFlash/densEspacial_19982011mamTaxaFlashPolyfill}} \label{txMAM}}

  \subfloat[JJA]{{\includegraphics[scale=0.88, trim=0 7 0 0, clip]{img/TaxaFlash/densEspacial_19982011jjaTaxaFlashPolyfill}} \label{txJJA}}
  \subfloat[SON]{{\includegraphics[scale=0.88, trim=0 7 0 0, clip]{img/TaxaFlash/densEspacial_19982011sonTaxaFlashPolyfill}} \label{txSON}}
  }    
  \caption{Densidade espacial sazonal de raios.}
\label{TaxaFlash}
\end{figure} 


Na primavera austral (SON), início do SAMS, a intensificação dos alísios vindos do Atlântico Norte, e o aumento gradativo da evapotranspiração na Floresta Amazônica vão intensificando o transporte de umidade da bacia do Amazônias para a bacia do Prata \cite{marengo2004}.  Esse processo de início da configuração do SAMS provoca a estação com a maior taxa de raios do continente Sul Americano, e esta, ocorre em regiões no centro no continente principalmente a Leste da Cordilheira dos Andes: na Amazônia Central, Argentina, Paraguai e Sul do Brasil.


Neste período destaca-se a taxa de raios sobre o Lago Maracaibo durante SON, na Venezuela, que no acumulado dos 14 anos atingiu o valor de 30 raios por mês de observação por quilômetro quadrado em cada ponto da gade de 0.25 graus. Em \citeonline{albrecht2009tropical}, a região do Lago Maracaibo foi apontada como o máximo global das observações do TRMM. 


\begin{table}[!h]
\caption{Total de tempestades elétricas observadas entre 1998-2011, para cada período de três meses associados as estações do ano.}
\label{EstacaoQtd}
\centering
\small
\newcommand{\grayline}{\rowcolor[gray]{.88}}
\renewcommand {\tabularxcolumn }[1]{ >{\arraybackslash }m{#1}}
\newcolumntype{W}{>{\centering\arraybackslash}X}
\begin{tabularx}{\textwidth}{W W} %{|p{10cm}|X|X|X|X|X|X|X|X| }
\hline \hline 
Estação & Número de sistemas \\[1.5pt]
 \hline
\grayline DJF & 44,534\\[1.5pt]
MAM & 36,096\\[1.5pt]
\grayline JJA & 16,786\\[1.5pt] 
SON & 56,773\\[1.5pt]
\hline 
\end{tabularx}
\end{table}

%Durante DJF a circulação do JBN trazendo umidade da Amazônia é predominante. É quando é ativado os processos de eletrificações de nuvens entre 40$^{\circ}$--20$^{\circ}$ Sul e 70$^{\circ}$--60$^{\circ}$ Norte.

Durante DJF, os máximos de raios são observados em Mato Grosso do Sul; Sul de Mato Grosso; Sudeste Brasileiro, entre costa de Santa Catarina e o Vale do Ribeira em São Paulo, região de fronteira entre São Paulo, Minas Gerais e Rio de Janeiro, aonde localiza-se o Parque Nacional Itatiaia e o Pico das Agulhas Negras; interior de São Paulo; Goias; e na Bacia do Rio Tocantis. Apesar de observarmos o maior número de raios durante a estação de transição entre seca e chuvosa, essas regiões Centrais e Sudeste da AS possuem os processos de eletrificação regulados durante a estação chuvosa. 

Em \citeonline{petersen2002trmm}, é mostrado que mesmo que se tenha observado diminuição na taxa de raios e redução da intensidade convectiva durante o regime de vento de Oeste no experimento TRMM \textit{Large-scale Biosphere Atmosphere Experiment in Amazonia} (LBA), em outras regiões da AS durante o período chuvoso, há um aumento da taxa de raios.

Considerando que o regime de ventos de Leste e Oeste identificado no LBA está associado com as fases ativas e inativas do SAMS conforme descrevem \citeonline{carvalho2002intraseasonal}, pode-se considerar que as máximas taxas de raios apresentadas na figura \ref{txDJF} são moduladas pelas variações na circulação sinóptica associadas com o processos de formação e dissipação da SACZ \cite{petersen2002trmm,albrecht2011,silva2002lba}.

Durante MAM, quando o máximo de chuvas começa a retornar para o Hemisfério Norte, observamos as tempestades elétricas bastante concentradas na região Norte e Nordeste da AS, como mostra a figura \ref{tempestadesMAM}. Neste período, principalmente nas regiões das cidade de Belém, estado do Maranhão, Piauí, Rio Grande do Norte e Paraíba,  ocorrem: os máximos de chuva, os máximos de densidade de raios e os máximos de densidade de tempestades elétricas. Esse sincronismo não é comum.

Ao comparar as figuras \ref{TaxaFlash} e \ref{DensidadeTempestadesSazonal} observa-se que as regiões de máxima densidade espacial de raios não são as regiões de máxima densidade de tempestades elétricas. Os máximos de raios ficam situados em regiões de transição, deslocados dos máximos de sistemas, reforçando a hipótese de \citeonline{williams2002}, em que se espera maior atividade elétrica de nuvem em um ambiente de transição entre seco e úmido.

Por exemplo, a maior área continua da América do Sul com taxas anuais de raios superiores a 20 raios por ano por quilômetro quadrado, como mostra a figura \ref{raiosTotal}, ocorre na região Sul da AS. Tanto na figura \ref{tempestadestotal} quanto na figura \ref{DensidadeTempestadesSazonal}, podemos observar um forte gradiente de sistemas nesta região, que marca a transição entre o clima Desértico no Deserto do Atacama e Semi-árido na Argentina para o clima Subtropical úmido, promovendo um ambiente de transição seco/úmido permanente para os sistemas que iniciam-se principalmente na região da Serra de Córdoba na Argentina e se propagam para Noroeste.

\begin{figure}[!ht]
  \centering{
  \subfloat[DJF]{{\includegraphics[scale=0.88, trim=0 7 0 0, clip]{img/DensidadeTempestades/densEspacial19982011djfTempestadesPolyfill}} \label{tempestadesDJF}}
  \subfloat[MAM]{{\includegraphics[scale=0.88, trim=0 7 0 0, clip]{img/DensidadeTempestades/densEspacial19982011mamTempestadesPolyfill}} \label{tempestadesMAM}}

  \subfloat[JJA]{{\includegraphics[scale=0.88, trim=0 7 0 0, clip]{img/DensidadeTempestades/densEspacial19982011jjaTempestadesPolyfill}} \label{tempestadesJJA}}
  \subfloat[SON]{{\includegraphics[scale=0.88, trim=0 7 0 0, clip]{img/DensidadeTempestades/densEspacial19982011sonTempestadesPolyfill}} \label{tempestadesSON}}
  
  }    
  \caption{Densidade espacial sazonal das tempestades elétricas.}
\label{DensidadeTempestadesSazonal}
\end{figure} 

A partir do estudo das densidades de tempestades elétricas e raios, a figura \ref{eficiencia},  representa as regiões em que as tempestades elétricas são mais eficientes na produção de raios. Foi calculada a taxa de raios por tempestade elétrica por ano por quilômetro quadrado. Os maiores valores desta dimensão que associa-se com eficiência espacial que cada região de 0.25 graus tem em produzir raios, representam os locais em que se tem menor número de sistemas em relação ao número de raios durante os 14 anos de dados.

A região da bacia do Prata é a maior extensão contínua com os maiores valores de eficiência espacial de produção de raios. Porém destacam-se regiões menores como no Vale do Ribeira em São Paulo, Pico das Agulhas Negras em Minas Gerais, região serrana do Rio de Janeiro, parte Sul do Tocantis, parte Leste e Norte do Pará e Leste do estado do Amazonas. Estas regiões podem estar associadas com regiões de tempo severo. Locais em que a topografia ou a circulação local intensifica os sistemas.

Na região do  Parque Nacional Natural Paramillo na Colômbia e no Lago Maracaibo na Venezuela, a taxa de raios por em cada área de tempestade de 0.25 graus mostra valores com a mesma magnitude de regiões na Bacia do Prata, mesmo que o número de raios e de sistemas produzidos ao Norte sejam maiores.

Algumas regiões no pico da Cordilheira dos Andes são bastante eficientes, principalmente na região da cidade de Cochabamba na Bolívia.

\begin{figure}[!h]
\centering{\includegraphics[scale=0.88]{img/TaxaFlashTempestade/densEspacial19982011totalEficienciaPolyfill}}  
\caption{Eficiencia de tempestade}
\label{eficiencia}
\end{figure}