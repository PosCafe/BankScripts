\chapter{DADOS E METODOLOGIA}
\label{metodologia}

A Metodologia consiste fundamentalmente na construção de um subconjunto de dados das observações dos sensores VIRS, LIS e PR a bordo do satélite TRMM, durante o período entre 1998 e 2011. Além dos dados satelitais, as reanálises 2 do \textit{National Centers for Environmental Prediction -- Department of Energy
} (NCEP--DOE) em níveis de pressão foram utilizadas para calcular os valores de temperatura nos níveis correspondentes as altitudes de medidas do PR.
\sigla{name={NCEP--DOE},description={\textit{National Centers for Environmental Prediction -- Department of Energy}}} 


As informações dos diferentes sensores foram combinadas de maneira a identificar sistemas denominados como tempestades elétricas, definidas como nuvens que possuíram pelo menos um raio -- \textit{flash} -- detectado pelo LIS. 

%A seguir são apresentadas as principais características do TRMM 
%Para melhor entender as implicações que envolvem a construção de uma base de dados de sistemas individualmente a partir das observações do TRMM, inicialmente descreve-se algumas das principais características operacionais do satélite TRMM.

\section{O SATÉLITE TRMM}
\label{metodologiaTRMM}
\index{O Satélite TRMM}

O satélite \textit{Tropical Rainfall Measuring Mission} -- TRMM  faz parte de uma missão conjunta entre a \textit{National Aeronautics and Space Administration} (NASA) e a \textit{Japan Aerospace Exploration Agency} (JAXA),  com o objetivo de estimar a distribuição espaço-temporal da chuva e do fluxo de calor latente para a região tropical e subtropical terrestre. Estas informações são fundamentais para avaliar modelos atmosféricos globais e climáticos principalmente quando se trata de previsão do tempo e clima nos trópicos \cite{kummerok1998,simpson1988}.

\sigla{name={NASA},description={\textit{National Aeronautics and Space Administration}}}   
\sigla{name={JAXA},description={\textit{Japan Aerospace Exploration Agency}}}
  

O satélite TRMM foi lançado em 28 de novembro de 1997 entrando em órbita circular baixa de 350 km de altitude com inclinação de 35$^{\circ}$ e período de 90 minutos. Originalmente a missão TRMM teria 3 anos, porém, devido ao sucesso até o ano de 2000, seu tempo foi prolongado e em agosto de 2001 sua órbita foi elevada até 402,5 km de altitude. Devido as suas características orbitais, o TRMM sobrevoa 2 vezes ao dia uma região de 10$^{\circ}$ $\times$ 10$^{\circ}$ de latitude e longitude \cite{simpson1988}. Em outubro de 2014, o satélite TRMM começou a descender e em 08 de Abril de 2015 atingiu a órbita de descomissionamento \cite{TRMMgoodbye,TRMMend}.

%, sendo o LIS o último sensor a ser desligado
% Em 2014, verificou-se que o combustível do satélite TRMM está no fim e a manobra de reentrada na atmosfera e aterrissagem deverá ocorrer por volta de abril de 2015, conforme notícia divulgada \cite{TRMMgoodbye}.
  
Os instrumentos a bordo do TRMM são: radar de precipitação (\textit{Precipitation Radar} -- PR), radiômetro de micro-ondas (\textit{TRMM Microwave Imager} -- TMI), radiômetro no visível e no infravermelho (\textit{Visible and Infrared Scanner} -- VIRS), radiômetro para medir a energia radiante da terra e das nuvens (\textit{Clouds and the Earth's Radiant Energy System} -- CERES) e sensor para imageamento de relâmpagos (\textit{Lightning Imaging Sensor} -- LIS). A figura \ref{figtrmm}, ilustra algumas das principais características de varreduras \cite{kummerok1998}.

\begin{figure}[!ht]
  \centering{
  {{\includegraphics[height=14.5cm]{img/TRMM/sensorPackageTraduzido}}}
  }
\caption{Ilustração do satélite TRMM e as  principais características de varredura dos sensores (adaptada de \citeonline{kummerok1998}).}
\label{figtrmm} 
\end{figure} 


\subsection{Radar de Precipitação}

O radar de precipitação (PR) é o principal instrumento do satélite TRMM. Trata-se do primeiro radar meteorológico lançado no espaço sendo a maior inovação apresentada pela missão TRMM. Os objetivos do PR são prover a estrutura tridimensional da precipitação e quantificar as taxas de precipitação sobre os continentes e oceanos.  \cite{kummerok1998}. 


\begin{table}[!ht]
\caption{Principais parâmetros do sinal eletromagnético transmitido e recebido pelo PR (adaptada de\citeonline{kummerok1998,trmmhandbook}).}
\label{PRparametros}
\centering
\small
\newcommand{\grayline}{\rowcolor[gray]{.88}}
\renewcommand {\tabularxcolumn }[1]{ >{\arraybackslash }m{#1}}
\newcolumntype{W}{>{\centering\arraybackslash}X}
\begin{tabularx}{\textwidth}{l W } %{|p{10cm}|X|X|X|X|X|X|X|X| }
\hline\hline 
  Item & Especificações \\[1.5pt]
\hline
\grayline Frequência & 13,796, 13,802 GHz\\[1.5pt]
Sensibilidade & $\leq$0,7 mm h$^{-1}$ (Sinal/Ruído por pulso $\simeq$ 0 dB)\\[1.5pt]
\grayline  Transmissor/Receptor: & \\[1.5pt]
\grayline {~~~~~~~~~} Tipo & SSPA and LNA (128 channels)\\[1.5pt]
\grayline {~~~~~~~~~} Potência máxima & $\geq$500 W \\[1.5pt]
\grayline {~~~~~~~~~} Largura do pulso & 1,6 $\mu$s $\times$ 2 canais \\[1.5pt]
\grayline {~~~~~~~~~} Frequência de repetição do pulso (PRF) & 2776 Hz \\[1.5pt]
\hline 
\end{tabularx}
\end{table}


Os principais parâmetros relacionados ao feixe eletromagnético do PR são listados na tabela \ref{PRparametros}. Em relação à antena do PR, possui uma largura de feixe de 0,71$^{\circ}$ $\times$ 0,71$^{\circ}$ disposta em um painel com abertura de 2,0 m $\times$ 2,0 m. Sua varredura transversal (\textit{cross-track}) de $\pm$17$^{\circ}$ é composta por 49 feixes. Após (antes) a elevação do satélite TRMM, o PR observa uma faixa na superfície de 247 km (215~km) e resolução horizontal no nadir de 5,0 km (4,3~km). Verticalmente o PR registra 80 medidas ao longo de uma faixa de 20 km a partir da superfície, com resolução de 250 m. O PR realiza $\simeq$9150 varreduras por órbita, o que corresponde a uma matriz tridimensional de 49 $\times$ 80 $\times$ 9150   \textit{gates}: 49 feixes na varredura horizontal, com 80 níveis verticais e $\simeq$9150 varreduras horizontais.

As medidas de potência recebida ($P_r$), associadas com a secção transversal de retro-espalhamento sem correção de atenuação e  \textit{ground clutter}, são armazenadas no produto 1B21. A equação do radar conforme as especificações do PR é aplicada nos dados de $P_r$ e são convertidos para o fator de refletividade ($Z_m$) em dBZ, definindo então o produto 1C21.

\simbolo{name={$Z_m$},description={Fator de refletividade sem correção de atenuação e \textit{ground clutter}}}

No produto 2A25 calcula-se a taxa de precipitação (R). É quando faz-se necessário corrigir o efeito de atenuação e \textit{ground clutter} de $Z_m$ obtendo o fator refletividade corrigido ($Z_c$) \cite{iguchi1994atenua,meneghini2000,iguchi2000rain}. No cálculo de R são utilizadas as classificações dos perfis verticais como convectivo, estratiforme e outros do produto 2A23 \cite{awaka1997}. 
\simbolo{name={$Z_c$},description={Fator de refletividade corrigido}}
\simbolo{name={R},description={Taxa de precipitação}}

%Os dados do PR desta pesquisa fazem parte do produto 2A25, que depende do 1C21, 2A21 e 2A23. No produto 1C21, informações a respeito da  degradação do feixe com a distância e do tipo de superfície (oceano, costa ou continente) são adicionadas. A partir do produto 1C21, no produto 2A25, a correção por atenuação do feixe é aplicada nos valores de $Z_m$, quando se obtém os valores de refletividade corrigida por atenuação $Z_c$ bem como a taxa de precipitação em cada instante do campo de visão do PR \cite{iguchi1994atenua,meneghini2000,iguchi2000rain}.   
%O produto 2A25 possui também as informações sobe o tipo de chuva, as quais são utilizadas nesta pesquisa. Esta classificação é gerada no produto 2A23 e incorporadas no 2A25, em que basicamente identifica-se a presença ou não de sinais de banda brilhante no perfil e busca classificar cada perfil vertical de $Z_m$ como convectivo estratiforme ou outros \cite{awaka1997}. 
% \cite{meneghini2000,iguchi2000rain,iguchi2009,PRv7}.
%Considerando as características do TRMM, no produto 
%Para esta pesquisa serão utilizados os dados do produto 2A25, o qual corrige a atenuação da refletividade do radar medida ($Z_m$) \simbolo{name={$Z_m$},description={Refletividade do radar medida}} e a partir do fator de refletividade corrigido por atenuação $Z_c$, estima a estrutura tridimensional da precipitação no instante da observação, bem como a taxa de precipitação em cada célula da resolução (46 $\times$ 80) do PR \cite{PRv7}. 
%Além da estrutura tridimensional da precipitação do produto 2A25, nesta tese, utiliza-se das classificações do tipo de chuva do produto 2A23: convectivo, estratiforme, outros, etc \cite{2A25,PRv7}.   

\sigla{name={PRF},description={Frequência de repetição do pulso, do inglês \textit{Pulse Repetition Frequency}}}
\simbolo{name={$f$},description={Distância focal}}

\subsection{Sensor imageador de raios}

O imageador de raios (LIS) é um sensor óptico capaz de detectar e localizar raios individualmente, a partir da emissão óptica resultante da dissociação, excitação e recombinação dos constituintes atmosféricos durante uma descarga atmosférica. 

O sistema de imageamento do LIS é constituído por um telescópio com razão focal de $f/1,6$ expandindo o feixe luminoso observado, que passa por um filtro de interferência no comprimento de onda de 777,4 nm e com largura de banda de 1 nm e atinge uma matriz de 128 $\times$ 128 CCDs\footnote{O CCD (\textit{charge-coupled device}) é um dispositivo eletrônico que mede corrente elétrica gerada por efeito fotoelétrico amplamente utilizado para obter imagens digitalmente.}. Acoplada a matriz de CCDs, uma lente angular proporciona um campo de visão panorâmico de 80$^{\circ}$ $\times$ 80$^{\circ}$, que corresponde a uma área na superfície terrestre de 668~km~$\times$~668~km depois da elevação, ou 580~km~$\times$~580~km antes da elevação. Um pixel do campo de visão do LIS no nadir possui resolução de 4,3~km depois da elevação e  3,7~km antes da elevação, sendo que, nas regiões mais externas da sua varredura a resolução depois (antes) da elevação atinge 12 km (10,4~km). Para identificar os raios o LIS utiliza um sistema de amostragem que captura 500 imagens por segundo \cite{christian2000LISalgorithm,boccippio1996science,trmmhandbook}. 

%\begin{figure}[!hb]
%  \centering
%  {{\includegraphics[height=6.cm]{img/TRMM/lis_pic.gif}}}
%\caption{LIS}
%\label{Lispic}
%\end{figure} 

Conforme descrito em \citeonline{christian2000LISalgorithm}, a identificação dos raios depende do brilho difuso e transiente observado no topo das tempestades elétricas. Dependendo das posições das CCDs que são sensibilizadas e do intervalo de tempo entre os brilhos subsequentes nas nuvens, o algoritmo de processamento de imagens do LIS identifica eventos, grupos e os raios. Os eventos, correspondem as posições das CCDs em que a irradiação detectada excede a irradiação do plano de fundo que é determinada a cada medida do LIS, os grupos são os agrupamentos de eventos que podem ser interpretados como descargas de retorno -- \textit{strokes} -- e os raios -- \textit{flashes} -- são  agrupamentos espaço-temporal de grupos de eventos. Os raios do LIS correspondem aos raios totais observados na atmosfera, i.e., intranuvens e nuvem-solo.

A figura \ref{LisImagemProcessa} ilustra como é feita a identificação dos  eventos, grupos até a caracterização de um raio. Observe na figura \ref{evgrfla} que em $t=0$ ms, as CCDs, 1, 2 e 3  foram sensibilizadas e o algoritmo definiu o grupo de evento $a$ candidato a ser um raio $A$.  Quando $t=100$ ms, as CCDs, 4, 5 e 6, são sensibilizadas e temos 2 grupos de eventos, $a$ e $b$, espacialmente e temporalmente ($<$330 ms) próximos, portanto, os grupos $a$ e $b$ integram o mesmo raio $A$, como ilustra a figura \ref{evgrflb}. Quando $t = 350$ ms, as CCDs 9 e 10 são sensibilizadas como mostra a figura \ref{evgrflc}. As CCDs 9 e 10 não estão próximas dos grupos $a$ e $b$ que compõe o raio $A$, portanto, o algoritmo define o grupo $d$ e um novo candidato a raio $B$. Na figura \ref{evgrfld} o raio $B$ possui mais dois grupos, $e$ e $f$. Na figura \ref{evgrfle}, note que a CCD 13 coincide com a posição 2, inicialmente em \ref{evgrfla}, mas como o intervalo de tempo entre as figuras \ref{evgrfla} e \ref{evgrfle} é superior a 330 ms, o grupo $g$ também definiu o novo raio $C$ \cite{christian2000LISalgorithm}.  

     
O LIS possui a capacidade de identificar descargas nuvem-solo e intranuvens, tanto no período diurno quanto noturno. Conforme \citeonline{boccippio2002} a eficiência de detecção de raios do LIS é maior no período noturno, com 93$\pm$4\%, enquanto que no período diurno é de 73$\pm$11\%.  Com a velocidade orbital de 11 km s$^{-1}$, o sensor LIS possui um campo de visão que permite a observação de um ponto (\textit{view time}) na Terra por até $\simeq$100 segundos no nadir, tempo suficiente para a estimativa da taxa de raios de uma tempestade \cite{christianTM,trmmhandbook}.

\sigla{name={LIS},description={Sensor imageador de raios, do inglês \textit{Lightning Imaging Sensor}.}}

\begin{figure}[!ht]
  \centering{
  \subfloat[]{{\includegraphics[height=3.3cm]{img/TRMM/time0_ptbr}} \label{evgrfla}}
  \subfloat[]{{\includegraphics[height=3.3cm]{img/TRMM/time100_ptbr}}\label{evgrflb}}
  
  \subfloat[]{{\includegraphics[height=3.3cm]{img/TRMM/time350_ptbr}}\label{evgrflc}}
  \subfloat[]{{\includegraphics[height=3.3cm]{img/TRMM/time400_ptbr}}\label{evgrfld}}

  \subfloat[]{{\includegraphics[height=3.3cm]{img/TRMM/time700_ptbr}} \label{evgrfle}}  
  }
\caption{Ilustração do algoritmo de identificação de eventos, grupos e os raios do LIS \cite{christian2000LISalgorithm}.}
\label{LisImagemProcessa} 
\end{figure} 


\subsection{Radiômetro no visível e infravermelho}

O Radiômetro no visível e infravermelho (VIRS) é um radiômetro de varredura transversal de $\pm$45$^{\circ}$, observando uma faixa de 720~km  na superfície terrestre com uma resolução de 2,11~km no nadir. Após a elevação do satélite, o VIRS passou a observar uma faixa de 833 km com 2,4~km de resolução no nadir. O VIRS mede a radiância em 5 bandas espectrais entre 0,63--12,03~$\mu$m, conforme mostra a tabela \ref{canaisVirs} \cite{kummerok1998}.
%: com comprimentos de onda de 0.63 $\mu$m e  1.61 $\mu$m, faixa do visível; 3.75 $\mu$m, infravermelho próximo; 10.8 $\mu$m e 12 $\mu$m, infravermelho. 

\begin{table}[!ht]
\caption{Canais do VIRS e objetivos das medidas de radiância espectral conforme cada comprimento de onda ($\lambda$) (adaptada de \citeonline{kummerok1998,trmmhandbook}).}
\label{canaisVirs}
\centering
\small
\newcommand{\grayline}{\rowcolor[gray]{.88}}
\renewcommand {\tabularxcolumn }[1]{ >{\arraybackslash }m{#1}}
\newcolumntype{W}{>{\centering\arraybackslash}X}
\begin{tabularx}{\textwidth}{W W W W W W} %{|p{10cm}|X|X|X|X|X|X|X|X| }
\hline\hline 
  & Canal 1 & Canal 2 & Canal 3 & Canal 4 & Canal 5\\[1.5pt]
\hline
\grayline $\lambda$ ($\mu$m) & 0,623$\pm$0,088 & 1,610$\pm$0,055 & 3,784$\pm$0,340 & 10,826$\pm$1,045 & 12,028$\pm$1,055 \\[1.5pt]
%Objetivos & identificar nuvens no período diurno & Identificar diferenças entre água e gelo & Vapor & Temperatura de topo de nuvem& Vapor \\[1.5pt]
\hline 
\end{tabularx}
\end{table}

O conjunto de dados das radiâncias espectrais calibradas, a partir das observações periódicas de referências ópticas como a Lua, o Sol e uma cavidade de corpo negro a bordo do satélite, representam o produto 1B01 \cite{kummerok1998}.

Nesta pesquisa, utilizamos apenas a radiância do canal 4 que corresponde ao comprimento de onda ($\lambda$) de 10,8 $\mu$m, pois considerando a Lei de Planck, os dados de radiância de $\lambda$=10,8 $\mu$m podem ser convertidos em temperatura de brilho ($T_b$) e a emissão do topo das nuvens associadas à emissão de um corpo negro. 

\sigla{name={VIRS},description={Radiômetro no visível e infravermelho, do inglês \textit{Visible and InfraRed Scanner}}}
\simbolo{name={$\lambda$},description={Comprimento de onda}}
\simbolo{name={$T_b$},description={Temperatura de brilho}}
%\subsection{Radiômetro de microondas}

%O TMI (\textit{TRMM Microwave Imager}) é um radiômetro passivo multicanal, 10,65 GHz, 19,35 GHz, 21,3 GHz, 37 GHz, e 85,5 GHz, com dupla polarização. Possui uma varredura cônica combinada com movimento de rotação de sua antena, a qual observa regiões elipsoidais quando projetadas na superfície \cite{kummerok1998}. Sua resolução horizontal varia entre 6-50 km, dependendo do ângulo entre o feixe e o nadir, e varredura de ~760 km \cite{trmmhandbook}. 
%\sigla{name={TMI},description={\textit{TRMM Microwave Imager}}}



\section{REANÁLISES (R2) DO NCEP-DOE}
\index{Reanálises (r2) do NCEP-DOE}

Os dados de altura geopotencial e temperatura são utilizados para converter a altura do feixe do PR em um eixo de temperatura. Para tanto, utiliza-se dos dados das reanálises 2 (R2) do NCEP-DOE.  \sigla{name={R2},description={Reanálises 2 do NCEP-DOE}} 

As reanálises são um conjunto de campos meteorológicos  consistidos dinamicamente e termodinamicamente em um modelo de circulação atmosférica global a partir de dados de radio-sonda, aviões e satélites \cite{kalnay1996ncep}. Os campos disponíveis são: magnitude e direção de ventos, temperatura, umidade relativa, altura geopotencial entre outros.

O projeto R2  -- \textit{NCEP-DOE Atmospheric Model Intercomparison Project (AMIP-II) reanalysis} -- representa correções aplicadas no projeto das reanálises R1, que busca corrigir erros humanos e erros de versões anteriores de modelos atmosféricos utilizados no processo de integração e assimilação que envolvem a construção das reanálises \cite{kanamitsu}.


\sigla{name={R1},description={Reanálises do NCEP-NCAR}}
\sigla{name={NCEP--NCAR},description={\textit{National Centers for Environmental Prediction -- National Center for Atmospheric Research }}}
\sigla{name={NCEP--DOE},description={\textit{National Centers for Environmental Prediction -- Department of Energy}}}
\sigla{name={NCEP},description={\textit{National Centers for Environmental Prediction}}}

\section{DADOS}

Os dados referentes as observações do TRMM foram obtidos a partir do servidor de FTP da NASA (ftp://disc2.nascom.nasa.gov) e do NCEP (ftp://ftp.cdc.noaa.gov).

Foram utilizados os dados de temperatura e altura geopotencial em 17 níveis de pressão das reanálises 2 do NCEP-DOE e os arquivos orbitais do TRMM, produto 1B01 e 2A25 ambos na versão 7, para o período entre janeiro de 1998 e dezembro de 2011. 

%Nesta etapa um conjunto de \textit{scripts} foi desenvolvido para download e verificação de integridade dos dados baixados. No total o volume de dados atingiu 28 TB.  %PR 16,5TB / VIRS 10TB / TMI 1,4TB /   
%órbitas
%primeira (1998) = 00539
%ultima (2011) = 80471
%total = 79932 (observados nos diretórios = 79924) 
%sobre a AS, subset = 63613(VIRS) 61229(PR)           


Os dados do LIS refentes ao tempo de visada (\textit{view time}), eventos, grupos e raios foram concedidos pela pesquisadora \citeonline{rachel}, quem processou estes dados na NASA, projetando os dados do tempo de visada do LIS para uma grade regular com resolução espacial de 0,25$^{\circ}$, próxima à resolução da varredura do PR. Originalmente os dados do tempo de visada do LIS são fornecidos na base de dados do LIS com resolução de 0,5$^{\circ}$. 

%No total, os dados brutos desta pesquisa representaram um volume de  aproximadamente 30 terabytes. 

Para este trabalho de pesquisa os dados do TRMM e da R2 foram amostrados sobre a região limitada entre 10$^{\circ}$~ Norte--40$^{\circ}$~Sul e 90$^{\circ}$--30$^{\circ}$~Oeste, que abrange toda a extensão da América do Sul e oceanos adjacentes. 
%Portanto foi feito um recorte nos dados orbitais apenas para esta região que cobre toda a América do Sul, o que reduziu bastante o volume de dados a serem utilizados e tornou o processamento possível perante a infraestrutura computacional do IAG-USP.

Na tabela \ref{varsTRMM} são apresentadas as medidas extraídas de cada sensor do TRMM e respectiva fonte.


\begin{table}[!h]
\centering
\small
\caption{Variáveis dos produtos do TRMM que foram utilizadas na identificação e descrição das tempestades elétricas.}
\label{varsTRMM}
\renewcommand {\tabularxcolumn }[1]{ >{\arraybackslash }m{#1}}
\newcolumntype{W}{>{\centering\arraybackslash }X}
\begin{tabularx}{\textwidth}{ p{7cm} W W }
\hline
\hline
\textbf{Variável} & \textbf{Sensor TRMM} & \textbf{Produto} \\[1.5pt]
\hline
 Latitude & VIRS & 1B01 \\[1.5pt]
Longitude & VIRS & 1B01 \\[1.5pt]
 Radiância (10,8 $\mu$m) & VIRS & 1B01 \\[1.5pt]
Latitude & PR & 2A25 \\[1.5pt]
 Longitude & PR & 2A25 \\[1.5pt]
Fator de refletividade $Z_c$ & PR & 2A25 \\[1.5pt]
Chuva na superfície  & PR & 2A25 \\[1.5pt]
 Tipo de chuva  &  PR  & 2A23 \\[1.5pt]
Latitude eventos/grupos/raios & LIS &  \cite{rachel} \\[1.5pt]
 Longitude eventos/grupos/raios&LIS& \cite{rachel} \\[1.5pt]
Tempo de visada 0,25$^{\circ}$ $\times$ 0,25$^{\circ}$ & LIS &  \cite{rachel} \\

\hline
\end{tabularx} 
\end{table} 
 

\section{TEMPESTADES ELÉTRICAS}
\label{identificaTempestades}
\index{Tempestades Elétricas}

%Após uma análise ponto a ponto, buscando associar cada raio com um perfil de refletividade do PR, partimos para uma análise de grupo, buscando identificar quais as tempestades elétricas que representam maior intensidade convectiva.

As tempestades elétricas, como já definido anteriormente, são nuvens que durante o seu ciclo de vida apresentaram ao menos um raio.

Dessa maneira, para criar o banco de dados de nuvens de tempestades elétricas deste trabalho de pesquisa, a equação de Planck foi aplicada nos dados de radiância espectral do produto 1B01, canal 4 do VIRS (10,8 $\mu$m) e as regiões com temperatura de brilho ($T_b$) inferiores à 258 K e com pelo menos um raio do LIS observado definiram as áreas das tempestades elétricas ($A_t$) \cite{morales2003}.

\simbolo{name={$A_t$},description={Área da tempestade elétrica}}

% Após, o algoritmo verifica se houve raios detectados pelo LIS na mesma área da nuvem. Havendo pelo menos um raio, o sistema era classificado como uma tempestade elétrica. 

A partir do agrupamento dos sistemas, \textit{clusters} com $T_b \leq 258$ K, foram extraídas as variáveis listadas na tabela \ref{varsTRMM} refentes as observações do PR e LIS para a mesma região em que o \textit{cluster} de tempestade elétrica foi observado.

%técnicas numéricas de mudança de base foram utilizadas para projetar 
Como os sensores do TRMM possuem diferentes resoluções espaciais, as observações orbitais do VIRS, PR e LIS, foram todas projetadas em uma grade regular com 0,05$^{\circ}$ $\times$ 0,05$^{\circ}$ de resolução, de maneira a verificar as medidas do PR, LIS e VIRS para uma mesma tempestade elétrica.

Cada tempestade elétrica identificada foi armazenada na forma de um arquivo HDF contendo medidas coincidentes do VIRS, LIS e PR. 

Inicialmente foram identificadas 154~189 tempestades elétricas. Entretanto, 331 tempestades elétricas não corresponderam a um único sistema convectivo ou multicelular, pois esses núcleos convectivos com raios estavam embebidos em grandes sistemas como Frentes e a ZCAS. 

Ao ajustar a área das tempestades elétricas a área de uma elipse, os 331 sistemas considerados ``enormes" ~possuíram distância focal (2c) maior do que 2500 km. Na figura \ref{nuvem221}, temos a representação de um dos sistemas considerados como ``enorme". 

Nestes 331 sistemas, foi feita uma redefinição do limiar de $T_b$ que define $A_t$, em que $A_t$ passou a ser definida como regiões com $T_b$ $\leq$ 221 K e com pelo menos um raio observado pelo LIS. Regiões com temperatura de brilho em infravermelho inferiores a 221 K são consideradas como a parte mais ativa dos sistemas convectivos de meso-escala identificados em \citeonline{Maddox1980}.  

%Note que, na parte superior e inferior da figura \ref{nuvem221}, há informações referentes a data e hora em que o sistema foi observado, número de raios/eventos (FL/EV), fração do sistema observado pelo PR, área do sistema (A), semieixo maior (a), menor (b), distância focal (2c) e excentricidade (e) de uma elipse ajustada às dimensões do sistema, a qual está plotada sobre a região geográfica. A barra de cores corresponde as temperaturas de brilho do topo da tempestade elétrica. Os valores de FTA e FT na parte superior da figura \ref{nuvem221} serão definidos em \ref{metodoFtaFt}, próxima seção.
% Este sistema foi dividido em 12 sistemas menores, com limiar de temperatura de brilho de 221 K e ocorrência de raio. 

\begin{figure}[hb]
\centering
\includegraphics[height=1.0cm]{img/grids/nucleosRaios/colorbar_virs}\\
\includegraphics[height=11.cm,trim=0 47cm 0 0,clip]{img/topSevero/EnormesInvalidas/018_Enormes_40791_0001}
\caption{Nuvem de tempestade elétrica considerada ``enorme". Na parte superior e inferior da figura, há informações referentes a data e hora em que o sistema foi observado, número de raios e eventos (FL/EV), fração do sistema observado pelo PR, área do sistema (A), semieixo maior (a), menor (b), distância focal (2c) e excentricidade (e) de uma elipse ajustada às dimensões do sistema, a qual está plotada sobre a região geográfica. A barra de cores corresponde as temperaturas de brilho em infravermelho (VIRS 10,8~$\mu$m) do topo da tempestade elétrica.}  
\label{nuvem221}
\end{figure}

Com a recategorização dos sistemas ``enormes", o número total de tempestades elétricas que passaram a integrar esta pesquisa é de 157~592.

\simbolo{name={A},description={Nas figuras que ilustram as tempestades elétricas, refere-se à área total ($A_t$) da tempestade elétrica.}}
\simbolo{name={a},description={Semieixo maior}}
\simbolo{name={b},description={Semieixo menor}}
\simbolo{name={2c},description={Distância focal}}
\simbolo{name={e},description={Excentricidade}}


\section{SEVERIDADE: TAXA DE RAIOS}
\label{metodoFtaFt}
\index{FTA}
\index{FT}

Condições de tempo severo, como frentes de rajadas, queda de granizo e tornados estão associados com um aumento abrupto na taxa de raios total das tempestades elétricas, principalmente governado por raios intra-nuvens \cite{macgorman1989, carey1998, williams1999, gatlin2010, schultz2011}.   

Portanto, a taxa de raios ([min$^{-1}$]) do LIS observada sobre a área que define uma tempestade elétrica pode indicar condições de tempo severo. Os sistemas precipitantes, (\textit{precipitation features} -- PFs) definidos em \citeonline{Nesbitt2000}, com as maiores 
taxas de raios por minuto, conforme mostram  \citeonline{cecil2005, zipser2006}, corresponderam aos sistemas com os maiores volumes de chuvas, mínimas temperaturas de brilho em mico-ondas, máximos valores de refletividade do PR, e de acordo com estes autores pode ser indicativo da presença de fortes correntes ascendentes. 

Neste trabalho de pesquisa a taxa de raios das tempestades elétricas será avaliada a partir de dois índices:

\begin{itemize}
\item FT -- A taxa de raios no tempo, sendo a razão entre o número de raios ($N_{fl}$) e o tempo médio ($VT_m$) de observação do LIS sobre a tempestade elétrica, conforme descreve a equação \ref{eqFT}.
\end{itemize}

\begin{equation}
FT = \frac{N_{fl} }{VT_m} 60 ~[\mathrm{min^{-1}}]  
\label{eqFT}  
\end{equation}
%31557600 ano

\begin{itemize}
\item FTA -- A taxa de raios por tempo normalizada pela área da tempestade elétrica, sendo a razão entre o número de raios ($N_{fl}$), o tempo médio ($VT_m$) observação do LIS e a extensão em área $A_t$ da tempestade elétrica observada, conforme descreve a equação \ref{eqFTA}.
\end{itemize}

\begin{equation}
FTA = \frac{N_{fl} }{VT_m A_t } 60 ~[\mathrm{min^{-1}~km^{-2}}]
\label{eqFTA}
\end{equation}

\simbolo{name={$N_{fl}$},description={Número de flashes }}
\simbolo{name={$VT_m$},description={Tempo médio de visada do LIS}}


Note que o fator 60 que multiplica tanto a equação \ref{eqFT} quanto a \ref{eqFTA} é aplicado para converter o tempo de visada do LIS de segundos ($VT_m$) para minutos de observações, pois, em geral a severidade é quantificada em raios por minuto.

%Para cada tempestade elétrica foram calculados os dois índices que podem estar associados com a severidade, o FT e FTA, conforme as equações \ref{eqFT} e \ref{eqFTA}. 

\simbolo{name={$FT$},description={Taxa de raios por tempo $[raios~minuto^{-1}]$}} \simbolo{name={$FTA$},description={Taxa de raios por tempo por área $[raios~dia^{-1}~km^{-2}]$}}.

\section{DISTRIBUIÇÕES GEOGRÁFICAS DE DENSIDADES}
\label{metodoPass}
% DE RAIOS E TEMPESTADES ELÉTRICAS

As distribuições geográficas são estudadas a partir da densidade de raios, densidade de tempestades elétricas e densidade de raios por tempestades elétricas. Estas três grandezas de densidades buscam mostrar as regiões de maior e menor ocorrência de raios e de tempestades elétricas e também identificar os locais em que as tempestades elétricas produzem mais (menos) raios por número de tempestades, i.e., apresentam processos de eletrificação mais (menos) eficazes. 

O que se torna fundamental na construção destes mapas da distribuição geográfica de densidades de ocorrências é considerar quantas vezes ou qual o tempo que o satélite ficou observando cada parte da região de estudo, pois latitudes mais altas possuem maior número de amostragens do que em latitudes mais baixas por causa das características da orbita baixa do satélite TRMM. Portanto, qualquer análise de densidade geográfica relacionada às observações do TRMM que não considere o número de passagens ou tempo em que o sensor observou a região projetada na superfície será tendenciosa. 

Mesmo que o satélite TRMM visite o mesmo lugar do globo duas vezes por dia em função de sua órbita com altitude de 402,5~km (350~km antes da elevação) e inclinada 35$^{\circ}$ \cite{kummerok1998}, entre o período de 1998--2011, o satélite passou {10 000} vezes mais sobre a região extra-tropical ($\geq$ 20$^{\circ}$ latitude) do que na região tropical, como mostra a figura \ref{VirsVT} que apresenta o número de órbitas sobrevoadas pelo VIRS em cada ponto da grade de 0,25$^{\circ}$  $\times$ 0,25$^{\circ}$ de latitude e longitude na América do Sul. Este efeito decorre da região onde o satélite atinge a latitude máxima.

\begin{figure}[!hb]
  \centering
  {{\includegraphics[height=13.5cm]{img/grids/passagens_virs_1998-2011}}}
\caption{Número de sobrevoos do VIRS em cada ponto da grade de 0,25$^{\circ}$  $\times$ 0,25$^{\circ}$ de latitude e longitude entre 1998--2011, ilustrando a matriz $\mathbf{VT}_{virs}$.}
\label{VirsVT}
\end{figure} 

%\label{gridAmostragem} 

Agora levando em consideração o tempo de amostragem do LIS (\textit{view time}), figura \ref{lisVT}, o número de dias de amostragem em cada ponto da grade regular (0,25$^{\circ}$  $\times$ 0,25$^{\circ}$) projetada sobre AS revela que durante os 14 anos o LIS observou 10 dias a mais na latitude 34$^{\circ}$ Sul do que em 0$^{\circ}$. Logo, se estas regiões com maior tempo de amostragem forem eletricamente ativas, é de se esperar um alto número de raios observados se comparadas com 0$^{\circ}$ de latitude.

\begin{figure}[!ht]
  \centering
  {{\includegraphics[height=13.5cm]{img/grids/vt_trmm}} }
  \caption{Tempo de amostragem (\textit{view time}) do LIS ente 1998--2011, para cada ponto da grade de  0,25$^{\circ}$  $\times$ 0,25$^{\circ}$ de latitude e longitude, ilustrando a matriz $\mathbf{VT}_{lis}$. }
\label{lisVT}
\end{figure} 

%Portanto, as figuras \ref{lisVT} e \ref{VirsVT} representam duas matrizes que correspondem aos pontos de uma grade com 0,25$^{\circ}$ $\times$ 0,25$^{\circ}$ de latitude e longitude sobre a América do Sul. são utilizadas para normalizar as medidas de raios observados pelo LIS e tempestades elétricas definidas por meio do canal 4 do VIRS.  

Portanto, as figuras \ref{lisVT} e \ref{VirsVT} representam duas matrizes: $\mathbf{VT}_{lis}$ (figura \ref{lisVT}) referente ao tempo de amostragem do LIS sobre cada ponto de grade (0,25$^{\circ}$  $\times$ 0,25$^{\circ}$ de latitude e longitude) e  $\mathbf{VT}_{virs}$ (figura \ref{VirsVT}), refente ao número de vezes que o VIRS sobrevoou cada ponto de grade (0,25$^{\circ}$  $\times$ 0,25$^{\circ}$ de latitude e longitude), as quais são utilizadas para normalizar as medidas de raios observados pelo LIS e tempestades elétricas definidas por meio do canal 4 do VIRS ($T_b$ $\leq$ 258 K).  

\simbolo{name={$\mathbf{VT}_{lis}$},description={Matriz do tempo total da visada do sensor LIS sobre a superfície}}

As medidas tanto de raios do LIS quanto de regiões sobre a superfície terrestre com ocorrência de tempestades elétricas definidas pelo VIRS ($T_b$ $\leq$ 258 K), também foram projetadas em uma grade de  0,25$^{\circ}$ $\times$ 0,25$^{\circ}$ de latitude e longitude, sobre à região de estudo. Foi obtida a matriz $\mathbf{FL}_{lis}$ em que cada posição da matiz corresponde a um ponto da grade (0,25$^{\circ}$ $\times$ 0,25$^{\circ}$ de latitude e longitude) com o acumulado de raios ocorridos entre 1998--2011, bem como foi obtida a matriz $\mathbf{P}_{te}$ com o total de tempestades elétricas ocorridas em cada ponto da grade (0,25$^{\circ}$ $\times$ 0,25$^{\circ}$ de latitude e longitude) sobre a AS.


%Com as mesmas dimensões e resolução de grade que o tempo de observação e o número de passagens do satélite foram acumulados em duas matrizes, os raios foram acumulados na matriz ($\mathbf{FL}_{lis}$) e todos os píxeis do VIRS com radiância espectral associada com temperaturas de brilho inferiores a 258 K e que definiram as áreas das tempestades elétricas, foram acumulados na matriz ($\mathbf{P}_{te}$) que representa os locais com maior cobertura de nuvens de tempestades elétricas.
%A matriz $\mathbf{FL}_{lis}$ projeta sobre a América do Sul está representada na figura \ref{gridFL} e a matriz $\mathbf{P}_{te}$, na figura \ref{gridTe}.

%A matriz $\mathbf{FL}_{lis}$ é ilustrada na figura \ref{taxatotalraios}, e a matriz $\mathbf{P}_{te}$ na figura \ref{taxaTotalTe}. 

Na figura \ref{taxaTotalTe}, que ilustra a matriz $\mathbf{P}_{te}$, há uma extensa região na parte Sul da AS em que se observa um elevado número de sistemas, mas este máximo não indica maior ocorrência de tempestades elétricas e sim maior frequência de passagem do satélite TRMM. No entanto, a ilustração da matriz $\mathbf{FL}_{lis}$ como mostra a figura \ref{taxatotalraios}, revela que no extremo norte da cordilheira dos Andes, mesmo que o satélite sobrevoe menos esta região próxima da linha do equador, os acumulados de raios são maiores do que sobre a região Sul da AS. 

%Portanto, faz-se necessário normalizar as medidas do TRMM pelo tempo de amostragem ou número de passagens.

\begin{figure}[!ht]
  \centering
  \subfloat[]{{\includegraphics[height=13.5cm]{img/grids/densEspacial_19982011acumuladoFlashPolyfill}} } 
\caption{Número total de raios (\textit{flashes}) observados pelo LIS em cada região de 0,25$^{\circ}$  $\times$ 0,25$^{\circ}$ de latitude e longitude entre 1998--2011, ilustrando a matriz $\mathbf{FL}_{lis}$.}
\label{taxatotalraios}
\end{figure}   
  
\begin{figure}[!ht]
  \centering 
  {{\includegraphics[height=13.5cm]{img/grids/densEspacial19982011acumuladoTempestadesPolyfill}}}
\caption{Número total de tempestades elétricas observadas em cada região de 0,25$^{\circ}$ $\times$ 0,25$^{\circ}$ de latitude e longitude entre 1998--2011, ilustrando a matriz $\mathbf{P}_{te}$.}
\label{taxaTotalTe}
\end{figure} 

Devemos também levar em consideração que os pontos de uma grade regular em graus de latitude e longitude não possuem áreas iguais, pois o comprimento de arco de 0,25$^{\circ}$ na direção zonal depende da latitude da região. Assim a matriz que corresponde à área da grade regular ($\mathbf{A}_g$) foi calculada e considera nos cálculos de densidades.

Portanto a partir das grandezas $\mathbf{FL}_{lis}$, $\mathbf{P}_{te}$, $\mathbf{VT}_{lis}$, $\mathbf{VT}_{virs}$ e $\mathbf{A}_g$, podemos determinar a densidade de raios $\mathbf{DE}_{fl}$, a densidade de tempestades elétricas $\mathbf{DE}_{te}$ e a densidade raios por tempestades elétricas $\mathbf{DE}_{rt}$. 

A densidade de raios $\mathbf{DE}_{fl}$ é calculada conforme a equação \ref{defl}, que apresenta a razão entre $\mathbf{FL}_{lis}$, $\mathbf{VT}_{lis}$ e $\mathbf{A}_g$ multiplicada por 24 horas $\times$ 60 minutos $\times$ 60 segundos $\times$ 365,25 dias, o que converte o tempo de observação do LIS originalmente em segundos para anos. A densidade de raios, portanto, é uma grandeza que representa o número de raios por ano por quilômetro quadrado ([ano$^{-1}$] [km$^{-2}$]).

\begin{equation}
\mathbf{DE}_{fl} = \frac{\mathbf{FL}_{lis}}{\mathbf{VT}_{lis} \mathbf{A}_g} 31557600     
\label{defl}
\end{equation}
\index{Distribuição geográfica de densidade! de raios}


Analogamente a densidade de tempestades elétricas ($\mathbf{DE}_{te}$) é obtida. Porém, conforme descrito em \ref{metodologiaTRMM}, o tempo de amostragem do LIS e do VIRS são distintos. Enquanto o LIS é um sistema de imageamento, o VIRS é um radiômetro que realiza varreduras durante a trajetória do satélite. Portanto, na obtenção da densidade de tempestades elétricas é considerado o número de sobrevoos do VIRS em cada ponto da grade de 0,25$^{\circ}$  $\times$ 0,25$^{\circ}$ de latitude e longitude. Desta forma, $\mathbf{DE}_{te}$ é obtida conforme a equação \ref{dete}, que define uma grandeza que representa o número de tempestades elétricas por órbita  por quilômetro quadrado ([km$^{-2}$]).

%Na figura \ref{VirsVT} temos apenas o acumulado de passagens do VIRS, porém, é conveniente converter o número de vezes que o VIRS observou cada ponto da grade de 0.25$^{\circ}$  $\times$ 0.25$^{\circ}$ em unidade de tempo, pois desta forma teremos as mesmas dimensões físicas tanto para a $\mathbf{DE}_{te}$ quanto para a $\mathbf{DE}_{fl}$.
%Conforme descrito em \ref{metodologiaTRMM}, em função da velocidade e órbita do satélite TRMM, é possível sobrevoar uma mesma região tropical duas vezes por dia. Então podemos considerar que cada ponto da grande de 0.25$^{\circ}$  $\times$ 0.25$^{\circ}$ da figura \ref{VirsVT} é observado pelo VIRS 2 vezes por dia, portanto se multiplicarmos 2 passagens por 365.25 dias podemos concluir o satélite observa cada ponto da grade aproximadamente 730.5 vezes por ano.
%Porém a constante de conversão de tempo na equação \ref{dete} é diferente da equação \ref{defl}, pois o tempo que o VIRS observou a AS, foi estimado a partir do número de vezes que o satélite passou sobre a AS e considerando que cada ponto de grade na orbita foi observado por 90 segundos. 



% a densidade de tempestade elétrica $\mathbf{DE}_{te}$, como descreve a equação \ref{dete} é normalizada pelo número de vezes que o VIRS sobrevoou cada ponto da grade de 0.25$^{\circ}$  $\times$ 0.25$^{\circ}$, o que corresponde a dimensão física do número de [sistemas] por  [observação] por [quilômetro quadrado] (sistemas observaçoes$^{-1}$ km$^{-2}$). Essa grandeza revela, por exemplo, que a cada 10,000 observações do VIRS, temos entre 1-4 tempestades elétricas observadas na América do Sul.

\begin{equation}
\mathbf{DE}_{te} = \frac{\mathbf{P}_{te}}{\mathbf{VT}_{virs} \mathbf{A}_g}    
\label{dete}
\end{equation}
\index{Distribuição geográfica de densidade! de tempestades}

A partir da metodologia do cálculo de $\mathbf{DE}_{fl}$, equação \ref{defl} e $\mathbf{DE}_{te}$, equação \ref{dete}, cria-se a densidade raios por tempestades elétricas $\mathbf{DE}_{rt}$, em que 

\begin{equation}
\mathbf{DE}_{rt} = \frac{\mathbf{FL}_{lis}}{\mathbf{VT}_{lis} \mathbf{A}_g\mathbf{P}_{te}} 31557600.  
\label{dert}
\end{equation}

\simbolo{name={$\mathbf{DE}_{rt}$},description={Matriz de densidade de raios por tempestades}}

\index{Distribuição geográfica de densidade! de de raios por tempestades}

\section{MORFOLOGIA DA ESTRUTURA TRIDIMENSIONAL DA PRECIPITAÇÃO}
\label{metodologiaPrec3D}

O estudo para descrever a morfologia da precipitação foi realizado com base nas observações do PR, buscando avaliar como a precipitação está distribuída nos níveis de altitude e como os perfis de $Z_c$ estão associados com os processos de crescimento de hidrometeoros e consequentemente de eletrificação.  

\subsection{Distribuições de probabilidades com a altitude}

A partir dos perfis de $Z_c$ selecionados pela metodologia de identificação de tempestades elétricas, conforme descrita em \ref{identificaTempestades}, foi estudada a probabilidade de ocorrência de $Z_c$ por altitude através dos Diagramas de Contorno de Frequência por Altitude (CFAD, do inglês, \textit{Contour Frequency by Altitude Diagram}) \cite{yuter1995}.
\sigla{name={CFAD},description={Diagrama de Contorno de Frequência por Altitude, do inglês, \textit{Contour Frequency by Altitude Diagram}}}
\index{Diagrama de Contorno! de Frequência por Altitude -- CFAD}

Conforme descrevem \citeonline{yuter1995}, primeiramente obteve-se uma função de densidade de probabilidade com duas variáveis ($f_{pdf}(x,y)$), cuja  dimensão $x$ correspondeu aos valores de $Z_{c}$ e $y$ os níveis de altitude do PR. Neste estudo, a função $f_{pdf}(x,y)$ foi representada numericamente por uma matriz bidimensional com a granularidade de 1 dBZ para cada 250 m de altitude. Para a obtenção dos diagramas de probabilidade normalizados por nível de altitude -- CFADs --, cada nível $y$ da função $f_{pdf}(x,y)$ foi normalizado pelo número total de ocorrências de $Z_c$. 

Os níveis $y$ de altitude com número total de ocorrências de $Z_c$ menor do que 10\% do nível $y$ com a máxima ocorrência de $Z_c$ foram desconsiderados de todos os CFADs, pois quando a amostragem de valores de $Z_c$ é baixa em relação ao nível $y$ de máxima ocorrência de $Z_c$ (níveis mais elevados de $y$ $\simeq$15--20 km),  pode-se obter valores de probabilidade próximos de 100\% e que não são representativos para os CFADs \cite[Apêndice~A]{yuter1995}.

\simbolo{name={$f_{pdf}(x,y)$},description={Função densidade de probabilidade com duas variáveis}}

Com base na $f_{pdf}(x,y)$ que definiu cada CFAD, foi calculada a função densidade de probabilidade cumulativa $f_{cdf}(x,y)$ de $Z_c$ por altitude, que originaram os Diagramas de Contorno de Frequência Cumulativa por Altitude (CCFAD\footnote{Sigla criada a partir do inglês, \textit{Contour Cumulative Frequency by Altitude Diagram}}).  \sigla{name={CCFAD},description={Diagramas de Contorno de Frequência Cumulativa por Altitude, do inglês, \textit{Contour Cumulative Frequency by Altitude Diagram} }}   

\index{Diagrama de Contorno! de Frequência Cumulativa por Altitude -- CCFAD}

\simbolo{name={$f_{cdf}(x,y)$},description={Função densidade de probabilidade cumulativa com duas variáveis}}

Os CCFADs auxiliam a investigar quais as diferenças entre os perfis de $Z_c$ associados a diferentes percentis da amostra de probabilidade, elucidando as informações contidas nos CFADs.	

%Então, objetivando uma análise dos processos de crescimento de hidrometeoros no perfil atmosférico e na mesma óptica de trabalhos como \citeonline{Takahashi1978,Saunders1999,Takahashi2002,avila2009}, ou seja, em função de diferentes condições de temperatura, nesta pesquisa construímos o diagrama denominado como Diagrama de Contorno de Frequência por Temperatura (CFTD), \sigla{name={CFTD},description={Diagrama de Contorno de Frequência por Temperatura}}.

\subsection{Distribuições de probabilidades em função da  temperatura}

A distribuição vertical da precipitação está associada com o desenvolvimento vertical, porém o tipo de hidrometeoro de uma  determinada altitude é função da temperatura, probabilidade de colisão e da razão de saturação naquela altura \cite[p.~263]{mason1971_2ed}. Os processos de eletrificação por outro lado dependem do conteúdo de água líquida, temperatura, velocidade terminal, velocidade vertical, tamanho e tipo das partículas \cite{Takahashi1978,Saunders1999,Takahashi2002,avila2009}.

Portanto, saber como o perfil da precipitação varia com a temperatura é preponderante para identificar qual mecanismo de crescimento dos hidrometeoros está dominando conforme o desenvolvimento e a severidade das tempestades elétricas. Neste sentido foi elaborado o Diagrama de Contorno de Frequência por Temperatura (CFTD). \sigla{name={CFTD},description={Diagrama de Contorno de Frequência por Temperatura}} 

\index{Diagrama de Contorno! de Frequência por Temperatura -- CFTD}


%Nos CFTDs, os níveis de temperatura não correspondem as condições controladas em laboratório, e sim às variações de temperatura do perfil atmosférico. 

Para converter os níveis de altitude do PR em níveis de temperatura, foram utilizados os dados de reanálises (R2) do NCEP--DOE entre 1998--2011, em 17 níveis de pressão, da altura geopotencial e temperatura, correspondentes aos horários mais próximos e anteriores ao horário de cada tempestades elétrica observada pelo TRMM.

Os perfis de altura geopotencial e temperatura mais próximos ou coincidentes com cada região de tempestade elétrica observada pelo VIRS foram extraídos. A partir dos 17 níveis verticais das R2, os 80 níveis de temperaturas associados aos 80 níveis de altitude das observações do PR foram calculados através de um método de mínimos quadrados.

Desta maneira, obteve-se a função $f_{pdf}(x,y)$, cuja a dimensão $x$ correspondeu à valores de $Z_{c}$ e $y$ os nível de temperaturas estimados a partir das R2 do NCEP--DOE. A função $f_{pdf}(x,y)$ de $Z_c$ por temperatura, foi representada por uma matriz bidimensional com a granularidade de 1 dBZ para cada 2 $^{\circ}$C. Nos CFTDs, os níveis superiores e inferiores foram definidos para temperaturas entre 20° C e -50° C.



Também foi calculada a função $f_{cdf}(x,y)$ de $Z_c$ por temperatura, que originaram os Diagramas de Contorno de Frequência Cumulativa por Temperatura (CCFTD).  \sigla{name={CCFTD},description={Diagrama de Contorno de Frequência Cumulativa por Temperatura}}   

\index{Diagrama de Contorno! de Frequência Cumulativa por Temperatura -- CCFTD}
